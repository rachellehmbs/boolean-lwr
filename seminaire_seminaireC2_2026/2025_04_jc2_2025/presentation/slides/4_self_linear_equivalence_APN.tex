% !TEX root = ../slides_jc2.tex
\section[A unified PoV on the known APN functions]{A unified point-of-view on the known APN functions}

{
\setbeamercolor{background canvas}{bg=ptblue}	
\begin{frame}[plain]
\vfill
\begin{center}
%\color{white} \Huge \Roman{section} -  \secname
\color{white} \Huge \secname
\end{center}
\vfill
\end{frame}
}


\subsection{One of the first non-power functions}

\begin{frame}
\frametitle{\subsecname}
\vspace{-.6cm}
% \begin{mybox}{Power function}{}{\small\purple{[Gold68, Nyberg94]}}
% $$F \from \FF_{2^{12}} \to \FF_{2^{12}} \quad x \mapsto x^{3}$$
% % $\blue{\lambda} \in \FFfield^{*}$. $F(\blue{\lambda} x) = \blue{\lambda}^{3} F(x)$ $\leadsto$ $\cyan{A} \comp F \comp \cyan{B} = F$ \hfill $\cyan{B}(x) = \blue{\lambda} x$, $\cyan{A}(x) = \blue{\lambda}^{-3} x$
% \end{mybox}
% \pause
\begin{mybox}{An APN binomial}{}{\small \purple{[BudCarLea08]}}
$$ G \from \orange{\FF_{2^{12}}} \to \orange{\FF_{2^{12}}} \quad x \mapsto x^3 + \orange{\alpha} x^{528}$$

$G(x) = x^{3}(1 + x^{525}) = x^{3}P(x^{15})$, where $P = 1+ X^{35}$ \hfill($525 = 35\times 15$)
%$\hspace*{5.5cm}F(x) =  x^3 + \alpha x^{528} = x^{3}P(x^{15}) \hfill P = 1 + x^{35}$


% \onslide<3->{
% \vspace{.3cm}
% $\blue{\lambda} \in \blue{\FF_{2^{4}}^{*}}$ (i.e. $\blue{\lambda}^{15} = 1$).  \hspace*{1.5cm} $F(\blue{\lambda})  =\blue{\lambda}^{3}P(\blue{\lambda}^{15}) = \blue{\lambda}^{3}P(1)$
% }


% \onslide<4->{
% \vspace{.3cm}
% \bulletpoint $F$ behaves as $x \mapsto x^{\red{3}}$ on each coset $\gamma\blue{\FF_{2^{4}}}$
% }

% \onslide<5->{
% \vspace{.1cm}
% \bulletpoint Multivariate point of view\hfill
% $\widetilde{F} \from (\blue{\FF_{2^{4}}})^{3} \to (\blue{\FF_{2^{4}}})^{3}\  (v_{1}, v_{2}, v_{3}) \mapsto \left(\widetilde{F_{1}}(v), \widetilde{F_{2}}(v), \widetilde{F_{3}}(v)\right)$
% }



% \onslide<6->{
% \vspace{.1cm}
% \hfill All coordinates $\widetilde{F}_{i}$ are \emph{homogeneous} of the \emph{same} \red{order 3}
% }

\end{mybox}\pause

\vspace{.2cm}
$\purple{\FF_{2^{4}}^{*}} \subset \orange{\FF_{2^{12}}^{*}}$. %$\quad\FF_{2^{12}}^{*} = \bigsqcup\limits_{\gamma \in \Gamma} \gamma\purple{\FF_{2^{4}}^{*}}$ for some system of representatives $\Gamma$.
\pause

$$\forall \ \purple{\phi} \in \purple{\FF_{2^{4}}^{*}}, \quad G(\purple{\phi}) = \purple{\phi}^{3}P(\purple{\phi}^{15}) = \purple{\phi}^{3}P(1).$$\pause
\vspace{.2cm}

\begin{proposition}
    For any $\gamma \in  \orange{\FF_{2^{12}}^{*}}$, the restriction of $G|_{\gamma\purple{\FF_{2^{4}}^{*}}}$ is (up to a constant) the power mapping $x \mapsto x^{3}$.
\end{proposition}


% \onslide<7->{
% \begin{mybox}{One of the first bivariate functions}{}{\purple{\small[ZhoPot13]}}
% $\hspace*{5.5cm}F \from \FF_{64}^{2} \to \FF_{64}^{2}, (x, y) \mapsto (xy, x^{3} + ay^{3})$


% % \vspace{.2cm}
% % $F_{0}$ homogeneous of order 2, $F_{1}$ homogeneous of order 3

% % \vspace{.2cm}
% % $\leadsto \cyan{A} \comp F \comp \cyan{B} = F$ with $\quad \cyan{B}(x, y) = (\blue{\lambda} x, \blue{\lambda} y), \cyan{A}(x, y) = (\blue{\lambda}^{-2} x, \blue{\lambda}^{-3} y)$
% \end{mybox}
% }

\end{frame}
\begin{frame}\frametitle{The multiplicative point of view}

\begin{mybox}{An APN binomial}{}{\small \purple{[BudCarLea08]}}
\begin{itemize}
    \item[\bulletpoint] $G \from \orange{\FF_{2^{12}}} \to \orange{\FF_{2^{12}}} \quad x \mapsto x^3 + \orange{\alpha} x^{528}$
    \item[\bulletpoint] $G|_{\purple{\FF_{2^{4}}}} : \purple{\phi} \mapsto c \purple{\phi}^{3}$
\end{itemize}
\end{mybox}\pause

\begin{mybox}{Multivariate point-of-view}{}{}
$G$ is linearly equivalent to $\widetilde{G} \from (\purple{\FF_{2^{4}}})^{3} \to (\purple{\FF_{2^{4}}})^{3}\  (x_{1}, x_{2}, x_{3}) \mapsto \left(\widetilde{G_{1}}(x), \widetilde{G_{2}}(x), \widetilde{G_{3}}(x)\right)$.

$$\widetilde{G}_{1}(x) = {\color{ptpurple!50}?}x_{1}^2x_{2} + {\color{ptpurple!50}?}x_{1}x_{2}^{2} + {\color{ptpurple!50}?}x_{2}^{3}+ {\color{ptpurple!50}?}x_{1}^2x_{3} + {\color{ptpurple!50}?}x_{2}^2x_{3} + {\color{ptpurple!50}?}x_{1}x_{3}^2 + {\color{ptpurple!50}?}x_{2}x_{3}^2 + {\color{ptpurple!50}?}x_{3}^3. $$

All coordinates of $\widetilde{G}$ are homogeneous of the same degree $3$.
\end{mybox}\pause

\begin{mybox}{An APN bivariate functions}{}{\purple{\small[ZhoPot13]}}
$$H \from \green{\FF_{64}^{2}} \to \green{\FF_{64}^{2}}, (x, y) \mapsto (xy, x^{3} + \green{a}y^{3})$$

\begin{itemize}
    \item[\bulletpoint{}] $H_{1}$ homogeneous of order 2.
    \item[\bulletpoint{}] $H_{2}$ homogeneous of order 3.
\end{itemize}
\end{mybox}


\end{frame}


\subsection{Linear self-equivalence}
\begin{frame}
\frametitle{\subsecname}
\vspace{-.7cm}
\begin{mybox}{Power mapping}{}{}
$$F(x) = x^{e}$$
Let $ \blue{\lambda} \in \FFfield^{*}$. Then for all $x$, $F(\blue{\lambda} x) = \blue{\lambda}^{e}x^{e} = \blue{\lambda}^{e}F(x)$.

Thus
 $\orange{A} \comp F \comp \orange{B} = F\ $ \hfill with $\orange{B}(x)\vcentcolon= \blue{\lambda} x$, $\ \orange{A}(x) \vcentcolon= \blue{\lambda}^{-e} x$. 
\end{mybox}
\pause

\vspace{-.1cm}
\begin{mybox}{Cyclotomic mapping w.r.t a subfield}{}{\purple{\small[Wang07]}}
$$ G(x) = x^{e}P\left(x^{2^{k} - 1}\right), n = \ell k$$
Let $ \purple{\phi} \in \purple{\FF_{2^{k}}}$. Then for all $x $, $G(\purple{\phi} x) = \purple{\phi}^{e}x^{e}P\left(x^{2^{k}-1}\right) = \purple{\phi}^{e}G(x)$.

Thus
 $\orange{A} \comp G \comp \orange{B} = G\ $ \hfill with  $ \orange{B}(x)\vcentcolon= \purple{\phi} x$, $\ \orange{A}(x) \vcentcolon= \purple{\phi}^{-e} x$. 


% $\cyan{A} \comp F \comp \cyan{B} = F$ with $\quad \cyan{B}(x) = \blue{\lambda} x$, $\quad\cyan{A}(x) = \blue{\lambda}^{-e} x$ for any $\lambda \in \blue{\FF_{2^{k}}^{*}}$

% \vspace{.2cm}
% $\widetilde{\cyan{A}} \comp \widetilde{F} \comp \widetilde{\cyan{B}} = \widetilde{F}$ with $\quad\widetilde{\cyan{B}}(v) = (\blue{\lambda} v_{1}, \dotsc , \blue{\lambda} v_{\ell}),$ $\quad \widetilde{\cyan{A}}(v) = (\blue{\lambda}^{-e} v_{1}, \dotsc , \blue{\lambda}^{-e} v_{\ell})$
\end{mybox}
\pause

\vspace{-.1cm}
\begin{mybox}{$\ell$-projective mapping}{}{\purple{\small[\blue{B}CP24,Göloğlu22]}}

$$ H \from \FFinter \to \FFinter \ (x_{1}, \dotsc, x_{\ell}
) \mapsto (H_{1}(x), \dotsc, H_{\ell}(x)),$$
$ \forall \ i$, $H_{i}$ is homogeneous of order $\cyan{e_{i}}$. 

Thus $\orange{A} \comp H \comp \orange{B} = H$ \hfill with $\quad\orange{B}(x) = (\purple{\phi} x_{1}, \dotsc , \purple{\phi} x_{\ell}),$

\hfill $\orange{A}(x) = (\purple{\phi}^{-\cyan{e_{1}}} x_{1}, \dotsc , \purple{\phi}^{-\cyan{e_{\ell}}} x_{\ell})$
\end{mybox}
\end{frame}

\subsection{Our main result (1/2)}
\begin{frame}
\frametitle{\subsecname}

\begin{center}
\vspace{-.5cm}
Among the 22 known infinite APN families, 19 consist entirely of 
\emph{cyclotomic} or \emph{$\ell$-projective} mappings, \emph{up to linear equivalence}.
% !TEX root = ../slides.tex
\renewcommand\arraystretch{1.3} 
\begin{tabular}{c<{\onslide<2->}c<{\onslide}}
% \visible<2->{\toprule}
\rctt\textbf{Univariate} &  \textbf{Observations}\\
% \midrule
$x^{2^s + 1} + ax^{2^{(3-i)k + s} + 2^{ik}}$ & cyclotomic\\
\rct$x^{2^s + 1} + ax^{2^{(4-i)k + s} + 2^{ik}} $ & cyclotomic\\
$ax^{2^k + 1} + x^{2^s +1} + x^{2^{s + k} + 2^k} + bx^{2^{k + s} + 1} + b^{2^{k}}x^{2^s + 2^k}$ & $\lin$  biprojective\\
\rct$x^{3} + a^{-1}\tr[\FF_{2^n}][\FF_2](a^3x^9)$ & cyclotomic/($\lin$) frob.\\
$x^{3} + a^{-1}\tr[\FF_{2^n}][\FF_{2^3}](a^3x^9 + a^6x^{18})$ & cyclotomic/($\lin$) frob.\\
\rct$x^{3} + a^{-1}\tr[\FF_{2^n}][\FF_{2^3}](a^6x^{18} + a^{12}x^{36})$ &  cyclotomic/($\lin$) frob.\\
$ax^{2^s + 1} + a^{2^k}x^{2^{2k} + 2^{k + s}} + bx^{2^{2k} + 1} + ca^{2^k + 1}x^{2^{s} + 2^{k + s}}$ & cyclotomic\\
\rct$a^{2}x^{2^{2k + 1} + 1} + b^{2}x^{2^{k +1} + 1} + ax^{2^{2k} + 2} + bx^{2^{k} + 2} + dx^{3}$ & cyclotomic\\
$ x^3 + ax^{2^{s+i} + 2^i} + a^2x^{2^{k+1} + 2^k} + x^{2^{s + i + k} + 2^{i + k}}$ & $\lin$ biprojective\\
\rct$ a\tr[\FFfield][\subfield](bx^{2^i + 1}) + a^{2^k}\tr[\FFfield][\subfield](cx^{2^s + 1})$ & $\lin$ biprojective\\
$ L(x)^{2^k + 1} + bx^{2^k + 1} $ & ?\\
% \visible<2->{\bottomrule}
\end{tabular}
\end{center}
% % !TEX root = ../slides.tex
\renewcommand\arraystretch{.9} 
\begin{tabular}{c|c|}
% \toprule
\rctt\textbf{Multivariate} &  \textbf{Observations}\\
% \midrule
$(x,y) \mapsto \left(\begin{array}{c} x^{2^s + 1} + ay ^{(2^s+1)2^i}\\ xy\end{array}\right)$ & $\lin$ biprojective\\
\rct$(x,y) \mapsto \left(\begin{array}{c}x^{2^{2s} + 2^{3s}} + ax^{2^{2s}}y^{2^s} + by^{2^s+1}\\ xy\end{array}\right)$ & $\lin$ biprojective\\
$(x,y) \mapsto \left(\begin{array}{c}x^{2^s+1} + x^{2^{s + k/2}}y^{2^{k/2}} + axy^{2^s} + by^{2^s+1}\\ xy\end{array}\right)$ & $\lin$ 4-projective\\
\rct$(x,y) \mapsto \left(\begin{array}{c}x^{2^s+1} + xy^{2^{s}} + y^{2^s + 1}\\ x^{2^{2s}+1} + x^{2^{2s}}y + y^{2^{2s} + 1}\end{array}\right)$ & biprojective\\
$(x,y) \mapsto \left(\begin{array}{c}x^{2^s+1} + xy^{2^{s}} + y^{2^s + 1}\\ x^{2^{3s}}y + xy^{2^{3s}}\end{array}\right)$ & biprojective\\
\rct$(x,y) \mapsto \left(\begin{array}{c}x^{2^s+1} + by^{2^s + 1}\\ x^{2^{s + k/2}}y + \frac{a}{b}xy^{2^{s + k/2}}\end{array}\right)$ &biprojective\\
$(x,y) \mapsto \left(\begin{array}{c}x^{2^s + 1} + xy^{2^s} + ay^{2^s +1}\\ x^{2^{2s} + 1} + ax^{2^{2s}}y + (1 + a)^{2^s}xy^{2^{2s}} + ay^{2^{2s} + 1}\end{array}\right)$ & biprojective\\
\rct$(x,y,z) \mapsto \left(\begin{array}{c}x^{2^s+1} + x^{2^s}z + yz^{2^s}\\x^{2^s}z + y^{2^s+1}\\xy^{2^s} + y^{2^s}z + z^{2^s+1}\end{array}\right)$ & $\begin{array}{c}\text{3-projective}\\ \lin \text{cyclotomic}\end{array}$\\
$(x,y,z) \mapsto \left(\begin{array}{c}x^{2^s+1} + xy^{2^s} + yz^{2^s}\\xy^{2^s} + z^{2^s+1}\\x^{2^s}z + y^{2^s+1} + y^{2^s}z\end{array}\right)$ & $\begin{array}{c}\text{3-projective}\\ \lin \text{cyclotomic}\end{array}$\\
% \bottomrule
\end{tabular}
\end{frame}

\subsection{Our main result (2/2)}
\begin{frame}
\frametitle{\subsecname}

\begin{center}
\vspace{-.5cm}
Among the 22 known infinite APN families, 19 consist entirely of 

\emph{cyclotomic} or \emph{$\ell$-projective} mappings, \emph{up to linear equivalence}.
\scalebox{.75}{
% !TEX root = ../slides.tex
\renewcommand\arraystretch{.9} 
\begin{tabular}{c|c|}
% \toprule
\rctt\textbf{Multivariate} &  \textbf{Observations}\\
% \midrule
$(x,y) \mapsto \left(\begin{array}{c} x^{2^s + 1} + ay ^{(2^s+1)2^i}\\ xy\end{array}\right)$ & $\lin$ biprojective\\
\rct$(x,y) \mapsto \left(\begin{array}{c}x^{2^{2s} + 2^{3s}} + ax^{2^{2s}}y^{2^s} + by^{2^s+1}\\ xy\end{array}\right)$ & $\lin$ biprojective\\
$(x,y) \mapsto \left(\begin{array}{c}x^{2^s+1} + x^{2^{s + k/2}}y^{2^{k/2}} + axy^{2^s} + by^{2^s+1}\\ xy\end{array}\right)$ & $\lin$ 4-projective\\
\rct$(x,y) \mapsto \left(\begin{array}{c}x^{2^s+1} + xy^{2^{s}} + y^{2^s + 1}\\ x^{2^{2s}+1} + x^{2^{2s}}y + y^{2^{2s} + 1}\end{array}\right)$ & biprojective\\
$(x,y) \mapsto \left(\begin{array}{c}x^{2^s+1} + xy^{2^{s}} + y^{2^s + 1}\\ x^{2^{3s}}y + xy^{2^{3s}}\end{array}\right)$ & biprojective\\
\rct$(x,y) \mapsto \left(\begin{array}{c}x^{2^s+1} + by^{2^s + 1}\\ x^{2^{s + k/2}}y + \frac{a}{b}xy^{2^{s + k/2}}\end{array}\right)$ &biprojective\\
$(x,y) \mapsto \left(\begin{array}{c}x^{2^s + 1} + xy^{2^s} + ay^{2^s +1}\\ x^{2^{2s} + 1} + ax^{2^{2s}}y + (1 + a)^{2^s}xy^{2^{2s}} + ay^{2^{2s} + 1}\end{array}\right)$ & biprojective\\
\rct$(x,y,z) \mapsto \left(\begin{array}{c}x^{2^s+1} + x^{2^s}z + yz^{2^s}\\x^{2^s}z + y^{2^s+1}\\xy^{2^s} + y^{2^s}z + z^{2^s+1}\end{array}\right)$ & $\begin{array}{c}\text{3-projective}\\ \lin \text{cyclotomic}\end{array}$\\
$(x,y,z) \mapsto \left(\begin{array}{c}x^{2^s+1} + xy^{2^s} + yz^{2^s}\\xy^{2^s} + z^{2^s+1}\\x^{2^s}z + y^{2^s+1} + y^{2^s}z\end{array}\right)$ & $\begin{array}{c}\text{3-projective}\\ \lin \text{cyclotomic}\end{array}$\\
% \bottomrule
\end{tabular}
}
\end{center}
\end{frame}

\begin{comment}

\subsection{Sketch of proof}
\begin{frame}
\frametitle{\subsecname}
\vspace{-.7cm}
% \begin{recap}[Conjugacy, again]
% The conjugate of a composition is the composition of the conjugates.
% $$ F = F_{3} \comp F_{2} \comp F_{1} \quad \iff \quad F^{\changevar} = F_{3}^{\changevar} \comp F_{2}^{\changevar} \comp F_{1}^{\changevar}$$
% \end{recap}

\onslide<1->{
\begin{mybox}{Linear self-equivalence \& conjugacy}{}{}
Let $F$ be linearly self-equivalent: $\quad F = \orange{A} \comp F \comp \orange{B}$.

Let $G$ be linearly equivalent to $F$: $\quad G = \cyan{P} \comp F \comp \cyan{Q}$.

\vspace{.4cm}
Then $G$ is linearly self-equivalent:
\vspace{-.2cm}
 $$ G = (\cyan{P} \comp \orange{A} \comp \cyan{P})^{-1} \comp G \comp (\cyan{Q}^{-1}\comp \orange{B} \comp \cyan{Q}) $$

\onslide<2->{
Furthermore, $\orange{A}$ and  $\cyan{P} \comp \orange{A} \comp \cyan{P}^{-1}$ are \emph{similar} and thus share the same \emph{elementary divisors}. }
\end{mybox}}

\onslide<3->{
$$G = \cyan{P} \comp F \comp \cyan{Q} = \cyan{P} \comp \orange{A} \comp F \comp \orange{B} \comp \cyan{Q}  = \cyan{P} \comp \orange{A} \comp \cyan{P}^{-1} \comp G \comp \cyan{Q}^{-1}\comp \orange{B} \comp \cyan{Q} $$}

\onslide<4>{
\begin{theorem}[Alternative formulation]
Most of the known infinite APN families are made of \emph{linearly self-equivalent mappings} with \emph{very specific} mappings $\orange{A}, \orange{B}$. This can be detected independently of the representation.
\end{theorem}
}
\end{frame}

\begin{frame}[t]\frametitle{Example: Cyclotomic mappings}
\begin{recap}
$$ F(x) = x^{e}P\left(x^{2^{k} - 1}\right), n = \ell k$$


Univariate: $\orange{A} \comp F \comp \orange{B} = F$ with $\quad \orange{B}(x) = \blue{\lambda} x$, $\quad\orange{A}(x) = \blue{\lambda}^{-e} x$ for any $\lambda \in \blue{\FF_{2^{k}}^{*}}$

\vspace{.2cm}
Multivariate: $\widetilde{\orange{A}} \comp \widetilde{F} \comp \widetilde{\orange{B}} = \widetilde{F}$ with $\quad\widetilde{\orange{B}}(v) = (\blue{\lambda} v_{1}, \dotsc , \blue{\lambda} v_{\ell}),$ $\quad \widetilde{\orange{A}}(v) = (\blue{\lambda}^{-e} v_{1}, \dotsc , \blue{\lambda}^{-e} v_{\ell})$
\end{recap}\pause

\begin{proposition}[Up to linear equivalence]
    $F \from \FFspace \to \FFspace$. $F$ is linearly equivalent to a cyclotomic mapping w.r.t a subfield $\FF_{2^{k}}$ iff:

    \vspace{.2cm}
     $\exists \ \orange{A},\orange{B}$ such that $\orange{A} \comp F \comp \orange{B} = F$ and:

    \begin{itemize}
        \item[\bulletpoint] $\min(\orange{A}), \min(\orange{B})$ are \emph{irreducible} polynomials
        \item[\bulletpoint] $\ord(\orange{B}) = 2^{k} - 1$ and $\ord(\orange{A}) \mid \ord(\orange{B})$
    \end{itemize}
\end{proposition}




\end{frame}

\begin{frame}\frametitle{Linear self-equivalence and APN functions}

\begin{mybox}{Sum up}{}{}
\begin{itemize}
    \item[\bulletpoint] \emph{Pen-and-paper} functions:  linearly self-equivalent with \emph{very specific} $\orange{A},\orange{B}$
    \item[\bulletpoint] From \emph{computer searches}: most are linearly self-equivalent with \emph{less structured} $\orange{A},\orange{B}$.
\end{itemize}
\end{mybox}\pause

\begin{mybox}{The only solution to the big APN problem}{}{}
A single bijective APN mapping is known when $n$ is even. It is \emph{CCZ-equivalent} to the ``Kim mapping'':
$$ \kappa \from \FF_{2^{6}} \to \FF_{2^{6}}, X \mapsto X^{3} + X^{10} + uX^{24},$$
for some specific $u \in \FF_{2^{6}}$.

\pause
\vspace{.3cm}
$\kappa(X) = X^{3}(1 + X^{7} + uX^{21}) = X^{3}P(X^{2^{3}-1})$ \hfill \emph{cyclotomic w.r.t $\FF_{2^{3}}$}.
\end{mybox}


\end{frame}
\end{comment}

\begin{comment}
\begin{frame}[t]\frametitle{A (re)open problem}


\begin{mybox}{Question}{}{}
For an APN function $F$, does there always exist a \emph{CCZ-equivalent} function $G$ which is linear self-equivalent ($\orange{A} \comp G \comp \orange{B} = G$) ?
\end{mybox}\pause

\begin{mybox}{Element of answers}{}{}
\begin{itemize}
    \item[\bulletpoint] A \emph{data base} of the known functions (sporadic / infinite families) for small $n$.
    \item[\bulletpoint] Some of the properties of $\orange{A},\orange{B}$ are still preserved by \emph{affine and CCZ equivalences}.
\end{itemize}

\end{mybox}

\end{frame}



\begin{frame}\frametitle{More self-equivalent APN functions ?}

% \begin{mybox}{Linear self-equivalences of APN functions}{}{}
% \begin{itemize}
%     \item[\bulletpoint] (almost) All pen-and-paper functions are  linearly self-equivalent with \emph{very specific} $\orange{A},\orange{B}$
%     \item[\bulletpoint] A lot (all ?) functions from \emph{computer searches} are linearly self-equivalent with \emph{less structured} $\orange{A},\orange{B}$.
% \end{itemize}
% \end{mybox}

\begin{mybox}{Previous works}{}{}
Linearly self-equivalence to \emph{speed up searches} \hfill \purple{\small [BeiBriLea21,BeiLea22]}.
\end{mybox}\pause

\begin{mybox}{Toward new APN functions ?}{}{}
\begin{itemize}
    \item[\bulletpoint] \emph{Non-quadratic }linearly self-equivalent functions for $n =6$ ?
    \item[\bulletpoint] Cyclotomic mappings $F(x) = x^{e}P\left(x^{2^{k} - 1}\right)$ with \emph{non-quadratic} $e$ ?
    \item[\bulletpoint] $\ell$-projective mappings with \emph{$\ell > 4$} ?
\end{itemize}

\end{mybox}

\end{frame}
% \onslide<3->{
% \begin{mybox}{Sketch theorem}{}{}
% $F$ $\lin$ to a cyclotomic mapping iff \quad $\exists \ \cyan{A}, \cyan{B}$, such that $F = \cyan{A} \comp F \comp \cyan{B}$ with:

% \vspace{.1cm}
% $\cyan{B}$ (resp. $\cyan{A}$) with elementary divisors of $M_{\blue{\lambda}} : x \mapsto \blue{\lambda} x$ (resp. $M_{\blue{\lambda^{e}}}$ )
% % \begin{itemize}
% % 	\item[-] 
% % 	\item[-] $\cyan{A}$ with elementary divisors of $M_{\blue{\lambda^{e}}}: x \mapsto \blue{\lambda^{e}} x$ for some $e$.
% % \end{itemize}

% \onslide<4>{
% \vspace{.1cm}
% \hfill$\leadsto$ related to \emph{minimal polynomials} of $\cyan{A}, \cyan{B}$.
% }
% \end{mybox}
% }
\end{comment}


\subsection{Take away}
\begin{frame}
\frametitle{\subsecname}

\begin{theorem}[]
Among the 22 known infinite APN families, 19 consist entirely of 

\emph{cyclotomic} or \emph{$\ell$-projective} mappings, \emph{up to linear equivalence}.
\end{theorem}

\begin{mybox}{Sum up}{}{}
\begin{itemize}
\item[-] Characterization of \emph{very specific} self-equivalences
\item[-] Unify most of the approaches
\item[-] Partial answer to the \emph{detection} of such structures up to equivalence
\end{itemize}
\end{mybox}

\pause
\begin{mybox}{Open questions}{}{}
\begin{itemize}
\item[-] Link between self-equivalence and APN-ness\hfill\purple{\small [BeiBriLea21, Conjecture 1]}
% \item[-] Characterization up to CCZ equivalence?
\item[-] Cyclotomic mappings outside the known classes? (from \emph{non-quadratic} APN monomial)
\item[-] Projective mappings outside the known classes? (with \emph{more} coordinates)

\end{itemize}
\end{mybox}

\end{frame}