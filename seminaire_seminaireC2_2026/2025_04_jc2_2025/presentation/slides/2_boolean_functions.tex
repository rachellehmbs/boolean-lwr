% !TEX root = ../slides_jc2.tex
\section{Polynomial representations of Boolean functions}

\begin{frame}
\frametitle{Linear self-equivalence : a unifying PoV on the known families of APN functions}

% !TEX root = ../slides.tex

\begin{columns}[c]
\begin{column}{0.5\textwidth}
\renewcommand\arraystretch{1.3} 
\scalebox{.88}{
\begin{tabular}{|c|}
\toprule
\rctt\textbf{Univariate}\\
\midrule
$x^{2^s + 1} + ax^{2^{(3-i)k + s} + 2^{ik}}$\\
\rct$x^{2^s + 1} + ax^{2^{(4-i)k + s} + 2^{ik}} $\\
$ax^{2^k + 1} + x^{2^s +1} + x^{2^{s + k} + 2^k} + bx^{2^{k + s} + 1} + b^{2^{k}}x^{2^s + 2^k}$\\
\rct$x^{3} + a^{-1}\tr[\FF_{2^n}][\FF_2](a^3x^9)$\\
$x^{3} + a^{-1}\tr[\FF_{2^n}][\FF_{2^3}](a^3x^9 + a^6x^{18})$\\
\rct$x^{3} + a^{-1}\tr[\FF_{2^n}][\FF_{2^3}](a^6x^{18} + a^{12}x^{36})$\\
$ax^{2^s + 1} + a^{2^k}x^{2^{2k} + 2^{k + s}} + bx^{2^{2k} + 1} + ca^{2^k + 1}x^{2^{s} + 2^{k + s}}$\\
\rct$a^{2}x^{2^{2k + 1} + 1} + b^{2}x^{2^{k +1} + 1} + ax^{2^{2k} + 2} + bx^{2^{k} + 2} + dx^{3}$\\
$ x^3 + ax^{2^{s+i} + 2^i} + a^2x^{2^{k+1} + 2^k} + x^{2^{s + i + k} + 2^{i + k}}$\\
\rct$ a\tr[\FFfield][\subfield](bx^{2^i + 1}) + a^{2^k}\tr[\FFfield][\subfield](cx^{2^s + 1})$\\
$ L(x)^{2^k + 1} + bx^{2^k + 1} $\\
\bottomrule
\end{tabular}}
\end{column}
\begin{column}{0.5\textwidth}
% !TEX root = ../slides.tex
\renewcommand\arraystretch{.9} 
\scalebox{.77}{
\begin{tabular}{|c|}
\toprule
\rctt \multicolumn{1}{c|}{\textbf{Multivariate}}\\ 
\midrule
$(x,y) \mapsto \left(\begin{array}{c} x^{2^s + 1} + ay ^{(2^s+1)2^i}\\ xy\end{array}\right)$\\
\rct$(x,y) \mapsto \left(\begin{array}{c}x^{2^{2s} + 2^{3s}} + ax^{2^{2s}}y^{2^s} + by^{2^s+1}\\ xy\end{array}\right)$\\
$(x,y) \mapsto \left(\begin{array}{c}x^{2^s+1} + x^{2^{s + k/2}}y^{2^{k/2}} + axy^{2^s} + by^{2^s+1}\\ xy\end{array}\right)$\\
\rct$(x,y) \mapsto \left(\begin{array}{c}x^{2^s+1} + xy^{2^{s}} + y^{2^s + 1}\\ x^{2^{2s}+1} + x^{2^{2s}}y + y^{2^{2s} + 1}\end{array}\right)$\\
$(x,y) \mapsto \left(\begin{array}{c}x^{2^s+1} + xy^{2^{s}} + y^{2^s + 1}\\ x^{2^{3s}}y + xy^{2^{3s}}\end{array}\right)$\\
\rct$(x,y) \mapsto \left(\begin{array}{c}x^{2^s+1} + by^{2^s + 1}\\ x^{2^{s + k/2}}y + \frac{a}{b}xy^{2^{s + k/2}}\end{array}\right)$\\
$(x,y) \mapsto \left(\begin{array}{c}x^{2^s + 1} + xy^{2^s} + ay^{2^s +1}\\ x^{2^{2s} + 1} + ax^{2^{2s}}y + (1 + a)^{2^s}xy^{2^{2s}} + ay^{2^{2s} + 1}\end{array}\right)$\\
\rct$(x,y,z) \mapsto \left(\begin{array}{c}x^{2^s+1} + x^{2^s}z + yz^{2^s}\\x^{2^s}z + y^{2^s+1}\\xy^{2^s} + y^{2^s}z + z^{2^s+1}\end{array}\right)$\\
$(x,y,z) \mapsto \left(\begin{array}{c}x^{2^s+1} + xy^{2^s} + yz^{2^s}\\xy^{2^s} + z^{2^s+1}\\x^{2^s}z + y^{2^s+1} + y^{2^s}z\end{array}\right)$\\
\bottomrule
\end{tabular}
}
\end{column}
\end{columns}

% \onslide<2->{
% \begin{tikzpicture}[overlay]
%      \node[fill=white] at (11.5,.6) {\inlinebox{ + a lot of sporadic examples for small $n$}};
%  \end{tikzpicture}}

 \onslide<2>{
\begin{tikzpicture}[overlay]
    %  \node[fill=white] at (8,6) {\inlinebox{\red{Where to look for a new function ?}}};
    % \node[fill=white] at (8,3) {\inlinebox{\red{Intersection between families ?}}};
     \node[fill=white] at (8,4.5) {\inlinebox{\orange{\large Hopefully clearer in 12 min ?}}};
 \end{tikzpicture}}

\end{frame}

{
\setbeamercolor{background canvas}{bg=ptblue}	
\begin{frame}[plain]
\vfill
\begin{center}
%\color{white} \Huge \Roman{section} -  \secname
\color{white} \Huge \secname
\end{center}
\vfill
\end{frame}
}
% \begin{frame}\frametitle{Representing a vectorial Boolean function}
% \vspace{-10pt}
% $$F \from \FFspace \to \FFspace, \begin{pmatrix}x_{1}\\  \vdots\\  x_{n}
	
% \end{pmatrix} \mapsto \begin{pmatrix}F_{1}(x_{1}, \dotsc, x_{n})\\  \vdots\\  F_{n}(x_{1}, \dotsc, x_{n})
	
% \end{pmatrix}.$$

% Each $F_{i}: \FFspace \to \FF_{2}$ is a \emph{coordinate}.\pause

% \vspace{.5cm}
% A \emph{component} of $F$ is a linear combination of coordinate: $\alpha \cdot F \vcentcolon= \sum_{i=0}^{n-1} \alpha_{i} F_{i}$.\pause

% \vspace{.5cm}
% \begin{mybox}{Representations we won't look at}{}{}
% \begin{itemize}
% 	\item[\bulletpoint] Truth table / \emph{graph} of $F$: $\graph_{F} = \set{(x, F(x)), x \in \FFspace}$
% 	\item[\bulletpoint] \emph{Walsh transform}: Fourier transform of all components $\alpha \cdot F: \FFspace \to \FF_{2}  \subset \CC $
% \end{itemize}
% \end{mybox}

% \end{frame}


\begin{frame}\frametitle{Polynomial representations (1/2)}
\vspace{-10pt}
$$F \from \cFFspace \to \cFFspace, \begin{pmatrix}x_{1}\\  \vdots\\  x_{n}
    
\end{pmatrix} \mapsto \begin{pmatrix}F_{1}(x_{1}, \dotsc, x_{n})\\  \vdots\\  F_{n}(x_{1}, \dotsc, x_{n})
    
\end{pmatrix}.$$
\pause
\begin{theorem}[Lagrange multivariate interpolation]
$f \from (\FF_{q})^{m} \to \FF_{q}$ admits a polynomial representation in $\FF_{q}[X_{1}, \dotsc, X_{m}]/ (X_{1}^{q} + X_{1}, \dotsc, X_{m}^{q} + X_{m})$.
\end{theorem}\pause


\begin{mybox}{Algebraic Normal Form (ANF)}{}{}
$(q = 2, m=n)$. Each coordinate is a polynomial of $\FF_{2}[X_{1}, \dotsc, X_{n}]/ (X_{1}^{2} + X_{1}, \dotsc, X_{n}^{2} + X_{n})$
\end{mybox}\pause

\begin{center}
$F \from \cyan{\FF_{2}^{4}} \to \cyan{\FF_{2}^{4}}, \begin{pmatrix}
x_{0}\\ x_{1}\\ x_{2}\\ x_{3}
\end{pmatrix} \mapsto \begin{pmatrix}
    x_0x_2 + x_0 + x_1x_2 + x_1x_3\\ 
x_0x_1 + x_0x_2 + x_2x_3 + x_3\\ 
x_0x_1 + x_0x_2 + x_0x_3 + x_1x_2 + x_1x_3 + x_2x_3 + x_2\\ 
x_1x_3 + x_1 + x_2x_3 + x_2 + x_3
\end{pmatrix}$
\end{center}

%\emph{Algebraic degree} : $\algdeg(F) \vcentcolon= \max\limits_{1 \leq i \leq n} \deg(F_{i})$.\hfill Here $\algdeg(F) = 2$

\end{frame}


\begin{frame}\frametitle{Polynomial representations (2/2)}
\vspace{-.5cm}
\begin{theorem}[Lagrange multivariate interpolation]
$f \from (\FF_{q})^{m} \to \FF_{q}$ admits a polynomial representation in $\FF_{q}[X_{1}, \dotsc, X_{m}]/ (X_{1}^{q} + X_{1}, \dotsc, X_{m}^{q} + X_{m})$.
\end{theorem}\pause


\begin{mybox}{$\FF_{2}$-space isomorphisms}{}{}
$$\quad \quad \cFFspace \quad \simeq \quad \cFFfield \quad \simeq \quad \cFFinter, \text{ with } n = \ell k.$$
\end{mybox}\pause

\begin{columns}[t]
\begin{column}{0.5\textwidth}
\begin{mybox}{Univariate representations}{$q = 2^{n}, m = 1$}{}
$F \from \cFFspace \to \cFFspace$ can be seen as $\widetilde{F} \from \cFFfield \to \cFFfield$.


\begin{align*}
	\widetilde{F} \from \orange{\FF_{2^{4}}} &\to \orange{\FF_{2^{4}}}\\ 
	X &\mapsto \orange{\alpha_{0}}X^{12} + \orange{\alpha_{1}}X^{6} + \orange{\alpha_{2}}X^{3}
\end{align*}
\end{mybox}

\end{column}\pause
\begin{column}{0.5\textwidth}

\begin{mybox}{Multivariate representations}{$q = 2^{k}, m = \ell$}{}
$F \from \cFFspace \to \cFFspace$ can be seen as $\widetilde{F} \from \cFFinter \to \cFFinter$.


\begin{align*}\widetilde{F} \from \purple{\FF_{2^{2}}^{2}} &\to \purple{\FF_{2^{2}}^{2}}\\ \begin{pmatrix}x_{0}\\  x_{1}\end{pmatrix} &\mapsto \begin{pmatrix}\purple{\alpha_{0}}x_{0}^3 + x_{0}^2x_{1} + \purple{\alpha_{1}}x_{0}x_{1}^2 + \purple{\alpha_{2}}x_{1}^3\\
\purple{\alpha_{3}}x_{0}^3 + \purple{\alpha_{4}}x_{0}^2x_{1} + \purple{\alpha_{5}}x_{0}x_{1}^2\end{pmatrix}
\end{align*}
\end{mybox}
\end{column}
\end{columns}

\begin{center}
    \emph{Up to a choice of bases!}
\end{center}
\end{frame}

\begin{frame}\frametitle{Polynomial representations and APN functions}
\vspace{-.5cm}
 $$ \delta_{F}(\din, \dout) = \card{\set{x, F(x + \din) + F(x) = \dout}} $$\pause

 $\orange{A} \from (\FFspace, +) \to (U, \black{+_{_{U}}})$ and $\orange{B} \from (V, \black{+_{_{V}}}) \to (\FFspace, +)$ linear bijective mappings.

 Then $\orange{A}\comp F\comp \orange{B} \from (V, \black{+_{_{V}}}) \to (U, \black{+_{_{U}}})$\pause



% $$
% \begin{array}{ccccc}
% \orange{A}\comp F\comp \orange{B}(x \ \black{+_{_{V}}} \ \din) \ &\black{+_{_{U}}}& \  \orange{A}\comp F\comp \orange{B}(x) &=& \dout\\\pause 
% %\orange{G} (F\comp \red{H}(x \ \red{+_{_{V}}} \ \din) &+& F\comp \red{H}(x)) &=& \dout \\ 
%  F\comp \orange{B}(x \ \black{+_{_{V}}} \ \din) &+& F\comp \orange{B}(x) &=& \orange{A}^{-1}(\dout) \\\pause
%   F( \orange{B}(x) + \orange{B}(\din)) &+& F\comp \orange{B}(x) &=& \orange{A}^{-1}(\dout) \\\pause
%     %F( y + \red{H}(\din)) &+& F(y) &=& \orange{G}^{-1}(\dout)
% \end{array}
% $$

\begin{proposition}
\begin{itemize}
	\item[\bulletpoint] $\forall \din, \dout, \quad \delta_{F}(\orange{B}(\din), \orange{A}^{-1}(\dout)) = \delta_{\orange{A}F\orange{B}} (\din, \dout)$ 
	\item[\bulletpoint] $F$ is APN if and only if $\orange{A}\comp F\comp \orange{B}$ is APN.
\end{itemize}
\end{proposition}

\begin{definition}[Linear equivalence]
    $F_{1} \lin F_{2}$ if $\quad \exists \ \orange{A}, \orange{B}$, bijective linear s.t. $\quad \orange{A} \comp F_{1} \comp \orange{B} = F_{2}$.
\end{definition}

% \begin{corollary}[Freedom of choice]
% 	Sometimes easier to prove that $F$ is APN using \emph{another representation}.
% \end{corollary}

\end{frame}

\begin{comment}
    


\subsection{Equivalence relations}
\begin{frame}
\frametitle{\subsecname}
\vspace{-.4cm}

\begin{columns}[c]
\begin{column}{0.7\textwidth}

\begin{mybox}{Linear equivalence}{}{}
$F_{1} \lin F_{2}$ if $\quad \exists \ \orange{A}, \orange{B}$, bijective \emph{linear} s.t. $\quad \orange{A} \comp F_{1} \comp \orange{B} = F_{2}$.
\end{mybox}

\onslide<2->{
\begin{mybox}{Affine equivalence}{}{}
$F_{1} \aff F_{2}$ if $\quad \exists \ \blue{A}, \blue{B}$, bijective \emph{affine} s.t. $\quad \blue{A} \comp F_{1} \comp \blue{B} = F_{2}$.
\end{mybox}
}
\onslide<3->{
\begin{mybox}{CCZ equivalence}{}{\purple{\small[CCZ98]}}
$F_{1} \from \FFspace \to \FFspace$  $\ccz$  $F_{2} \from \FFspace \to \FFspace$ if: $\exists \ \purple{\mathcal{A}} \from \FFspace \times \FFspace \to \FFspace \times \FFspace$

 bijective \emph{affine} s.t.
\vspace{-.2cm}
$$ \purple{\mathcal{A}} \left(\graph_{F_{1}}\right) = \graph_{F_{2}},$$

where $\graph_{F} \vcentcolon= \set{(x, F(x), x \in \FFspace)}$.
\end{mybox}}

\onslide<4->{
\begin{proposition}
    If  $F_{1} \ccz F_{2}$, then $\quad F_{1}$ APN $\iff$ $F_{2}$ APN.
\end{proposition}
}

\end{column}

\begin{column}{0.3\textwidth}

\vspace{.6cm}
\begin{tikzpicture}[scale=.8]
    %   \scriptsize     
        % \coordinate (nw) at (0, 5);
        % \coordinate (sw) at (0, 0);
        % \coordinate (ne) at (6, 5);
        % \coordinate (se) at (6, 0);
        
        \onslide<2->{
        	\draw[very thick,oiblue, rounded corners=5pt] (0, -.2) rectangle node {Affine} (4, -3.8) ;
            \draw[very thick,oiblue,rounded corners=5pt] (0, 0) rectangle (4,4) ;
         \draw[very thick,oiorange,rounded corners=5pt] (.1, 2.05) rectangle +(3.8,.875) ;
          \draw[very thick,oiorange,rounded corners=5pt] (.1, 3.025) rectangle +(3.8,.875) ;
        \draw[very thick,oiorange,rounded corners=5pt] (.1, 1.075) rectangle +(3.8,.875) ;
        \draw[very thick,oiorange,rounded corners=5pt] (.1, 0.1) rectangle node {Linear} +(3.8,.875) ;
        }

       	\onslide<3->{
        \draw[very thick,oipurple,rounded corners=5pt] (-0.2, 4.2) rectangle (4.2,-4) ;
        \draw[very thick, oipurple] (0.2, -4) -- +(.5, -.5) node[right=.4] {CCZ};}

        % \draw[very thick,oiorange,rounded corners=15pt] ($(nw)+(3, -1.5)$) rectangle ($(se)+(-1.3, 1.5)$) ;


        % \draw (3.7,1.9) node[] {{\tiny X} $E_{\red{k}}$};

        
        % \draw[very thick,oiblue] ($ (ne) + (-.25, -1.4) $) --++(.6,0.7) node[right=1]{$\mathrm{aff}(\FF_2^n)$};
        % \node[very thick,oiorange] at ($ (ne) + (-1.1, -2) $) {\orange{$\mathcal{E}$}};
\end{tikzpicture}


\end{column}

\end{columns}

\end{frame}
\end{comment}


\subsection{Proper representatives for easier proofs}
\begin{frame}
\frametitle{\subsecname}

% \begin{center}
% $4$ bits $\approx$ a pair of $2$-bit words $\approx$ a $4$-bit word    
% \end{center}
\vspace{-.5cm}
\begin{mybox}{4 linearly-equivalent functions}{}{}

$F \from \cyan{\FF_{2}^{4}} \to \cyan{\FF_{2}^{4}}, \begin{pmatrix}
x_{0}\\ x_{1}\\ x_{2}\\ x_{3}
\end{pmatrix} \mapsto \begin{pmatrix}
    x_0x_2 + x_0 + x_1x_2 + x_1x_3\\ 
x_0x_1 + x_0x_2 + x_2x_3 + x_3\\ 
x_0x_1 + x_0x_2 + x_0x_3 + x_1x_2 + x_1x_3 + x_2x_3 + x_2\\ 
x_1x_3 + x_1 + x_2x_3 + x_2 + x_3
\end{pmatrix}$
\vspace{.5cm}

$F \from \purple{\FF_{4}^{2}} \to \purple{\FF_{4}^{2}},  \begin{pmatrix}x_{0}\\  x_{1}\end{pmatrix} \mapsto \begin{pmatrix}\purple{\alpha_{0}}x_{0}^3 + x_{0}^2x_{1} + \purple{\alpha_{1}}x_{0}x_{1}^2 + \purple{\alpha_{2}}x_{1}^3\\
\purple{\alpha_{3}}x_{0}^3 + \purple{\alpha_{4}}x_{0}^2x_{1} + \purple{\alpha_{5}}x_{0}x_{1}^2\end{pmatrix}$

\vspace{.5cm}
$F \from \orange{\FF_{16}} \to \orange{\FF_{16}}, X \mapsto \orange{\alpha_{0}}X^{12} + \orange{\alpha_{1}}X^{6} + \orange{\alpha_{2}}X^{3}$

\onslide<2->{
\vspace{.5cm}
$F \from \orange{\FF_{16}} \to \orange{\FF_{16}}, X \mapsto X^{3}$}
\end{mybox}
\vspace{-1cm}
\begin{align*}
	\onslide<3->{F(X+ \plaindiff) + F(X) = \dout}\\
	\onslide<4->{(X + \plaindiff)^{3} + X^{3} = \dout}\\ 
	\onslide<5->{\plaindiff X^{2} + \plaindiff^{2} X + \plaindiff^{3} + \dout = 0}
\end{align*}
%$$  = (X + \plaindiff)^{3} + X^{3} = \plaindiff X^{2} + \plaindiff^{2} X + \plaindiff^{3}$$

\onslide<6>{$\implies$ at most 2 solutions $\implies$ APN !}
\end{frame}

\begin{frame}
\frametitle{Linear self-equivalence : a unifying PoV on the known families of APN functions}

% !TEX root = ../slides.tex

\begin{columns}[c]
\begin{column}{0.5\textwidth}
\renewcommand\arraystretch{1.3} 
\scalebox{.88}{
\begin{tabular}{|c|}
\toprule
\rctt\textbf{Univariate}\\
\midrule
$x^{2^s + 1} + ax^{2^{(3-i)k + s} + 2^{ik}}$\\
\rct$x^{2^s + 1} + ax^{2^{(4-i)k + s} + 2^{ik}} $\\
$ax^{2^k + 1} + x^{2^s +1} + x^{2^{s + k} + 2^k} + bx^{2^{k + s} + 1} + b^{2^{k}}x^{2^s + 2^k}$\\
\rct$x^{3} + a^{-1}\tr[\FF_{2^n}][\FF_2](a^3x^9)$\\
$x^{3} + a^{-1}\tr[\FF_{2^n}][\FF_{2^3}](a^3x^9 + a^6x^{18})$\\
\rct$x^{3} + a^{-1}\tr[\FF_{2^n}][\FF_{2^3}](a^6x^{18} + a^{12}x^{36})$\\
$ax^{2^s + 1} + a^{2^k}x^{2^{2k} + 2^{k + s}} + bx^{2^{2k} + 1} + ca^{2^k + 1}x^{2^{s} + 2^{k + s}}$\\
\rct$a^{2}x^{2^{2k + 1} + 1} + b^{2}x^{2^{k +1} + 1} + ax^{2^{2k} + 2} + bx^{2^{k} + 2} + dx^{3}$\\
$ x^3 + ax^{2^{s+i} + 2^i} + a^2x^{2^{k+1} + 2^k} + x^{2^{s + i + k} + 2^{i + k}}$\\
\rct$ a\tr[\FFfield][\subfield](bx^{2^i + 1}) + a^{2^k}\tr[\FFfield][\subfield](cx^{2^s + 1})$\\
$ L(x)^{2^k + 1} + bx^{2^k + 1} $\\
\bottomrule
\end{tabular}}
\end{column}
\begin{column}{0.5\textwidth}
% !TEX root = ../slides.tex
\renewcommand\arraystretch{.9} 
\scalebox{.77}{
\begin{tabular}{|c|}
\toprule
\rctt \multicolumn{1}{c|}{\textbf{Multivariate}}\\ 
\midrule
$(x,y) \mapsto \left(\begin{array}{c} x^{2^s + 1} + ay ^{(2^s+1)2^i}\\ xy\end{array}\right)$\\
\rct$(x,y) \mapsto \left(\begin{array}{c}x^{2^{2s} + 2^{3s}} + ax^{2^{2s}}y^{2^s} + by^{2^s+1}\\ xy\end{array}\right)$\\
$(x,y) \mapsto \left(\begin{array}{c}x^{2^s+1} + x^{2^{s + k/2}}y^{2^{k/2}} + axy^{2^s} + by^{2^s+1}\\ xy\end{array}\right)$\\
\rct$(x,y) \mapsto \left(\begin{array}{c}x^{2^s+1} + xy^{2^{s}} + y^{2^s + 1}\\ x^{2^{2s}+1} + x^{2^{2s}}y + y^{2^{2s} + 1}\end{array}\right)$\\
$(x,y) \mapsto \left(\begin{array}{c}x^{2^s+1} + xy^{2^{s}} + y^{2^s + 1}\\ x^{2^{3s}}y + xy^{2^{3s}}\end{array}\right)$\\
\rct$(x,y) \mapsto \left(\begin{array}{c}x^{2^s+1} + by^{2^s + 1}\\ x^{2^{s + k/2}}y + \frac{a}{b}xy^{2^{s + k/2}}\end{array}\right)$\\
$(x,y) \mapsto \left(\begin{array}{c}x^{2^s + 1} + xy^{2^s} + ay^{2^s +1}\\ x^{2^{2s} + 1} + ax^{2^{2s}}y + (1 + a)^{2^s}xy^{2^{2s}} + ay^{2^{2s} + 1}\end{array}\right)$\\
\rct$(x,y,z) \mapsto \left(\begin{array}{c}x^{2^s+1} + x^{2^s}z + yz^{2^s}\\x^{2^s}z + y^{2^s+1}\\xy^{2^s} + y^{2^s}z + z^{2^s+1}\end{array}\right)$\\
$(x,y,z) \mapsto \left(\begin{array}{c}x^{2^s+1} + xy^{2^s} + yz^{2^s}\\xy^{2^s} + z^{2^s+1}\\x^{2^s}z + y^{2^s+1} + y^{2^s}z\end{array}\right)$\\
\bottomrule
\end{tabular}
}
\end{column}
\end{columns}

% \onslide<2->{
% \begin{tikzpicture}[overlay]
%      \node[fill=white] at (11.5,.6) {\inlinebox{ + a lot of sporadic examples for small $n$}};
%  \end{tikzpicture}}

 \onslide<2>{
\begin{tikzpicture}[overlay]
    %  \node[fill=white] at (8,6) {\inlinebox{\red{Where to look for a new function ?}}};
    % \node[fill=white] at (8,3) {\inlinebox{\red{Intersection between families ?}}};
     \node[fill=white] at (8,4.5) {\inlinebox{\orange{\large Hopefully clearer in 5 min ?}}};
 \end{tikzpicture}}

\end{frame}

\begin{comment}
\begin{frame}\frametitle{The APN family tree}
\vspace{-.5cm}
\begin{mybox}{A common descent}{}{\small \purple{[Nyberg93]}}
The function $F \from \FFfield \to \FFfield, X \mapsto X^{3}$ is APN.
\begin{itemize}
	\item[\bulletpoint] $F$ is a \blue{power mapping}
	\item[\bulletpoint] $F$ is \orange{quadratic}: $\algdeg(F) = \hamming(3) = 2$
\end{itemize}
\end{mybox}\pause
\vspace{-.5cm}

\begin{chronology}[5]{1992}{2024}{15cm}[\textwidth]
\eventpoint{1992}{ \ \ [NybKnu92]}[black][1][1]
\eventpoint{1993}{ \ \ [Nyberg93]}[black][1][1]
\eventspan{1992}{2001}{}[oiblue][.7][.2]
\eventpoint{2006}{ \ \ [EKP06]}[black][1][1] %BCP06, 
\eventspan{2006}{2024}{}[oiorange][.7][.2]
\end{chronology}

\begin{mybox}{Descendants}{}{}
\begin{itemize}
	\item[\bulletpoint] 6 infinite families of APN \blue{power mappings}, some are \emph{not quadratic}. %\small \purple{[Nyberg93, Dobbertin99a,99b,01]
	\item[\bulletpoint] About 20 infinite families of \orange{quadratic} APN mappings.
% 	\item[\bulletpoint] A lot of sporadic examples for $n \leq 9$
\end{itemize}
\end{mybox}\pause

% \begin{mybox}{Descendant branch \#2: quadratic mappings}{}{}
% \begin{itemize}
% 	\item[\bulletpoint] About 20 infinite families of quadratic APN mappings.
% 	\item[\bulletpoint] A lot of sporadic examples for $n \leq 9$
% \end{itemize}
% \end{mybox}

\begin{mybox}{A single counter-example}{}{\small \purple{[BriLea08,EdePot09]}}
A single APN function \emph{inequivalent} to a power mapping or a quadratic mapping is known.
\end{mybox}

\end{frame}
\end{comment}


\begin{comment}
    

\begin{frame}
\frametitle{Infinite families of quadratic APN mappings}

% !TEX root = ../slides.tex

\begin{columns}[c]
\begin{column}{0.5\textwidth}
\renewcommand\arraystretch{1.3} 
\scalebox{.88}{
\begin{tabular}{|c|}
\toprule
\rctt\textbf{Univariate}\\
\midrule
$x^{2^s + 1} + ax^{2^{(3-i)k + s} + 2^{ik}}$\\
\rct$x^{2^s + 1} + ax^{2^{(4-i)k + s} + 2^{ik}} $\\
$ax^{2^k + 1} + x^{2^s +1} + x^{2^{s + k} + 2^k} + bx^{2^{k + s} + 1} + b^{2^{k}}x^{2^s + 2^k}$\\
\rct$x^{3} + a^{-1}\tr[\FF_{2^n}][\FF_2](a^3x^9)$\\
$x^{3} + a^{-1}\tr[\FF_{2^n}][\FF_{2^3}](a^3x^9 + a^6x^{18})$\\
\rct$x^{3} + a^{-1}\tr[\FF_{2^n}][\FF_{2^3}](a^6x^{18} + a^{12}x^{36})$\\
$ax^{2^s + 1} + a^{2^k}x^{2^{2k} + 2^{k + s}} + bx^{2^{2k} + 1} + ca^{2^k + 1}x^{2^{s} + 2^{k + s}}$\\
\rct$a^{2}x^{2^{2k + 1} + 1} + b^{2}x^{2^{k +1} + 1} + ax^{2^{2k} + 2} + bx^{2^{k} + 2} + dx^{3}$\\
$ x^3 + ax^{2^{s+i} + 2^i} + a^2x^{2^{k+1} + 2^k} + x^{2^{s + i + k} + 2^{i + k}}$\\
\rct$ a\tr[\FFfield][\subfield](bx^{2^i + 1}) + a^{2^k}\tr[\FFfield][\subfield](cx^{2^s + 1})$\\
$ L(x)^{2^k + 1} + bx^{2^k + 1} $\\
\bottomrule
\end{tabular}}
\end{column}
\begin{column}{0.5\textwidth}
% !TEX root = ../slides.tex
\renewcommand\arraystretch{.9} 
\scalebox{.77}{
\begin{tabular}{|c|}
\toprule
\rctt \multicolumn{1}{c|}{\textbf{Multivariate}}\\ 
\midrule
$(x,y) \mapsto \left(\begin{array}{c} x^{2^s + 1} + ay ^{(2^s+1)2^i}\\ xy\end{array}\right)$\\
\rct$(x,y) \mapsto \left(\begin{array}{c}x^{2^{2s} + 2^{3s}} + ax^{2^{2s}}y^{2^s} + by^{2^s+1}\\ xy\end{array}\right)$\\
$(x,y) \mapsto \left(\begin{array}{c}x^{2^s+1} + x^{2^{s + k/2}}y^{2^{k/2}} + axy^{2^s} + by^{2^s+1}\\ xy\end{array}\right)$\\
\rct$(x,y) \mapsto \left(\begin{array}{c}x^{2^s+1} + xy^{2^{s}} + y^{2^s + 1}\\ x^{2^{2s}+1} + x^{2^{2s}}y + y^{2^{2s} + 1}\end{array}\right)$\\
$(x,y) \mapsto \left(\begin{array}{c}x^{2^s+1} + xy^{2^{s}} + y^{2^s + 1}\\ x^{2^{3s}}y + xy^{2^{3s}}\end{array}\right)$\\
\rct$(x,y) \mapsto \left(\begin{array}{c}x^{2^s+1} + by^{2^s + 1}\\ x^{2^{s + k/2}}y + \frac{a}{b}xy^{2^{s + k/2}}\end{array}\right)$\\
$(x,y) \mapsto \left(\begin{array}{c}x^{2^s + 1} + xy^{2^s} + ay^{2^s +1}\\ x^{2^{2s} + 1} + ax^{2^{2s}}y + (1 + a)^{2^s}xy^{2^{2s}} + ay^{2^{2s} + 1}\end{array}\right)$\\
\rct$(x,y,z) \mapsto \left(\begin{array}{c}x^{2^s+1} + x^{2^s}z + yz^{2^s}\\x^{2^s}z + y^{2^s+1}\\xy^{2^s} + y^{2^s}z + z^{2^s+1}\end{array}\right)$\\
$(x,y,z) \mapsto \left(\begin{array}{c}x^{2^s+1} + xy^{2^s} + yz^{2^s}\\xy^{2^s} + z^{2^s+1}\\x^{2^s}z + y^{2^s+1} + y^{2^s}z\end{array}\right)$\\
\bottomrule
\end{tabular}
}
\end{column}
\end{columns}

% \onslide<2->{
% \begin{tikzpicture}[overlay]
%      \node[fill=white] at (11.5,.6) {\inlinebox{ + a lot of sporadic examples for small $n$}};
%  \end{tikzpicture}}

 \onslide<2>{
\begin{tikzpicture}[overlay]
     \node[fill=white] at (8,6) {\inlinebox{\red{Where to look for a new function ?}}};
  	\node[fill=white] at (8,3) {\inlinebox{\red{Intersection between families ?}}};
     \node[fill=white] at (8,4.5) {\inlinebox{\red{How to prove that a new $F$ is actually new ? }}};
 \end{tikzpicture}}

\end{frame}
\end{comment}
