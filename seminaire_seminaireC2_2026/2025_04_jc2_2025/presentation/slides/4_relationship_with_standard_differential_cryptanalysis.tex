% !TEX root = ../slides.tex
\section{Relationship with standard differential cryptanalysis}

{
\setbeamercolor{background canvas}{bg=ptblue}	
\begin{frame}[plain]
\vfill
\begin{center}
%\color{white} \Huge \Roman{section} -  \secname
\color{white} \Huge \secname
\end{center}
\vfill
\end{frame}
}

% \subsection{XXXX}
% \begin{frame}\frametitle{\subsecname}



% \end{frame}

\subsection{From commutative cryptanalysis back to differential cryptanalysis}
\begin{frame}[fragile]
	\frametitle{\subsecname}
\vspace{-.5cm}
	\begin{recap}[Commutative interpretation for ``almost''-Midori]
	Under weak-key hypothesis, there exists an affine bijective mapping $\alayer$ such that:
	\vspace{-.2cm}
		$$\alayer \comp \purple{F} = \purple{F} \comp \alayer\quad \text{for every layer } \purple{F}.$$
	\end{recap}

		\begin{center}
		\adjustbox{scale=1.1}{		\begin{tikzcd}
				x^{(0)}  \arrow[]{d}{\alayer} \arrow[]{r}[name=uparrow]{F_{\red{k^{(0)}}}} & x^{(1)} \arrow[]{d}{\alayer} \arrow[dashed]{r}{} &  x^{(R-1)} \arrow[]{r}{F_{\red{k^{(0)}}}} \arrow[]{d}{}{\alayer} &  E_{\red{k}}(x^{(0)}) \arrow[]{d}{\alayer}
				\\
				y^{(0)} \arrow[]{r}{}[swap,name=bottomarrow]{F_{\red{k^{(0)}}}} & y^{(1)}  \arrow[dashed]{r}{}[swap]{} &  y^{(R-1)} \arrow[swap]{r}{}{F_{\red{k^{(0)}}}} &   E_{\red{k}}(y^{(0)})\\
				%			\arrow[to path={(uparrow) node[midway, scale=1.2, left=0.05cm] {$\circlearrowleft$} (bottomarrow)}]{}
				%\orange{\Delta_0} \arrow[]{r}{?} &  \orange{\Delta_1} \arrow[dashed]{r}{?} & \orange{\Delta_{r-1}} \arrow[]{r}{?} & \orange{\Delta_r}
			\end{tikzcd} }
	\end{center}
\vspace{-1cm}
\pause
\begin{mybox}{Differential cryptanalysis}{}{}
Commutative cryptanalysis restricted to $\alayer(x) = \id(x) + \plaindiff$
\end{mybox}
\vspace{-.6cm}
\begin{center}
	\begin{tikzcd}
			x^{(0)}  \arrow[leftrightarrow]{d}{\din}[]{} \arrow[]{r}[name=uparrow]{F_{\red{k^{(0)}}}} & x^{(1)} \arrow[leftrightarrow]{d}{\seconddiff} \arrow[dashed]{r}{} &  x^{(R-1)} \arrow[]{r}{F_{\red{k^{(R-1)}}}} \arrow[leftrightarrow]{d}{}{\difference{R-1}} &  \black{E}_{\red{k}}(x^{(0)}) \arrow[leftrightarrow]{d}{\dout}[]{}
						\\
						y^{(0)} \arrow[]{r}{}[swap,name=bottomarrow]{F_{\red{k^{(0)}}}} & y^{(1)}  \arrow[dashed]{r}{}[swap]{} &  y^{(R-1)} \arrow[swap]{r}{}{F_{\red{k^{(R-1)}}}} &   \black{E}_{\red{k}}(y^{(0)})
						%\arrow[to path={(uparrow) node[midway, scale=1.2, left=0.05cm] {$\circlearrowleft$} (bottomarrow)}]{}
\end{tikzcd}
	\end{center}


	% \begin{center}
	% 	\adjustbox{scale=1.1}{		\begin{tikzcd}
	% 			x_0  \arrow[]{d}{\red{\alayer}^{\red{\star}}} \arrow[]{r}[name=uparrow]{R_0} & x_1 \arrow[]{d}{\red{\alayer}^{\red{\star}}} \arrow[dashed]{r}{} &  x_{r-1} \arrow[]{r}{R_{r-1}} \arrow[]{d}{}{\red{\alayer}^{\red{\star}}} &  E(x_0) \arrow[]{d}{\red{\alayer}^{\red{\star}}}
	% 			\\
	% 			z_0 \arrow[]{r}{}[swap,name=bottomarrow]{R_0} & z_1  \arrow[dashed]{r}{}[swap]{} &  z_{r-1} \arrow[swap]{r}{}{R_{r-1}} &   E(z_0)\\
	% 			%			\arrow[to path={(uparrow) node[midway, scale=1.2, left=0.05cm] {$\circlearrowleft$} (bottomarrow)}]{}
	% 			%\orange{\Delta_0} \arrow[]{r}{?} &  \orange{\Delta_1} \arrow[dashed]{r}{?} & \orange{\Delta_{r-1}} \arrow[]{r}{?} & \orange{\Delta_r}
	% 		\end{tikzcd} }
	% \end{center}
	
	\begin{mybox}{}{}{}
		$$ \mathbb{P}_{x \xleftarrow{\$} X}(\underbrace{\red{\alayer}^{\red{\star}} \to \red{\alayer}^{\red{\star}} \to \dots \to \red{\alayer}^{\red{\star}}}_{r \text{ times}}) = 1, \quad \text{for \red{any} } r ! $$
	\end{mybox}

\end{frame}

\subsection{Differential interpretation of a commutative distinguisher}
\begin{frame}[fragile]
	\frametitle{\subsecname}
	
	\begin{columns}[c, totalwidth=15cm]
		\begin{column}[c]{6.5cm}
			\adjustbox{scale=1.1}{		\begin{tikzcd}
					x^{(0)}  \arrow[]{d}{\alayer}[swap]{\difference{0}} \arrow[]{r}[name=uparrow]{F_{\red{k^{(0)}}}} & x^{(1)} \arrow[]{d}{\alayer}[swap]{\difference{1}} \arrow[dashed]{r}{} &  x^{(R-1)} \arrow[]{r}{F_{\red{k^{(R-1)}}}} \arrow[]{d}{}{\alayer}[swap]{\difference{R-1}} &  E_{\red{k}}(x^{(0)}) \arrow[]{d}{\alayer}[swap]{\difference{R}}
					\\
					y^{(0)} \arrow[]{r}{}[swap,name=bottomarrow]{F_{\red{k^{(0)}}}} & y^{(1)}   \arrow[dashed]{r}{}[swap]{} &  y^{(R-1)}  \arrow[swap]{r}{}{F_{\red{k^{(R-1)}}}} &   E_{\red{k}}(y^{(0)})\\
					%			\arrow[to path={(uparrow) node[midway, scale=1.2, left=0.05cm] {$\circlearrowleft$} (bottomarrow)}]{}
					%\orange{\Delta_0} \arrow[]{r}{?} &  \orange{\Delta_1} \arrow[dashed]{r}{?} & \orange{\Delta_{r-1}} \arrow[]{r}{?} & \orange{\Delta_r}
				\end{tikzcd}
			}
		\end{column}
		\begin{column}[c]{5.5cm}
			$\difference{i} \vcentcolon = x^{(i)} \xor y^{(i)} = x^{(i)} \xor \alayer(x^{(i)})$
		\end{column}
		
	\end{columns}
	\pause
	\vspace{-15pt}
	\begin{mybox}{Observation}{}{}
	Let $C \from x \mapsto x \xor \pcomA(x)$. Then $C(\FF_{2}^{4}) = \set{\orange{\delta}, \orange{\delta'}}$ where $\orange{\delta} \neq \orange{\delta'}$.
%		$\blue{\delta}  = \blue{\mathtt{0xf}}, \quad \blue{\Delta} = \blue{\delta}^{\otimes 16}, \quad \orange{\delta'}  = \orange{\mathtt{0xa}}, \quad \orange{\Delta'} = \orange{\delta'}^{\otimes 16}.$
		% $\blue{\delta}  = \blue{\mathtt{0xf}}, \quad \orange{\delta'}  = \orange{\mathtt{0xa}}.$
		\vspace{10pt}
%		\begin{itemize}%[leftmargin=0.2cm]
%			\item[-]$\mathbb{P}_{x \xleftarrow{\$} X}\left(\red{A}^{\red{\star}}(x) = x + \blue{\delta}\right) = \frac{1}{2} \quad \mathbb{P}_{x \xleftarrow{\$} X}\left(\red{A}^{\red{\star}}(x) = x + \orange{\delta'}\right) = \frac{1}{2}$.
%			
%			\item[-]$\forall x, \quad x + \red{\alayer}^{\red{\star}}(x) \in \{ \blue{\delta}, \orange{\delta'}\}^{16}$.
%		\end{itemize}


$\forall \ \blue{\Delta} \in  \{ \orange{\delta}, \orange{\delta'}\}^{16}, \  \mathbb{P}_{x \xleftarrow{\$} \FF_{2}^{64}}\left(x + \alayer(x) = \blue{\Delta}\right) = 2^{-16}$
\end{mybox}
\pause
\begin{mybox}{Surprising differential interpretation}{}{}
A differential pair $(x, x + \blue{\Delta})$ coincides with a commutative pair $(x, \alayer(x))$ with proba $2^{-16}$
		\vspace{5pt}
$$ \blue{\Delta} \xrightarrow{2^{-16}} \alayer \xrightarrow{1} \cdots \xrightarrow{1} \alayer  \xrightarrow{2^{-16}} \blue{\Delta} $$
	\end{mybox}
	
\end{frame}

\subsection{Weak-key differential interpretation}
\begin{frame}
	\frametitle{\subsecname}
	\vspace{-.5cm}
	\begin{mybox}{Recap}{}{}
		Under weak-key hypothesis:
		\begin{itemize}%[leftmargin=0.5cm]
		\item[-] $\mathbb{P}_{x \xleftarrow{\$} X}\left(\plaindiff \rightarrow \{ \orange{\delta}, \orange{\delta'}\}^{16}\right) \geq 2^{-16}$ for any $\plaindiff \in \{ \orange{\delta}, \orange{\delta'}\}^{16}$.
		\item[-] If output differences are uniformly distributed, then:

		$\mathbb{P}_{x \xleftarrow{\$} X}\left(\plaindiff \to \plaindiff'\right) \approx 2^{-32}$ for any $\plaindiff, \plaindiff' \in \{ \orange{\delta}, \orange{\delta'}\}^{16}$ 
			\item[-] Holds for \emph{infinitely many rounds} !
		\end{itemize}
	\end{mybox}
	\pause
	\begin{columns}[T, totalwidth=15cm]
		\begin{column}[T]{7.5cm}
			Standard case : \emph{quite low} $\mathbb{P}_{\red{k},x}$
			%% Document
\begin{tikzpicture}[scale=0.1]
  \everymath{\scriptstyle}

  %\tikzset{shadows=no}        % Option: add shadows to XOR, ADD, etc.
  \tikzset{edge/.style=-latex new, arrow head=6pt, thick};

  %% Plaintext
  \begin{scope}[yshift=19cm]

    \begin{scope}[xshift=3.5cm]
    
	  \fill[color=pink] 				 (0,3) rectangle +(1,1);
	  \fill[pattern=north east lines] (0,2) rectangle +(1,1);

	  \fill[pattern=north east lines] (1,1) rectangle +(1,1);
	  
	  \fill[color=orange]			 (2,3) rectangle +(1,1);
	  \fill[color=green]				 (2,2) rectangle +(1,1);
	  \fill[color=red] 				 (2,1) rectangle +(1,1);
	  \fill[pattern=north east lines] (2,0) rectangle +(1,1);
	  
	  \fill[pattern=north east lines] (3,3) rectangle +(1,1);
	  \fill[pattern=north east lines] (3,2) rectangle +(1,1);
	  
      \draw (0,0) rectangle (4,4);
      \draw (0,1) -- +(4,0);
      \draw (0,2) -- +(4,0);
      \draw (0,3) -- +(4,0);
      \draw (1,0) -- +(0,4);
      \draw (2,0) -- +(0,4);
      \draw (3,0) -- +(0,4);
      \path (-2.5,2) node {$\Delta_{IN}$};

      \draw (2,0) -- ++(0,-3.4);
    \end{scope}
  \end{scope}


  %% Round 0
  \begin{scope}[yshift=14cm]

    \draw (5.5,2) circle (0.3);
    \draw[edge] (1,0) -- node[left] {\tiny KS} ++(0,-3);

    \begin{scope}[xshift=-1cm]
    
	  \fill[color=pink] (0,3) rectangle +(1,1);
	  
	  \fill[color=cyan] (2,3) rectangle +(1,1);
	  \fill[color=green](2,2) rectangle +(1,1);
	  \fill[color=red]  (2,1) rectangle +(1,1);
	  \fill[color=red]  (2,0) rectangle +(1,1);
	  
	  \fill[color=blue] (3,2) rectangle +(1,1);

      \draw (0,0) rectangle (4,4);
      \draw (0,1) -- +(4,0);
      \draw (0,2) -- +(4,0);
      \draw (0,3) -- +(4,0);
      \draw (1,0) -- +(0,4);
      \draw (2,0) -- +(0,4);
      \draw (3,0) -- +(0,4);
      \path (-2,2) node {$k_0$};
      
      \draw[edge] (4,2) -- node[below] {\tiny AK} +(5,0);
    \end{scope}

    \begin{scope}[xshift=8cm]
    
	  \fill[pattern=north east lines] (0,2) rectangle +(1,1);

	  \fill[pattern=north east lines] (1,1) rectangle +(1,1);
	  
	  \fill[pattern=north east lines] (2,0) rectangle +(1,1);
	  
	  \fill[pattern=north east lines] (3,3) rectangle +(1,1);
	  \fill[pattern=north east lines] (3,2) rectangle +(1,1);  

      \draw (0,0) rectangle (4,4);
      \draw (0,1) -- +(4,0);
      \draw (0,2) -- +(4,0);
      \draw (0,3) -- +(4,0);
      \draw (1,0) -- +(0,4);
      \draw (2,0) -- +(0,4);
      \draw (3,0) -- +(0,4);
%      \path (2,4.5) node {\scriptsize$x_0$};
      
      \draw[edge] (4,2) -- node[above] {\tiny SB} +(4,0);
    \end{scope}


    \begin{scope}[xshift=16cm]
    
	  \fill[pattern=north east lines] (0,2) rectangle +(1,1);

	  \fill[pattern=north east lines] (1,1) rectangle +(1,1);
	  
	  \fill[pattern=north east lines] (2,0) rectangle +(1,1);
	  
	  \fill[pattern=north east lines] (3,3) rectangle +(1,1);
	  \fill[color=red] 				 (3,2) rectangle +(1,1);

      \draw (0,0) rectangle (4,4);
      \draw (0,1) -- +(4,0);
      \draw (0,2) -- +(4,0);
      \draw (0,3) -- +(4,0);
      \draw (1,0) -- +(0,4);
      \draw (2,0) -- +(0,4);
      \draw (3,0) -- +(0,4);
%      \path (2,4.5) node {\scriptsize$y_0$};
      
      \draw[edge] (4,2) -- node[above] {\tiny SR} +(4,0);
    \end{scope}


    \begin{scope}[xshift=24cm]
        
	  \fill[color=red] 				 (2,2) rectangle +(1,1);
	  \fill[pattern=north east lines] (3,0) rectangle +(1,4); 
	  
      \draw (0,0) rectangle (4,4);
      \draw (0,1) -- +(4,0);
      \draw (0,2) -- +(4,0);
      \draw (0,3) -- +(4,0);
      \draw (1,0) -- +(0,4);
      \draw (2,0) -- +(0,4);
      \draw (3,0) -- +(0,4);
%      \path (2,4.5) node {\scriptsize$z_0$};
      
      \draw[edge] (4,2) -- node[above] {\tiny MC} +(4,0);
    \end{scope}


    \begin{scope}[xshift=32cm]
        
	  \fill[color=blue] 				 (2,3) rectangle +(1,1);
	  \fill[color=green]				 (2,2) rectangle +(1,1);
	  \fill[color=red] 				 (2,1) rectangle +(1,1);
	  \fill[color=red] 				 (2,0) rectangle +(1,1);
	  
	  \fill[pattern=north east lines] (3,0) rectangle +(1,4);

      \draw (0,0) rectangle (4,4);
      \draw (0,1) -- +(4,0);
      \draw (0,2) -- +(4,0);
      \draw (0,3) -- +(4,0);
      \draw (1,0) -- +(0,4);
      \draw (2,0) -- +(0,4);
      \draw (3,0) -- +(0,4);
%      \path (2,4.5) node {\scriptsize$w_0$};
      
      \draw (4,2) -- ++(2,0) -- ++(0,-3.5) -- ++(-32.5,0) -- ++(0,-3.9);
      \path (11,2) node {\scriptsize Round 0};
    \end{scope}
  \end{scope}


  %% Round 1
  \begin{scope}[yshift=7cm]

    \draw (5.5,2) circle (0.3);
    \draw[edge] (1,0) -- node[left] {\tiny KS} ++(0,-3);

    \begin{scope}[xshift=-1cm]
        
	  \fill[color=cyan] 				 (2,3) rectangle +(1,1);
	  \fill[color=green]				 (2,2) rectangle +(1,1);
	  \fill[color=red] 				 (2,1) rectangle +(1,1);
	  \fill[color=red] 				 (2,0) rectangle +(1,1);
        
	  \fill[color=cyan] 				 (3,3) rectangle +(1,1);
	  \fill[color=red] 				 (3,2) rectangle +(1,1);
	  \fill[color=red] 				 (3,1) rectangle +(1,1);
	  \fill[color=red] 				 (3,0) rectangle +(1,1);

      \draw (0,0) rectangle (4,4);
      \draw (0,1) -- +(4,0);
      \draw (0,2) -- +(4,0);
      \draw (0,3) -- +(4,0);
      \draw (1,0) -- +(0,4);
      \draw (2,0) -- +(0,4);
      \draw (3,0) -- +(0,4);
      \path (-2,2) node {$k_1$};
      
      \draw[edge] (4,2) -- node[below] {\tiny AK} +(5,0);
    \end{scope}

    \begin{scope}[xshift=8cm]
        
	  \fill[color=orange] 				 (2,3) rectangle +(1,1);
	  
	  \fill[pattern=north east lines] (3,0) rectangle +(1,4);

      \draw (0,0) rectangle (4,4);
      \draw (0,1) -- +(4,0);
      \draw (0,2) -- +(4,0);
      \draw (0,3) -- +(4,0);
      \draw (1,0) -- +(0,4);
      \draw (2,0) -- +(0,4);
      \draw (3,0) -- +(0,4);
%      \path (2,4.5) node {\scriptsize$x_1$};
      
      \draw[edge] (4,2) -- node[above] {\tiny SB} +(4,0);
    \end{scope}


    \begin{scope}[xshift=16cm]
        
	  \fill[pattern=north east lines] (2,3) rectangle +(1,1);
	  
	  \fill[pattern=north east lines] (3,0) rectangle +(1,4);

      \draw (0,0) rectangle (4,4);
      \draw (0,1) -- +(4,0);
      \draw (0,2) -- +(4,0);
      \draw (0,3) -- +(4,0);
      \draw (1,0) -- +(0,4);
      \draw (2,0) -- +(0,4);
      \draw (3,0) -- +(0,4);
%      \path (2,4.5) node {\scriptsize$y_1$};
      
      \draw[edge] (4,2) -- node[above] {\tiny SR} +(4,0);
    \end{scope}


    \begin{scope}[xshift=24cm]
        
	  \fill[pattern=north east lines] (0,0) rectangle +(1,1);
	  \fill[pattern=north east lines] (1,1) rectangle +(1,1);
	  \fill[pattern=north east lines] (2,2) rectangle +(1,1);
	  \fill[pattern=north east lines] (2,3) rectangle +(1,1);
	  \fill[pattern=north east lines] (3,3) rectangle +(1,1);

      \draw (0,0) rectangle (4,4);
      \draw (0,1) -- +(4,0);
      \draw (0,2) -- +(4,0);
      \draw (0,3) -- +(4,0);
      \draw (1,0) -- +(0,4);
      \draw (2,0) -- +(0,4);
      \draw (3,0) -- +(0,4);
%      \path (2,4.5) node {\scriptsize$z_1$};
      %\path (2,  -1) node {$S_{start}$};
      
      \draw[edge] (4,2) -- node[above] {\tiny MC} +(4,0);
    \end{scope}


    \begin{scope}[xshift=32cm]
        
	  \fill[pattern=north east lines] (0,0) rectangle +(4,4);

      \draw (0,0) rectangle (4,4);
      \draw (0,1) -- +(4,0);
      \draw (0,2) -- +(4,0);
      \draw (0,3) -- +(4,0);
      \draw (1,0) -- +(0,4);
      \draw (2,0) -- +(0,4);
      \draw (3,0) -- +(0,4);
%      \path (2,4.5) node {\scriptsize$w_1$};
      %\path (2,  -1) node {$S'_{start}$};
      
      \draw (4,2) -- ++(2,0) -- ++(0,-3.9) -- ++(-32.5,0) -- ++(0,-3.5);
      \path (11,2) node {\scriptsize Round 1};
    \end{scope}
  \end{scope}


  %% Round 2
  \begin{scope}[yshift=0cm]
    \draw (5.5,2) circle (0.3);
    \draw[edge] (1,0) -- node[left] {\tiny KS} ++(0,-3);

    \begin{scope}[xshift=-1cm]
        
	  \fill[color=blue] 				 (0,3) rectangle +(1,1);
	  \fill[color=red] 				 (0,2) rectangle +(1,1);
	  \fill[color=red] 				 (0,1) rectangle +(1,1);
	  \fill[color=red] 				 (0,0) rectangle +(1,1);
        
	  \fill[color=blue] 				 (1,3) rectangle +(1,1);
	  \fill[color=red] 				 (1,2) rectangle +(1,1);
	  \fill[color=red] 				 (1,1) rectangle +(1,1);
	  \fill[color=red] 				 (1,0) rectangle +(1,1);
        
	  \fill[color=orange]			 (2,3) rectangle +(1,1);
	  \fill[color=blue] 				 (2,2) rectangle +(1,1);
        
	  \fill[color=blue] 				 (3,3) rectangle +(1,1);
	  \fill[color=green]				 (3,2) rectangle +(1,1);
	  \fill[color=red] 				 (3,1) rectangle +(1,1);
	  \fill[color=red] 				 (3,0) rectangle +(1,1);

      \draw (0,0) rectangle (4,4);
      \draw (0,1) -- +(4,0);
      \draw (0,2) -- +(4,0);
      \draw (0,3) -- +(4,0);
      \draw (1,0) -- +(0,4);
      \draw (2,0) -- +(0,4);
      \draw (3,0) -- +(0,4);
      \path (-2,2) node {$k_2$};
      
      \draw[edge] (4,2) -- node[below] {\tiny AK} +(5,0);
    \end{scope}
    
  \begin{scope}[xshift=8cm]
        
	\fill[pattern=north east lines] (0,0) rectangle +(4,4);

    \draw (0,0) rectangle (4,4);
    \draw (0,1) -- +(4,0);
    \draw (0,2) -- +(4,0);
    \draw (0,3) -- +(4,0);
    \draw (1,0) -- +(0,4);
    \draw (2,0) -- +(0,4);
    \draw (3,0) -- +(0,4);
  %  \path (2,4.5) node {\scriptsize$x_2$};
    
    \draw[edge] (4,2) -- node[above] {\tiny SB} +(4,0);
  \end{scope}


  \begin{scope}[xshift=16cm]
        
	\fill[pattern=north east lines] (0,0) rectangle +(4,4);

    \draw (0,0) rectangle (4,4);
    \draw (0,1) -- +(4,0);
    \draw (0,2) -- +(4,0);
    \draw (0,3) -- +(4,0);
    \draw (1,0) -- +(0,4);
    \draw (2,0) -- +(0,4);
    \draw (3,0) -- +(0,4);
   % \path (2,4.5) node {\scriptsize$y_2$};
    
    \draw[edge] (4,2) -- node[above] {\tiny SR} +(4,0);
  \end{scope}


  \begin{scope}[xshift=24cm]
        
	\fill[pattern=north east lines] (0,0) rectangle +(4,4);
    
    \draw (0,0) rectangle (4,4);
    \draw (0,1) -- +(4,0);
    \draw (0,2) -- +(4,0);
    \draw (0,3) -- +(4,0);
    \draw (1,0) -- +(0,4);
    \draw (2,0) -- +(0,4);
    \draw (3,0) -- +(0,4);
  %  \path (2,4.5) node {\scriptsize$z_2$};
    
    \draw[edge] (4,2) -- node[above] {\tiny MC} +(4,0);
  \end{scope}


  \begin{scope}[xshift=32cm]
        
	\fill[pattern=north east lines] (0,3) rectangle +(1,1);
	\fill[pattern=north east lines] (0,0) rectangle +(1,1);

	\fill[pattern=north east lines] (1,3) rectangle +(1,1);
	\fill[pattern=north east lines] (1,2) rectangle +(1,1);

	\fill[pattern=north east lines] (2,2) rectangle +(1,1);
	\fill[pattern=north east lines] (2,1) rectangle +(1,1);

	\fill[pattern=north east lines] (3,1) rectangle +(1,1);
	\fill[pattern=north east lines] (3,0) rectangle +(1,1);

    \fill[color=red] 			   (1,1) rectangle +(1,1);
    \fill[color=red] 			   (1,0) rectangle +(1,1);

    \fill[color=blue] 			   (2,3) rectangle +(1,1);
    \fill[color=red] 			   (2,0) rectangle +(1,1);

    \draw (0,0) rectangle (4,4);
    \draw (0,1) -- +(4,0);
    \draw (0,2) -- +(4,0);
    \draw (0,3) -- +(4,0);
    \draw (1,0) -- +(0,4);
    \draw (2,0) -- +(0,4);
    \draw (3,0) -- +(0,4);
  %  \path (2,4.5) node {\scriptsize$w_2$};
    
    \draw (4,2) -- ++(2,0) -- ++(0,-3.5) -- ++(-32.5,0) -- ++(0,-3.9);
    \path (11,2) node {\scriptsize Round 2};
  \end{scope}
  \end{scope}

  %% Round 3
  \begin{scope}[yshift=-7cm]

    \draw (5.5,2) circle (0.3);
    \draw[edge] (1,0) -- node[left] {\tiny KS} ++(0,-3);

    \begin{scope}[xshift=-1cm]

    	  \fill[color=orange]		   (0,3) rectangle +(1,1);

    	  \fill[color=cyan] 			   (1,3) rectangle +(1,1);
    	  \fill[color=red] 			   (1,2) rectangle +(1,1);
    	  \fill[color=red] 			   (1,1) rectangle +(1,1);
    	  \fill[color=red] 			   (1,0) rectangle +(1,1);

    	  \fill[color=blue] 			   (2,3) rectangle +(1,1);
    	  \fill[color=green]			   (2,2) rectangle +(1,1);
    	  \fill[color=red] 			   (2,1) rectangle +(1,1);
    	  \fill[color=red] 			   (2,0) rectangle +(1,1);

      \draw (0,0) rectangle (4,4);
      \draw (0,1) -- +(4,0);
      \draw (0,2) -- +(4,0);
      \draw (0,3) -- +(4,0);
      \draw (1,0) -- +(0,4);
      \draw (2,0) -- +(0,4);
      \draw (3,0) -- +(0,4);
      \path (-2,2) node {$k_3$};
      
      \draw[edge] (4,2) -- node[below] {\tiny AK} +(5,0);
    \end{scope}

    \begin{scope}[xshift=8cm]	
    
      \fill[pattern=north east lines] (0,3) rectangle +(1,1);
      \fill[pattern=north east lines] (0,0) rectangle +(1,1);

      \fill[pattern=north east lines] (1,3) rectangle +(1,1);
      \fill[pattern=north east lines] (1,2) rectangle +(1,1);

      \fill[pattern=north east lines] (2,2) rectangle +(1,1);
      \fill[pattern=north east lines] (2,1) rectangle +(1,1);

      \fill[pattern=north east lines] (3,1) rectangle +(1,1);
      \fill[pattern=north east lines] (3,0) rectangle +(1,1);

      \draw (0,0) rectangle (4,4);
      \draw (0,1) -- +(4,0);
      \draw (0,2) -- +(4,0);
      \draw (0,3) -- +(4,0);
      \draw (1,0) -- +(0,4);
      \draw (2,0) -- +(0,4);
      \draw (3,0) -- +(0,4);
%      \path (2,4.5) node {\scriptsize$x_3$};
      
      \draw[edge] (4,2) -- node[above] {\tiny SB} +(4,0);
    \end{scope}


    \begin{scope}[xshift=16cm]

    	  \fill[color=LimeGreen] 		   (0,3) rectangle +(1,1);
    	  \fill[color=DarkOrchid] 		   (0,0) rectangle +(1,1);

    	  \fill[color=Rhodamine] 		   (1,3) rectangle +(1,1);
    	  \fill[color=violet] 			   (1,2) rectangle +(1,1);

    	  \fill[color=yellow] 			   (2,2) rectangle +(1,1);
    	  \fill[color=RedOrange] 		   (2,1) rectangle +(1,1);

    	  \fill[color=black] 			   (3,1) rectangle +(1,1);
    	  \fill[color=Maroon] 			   (3,0) rectangle +(1,1);

      \draw (0,0) rectangle (4,4);
      \draw (0,1) -- +(4,0);
      \draw (0,2) -- +(4,0);
      \draw (0,3) -- +(4,0);
      \draw (1,0) -- +(0,4);
      \draw (2,0) -- +(0,4);
      \draw (3,0) -- +(0,4);
%      \path (2,4.5) node {\scriptsize$y_3$};
      
      \draw[edge] (4,2) -- node[above] {\tiny SR} +(4,0);
    \end{scope}


    \begin{scope}[xshift=24cm]

    	  \fill[color=LimeGreen] 	   (0,3) rectangle +(1,1);
    	  \fill[color=violet] 		   (0,2) rectangle +(1,1);
    	  \fill[color=RedOrange] 	   (0,1) rectangle +(1,1);
    	  \fill[color=Maroon] 		   (0,0) rectangle +(1,1);

    	  \fill[color=Rhodamine] 	   (1,3) rectangle +(1,1);
    	  \fill[color=yellow] 		   (1,2) rectangle +(1,1);
    	  \fill[color=black] 		   (1,1) rectangle +(1,1);
    	  \fill[color=DarkOrchid] 	   (1,0) rectangle +(1,1);

      \draw (0,0) rectangle (4,4);
      \draw (0,1) -- +(4,0);
      \draw (0,2) -- +(4,0);
      \draw (0,3) -- +(4,0);
      \draw (1,0) -- +(0,4);
      \draw (2,0) -- +(0,4);
      \draw (3,0) -- +(0,4);
%      \path (2,4.5) node {\scriptsize$z_3$};
      %\path (2,-1) node {$S_{end}$};
      
      \draw[edge] (4,2) -- node[above] {\tiny MC} +(4,0);
    \end{scope}


    \begin{scope}[xshift=32cm]

    	  \fill[color=orange]		   (0,3) rectangle +(1,1);
    	  \fill[color=blue] 			   (0,2) rectangle +(1,1);

    	  \fill[color=blue] 			   (1,3) rectangle +(1,1);
    	  \fill[color=red] 			   (1,2) rectangle +(1,1);
    	  \fill[color=red] 			   (1,1) rectangle +(1,1);
    	  \fill[color=red] 			   (1,0) rectangle +(1,1);

      \draw (0,0) rectangle (4,4);
      \draw (0,1) -- +(4,0);
      \draw (0,2) -- +(4,0);
      \draw (0,3) -- +(4,0);
      \draw (1,0) -- +(0,4);
      \draw (2,0) -- +(0,4);
      \draw (3,0) -- +(0,4);
%      \path (2,4.5) node {\scriptsize$w_3$};
      
      \draw (4,2) -- ++(2,0) -- ++(0,-3.9) -- ++(-32.5,0) -- ++(0,-3.5);
      \path (11,2) node {\scriptsize Round 3};
    \end{scope}
  \end{scope}

  %% Round 4
  \begin{scope}[yshift=-14cm]

    \draw (5.5,2) circle (0.3);
    %\draw[edge] (1,0) -- node[left] {\tiny KS} ++(0,-3);

    \begin{scope}[xshift=-1cm]

    	  \fill[color=orange]		   (0,3) rectangle +(1,1);

    	  \fill[color=blue] 			   (1,3) rectangle +(1,1);
    	  \fill[color=red] 			   (1,2) rectangle +(1,1);
    	  \fill[color=red] 			   (1,1) rectangle +(1,1);
    	  \fill[color=red] 			   (1,0) rectangle +(1,1);

    	  \fill[color=blue] 			   (2,2) rectangle +(1,1);

    	  \fill[color=blue] 			   (3,2) rectangle +(1,1);

      \draw (0,0) rectangle (4,4);
      \draw (0,1) -- +(4,0);
      \draw (0,2) -- +(4,0);
      \draw (0,3) -- +(4,0);
      \draw (1,0) -- +(0,4);
      \draw (2,0) -- +(0,4);
      \draw (3,0) -- +(0,4);
      \path (-2,2) node {$k_4$};
      
      \draw[edge] (4,2) -- node[below] {\tiny AK} +(5,0);
    \end{scope}

    \begin{scope}[xshift=8cm]

    	  \fill[color=blue] 			   (0,2) rectangle +(1,1);
    	  \fill[color=blue] 			   (2,2) rectangle +(1,1);
    	  \fill[color=blue] 			   (3,2) rectangle +(1,1);

      \draw (0,0) rectangle (4,4);
      \draw (0,1) -- +(4,0);
      \draw (0,2) -- +(4,0);
      \draw (0,3) -- +(4,0);
      \draw (1,0) -- +(0,4);
      \draw (2,0) -- +(0,4);
      \draw (3,0) -- +(0,4);
%      \path (2,4.5) node {\scriptsize$x_4$};
      
      \draw[edge] (4,2) -- node[above] {\tiny SB} +(4,0);
    \end{scope}


    \begin{scope}[xshift=16cm]

    	  \fill[color=red] 			   (0,2) rectangle +(1,1);
    	  \fill[color=red] 			   (2,2) rectangle +(1,1);
    	  \fill[color=red] 			   (3,2) rectangle +(1,1);

      \draw (0,0) rectangle (4,4);
      \draw (0,1) -- +(4,0);
      \draw (0,2) -- +(4,0);
      \draw (0,3) -- +(4,0);
      \draw (1,0) -- +(0,4);
      \draw (2,0) -- +(0,4);
      \draw (3,0) -- +(0,4);
%      \path (2,4.5) node {\scriptsize$y_4$};
      
      \draw[edge] (4,2) -- node[above] {\tiny SR} +(4,0);
    \end{scope}


    \begin{scope}[xshift=24cm]

    	  \fill[color=red] 			   (1,2) rectangle +(1,1);
    	  \fill[color=red] 			   (2,2) rectangle +(1,1);
    	  \fill[color=red] 			   (3,2) rectangle +(1,1);

      \draw (0,0) rectangle (4,4);
      \draw (0,1) -- +(4,0);
      \draw (0,2) -- +(4,0);
      \draw (0,3) -- +(4,0);
      \draw (1,0) -- +(0,4);
      \draw (2,0) -- +(0,4);
      \draw (3,0) -- +(0,4);
%      \path (2,4.5) node {\scriptsize$z_4$};
      
      \draw[edge] (4,2) -- node[above] {\tiny MC} +(4,0);
    \end{scope}


    \begin{scope}[xshift=32cm]
    
    	  \fill[color=blue] 			   (1,3) rectangle +(1,1);
    	  \fill[color=green]			   (1,2) rectangle +(1,1);
    	  \fill[color=red] 			   (1,1) rectangle +(1,1);
    	  \fill[color=red] 			   (1,0) rectangle +(1,1);

    	  \fill[color=blue] 			   (2,3) rectangle +(1,1);
    	  \fill[color=green]			   (2,2) rectangle +(1,1);
    	  \fill[color=red] 			   (2,1) rectangle +(1,1);
    	  \fill[color=red] 			   (2,0) rectangle +(1,1);

    	  \fill[color=blue] 			   (3,3) rectangle +(1,1);
    	  \fill[color=green]			   (3,2) rectangle +(1,1);
    	  \fill[color=red] 			   (3,1) rectangle +(1,1);
    	  \fill[color=red] 			   (3,0) rectangle +(1,1);

      \draw (0,0) rectangle (4,4);
      \draw (0,1) -- +(4,0);
      \draw (0,2) -- +(4,0);
      \draw (0,3) -- +(4,0);
      \draw (1,0) -- +(0,4);
      \draw (2,0) -- +(0,4);
      \draw (3,0) -- +(0,4);
%      \path (2,4.5) node {\scriptsize$w_4$};
      
      %\draw (4,2) -- ++(2,0) -- ++(0,-3.5) -- ++(-32.5,0) -- ++(0,-3.9);
      \path (11,2) node {\scriptsize Round 4};
    \end{scope}
  \end{scope}

 
  


\end{tikzpicture}




			{\tiny Part of 9-round chosen-key distinguisher for AES-128.\\\vspace{-5pt}
				Figure by J. Jean, extracted from Tikz for Cryptographers [Jean16].}
		\end{column}
		\pause
		\begin{column}[T]{7.5cm}
			This work: \emph{high} $\mathbb{P}_{x}$ for \emph{some} $k$
			

%% Document
\begin{tikzpicture}[scale=0.1]
  \everymath{\scriptstyle}

  \tikzset{edge/.style=-latex new, arrow head=6pt, thick};

  %% Plaintext
  \begin{scope}[yshift=19cm]

    \begin{scope}[xshift=3.5cm]
    
    	\foreach \x in {0,...,3}
    	{ \foreach \y in {0,...,3}
    		{\fill[color=cblue] 	(\x,\y) rectangle +(1,1);}
   		}
	  
      \draw (0,0) rectangle (4,4);
      \draw (0,1) -- +(4,0);
      \draw (0,2) -- +(4,0);
      \draw (0,3) -- +(4,0);
      \draw (1,0) -- +(0,4);
      \draw (2,0) -- +(0,4);
      \draw (3,0) -- +(0,4);
      \path (-2.5,2) node {$\Delta_{IN}$};

      \draw (2,0) -- ++(0,-3.4);
    \end{scope}
  \end{scope}


  %% Round 0
  \begin{scope}[yshift=14cm]

    \draw (5.5,2) circle (0.3);
    \draw[edge] (1,0) -- node[left] {\tiny KS} ++(0,-3);

    \begin{scope}[xshift=-1cm]
   	

      \draw (0,0) rectangle (4,4);
      \draw (0,1) -- +(4,0);
      \draw (0,2) -- +(4,0);
      \draw (0,3) -- +(4,0);
      \draw (1,0) -- +(0,4);
      \draw (2,0) -- +(0,4);
      \draw (3,0) -- +(0,4);
      \path (-2,2) node {$k_0$};
      
      \draw[edge] (4,2) -- node[below] {\tiny AK} +(5,0);
    \end{scope}

    \begin{scope}[xshift=8cm]
    	
    	
\foreach \x in {0,...,3}
{ \foreach \y in {0,...,3}
{\fill[color=cred] 	(\x,\y) rectangle +(1,1);}
}

      \draw (0,0) rectangle (4,4);
      \draw (0,1) -- +(4,0);
      \draw (0,2) -- +(4,0);
      \draw (0,3) -- +(4,0);
      \draw (1,0) -- +(0,4);
      \draw (2,0) -- +(0,4);
      \draw (3,0) -- +(0,4);
%      \path (2,4.5) node {\scriptsize$x_0$};
      
      \draw[edge] (4,2) -- node[above] {\tiny SB} +(4,0);
    \end{scope}


    \begin{scope}[xshift=16cm]
    
	\foreach \x in {0,...,3}
	{ \foreach \y in {0,...,3}
		{\fill[color=cred] 	(\x,\y) rectangle +(1,1);}
	}


      \draw (0,0) rectangle (4,4);
      \draw (0,1) -- +(4,0);
      \draw (0,2) -- +(4,0);
      \draw (0,3) -- +(4,0);
      \draw (1,0) -- +(0,4);
      \draw (2,0) -- +(0,4);
      \draw (3,0) -- +(0,4);
%      \path (2,4.5) node {\scriptsize$y_0$};
      
      \draw[edge] (4,2) -- node[above] {\tiny SR} +(4,0);
    \end{scope}


    \begin{scope}[xshift=24cm]
\foreach \x in {0,...,3}
{ \foreach \y in {0,...,3}
	{\fill[color=cred] 	(\x,\y) rectangle +(1,1);}
}

      \draw (0,0) rectangle (4,4);
      \draw (0,1) -- +(4,0);
      \draw (0,2) -- +(4,0);
      \draw (0,3) -- +(4,0);
      \draw (1,0) -- +(0,4);
      \draw (2,0) -- +(0,4);
      \draw (3,0) -- +(0,4);
%      \path (2,4.5) node {\scriptsize$z_0$};
      
      \draw[edge] (4,2) -- node[above] {\tiny MC} +(4,0);
    \end{scope}


    \begin{scope}[xshift=32cm]
\foreach \x in {0,...,3}
{ \foreach \y in {0,...,3}
	{\fill[color=cred] 	(\x,\y) rectangle +(1,1);}
}


      \draw (0,0) rectangle (4,4);
      \draw (0,1) -- +(4,0);
      \draw (0,2) -- +(4,0);
      \draw (0,3) -- +(4,0);
      \draw (1,0) -- +(0,4);
      \draw (2,0) -- +(0,4);
      \draw (3,0) -- +(0,4);
%      \path (2,4.5) node {\scriptsize$w_0$};
      
      \draw (4,2) -- ++(2,0) -- ++(0,-3.5) -- ++(-32.5,0) -- ++(0,-3.9);
      \path (11,2) node {\scriptsize Round 0};
    \end{scope}
  \end{scope}


  %% Round 1
  \begin{scope}[yshift=7cm]

    \draw (5.5,2) circle (0.3);
    \draw[edge] (1,0) -- node[left] {\tiny KS} ++(0,-3);

    \begin{scope}[xshift=-1cm]
        

      \draw (0,0) rectangle (4,4);
      \draw (0,1) -- +(4,0);
      \draw (0,2) -- +(4,0);
      \draw (0,3) -- +(4,0);
      \draw (1,0) -- +(0,4);
      \draw (2,0) -- +(0,4);
      \draw (3,0) -- +(0,4);
      \path (-2,2) node {$k_1$};
      
      \draw[edge] (4,2) -- node[below] {\tiny AK} +(5,0);
    \end{scope}

    \begin{scope}[xshift=8cm]
\foreach \x in {0,...,3}
{ \foreach \y in {0,...,3}
	{\fill[color=cred] 	(\x,\y) rectangle +(1,1);}
}
      \draw (0,0) rectangle (4,4);
      \draw (0,1) -- +(4,0);
      \draw (0,2) -- +(4,0);
      \draw (0,3) -- +(4,0);
      \draw (1,0) -- +(0,4);
      \draw (2,0) -- +(0,4);
      \draw (3,0) -- +(0,4);
%      \path (2,4.5) node {\scriptsize$x_1$};
      
      \draw[edge] (4,2) -- node[above] {\tiny SB} +(4,0);
    \end{scope}


    \begin{scope}[xshift=16cm]
\foreach \x in {0,...,3}
{ \foreach \y in {0,...,3}
	{\fill[color=cred] 	(\x,\y) rectangle +(1,1);}
}

      \draw (0,0) rectangle (4,4);
      \draw (0,1) -- +(4,0);
      \draw (0,2) -- +(4,0);
      \draw (0,3) -- +(4,0);
      \draw (1,0) -- +(0,4);
      \draw (2,0) -- +(0,4);
      \draw (3,0) -- +(0,4);
%      \path (2,4.5) node {\scriptsize$y_1$};
      
      \draw[edge] (4,2) -- node[above] {\tiny SR} +(4,0);
    \end{scope}


    \begin{scope}[xshift=24cm]
\foreach \x in {0,...,3}
{ \foreach \y in {0,...,3}
	{\fill[color=cred] 	(\x,\y) rectangle +(1,1);}
}
      \draw (0,0) rectangle (4,4);
      \draw (0,1) -- +(4,0);
      \draw (0,2) -- +(4,0);
      \draw (0,3) -- +(4,0);
      \draw (1,0) -- +(0,4);
      \draw (2,0) -- +(0,4);
      \draw (3,0) -- +(0,4);
%      \path (2,4.5) node {\scriptsize$z_1$};
      %\path (2,  -1) node {$S_{start}$};
      
      \draw[edge] (4,2) -- node[above] {\tiny MC} +(4,0);
    \end{scope}


    \begin{scope}[xshift=32cm]
\foreach \x in {0,...,3}
{ \foreach \y in {0,...,3}
	{\fill[color=cred] 	(\x,\y) rectangle +(1,1);}
}
      \draw (0,0) rectangle (4,4);
      \draw (0,1) -- +(4,0);
      \draw (0,2) -- +(4,0);
      \draw (0,3) -- +(4,0);
      \draw (1,0) -- +(0,4);
      \draw (2,0) -- +(0,4);
      \draw (3,0) -- +(0,4);
%      \path (2,4.5) node {\scriptsize$w_1$};
      %\path (2,  -1) node {$S'_{start}$};
      
      \draw (4,2) -- ++(2,0) -- ++(0,-3.9) -- ++(-32.5,0) -- ++(0,-3.5);
      \path (11,2) node {\scriptsize Round 1};
    \end{scope}
  \end{scope}


  %% Round 2
  \begin{scope}[yshift=0cm]
    \draw (5.5,2) circle (0.3);
    \draw[edge] (1,0) -- node[left] {\tiny KS} ++(0,-3);

    \begin{scope}[xshift=-1cm]
        

      \draw (0,0) rectangle (4,4);
      \draw (0,1) -- +(4,0);
      \draw (0,2) -- +(4,0);
      \draw (0,3) -- +(4,0);
      \draw (1,0) -- +(0,4);
      \draw (2,0) -- +(0,4);
      \draw (3,0) -- +(0,4);
      \path (-2,2) node {$k_2$};
      
      \draw[edge] (4,2) -- node[below] {\tiny AK} +(5,0);
    \end{scope}
    
  \begin{scope}[xshift=8cm]
\foreach \x in {0,...,3}
{ \foreach \y in {0,...,3}
	{\fill[color=cred] 	(\x,\y) rectangle +(1,1);}
}
    \draw (0,0) rectangle (4,4);
    \draw (0,1) -- +(4,0);
    \draw (0,2) -- +(4,0);
    \draw (0,3) -- +(4,0);
    \draw (1,0) -- +(0,4);
    \draw (2,0) -- +(0,4);
    \draw (3,0) -- +(0,4);
  %  \path (2,4.5) node {\scriptsize$x_2$};
    
    \draw[edge] (4,2) -- node[above] {\tiny SB} +(4,0);
  \end{scope}


  \begin{scope}[xshift=16cm]
\foreach \x in {0,...,3}
{ \foreach \y in {0,...,3}
	{\fill[color=cred] 	(\x,\y) rectangle +(1,1);}
}
    \draw (0,0) rectangle (4,4);
    \draw (0,1) -- +(4,0);
    \draw (0,2) -- +(4,0);
    \draw (0,3) -- +(4,0);
    \draw (1,0) -- +(0,4);
    \draw (2,0) -- +(0,4);
    \draw (3,0) -- +(0,4);
   % \path (2,4.5) node {\scriptsize$y_2$};
    
    \draw[edge] (4,2) -- node[above] {\tiny SR} +(4,0);
  \end{scope}


  \begin{scope}[xshift=24cm]
\foreach \x in {0,...,3}
{ \foreach \y in {0,...,3}
	{\fill[color=cred] 	(\x,\y) rectangle +(1,1);}
}
    
    \draw (0,0) rectangle (4,4);
    \draw (0,1) -- +(4,0);
    \draw (0,2) -- +(4,0);
    \draw (0,3) -- +(4,0);
    \draw (1,0) -- +(0,4);
    \draw (2,0) -- +(0,4);
    \draw (3,0) -- +(0,4);
  %  \path (2,4.5) node {\scriptsize$z_2$};
    
    \draw[edge] (4,2) -- node[above] {\tiny MC} +(4,0);
  \end{scope}


  \begin{scope}[xshift=32cm]
\foreach \x in {0,...,3}
{ \foreach \y in {0,...,3}
	{\fill[color=cred] 	(\x,\y) rectangle +(1,1);}
}

    \draw (0,0) rectangle (4,4);
    \draw (0,1) -- +(4,0);
    \draw (0,2) -- +(4,0);
    \draw (0,3) -- +(4,0);
    \draw (1,0) -- +(0,4);
    \draw (2,0) -- +(0,4);
    \draw (3,0) -- +(0,4);
  %  \path (2,4.5) node {\scriptsize$w_2$};
    
    \draw (4,2) -- ++(2,0) -- ++(0,-3.5) -- ++(-32.5,0) -- ++(0,-3.9);
    \path (11,2) node {\scriptsize Round 2};
  \end{scope}
  \end{scope}

  %% Round 3
  \begin{scope}[yshift=-7cm]

    \draw (5.5,2) circle (0.3);
    \draw[edge] (1,0) -- node[left] {\tiny KS} ++(0,-3);

    \begin{scope}[xshift=-1cm]


      \draw (0,0) rectangle (4,4);
      \draw (0,1) -- +(4,0);
      \draw (0,2) -- +(4,0);
      \draw (0,3) -- +(4,0);
      \draw (1,0) -- +(0,4);
      \draw (2,0) -- +(0,4);
      \draw (3,0) -- +(0,4);
      \path (-2,2) node {$k_3$};
      
      \draw[edge] (4,2) -- node[below] {\tiny AK} +(5,0);
    \end{scope}

    \begin{scope}[xshift=8cm]	
\foreach \x in {0,...,3}
{ \foreach \y in {0,...,3}
	{\fill[color=cred] 	(\x,\y) rectangle +(1,1);}
}
      \draw (0,0) rectangle (4,4);
      \draw (0,1) -- +(4,0);
      \draw (0,2) -- +(4,0);
      \draw (0,3) -- +(4,0);
      \draw (1,0) -- +(0,4);
      \draw (2,0) -- +(0,4);
      \draw (3,0) -- +(0,4);
%      \path (2,4.5) node {\scriptsize$x_3$};
      
      \draw[edge] (4,2) -- node[above] {\tiny SB} +(4,0);
    \end{scope}


    \begin{scope}[xshift=16cm]
\foreach \x in {0,...,3}
{ \foreach \y in {0,...,3}
	{\fill[color=cred] 	(\x,\y) rectangle +(1,1);}
}

      \draw (0,0) rectangle (4,4);
      \draw (0,1) -- +(4,0);
      \draw (0,2) -- +(4,0);
      \draw (0,3) -- +(4,0);
      \draw (1,0) -- +(0,4);
      \draw (2,0) -- +(0,4);
      \draw (3,0) -- +(0,4);
%      \path (2,4.5) node {\scriptsize$y_3$};
      
      \draw[edge] (4,2) -- node[above] {\tiny SR} +(4,0);
    \end{scope}


    \begin{scope}[xshift=24cm]
\foreach \x in {0,...,3}
{ \foreach \y in {0,...,3}
	{\fill[color=cred] 	(\x,\y) rectangle +(1,1);}
}

      \draw (0,0) rectangle (4,4);
      \draw (0,1) -- +(4,0);
      \draw (0,2) -- +(4,0);
      \draw (0,3) -- +(4,0);
      \draw (1,0) -- +(0,4);
      \draw (2,0) -- +(0,4);
      \draw (3,0) -- +(0,4);
%      \path (2,4.5) node {\scriptsize$z_3$};
      %\path (2,-1) node {$S_{end}$};
      
      \draw[edge] (4,2) -- node[above] {\tiny MC} +(4,0);
    \end{scope}


    \begin{scope}[xshift=32cm]
\foreach \x in {0,...,3}
{ \foreach \y in {0,...,3}
	{\fill[color=cred] 	(\x,\y) rectangle +(1,1);}
}

      \draw (0,0) rectangle (4,4);
      \draw (0,1) -- +(4,0);
      \draw (0,2) -- +(4,0);
      \draw (0,3) -- +(4,0);
      \draw (1,0) -- +(0,4);
      \draw (2,0) -- +(0,4);
      \draw (3,0) -- +(0,4);
%      \path (2,4.5) node {\scriptsize$w_3$};
      
      \draw (4,2) -- ++(2,0) -- ++(0,-3.9) -- ++(-32.5,0) -- ++(0,-3.5);
      \path (11,2) node {\scriptsize Round 3};
    \end{scope}
  \end{scope}

  %% Round 4
  \begin{scope}[yshift=-14cm]

    \draw (5.5,2) circle (0.3);
    %\draw[edge] (1,0) -- node[left] {\tiny KS} ++(0,-3);

    \begin{scope}[xshift=-1cm]


      \draw (0,0) rectangle (4,4);
      \draw (0,1) -- +(4,0);
      \draw (0,2) -- +(4,0);
      \draw (0,3) -- +(4,0);
      \draw (1,0) -- +(0,4);
      \draw (2,0) -- +(0,4);
      \draw (3,0) -- +(0,4);
      \path (-2,2) node {$k_4$};
      
      \draw[edge] (4,2) -- node[below] {\tiny AK} +(5,0);
    \end{scope}

    \begin{scope}[xshift=8cm]
\foreach \x in {0,...,3}
{ \foreach \y in {0,...,3}
	{\fill[color=cred] 	(\x,\y) rectangle +(1,1);}
}

      \draw (0,0) rectangle (4,4);
      \draw (0,1) -- +(4,0);
      \draw (0,2) -- +(4,0);
      \draw (0,3) -- +(4,0);
      \draw (1,0) -- +(0,4);
      \draw (2,0) -- +(0,4);
      \draw (3,0) -- +(0,4);
%      \path (2,4.5) node {\scriptsize$x_4$};
      
      \draw[edge] (4,2) -- node[above] {\tiny SB} +(4,0);
    \end{scope}


    \begin{scope}[xshift=16cm]
\foreach \x in {0,...,3}
{ \foreach \y in {0,...,3}
	{\fill[color=cred] 	(\x,\y) rectangle +(1,1);}
}

      \draw (0,0) rectangle (4,4);
      \draw (0,1) -- +(4,0);
      \draw (0,2) -- +(4,0);
      \draw (0,3) -- +(4,0);
      \draw (1,0) -- +(0,4);
      \draw (2,0) -- +(0,4);
      \draw (3,0) -- +(0,4);
%      \path (2,4.5) node {\scriptsize$y_4$};
      
      \draw[edge] (4,2) -- node[above] {\tiny SR} +(4,0);
    \end{scope}


    \begin{scope}[xshift=24cm]
\foreach \x in {0,...,3}
{ \foreach \y in {0,...,3}
	{\fill[color=cred] 	(\x,\y) rectangle +(1,1);}
}

      \draw (0,0) rectangle (4,4);
      \draw (0,1) -- +(4,0);
      \draw (0,2) -- +(4,0);
      \draw (0,3) -- +(4,0);
      \draw (1,0) -- +(0,4);
      \draw (2,0) -- +(0,4);
      \draw (3,0) -- +(0,4);
%      \path (2,4.5) node {\scriptsize$z_4$};
      
      \draw[edge] (4,2) -- node[above] {\tiny MC} +(4,0);
    \end{scope}


    \begin{scope}[xshift=32cm]
\foreach \x in {0,...,3}
{ \foreach \y in {0,...,3}
	{\fill[color=cred] 	(\x,\y) rectangle +(1,1);}
}

      \draw (0,0) rectangle (4,4);
      \draw (0,1) -- +(4,0);
      \draw (0,2) -- +(4,0);
      \draw (0,3) -- +(4,0);
      \draw (1,0) -- +(0,4);
      \draw (2,0) -- +(0,4);
      \draw (3,0) -- +(0,4);
%      \path (2,4.5) node {\scriptsize$w_4$};
      
      %\draw (4,2) -- ++(2,0) -- ++(0,-3.5) -- ++(-32.5,0) -- ++(0,-3.9);
      \path (11,2) node {\scriptsize Round 4};
    \end{scope}
  \end{scope}

 
\begin{scope}[yshift=-19cm]

            % blue 0x8E
            \draw[fill=cblue]  (0,0) rectangle ++(1,1);
            \path (4.4,0.5) node {\scriptsize\tt 0xf};

            % red 0x7A
            \draw[fill=cred]  (0,-2) rectangle ++(1,1);
            \path (8.4,0.5-2) node {\scriptsize\tt 0xf or \tt 0xa};

            % WHITE No Difference
            \draw[draw=black]  (0,-4) rectangle ++(1,1);
            \path (7,0.5-4) node {\tt \scriptsize No diff.};

\end{scope}

\end{tikzpicture}




		\end{column}
	\end{columns}
	
\end{frame}

\subsection{Weak-key Differential interpretation, part 2}
\begin{frame}[c]
	\frametitle{\subsecname}

	\begin{center}
		\includegraphics[scale=.065]{figures/plot_differential_interpretation_mar_2024_2.png}
	\end{center}
	\pause
	\vspace{-.5cm}
	\begin{mybox}{Caution}{}{}
		\begin{itemize}
			\item[-] \emph{Same observations} for the CAESAR candidate \textsc{SCREAM}.
			\item[-] \emph{Same idea} can be used to \emph{hide} probability-1 differential trails. \hfill\purple{[C:BFLNS23]}
		\end{itemize}

	\end{mybox}
	\begin{mybox}{Good news}{}{}
		Probability-1 commutative trails can be \emph{automatically} detected !
	\end{mybox}

	
%	\begin{columns}[T, totalwidth=15cm]
%	\begin{column}[T]{8cm}
%
%		\begin{mybox}[cblue]{The designers' work}
%			\blue{Estimate} $\mathbb{E}_{\red{k} \xleftarrow{\$} \red{K}}\left( \#\set{x \ \mathrm{st.} \ E_{\red{k}}(x + \orange{\alpha}) = E_{\red{k}}(x) + \orange{\beta}}\right)$  and \blue{assume} representativeness.
%			
%			{\color{bluefig}Blue} curve.
%		\end{mybox}			
%	\end{column}
%	\begin{column}[T]{7.5cm}
%	\begin{mybox}[cred]{This work}
%		Find \red{non-average keys} with \\ \red{easily-distinguishable} property.
%		
%		{\color{violetfig}Purple} and {\color{redfig}red} curves.
%	\end{mybox}
%
%
%	\end{column}
%\end{columns}
%	
\end{frame}