% !TEX root = ../slides.tex

\begin{frame}[label=pi_figure_more_info]
\frametitle{New results on $\pi$}

\recapbox{$\cpi(\cgamma\cphi) = \orangeG(\cgamma) + \blueF(\cphi) \quad \quad$ when $\cgamma \in \cGamma \setminus \cFF, \cphi \in \cFF^*$.}


\begin{columns}[T]
\begin{column}{.4\textwidth}
\includegraphics[scale=.15]{figures/pi_trace}
\centering
\small $\trlf(\cpi(b^{i + 17j}))$, where $\LL^{*} = \langle b \rangle$ 
\end{column}

\begin{column}{.6\textwidth}
\begin{mybox}[cblue]{A few novelties}
\vspace{5pt}
\begin{itemize}
	\item[-] The choice of $\cO$ is understood.
	\item[-] $\orangeG$ is understood.
	\item[-] $\cpi|_\cFF$ and $\orangeG$ behaves in the ``same way''.
\end{itemize}
\end{mybox}
\end{column}
\end{columns}
\pause
\begin{mybox}[cred]{\centering \Large \bfseries  Can we say more about this structure ?}
\end{mybox}

\end{frame}