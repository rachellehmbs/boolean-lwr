% !TEX root = ../slides.tex

\begin{frame}
	\frametitle{Block ciphers in practice}
	\vfill


\begin{definition}[hihi]\label{def:xxx}
    asd
\end{definition}

\begin{proposition}[hihi]\label{def:xxx}
    asd
\end{proposition}

\begin{lemma}[hihi]\label{def:xxx}
    asd
\end{lemma}

\begin{theorem}[hihi]\label{def:xxx}
    asd
\end{theorem}

\begin{remark}[ASD]\label{rem:}
asd
\end{remark}


\end{frame}


\begin{frame}[plain, label=title]
\vfill

\blue{This is a test}


\cyan{This is a test}


\yellow{This is a test}


\orange{This is a test}


\red{This is a test}


\purple{This is a test}

		\vspace{-10pt}
		$$ \orange{y} = E_{\red{k}}(\purple{x}) \quad \quad \iff \quad \quad \purple{x} = (E_{\red{k}})^{-1}(\orange{y}) $$
\vspace{5pt}		
\begin{mybox}[Definition]{Indistinguishability}
$[\ E \xleftarrow{\$} \orange{\mathcal{E}} \ ]$ The block is \emph{indistinguishable} from a random bijection if $[ \ F \xleftarrow{\$} \blue{\mathrm{Bij}}(\FF_2^{n}) \ ].$
	\end{mybox}

\begin{mybox}[Theorem]{Major constraints}
$\card{\red{K}} = 2^{128}$, $n = 64, 128$
\begin{itemize}
	\item[-] $\orange{\mathcal{E}}$ should be \emph{easily implemented},
	\item[-] $\orange{\mathcal{E}}$ should be \emph{``easily'' analyzed}.
\end{itemize}
\end{mybox}

\end{frame}
