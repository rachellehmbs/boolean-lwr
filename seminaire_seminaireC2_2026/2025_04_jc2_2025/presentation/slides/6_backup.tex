% !TEX root = ../slides.tex
\section{Appendix}
{
\setbeamercolor{background canvas}{bg=ptblue}	
\begin{frame}[plain]
\vfill
\begin{center}
%\color{white} \Huge \Roman{section} -  \secname
\color{white} \Huge \secname
\end{center}
\vfill
\end{frame}
}

	\subsection{The permutation}
	\begin{frame}[label=page3]
		\frametitle{\subsecname}
		\vspace{-.3cm}
		\begin{columns}
			\begin{column}{0.5\textwidth}
				\begin{mybox}{\centering A confusion/diffusion structure\dots}{}{}
				
						\begin{textblock*}{7cm}(.2cm,2.4cm)
							\centering
							{\scriptsize The state}
							\scalebox{0.4}{%%%%%%%%%%%%%%%%%%%%%%%%%%%%%%%%%%%%%%%%%%%%%%%%%%%%%%%%%%%%%%%%%%%%%%%%%%%%%%%%%%
% The Ascon state
%
% public domain (CC0 1.0 https://creativecommons.org/publicdomain/zero/1.0/)
%%%%%%%%%%%%%%%%%%%%%%%%%%%%%%%%%%%%%%%%%%%%%%%%%%%%%%%%%%%%%%%%%%%%%%%%%%%%%%%%%%



\newif\ifsans
\newif\iftext
\newif\ifcolor

%%% CONFIGURATION %%%%%%%%%%%%%%%%%%%%%%%%%%%%%%%%%%%%%%%%%%%%%%%%%%%%%%%%%%%%%%%%
%\sanstrue   % for sans-serif fonts (slides, web)
\sansfalse % for serif fonts (article)

\texttrue   % include phase description
%\textfalse % no phase description

\colortrue   % use color for highlights
%\colorfalse % no color

%\horizontaltrue % highlight horizontal word
%\verticaltrue % highlight vertical slice
%\constanttrue % highlight vertical slice
%%%%%%%%%%%%%%%%%%%%%%%%%%%%%%%%%%%%%%%%%%%%%%%%%%%%%%%%%%%%%%%%%%%%%%%%%%%%%%%%%%

\usetikzlibrary{shadows}
\ifcolor
  \definecolor{webred}{HTML}{D35400}
\else
  \definecolor{webred}{HTML}{444444}
\fi

\begin{tikzpicture}[thick, scale=0.5]
  \draw[fill=white, drop shadow] (0,0) rectangle (32,3.5);
  \draw (0,.7) -- (32,.7)
                (0,1.4) -- (32,1.4)
                (0,2.1) -- (32,2.1)
                (0,2.8) -- (32,2.8);
                
  \foreach \x in {1,...,63}
  \draw (0.5*\x, 0) -- (.5*\x, 3.5);
  
  \iftext
    \draw (33,.35) node {$X_4$}
         ++(0,.7) node {$X_3$}
         ++(0,.7) node {$X_2$}
         ++(0,.7) node {$X_1$}
         ++(0,.7) node {$X_0$};
  \fi

\end{tikzpicture}}
							\vspace{-.2cm}
							\[  p = {\color{oigreen} p_L} \comp {\color{oiorange} p_S}  \comp {\color{oiblue} p_C} \]
						\end{textblock*}
			
						\begin{textblock*}{7cm}(0.2cm,4.5cm)
							\centering
							\scriptsize The constant addition {\color{oiblue} $ p_C $}
							\scalebox{0.4}{%%%%%%%%%%%%%%%%%%%%%%%%%%%%%%%%%%%%%%%%%%%%%%%%%%%%%%%%%%%%%%%%%%%%%%%%%%%%%%%%%%
% The Ascon state
%
% public domain (CC0 1.0 https://creativecommons.org/publicdomain/zero/1.0/)
%%%%%%%%%%%%%%%%%%%%%%%%%%%%%%%%%%%%%%%%%%%%%%%%%%%%%%%%%%%%%%%%%%%%%%%%%%%%%%%%%%

\usetikzlibrary{shadows}


\begin{tikzpicture}[thick, scale=0.5]
	  
	
  \draw[thick, fill = white, drop shadow] (0,0) rectangle (32,3.5);
  \draw (0,.7) -- (32,.7)
                (0,1.4) -- (32,1.4)
                (0,2.1) -- (32,2.1)
                (0,2.8) -- (32,2.8);
                
  \foreach \x in {1,...,63}
  \draw (0.5*\x, 0) -- (.5*\x, 3.5);
  
    \draw (33,.35) node {$X_4$}
         ++(0,.7) node {$X_3$}
         ++(0,.7) node {$X_2$}
         ++(0,.7) node {$X_1$}
         ++(0,.7) node {$X_0$};

 
  \draw[line width =2, color=cblue,fill=white,drop shadow={shadow yshift=-.7ex}] (27.9,1.3) rectangle +(4.2,.9);
  \foreach \x in {1,...,7}
  \draw[line width = 1.5, color=cblue] (32-\x*0.5,1.3) -- (32-\x*0.5,2.2);

  \foreach \x in {1,...,8}
  \draw[color=cblue] (27.75+\x*0.5,1.75) node { $ \Plus $};

\end{tikzpicture}}
						\end{textblock*}
			

						\begin{textblock*}{7cm}(.2cm,5.9cm)
							\centering
							\scriptsize The substitution layer {\color{oiorange} $ p_S $} \scalebox{0.4}{%%%%%%%%%%%%%%%%%%%%%%%%%%%%%%%%%%%%%%%%%%%%%%%%%%%%%%%%%%%%%%%%%%%%%%%%%%%%%%%%%%
% The Ascon state
%
% public domain (CC0 1.0 https://creativecommons.org/publicdomain/zero/1.0/)
%%%%%%%%%%%%%%%%%%%%%%%%%%%%%%%%%%%%%%%%%%%%%%%%%%%%%%%%%%%%%%%%%%%%%%%%%%%%%%%%%%
\usetikzlibrary{shadows}

\begin{tikzpicture}[thick, scale=0.5]
  \draw[fill=white, drop shadow] (0,0) rectangle (32,3.5);
  \draw (0,.7) -- (32,.7)
                (0,1.4) -- (32,1.4)
                (0,2.1) -- (32,2.1)
                (0,2.8) -- (32,2.8);
                
  \foreach \x in {1,...,63}
  \draw (0.5*\x, 0) -- (.5*\x, 3.5);
  
    \draw (33,.35) node {$X_4$}
         ++(0,.7) node {$X_3$}
         ++(0,.7) node {$X_2$}
         ++(0,.7) node {$X_1$}
         ++(0,.7) node {$X_0$};
  	
    \draw[line width=2, color=corange,fill=white,drop shadow={shadow xshift=.7ex}] (2.4,-.6) rectangle +(.7, 4.7);
    \draw[line width =1.5, color=corange,<->] (2.75, -.1) -- (2.75, 3.6);
\end{tikzpicture}}

						\end{textblock*}
										
				
						\begin{textblock*}{7cm}(.2cm,7.4cm)
							\centering
							\scriptsize The linear layer {\color{oigreen} $ p_L $}
							\scalebox{0.4}{%%%%%%%%%%%%%%%%%%%%%%%%%%%%%%%%%%%%%%%%%%%%%%%%%%%%%%%%%%%%%%%%%%%%%%%%%%%%%%%%%%
% The Ascon state
%
% public domain (CC0 1.0 https://creativecommons.org/publicdomain/zero/1.0/)
%%%%%%%%%%%%%%%%%%%%%%%%%%%%%%%%%%%%%%%%%%%%%%%%%%%%%%%%%%%%%%%%%%%%%%%%%%%%%%%%%%

\usetikzlibrary{shadows}

\begin{tikzpicture}[thick, scale=0.5]
  \draw[fill=white, drop shadow] (0,0) rectangle (32,3.5);
  \draw (0,.7) -- (32,.7)
                (0,1.4) -- (32,1.4)
                (0,2.1) -- (32,2.1)
                (0,2.8) -- (32,2.8);
                
  \foreach \x in {1,...,63}
  \draw (0.5*\x, 0) -- (.5*\x, 3.5);
  

    \draw (33,.35) node {$X_4$}
         ++(0,.7) node {$X_3$}
         ++(0,.7) node {$X_2$}
         ++(0,.7) node {$X_1$}
         ++(0,.7) node {$X_0$};

    \draw[line width=2 , color=cgreen,fill=white,drop shadow={shadow yshift=-.7ex}] (-.4,2.05) rectangle +(32.8,.9);
    \draw[line width=1.5,color=cgreen,<->] (-.2, 2.5) -- (32.2, 2.5);
\end{tikzpicture}}
						\end{textblock*}
					
				\end{mybox}
			\end{column}

			\begin{column}{0.36\textwidth}
				\begin{mybox}{\centering \dots studied algebraically}{}{}
					\tiny
					\vspace{15pt}
					\begin{align*}
						y_0 &= {\color{oired} \bm{x_4x_1}} + x_3 + {\color{oired} \bm{x_2x_1}} + x_2 + {\color{oired} \bm{x_1x_0}} + x_1 + x_0 \\
						y_1 &= x_4 + {\color{oired} \bm{x_3x_2}} + {\color{oired} \bm{x_3x_1}} + x_3 + {\color{oired} \bm{x_2x_1}} + x_2 + x_1 + x_0 \\
						y_2 &= {\color{oired} \bm{x_4x_3}} + x_4 + x_2 + x_1 + 1\\
						y_3 &= {\color{oired} \bm{x_4x_0}} + x_4 + {\color{oired} \bm{x_3x_0}} + x_3 + x_2 + x_1 + x_0 \\
						y_4 &= {\color{oired} \bm{x_4x_1}} + x_4 + x_3 + {\color{oired} \bm{x_1x_0}} + x_1
					\end{align*}
					%\vspace{-22pt}
					\begin{center}
						\scriptsize Algebraic Normal Form (ANF) of the {\color{oiorange} S-box}
					\end{center}
					%\pause
					\vspace{5pt}
					\begin{align*}
						X_0 &= X_0 \xor \ (X_0 \ggg 19) \ \xor \ (X_0 \ggg 28) \\
						X_1 &= X_1 \xor \ (X_1 \ggg 61) \ \xor \ (X_1 \ggg 39)  \\
						X_2 &= X_2 \xor \ (X_2 \ggg 1) \ \xor \ (X_2 \ggg 6)  \\
						X_3 &= X_3 \xor \ (X_3 \ggg 10) \ \xor \ (X_3 \ggg 17)  \\
						X_4 &= X_4 \xor \ (X_4 \ggg 7) \ \xor \ (X_4 \ggg 41)
					\end{align*}
					%\vspace{-22pt}
					\begin{center}
						\scriptsize ANF of the linear layer {\color{oigreen} $ p_L $}
					\end{center}

				\end{mybox}
			
			\end{column}
		\end{columns}

	\end{frame}

	\subsection{The nonce-misuse scenario}
	\begin{frame}[label=page4]
		\frametitle{\subsecname}
		\begin{mybox}{Simplified setting of \Ascon-128}{}{}
			\begin{center}
				\scalebox{0.7}{\begin{tikzpicture}
	
	\definecolor{cblue}{HTML}{3F71A1}
	\definecolor{corange}{HTML}{E09F1F}
	\definecolor{cgreen}{HTML}{36C190}
	\definecolor{cred}{HTML}{C2554F}
	\definecolor{deepmagenta}{HTML}{9F0162}
	
	 \node[draw] at (8.5, 0) (init) [rounded corners=1ex, align=center,draw]  { \begin{tabular}{c}
			\color{cgreen} $ v_0, \ \cdots, v_{63} $	\\ \hline \hline
			\color{corange} $ a_0, \ \cdots, a_{63}$ 	\\ \hline
			\color{corange}$ b_0, \ \cdots, b_{63} $  	\\ \hline
			\color{corange}$ c_0, \ \cdots, c_{63} $  	\\ \hline
			\color{corange}$ d_0, \ \cdots, d_{63} $  	\\	
	\end{tabular}};


	\node (capacity) [align=center] [right of=init, node distance=3.3cm] {\small Unknown internal state};
	\node (externalstate) [align=center] [above of=capacity, node distance=.9cm] {\small Chosen external state};
	\node (caption) [spongephase, align = center, below of = capacity, left of = externalstate, left=0.4cm, below = 1.67cm] {{\color{cred}$ \Sigma_{\mathrm{E}} $} State before encryption};

    \SpongeInitInner{$k \| N$}{}
   \draw (P.south) +(-.6,-1) node[spongephase] (phase) {Initialization};
   %\SpongeXorInner{$\hskip-4pt 0^* \| k$}
   \SpongeStep\SpongePhaseSep[cred]{}
   
\SpongeEncrypt{$ \mathbf{P_0 }$~~}{~~$C_0 $}{$ p^6 $}{}{}
\draw (P.south) +(-1.5,-0.63) node[spongephase] (phase) {\color{cred}{$ \Sigma_{\mathrm{E}} $}};
\draw (P.south) +(0,-1.04) node[spongephase] (phase) {Encryption};
\SpongeEncrypt{$ 0^* $~~}{~~$ \mathbf{C_1} $}{}{}{}



\end{tikzpicture}
}
			\end{center}
		\vspace{10pt}
			\begin{itemize}[leftmargin=0cm]
				\item[-] Many reuse of the {\color{oiorange} same $ (k, N) $ pair}.
				\item[-] State recovery = {\color{oired} compromised confidentiality without interaction}.
				\item[-] {\color{oigreen} No trivial key-recovery nor forgery} in that case.
				\item[-] Different from the generic attack [ACNS:VauViz18].

			\end{itemize}
		\end{mybox}
	
	\end{frame}


\subsection{Conditional cube}
\begin{frame}
	\frametitle{\subsecname}
	\vspace{10pt}
%	\begin{center}
		\begin{itemize}[leftmargin=0.3cm]
			\item[-] We look for $ {\color{oiorange}\alpha_u} $ with a {\color{oiblue} simple divisor}: $ \color{oiorange}\beta_0 $.
			\item[-]  $\color{oiorange} \alpha_u $ {\color{oired} mostly unknown}, but we still get: $ {\color{oiorange}\alpha_u } = 1 \implies {\color{oiorange}\beta_0 } = 1$. 
			\item[-] If $ \color{oiorange}\beta_0  $ is linear, we get a { \color{oiblue} linear system}.
			
		\end{itemize}
%	\end{center}

		% \pause
				
		% 	\only<2>{\begin{center}
		% 		\scalebox{0.8}{\input{figures/trees_4.tex}}\end{center}}
		\begin{center}
				\scalebox{0.8}{\begin{tikzpicture}
	\node[inner sep=0pt] (s5) at (3.5,0) {${\color{oiorange}\alpha_u} v_0v_1v_2v_3v_4v_5v_{6}v_{7} $};
	\node[inner sep=0pt] (s52) at (3.4,-1.1) {${\color{oiorange}\alpha_u} = {\color{oiorange}\beta_0} \left( {\color{oiblue} \prod\limits_{i = 1}^7  ?} + {\color{oired} \prod\limits_{i = 1}^7  ?}  + \cdots \right)$};
	\node[inner sep=0pt] (r4) at (3.5,-2.7) {$ R_4 $};
	
	
	
	\node[inner sep=0pt] (trail0) at (0,2.5) {Trail $ t_0 $};
	\node[inner sep=0pt] (s41) at (0,1.2) {$ v_0v_1v_2v_3 $};
	\node[inner sep=0pt] (s42) at (0,-1.2) {$ v_4v_5v_6v_7 $};
	\node[inner sep=0pt] (r3l) at (0,-2.7) {$ R_3 $};
	
	
	\node[inner sep=0pt] (trail1) at (7,2.5) {Trail $ t_1 $};
	\node[inner sep=0pt] (s43) at (7,1.2) {$ v_0v_1v_6v_7 $};
	\node[inner sep=0pt] (s44) at (7,-1.2) {$ v_2v_3v_4v_5 $};
	\node[inner sep=0pt] (r3r) at (7,-2.7) {$ R_3 $};
	
	
	
	\node[inner sep=0pt] (s31) at (-2,1.8) {$ v_0v_1 $};
	\node[inner sep=0pt] (s32) at (-2,0.6) {$ v_2v_3 $};
	
	\node[inner sep=0pt] (s33) at (-2,-0.6) {$ v_4v_5 $};
	\node[inner sep=0pt] (s34) at (-2,-1.8) {$ v_6v_7 $};
	
	\node[inner sep=0pt] (r2l) at (-2,-2.7) {$ R_2 $};
	
	
	
	
	\node[inner sep=0pt] (s35) at (9,1.8) {$ v_0v_7 $};
	\node[inner sep=0pt] (s36) at (9,0.6) {$ v_1v_6 $};
	
	\node[inner sep=0pt] (s37) at (9,-1.8) {$ v_2v_5 $};
	\node[inner sep=0pt] (s38) at (9,-0.6) {$ v_3v_4 $};
	
	\node[inner sep=0pt] (r2r) at (9,-2.7) {$ R_2 $};
	
	
	
	
	\node[inner sep=0pt] (v01) at (-3.2,2) {$ {\color{oiorange} \beta_{0}}v_0 $};
	\node[inner sep=0pt] (v11) at (-3.2,1.6) {{\color{oiblue} ?}$ v_1 $};
	\draw[-,thick] (v01.east) -- (s31.west)
	node[above=10pt,midway,] {};
	\draw[-,thick] (v11.east) -- (s31.west)
	node[above=10pt,midway,] {};
	
	\node[inner sep=0pt] (v21) at (-3.2,0.8) {{\color{oiblue} ?}$v_2 $};
	\node[inner sep=0pt] (v31) at (-3.2,0.4) {{\color{oiblue} ?}$v_3 $};
	\draw[-,thick] (v21.east) -- (s32.west)
	node[above=10pt,midway,] {};
	\draw[-,thick] (v31.east) -- (s32.west)
	node[above=10pt,midway,] {};
	
	\node[inner sep=0pt] (v41) at (-3.2,-0.4) {{\color{oiblue} ?}$ v_4 $};
	\node[inner sep=0pt] (v51) at (-3.2,-0.8) {{\color{oiblue} ?}$ v_5 $};
	\draw[-,thick] (v41.east) -- (s33.west)
	node[above=10pt,midway,] {};
	\draw[-,thick] (v51.east) -- (s33.west)
	node[above=10pt,midway,] {};
	
	\node[inner sep=0pt] (v61) at (-3.2,-2) {{\color{oiblue} ?}$ v_6 $};
	\node[inner sep=0pt] (v71) at (-3.2,-1.6) {{\color{oiblue} ?}$ v_7 $};
	\draw[-,thick] (v61.east) -- (s34.west)
	node[above=10pt,midway,] {};
	\draw[-,thick] (v71.east) -- (s34.west)
	node[above=10pt,midway,] {};
	
	\node[inner sep=0pt] (r1l) at (-3.2,-2.7) {$ R_1 $};
	
	
	
	
	
	\node[inner sep=0pt] (v02) at (10.2,2) {${\color{oiorange} \beta_{0}} v_0 $};
	\node[inner sep=0pt] (v12) at (10.2,1.6) {{\color{oired} ?}$ v_7 $};
	\draw[-,thick] (v02.west) -- (s35.east)
	node[above=10pt,midway,] {};
	\draw[-,thick] (v12.west) -- (s35.east)
	node[above=10pt,midway,] {};
	
	\node[inner sep=0pt] (v22) at (10.2,0.8) {{\color{oired} ?}$v_1 $};
	\node[inner sep=0pt] (v32) at (10.2,0.4) {{\color{oired} ?}$ v_6 $};
	\draw[-,thick] (v22.west) -- (s36.east)
	node[above=10pt,midway,] {};
	\draw[-,thick] (v32.west) -- (s36.east)
	node[above=10pt,midway,] {};
	
	\node[inner sep=0pt] (v42) at (10.2,-0.4) {{\color{oired} ?}$ v_3 $};
	\node[inner sep=0pt] (v52) at (10.2,-0.8) {{\color{oired} ?}$ v_4 $};
	\draw[-,thick] (v42.west) -- (s38.east)
	node[above=10pt,midway,] {};
	\draw[-,thick] (v52.west) -- (s38.east)
	node[above=10pt,midway,] {};
	
	\node[inner sep=0pt] (v62) at (10.2,-2) {{\color{oired} ?}$ v_2 $};
	\node[inner sep=0pt] (v72) at (10.2,-1.6) {{\color{oired} ?}$ v_5$};
	\draw[-,thick] (v62.west) -- (s37.east)
	node[above=10pt,midway,] {};
	\draw[-,thick] (v72.west) -- (s37.east)
	node[above=10pt,midway,] {};
	
	\node[inner sep=0pt] (r1r) at (10.2,-2.7) {$ R_1 $};
	
	
	\draw[-,thick] (s31.east) -- (s41.west)
	node[above=10pt,midway,] {};
	\draw[-,thick] (s32.east) -- (s41.west)
	node[above=10pt,midway,] {};
	
	\draw[-,thick] (s33.east) -- (s42.west)
	node[above=10pt,midway,] {};
	\draw[-,thick] (s34.east) -- (s42.west)
	node[above=10pt,midway,] {};
	
	\draw[-,thick] (s35.west) -- (s43.east)
	node[above=10pt,midway,] {};
	\draw[-,thick] (s36.west) -- (s43.east)
	node[above=10pt,midway,] {};
	
	\draw[-,thick] (s37.west) -- (s44.east)
	node[above=10pt,midway,] {};
	\draw[-,thick] (s38.west) -- (s44.east)
	node[above=10pt,midway,] {};
	
	
	\draw[-,thick] (s41.east) -- (s5.west)
	node[above=10pt,midway,] {};
	\draw[-,thick] (s42.east) -- (s5.west)
	node[above=10pt,midway,] {};
	
	\draw[-,thick] (s43.west) -- (s5.east)
	node[above=10pt,midway,] {};
	\draw[-,thick] (s44.west) -- (s5.east)
	node[above=10pt,midway,] {};
	
	
		\end{tikzpicture}}\end{center}

\end{frame}

	\subsection{Choosing conditional cubes by forcing linear divisors}
	\begin{frame}[label=page5]
		\frametitle{\subsecname}
		\vspace{-.7cm}
		\begin{mybox}{\nth{1} round}{}{}
				\begin{center}
				\scalebox{0.85}{
    \begin{tikzpicture}
	
	
	\definecolor{cblue}{HTML}{3F71A1}
	\definecolor{corange}{HTML}{E09F1F}
	\definecolor{cgreen}{HTML}{36C190}
	\definecolor{cred}{HTML}{C2554F}
        
        \node (init)[rounded corners=1ex, align=center,draw]  [node distance=1.5cm]{\footnotesize \begin{tabular}{c}
        	{\color{cgreen} $ v_0 $}	\\ \hline
        	{\color{corange} $ a_0 $} 	\\
        	{\color{corange} $ b_0 $}  	\\ 
        	{\color{corange} $ c_0 $}   	\\ 
        	{\color{corange} $ d_0 $}   	\\	
        \end{tabular}};
        
        % \node (capacity) [align=center] [left of=init, node distance=2.2cm] {\scriptsize{{Unknown capacity}}};
   % \node (externalstate) [,align=center] [above of=capacity, node distance=.7cm] {\scriptsize{Chosen external state}};

    
    \node (s1)[right of = init, rounded corners=1ex, align=center,draw]  [node distance=3cm]{\footnotesize\begin{tabular}{c}
    		{\color{corange}$ (a_0 + 1)$}{\color{cgreen}$ v_0 $} $+ \cdots $		\\ \hline
    		{\color{cgreen}$ v_0 $} $ + \cdots$ 				\\ \hline
    		$ \cdots $  					\\ \hline
    		{\color{corange}$ (c_0 + d_0 + 1)$}{\color{cgreen}$ v_0$} $ + \cdots $  	\\ \hline
    		{\color{corange}$ a_0$}{\color{cgreen}$ v_0 $}$ + \cdots $
    \end{tabular}};

    
    \draw[-latex] ($ (s1.east) + (0.9, -0.3)$) -- ($ (s1.east) + (0.3, -0.3)$)  node[midway, right]{$\ \ \ {\color{corange}  \gamma_0} \vcentcolon = {\color{corange}c_0 + d_0 + 1} $};
        \draw[-latex] ($ (s1.east) + (0.9, 0.6)$) -- ($ (s1.east) + (0.3, 0.6)$)  node[midway, right]{$\ \ \  {\color{corange} \beta_0 } \vcentcolon = {\color{corange} a_0 + 1} $};
   

\draw[-latex] ($ (init.east) + (0.1,0) $ )-- ($ (s1.west) - (0.1, 0)$) node[midway, below]{ \footnotesize $ S_1 $};
	\end{tikzpicture}}\\
			\end{center}
		\end{mybox}
	\vspace{-8pt}
	\pause
		\begin{mybox}{\nth{2} round}{}{}
			\begin{itemize}[leftmargin=0.3cm]
				\item[-]For any $\color{oigreen} v_0v_i $, $  i \neq 0 $:
				$ {\color{oiorange} \beta_0}P + {\color{oiorange} 1}Q  + {\color{oiorange} \gamma_0}R +  ({\color{oiorange}\beta_0 + 1})S$. 
				\item[-] But for {\color{oired} some} $ i $: $ {\color{oiorange} \beta_0}P $ or $ {\color{oiorange} \gamma_0}R $.
			\end{itemize}
			
%			\begin{itemize}[leftmargin=0.5cm]
%				\item[-] For any $\color{oigreen} v_0v_i $, $  i \neq 0 $:
%				$ {\color{oiorange} \beta_0}P + {\color{oiorange} 1}Q  + {\color{oiorange} \gamma_0}R +  ({\color{oiorange}\beta_0 + 1})S$.
%				\vspace{5pt}
%				\pause
%				\item[-] coefficient for {\color{oired} some} $ i $: $ {\color{oiorange} \beta_0}P $ or $ {\color{oiorange} \gamma_0}R $.
%			\end{itemize}
		\end{mybox}



	\end{frame}


	\begin{frame}
	    
	
	\begin{mybox}{\nth{6} round}{}{}
		\frametitle{\subsecname}

					\begin{itemize}[leftmargin=0.3cm]
				\item[-] With {\color{oired} chosen u},
				$ {\color{oiorange}\alpha_{u, \ j}} = {\color{oiorange}\beta_0}(\dots) + {\color{oiorange}\gamma_0}(\dots) $ , for all output coordinates. % ($ j \in \intset{0, 63}$).
				\pause
%				\vspace{5pt}
				\item[-] $ \left({\color{oiorange}\alpha_{u, 0} }, \cdots,{\color{oiorange}\alpha_{u, 63} }\right) \neq (0,\cdots, 0) \implies {\color{oiorange}\beta_0} = 1 $ or ${\color{oiorange}\gamma_0} = 1  $
				\pause
				\item[-] In practice, {\color{oired} reciprocal also true!} $ [\,{\color{oiorange}\alpha_{u, \ j}} = 0, \ \forall \ j\ ]\,\ \implies {\color{oiorange}\beta_0} = 0 $ and ${\color{oiorange}\gamma_0 } = 0   $
				\vspace{5pt}

				\begin{columns}
			\begin{column}{5cm}
				\includegraphics[width=5cm]{figures/test0} %\\
				%\scriptsize \centering Individual cancellations of each $ \color{oiorange} \alpha_{u,j} $\\ (1000 random internal states)
			\end{column}
			
			\begin{column}{5cm}
			\includegraphics[width=5cm]{figures/test1} %\\
%				\scriptsize \centering Hamming weight of the cube-sum vectors \\(1000 random internal states)
			\end{column}
		\end{columns}

\end{itemize}
		\end{mybox}
	\end{frame}


\begin{frame}\frametitle{Midori}
\includegraphics[width=14.5cm]{figures/midori_1}
\end{frame}
\begin{frame}\frametitle{Midori}
\includegraphics[width=14.5cm]{figures/midori_2}
\end{frame}
\begin{frame}\frametitle{Midori}
\includegraphics[width=14.5cm]{figures/midori_3}
\end{frame}

\begin{frame}\frametitle{Walsh spectrum of cyclotomic mappings}
\includegraphics[width=14.5cm]{figures/wcc_1}
\end{frame}
\begin{frame}\frametitle{Streebog}
\includegraphics[width=14.5cm]{figures/wcc_2}
\end{frame}

\begin{frame}\frametitle{}
\includegraphics[width=14.5cm]{figures/thesis_1}
\end{frame}
\begin{frame}\frametitle{}
\includegraphics[width=14.5cm]{figures/thesis_2}
\end{frame}
\begin{frame}\frametitle{}
\includegraphics[width=14.5cm]{figures/thesis_3}
\end{frame}

\begin{frame}\frametitle{Brinckmann-Leander-Edel-Pott APN cubic}
\begin{itemize}
	\item[-] 7 non-trivial automorphisms
	\item[-] Elementary divisors for $\calLL$ : $(X+1) $multiplicity 2, $(X+1)^{2}$ multiplicity 5
	\item[-]  If $\calLL \sim \diag(A, B)$ then $(X+1)^{2}$ is among the elementary divisors of $A$ $\implies$ $\min(A) = (X+1)^{2}$ not irreducible.
	\item[-] Cannot be cyclotomic mappings nor $\ell$-projective mappings.
\end{itemize}
\end{frame}

\begin{frame}\frametitle{}
\includegraphics[width=13cm]{figures/thesis_4}
\end{frame}
\begin{frame}\frametitle{}
\includegraphics[width=13cm]{figures/thesis_5}
\end{frame}