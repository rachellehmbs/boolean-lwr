% !TEX root = ../slides_jc2.tex

\section{Introduction}


\begin{frame}
\frametitle{Linear self-equivalence : a unifying PoV on the known families of APN functions}

% !TEX root = ../slides.tex

\begin{columns}[c]
\begin{column}{0.5\textwidth}
\renewcommand\arraystretch{1.3} 
\scalebox{.88}{
\begin{tabular}{|c|}
\toprule
\rctt\textbf{Univariate}\\
\midrule
$x^{2^s + 1} + ax^{2^{(3-i)k + s} + 2^{ik}}$\\
\rct$x^{2^s + 1} + ax^{2^{(4-i)k + s} + 2^{ik}} $\\
$ax^{2^k + 1} + x^{2^s +1} + x^{2^{s + k} + 2^k} + bx^{2^{k + s} + 1} + b^{2^{k}}x^{2^s + 2^k}$\\
\rct$x^{3} + a^{-1}\tr[\FF_{2^n}][\FF_2](a^3x^9)$\\
$x^{3} + a^{-1}\tr[\FF_{2^n}][\FF_{2^3}](a^3x^9 + a^6x^{18})$\\
\rct$x^{3} + a^{-1}\tr[\FF_{2^n}][\FF_{2^3}](a^6x^{18} + a^{12}x^{36})$\\
$ax^{2^s + 1} + a^{2^k}x^{2^{2k} + 2^{k + s}} + bx^{2^{2k} + 1} + ca^{2^k + 1}x^{2^{s} + 2^{k + s}}$\\
\rct$a^{2}x^{2^{2k + 1} + 1} + b^{2}x^{2^{k +1} + 1} + ax^{2^{2k} + 2} + bx^{2^{k} + 2} + dx^{3}$\\
$ x^3 + ax^{2^{s+i} + 2^i} + a^2x^{2^{k+1} + 2^k} + x^{2^{s + i + k} + 2^{i + k}}$\\
\rct$ a\tr[\FFfield][\subfield](bx^{2^i + 1}) + a^{2^k}\tr[\FFfield][\subfield](cx^{2^s + 1})$\\
$ L(x)^{2^k + 1} + bx^{2^k + 1} $\\
\bottomrule
\end{tabular}}
\end{column}
\begin{column}{0.5\textwidth}
% !TEX root = ../slides.tex
\renewcommand\arraystretch{.9} 
\scalebox{.77}{
\begin{tabular}{|c|}
\toprule
\rctt \multicolumn{1}{c|}{\textbf{Multivariate}}\\ 
\midrule
$(x,y) \mapsto \left(\begin{array}{c} x^{2^s + 1} + ay ^{(2^s+1)2^i}\\ xy\end{array}\right)$\\
\rct$(x,y) \mapsto \left(\begin{array}{c}x^{2^{2s} + 2^{3s}} + ax^{2^{2s}}y^{2^s} + by^{2^s+1}\\ xy\end{array}\right)$\\
$(x,y) \mapsto \left(\begin{array}{c}x^{2^s+1} + x^{2^{s + k/2}}y^{2^{k/2}} + axy^{2^s} + by^{2^s+1}\\ xy\end{array}\right)$\\
\rct$(x,y) \mapsto \left(\begin{array}{c}x^{2^s+1} + xy^{2^{s}} + y^{2^s + 1}\\ x^{2^{2s}+1} + x^{2^{2s}}y + y^{2^{2s} + 1}\end{array}\right)$\\
$(x,y) \mapsto \left(\begin{array}{c}x^{2^s+1} + xy^{2^{s}} + y^{2^s + 1}\\ x^{2^{3s}}y + xy^{2^{3s}}\end{array}\right)$\\
\rct$(x,y) \mapsto \left(\begin{array}{c}x^{2^s+1} + by^{2^s + 1}\\ x^{2^{s + k/2}}y + \frac{a}{b}xy^{2^{s + k/2}}\end{array}\right)$\\
$(x,y) \mapsto \left(\begin{array}{c}x^{2^s + 1} + xy^{2^s} + ay^{2^s +1}\\ x^{2^{2s} + 1} + ax^{2^{2s}}y + (1 + a)^{2^s}xy^{2^{2s}} + ay^{2^{2s} + 1}\end{array}\right)$\\
\rct$(x,y,z) \mapsto \left(\begin{array}{c}x^{2^s+1} + x^{2^s}z + yz^{2^s}\\x^{2^s}z + y^{2^s+1}\\xy^{2^s} + y^{2^s}z + z^{2^s+1}\end{array}\right)$\\
$(x,y,z) \mapsto \left(\begin{array}{c}x^{2^s+1} + xy^{2^s} + yz^{2^s}\\xy^{2^s} + z^{2^s+1}\\x^{2^s}z + y^{2^s+1} + y^{2^s}z\end{array}\right)$\\
\bottomrule
\end{tabular}
}
\end{column}
\end{columns}

% \onslide<2->{
% \begin{tikzpicture}[overlay]
%      \node[fill=white] at (11.5,.6) {\inlinebox{ + a lot of sporadic examples for small $n$}};
%  \end{tikzpicture}}

 \onslide<2>{
\begin{tikzpicture}[overlay]
    %  \node[fill=white] at (8,6) {\inlinebox{\red{Where to look for a new function ?}}};
    % \node[fill=white] at (8,3) {\inlinebox{\red{Intersection between families ?}}};
     \node[fill=white] at (8,4.5) {\inlinebox{\orange{\large Hopefully clearer in 20 min ?}}};
 \end{tikzpicture}}

\end{frame}





% \begin{frame}\frametitle{Searching for ideal components}

% \begin{center}
% \begin{tikzpicture}
%     %   \scriptsize     
%         \coordinate (nw) at (0, 5);
%         \coordinate (sw) at (0, 0);
%         \coordinate (ne) at (6, 5);
%         \coordinate (se) at (6, 0);

%             \draw[very thick,oiblue,rounded corners=30pt] ($(nw)+(2, -.6)$) rectangle ($(se)+(-.2, .6)$) ;
%         \draw[very thick,oiorange,rounded corners=10pt] ($(nw)+(2.2, -1.5)$) rectangle ($(se)+(-3.1, 2.5)$) ;

%         \draw[very thick,oired,rounded corners=2pt] ($(nw)+(2.3, -2.1)$) rectangle + (.2, -.2) ;

%         % \draw (3.5,2.7) node[] {{\tiny X} $E_{\red{k}}$};

        
%         \draw[very thick,oiblue] ($ (ne) + (-.25, -1.4) $) --++(.6,0.7) node[right=1]{Cryptography};
%         \draw[very thick,oiorange] ($ (nw)+(2.3, -1.6) $) --++(-.6,0.7) node[left=1]{Symmetric cryptography};
%         \draw[very thick,oired] ($ (nw)+(2.3, -2.2) $) --++(-.6,-0.7) node[left=1]{Cryptographic Boolean functions};
% \end{tikzpicture}
% \end{center}

% \begin{mybox}{Using optimal components}{}{}
% \begin{itemize}
%     \item[-] to reach a high security at \emph{lower costs}
%     \item[-] to achieve ideal properties \emph{assumed in security proofs}
% \end{itemize}
% \end{mybox}

% \begin{columns}[t]
% \begin{column}{0.3\textwidth}
% \begin{mybox}{Cryptanalysis}{}{}
% \begin{itemize}
%     \item[-] A specific attack\dots
%     \item[-] \dots then generalized
% \end{itemize}
% \end{mybox}

% \end{column}
% \begin{column}{0.4\textwidth}
% \onslide<2>{
%     \begin{mybox}{Theoretical study}{}{}
% \begin{itemize}
% \item[-] Definition of ``resistance''
% \item[-] Study of optimal objects
% \end{itemize}
% \end{mybox}
% }
% \end{column}

% \begin{column}{0.3\textwidth}
% \begin{mybox}{Design}{}{}
% \begin{itemize}
%     \item[-] Security arguments
% \end{itemize}
% \end{mybox}
% \end{column}
% \end{columns}

%\end{frame}

\begin{frame}\frametitle{Outline}

\begin{itemize}
    \item[\bulletpoint] From Differential cryptanalysis to APN functions
    \item[\bulletpoint] Polynomial representations of vectorial Boolean functions
    \item[\bulletpoint] APN state of the art
    \item[\bulletpoint] Our unified point of view on the known APN functions

\end{itemize}

\end{frame}

% \section{Symmetric encryption}
% %========================= Encryption ========================
% \subsection{}
% \begin{frame}
% \frametitle{Symmetric encryption}
% \vspace{-.3cm}
% \begin{goal}
% Ensure \emph{confidentiality} under the assumption of a \emph{shared secret} \red{\faKey}.
% \end{goal}

% \begin{center}
% \begin{tikzpicture}
% \node (A) at (0,0)
%     {\includegraphics[height=1.24cm]{figures/shadok_A3} \blue{A}, \red{\faKey}};
% \node (B) at (6,0)
%     {\purple{B},\red{\faKey} \includegraphics[height=1.25cm]{figures/shadok_B3}};

% \draw[<->,thick] (A.east) -- (B.west) node[midway] (mid) {} ;

% \onslide<2>{
% \draw[thick, draw=oigray ,fill=white,drop shadow] (.6,-2.2) rectangle node[midway] {\large \black{$E$}} +(.8,.8) ;
% \draw[->, thick] (1,-.8) -- +(0,-.5);
% \draw[draw=oigray, fill=oigray] (.8, -1.4) -- (1, -1.55) -- (1.2, -1.4);

% \draw[thick, draw=oigray ,fill=white,drop shadow] (5.2,-2.2) rectangle node[midway] {\large \black{$E^{-1}$}} +(.8,.8) ;
% \draw[->, thick] (5.6,-.8) -- +(0,-.5);
% \draw[draw=oigray, fill=oigray] (5.4, -1.4) -- (5.6, -1.55) -- (5.8, -1.4);


% \node at (-.5, -1.8) {\blue{\faFileTextO}};
% \node at (7.1, -1.8) {\blue{\faFileTextO}};

% \draw[->, thick] (-.25, -1.8) -- +(0.8,0);
% \draw[<-, thick] (6.9, -1.8) -- +(-0.8,0);
% \draw[->, thick] (1.5, -1.8) -- node[midway, fill=white] {\red{\faEnvelope}} +(3.6,0);
% }
% \end{tikzpicture}

% \end{center}

% \onslide<2>{
% \begin{constraints}
% \begin{itemize}
%     \item[\bulletpoint] Secure
%     \item[\bulletpoint] Easily implemented
%     \item[\bulletpoint] Arbitrary-long messages
% \end{itemize}
% \end{constraints}
% }

% \end{frame}

% \begin{frame}\frametitle{Building a symmetric encryption scheme}

% \begin{center}
%     \begin{tikzpicture}
%     %[thick, draw=oigray ,fill=white,drop shadow] % prev rounded corners=1ex,fill=red!20,draw
%         \foreach \x in {0, 1, 2} {
%             \node (f\x) at ($\x*(2.5cm,0)$) [minimum size=1cm,thick, draw=oigray, fill=white, drop shadow] {$E_{\red{k}}$};
%             \node (p\x) [above of=f\x, node distance=1.5cm, circle, draw] {};
%             \node (m\x) [above of=p\x, node distance=1cm] {$m_\x$};
%             %\node (k\x) [left of=f\x, node distance=1.5cm] {$k$};
%             \node (c\x) [below of=f\x, node distance=1.5cm] {$c_\x$};
%             \draw[-] (p\x.north) -- (p\x.south);
%             \draw[-] (p\x.east) -- (p\x.west);
%             \draw[-latex] (m\x) -- (p\x);
%             \draw[-latex] (p\x) -- (f\x);
%             %\draw[-latex] (k\x) -- (f\x);
%             \draw[-latex] (f\x) -- (c\x);
%         }
%         \node (iv) [left of=p0, node distance=1.5cm] {$IV$};
%         \draw[-latex] (iv) -- (p0);

%         \foreach \x in {0, 1} {
%         \draw[-latex] ($(c\x) + (0,0.6cm)$) -| +(0.8cm,2.4cm) -- ($(p\x) + (2.4cm,0)$);

%         \begin{scope}
%             \node at (6.4,0) {$\cdots\cdots$};
%         \end{scope}

%         \begin{scope}
%             \node (f) at (9.6cm,0) [minimum size=1cm,thick, draw=oigray, fill=white, drop shadow] {$E_{\red{k}}$};
%             \node (p) [above of=f, node distance=1.5cm, circle, draw] {};
%             \node (m) [above of=p, node distance=1cm] {$m_n$};
%             %\node (k) [left of=f, node distance=1.5cm] {$k$};
%             \node (c) [below of=f, node distance=1.5cm] {$c_n$};
%             \draw[-] (p.north) -- (p.south);
%             \draw[-] (p.east) -- (p.west);
%             \draw[-latex] (m) -- (p);
%             \draw[-latex] (p) -- (f);
%             %\draw[-latex] (k) -- (f);
%             \draw[-latex] (f) -- (c);
%             \draw[-] ($(c) + (-2.5,0.6cm)$) -- ($(c) + (-1.7,0.6cm)$);
%             \draw[-latex] ($(c) + (-1.7,0.6cm)$) |- + (0cm,2.4cm) -- (p);
%         \end{scope}
%         }

%     \end{tikzpicture}
%     \end{center}

%     \begin{mybox}{Ingredients}{}{}
%         \begin{itemize}
%             \item[\bulletpoint] a \red{key-dependent} transformation of $n$-bit words (\textit{e.g.} $n = 128$).\hfill \emph{Block cipher}
%             \item[\bulletpoint] a chaining method to handle arbitrary-long messages \hfill \emph{Mode of operation}
%         \end{itemize}
%     \end{mybox}

% \end{frame}


\section{From differential cryptanalysis to APN functions}
\subsection{}
\begin{frame}
\frametitle{Security of block ciphers}

\vspace{-.8cm}

\vspace{.5cm}
\begin{mybox}{Block cipher}{}{}
A family of bijections $\blockcipher$ of $\FFspace$.
$$\blockcipher = \left(E_{\red{k}} \from \FFspace \bij \FFspace\right)_{\red{k} \in \keyspace}$$
\end{mybox}

\begin{center}
\begin{tikzpicture}

\draw[thick, draw=oigray ,fill=white,drop shadow] (.6,-2.2) rectangle node[midway] {\large $E$} +(.8,.8) ;
\draw[->, thick] (1,-.8) node[above] {$\red{k} = $ \red{\faKey}} -- +(0,-.5);
\draw[draw=oigray, fill=oigray] (.8, -1.4) -- (1, -1.55) -- (1.2, -1.4);

\draw[thick, draw=oigray ,fill=white,drop shadow] (5.2,-2.2) rectangle node[midway] {\large \black{$E^{-1}$}} +(.8,.8) ;
\draw[->, thick] (5.6,-.8) node[above] {\red{$k$}} -- +(0,-.5);
\draw[draw=oigray, fill=oigray] (5.4, -1.4) -- (5.6, -1.55) -- (5.8, -1.4);


\node at (-.5, -1.8) {\gray{\faFileTextO}};
\node at (7.1, -1.8) {\gray{\faFileTextO}};

\draw[->, thick] (-.25, -1.8) -- +(0.8,0);
\draw[<-, thick] (6.9, -1.8) -- +(-0.8,0);
\draw[->, thick] (1.5, -1.8) -- node[midway, fill=white] {\red{\faEnvelope}} +(3.6,0);
\end{tikzpicture}

\end{center}\pause

\begin{mybox}{Ideal block cipher}{}{}
A \emph{random} family of bijections.

\vspace{.4cm}
In practice, $\blockcipher$ should be \emph{indistinguishable} from a random family of bijections

\vspace{-.1cm}
\begin{itemize}
    \item[\bulletpoint] to satisfy assumptions of security proofs
    \item[\bulletpoint] to avoid stronger attack (\eg{} key recoveries)
\end{itemize}
\end{mybox}


\end{frame}

\begin{comment}
    
\subsection{Security of block ciphers}
\begin{frame}\frametitle{\subsecname}
\vspace{-.5cm}
\only<1>{
\begin{definition}[Block cipher]
A family of bijections $\blockcipher = \left(\black{E}_{\red{k}} \from \FFspace \bij \FFspace\right)_{\red{k} \in \keyspace}$
\end{definition}
}
\only<2->{
\begin{definition}[\emph{Secure} block cipher]
A family of bijections $\blockcipher = \left(\black{E}_{\red{k}} \from \FFspace \bij \FFspace\right)_{\red{k} \in \keyspace}$ which \emph{behaves ideally}.
\end{definition}
}

\onslide<3>{
\begin{center}

\scalebox{.8}{
\begin{tikzpicture}
    %   \scriptsize     
        \coordinate (nw) at (0, 5);
        \coordinate (sw) at (0, 0);
        \coordinate (ne) at (6, 5);
        \coordinate (se) at (6, 0);
        


            \draw[very thick,oipurple,rounded corners=30pt] ($(nw)+(2, -.6)$) rectangle ($(se)+(-.2, .6)$) ;
        \draw[very thick,oiorange,rounded corners=15pt] ($(nw)+(3, -1.5)$) rectangle ($(se)+(-1.9, 2.3)$) ;

        \draw (3.5,2.7) node[] {{\tiny X} $E_{\red{k}}$};

        
        \draw[very thick,oipurple] ($ (ne) + (-.25, -1.4) $) --++(.6,0.7) node[right=1]{$\mathrm{Bij}(\FF_2^n)$};
        \node[very thick,oiorange] at ($ (ne) + (-1.6, -2) $) {\orange{$\mathcal{E}$}};
\end{tikzpicture}
}
\end{center}

\begin{definition}[Indistinguishability]
A \emph{random-looking} family: $[\ \orange{E \xleftarrow{\$} \mathcal{E}} \ ]$ \emph{indistinguishable} from  $[ \ \purple{F \xleftarrow{\$} \mathrm{Bij}(\FF_2^{n})} \ ].$

Otherwise:
\vspace{-.1cm}
\begin{itemize}
    \item[\bulletpoint] contradicts the assumptions of modes
    \item[\bulletpoint] leads to key recoveries.
\end{itemize}
\end{definition}
}
\end{frame}
\end{comment}



% \subsection{Iterated block ciphers}
% \begin{frame}
% \frametitle{\subsecname}
% \vspace{-.5cm}
% \begin{mybox}{Block cipher}{}{}
% A family of bijections $\blockcipher = \left(\black{E}_{\red{k}} \from \FFspace \bij \FFspace\right)_{\red{k} \in \keyspace}$.
% \end{mybox}
% %\scalebox{.65}{% !TeX root = ../these.tex
\begin{tikzpicture}[scale=1]
	
	\foreach \xshift in {0, 2.5, 8} 
	{
	\begin{scope}[xshift=\xshift cm]

			
			%wires
			\begin{scope}[decoration={
    markings,
    mark=at position 0.8 with {\arrow{Stealth}}}
    ] 

			\draw[postaction={decorate}, thick] (.5, 1.7) -- +(0, -.7);		

			\draw[postaction={decorate}, thick] (-1.5, .5) -- +(1.5, 0);

			\ifthenelse{\not{\equal{\xshift}{0}}}{	\draw[postaction={decorate}, thick] (1, .5) -- +(1.5, 0);}

			\end{scope}

		
			%sbox
			\draw[thick, draw=oigray ,fill=white,drop shadow] (0,0) rectangle node[midway] {\huge \gray{$F$}} +(1 , 1) ;

	\end{scope}

	\draw[dashed, thick] (5, .5) -- +(1.5, 0);
				\begin{scope}[decoration={
    markings,
    mark=at position 0.5 with {\arrow{Stealth}}}
    ] 
			% \draw[postaction={decorate}] (1, 1.5) -- +(2, 0);
			\end{scope}


}
	\begin{scope}[decoration={
    markings,
    mark=at position 0.5 with {\arrow{Stealth}}}
    ] 
	\draw[postaction={decorate}, thick] (5, 2.8) node[above]{\red{$k$}} -- +(0, -.5);
	\end{scope}

	% \draw[draw=oired, thick,decorate,decoration={brace, amplitude=10pt}] (1.25, -.25) -- node[midway, below=.3cm] {{Round $0$}} +(-1.5, 0);
	% \draw[draw=oired, thick,decorate,decoration={brace, amplitude=10pt}] (3.75, -.25) -- node[midway, below=.3cm] {{Round $1$}} +(-1.5, 0);
	% \draw[draw=oired, thick,decorate,decoration={brace, amplitude=10pt}] (9.25, -.25) -- node[midway, below=.3cm] {{Round $R-1$}} +(-1.5, 0);

	\node[oired] at (1, 1.35) {$\roundk{0}$};
	\node[oired] at (3.5, 1.35) {$\roundk{1}$};
	\node[oired] at (9.25, 1.35) {$\roundk{R-1}$};

	\node at (-1.8, 0.5) {\Large $\gray{m}$};
	\node at (10.7, 0.5) {\Large $\gray{c}$};

	\draw[thick, draw=oired ,fill=white,drop shadow] (0,1.7) -- node[midway, above=.01] {\large \red{Key schedule}} (9,1.7) -- (8, 2.3) -- (1, 2.3) -- (0,1.7); 
	% rectangle node[midway] {\huge \green{$F$}} +(1 , 3) ;
	\draw[draw=oigray, thick,decorate,decoration={brace, amplitude=10pt}] (9, -.3) -- node[midway, below=.3cm] {\huge $\gray{E_{\red{k}}} = \gray{F}_{\red{k^{(R-1)}}}  \comp \cdots \comp \gray{F}_{\red{k^{(1)}}} \comp \gray{F}_{\red{k^{(0)}}} $} +(-9, 0);
\end{tikzpicture}
} 
% \vspace{.3cm}
% \only<1>{
% \begin{center}
% \toggletrue{figureSPNnoDetail}
% \scalebox{.6}{% !TeX root = ../slides_uvsq.tex
\begin{tikzpicture}[scale=1]
	
	\foreach \xshift in {0, 3.7, 10.4} 
	{
	\begin{scope}[xshift=\xshift cm]
		\foreach \y in {0, 1.5, 4.5}
		{
			\iftoggle{figureSPNnoDetail}{}{
			%sbox
			\draw[thick, draw=oicyan ,fill=white,drop shadow] (0.5,\y) rectangle node[midway] {\huge \cyan{$S$}} +(1,1) ;
			}
			
			%wires
			\foreach \smally in {1, 2, 3, 4} {
			\draw (-.1, \y + .2*\smally) -- +(.6, 0);
			\draw (1.5, \y + .2*\smally) -- +(.5, 0);
			\draw (3, \y + .2*\smally) -- +(.6, 0);
			
			%\ifthenelse{\not{\equal{\xshift}{0}}}{	\draw (3.7, \y + .2*\smally) -- +(.5, 0);}{}
			
			\iftoggle{figureSPNaddroundkey}{
			%xors
			\draw[thick, draw=oired] (3.6,\y + .2*\smally) circle (0.1cm);
			\draw[thick, draw=oired] (3.6, \y + .2*\smally) -- +(0, 0.1)
			(3.6, \y + .2*\smally) -- +(0, -0.1)
			(3.6, \y + .2*\smally) -- +(-.1, 0)
			(3.6, \y + .2*\smally) -- +(.1, 0);

			}{
						\draw[thick, draw=oired, opacity=0] (3.6,\y + .2*\smally) circle (0.1cm);
			\draw[thick, draw=oired, opacity=0] (3.6, \y + .2*\smally) -- +(0, 0.1)
			(3.6, \y + .2*\smally) -- +(0, -0.1)
			(3.6, \y + .2*\smally) -- +(-.1, 0)
			(3.6, \y + .2*\smally) -- +(.1, 0);
			}


			}
		}

		\iftoggle{figureSPNnoDetail}{
		\draw[thick, draw=oigray ,fill=white,drop shadow] (0.5,0) rectangle node[midway] {\huge \gray{$F$}} +(2.5, 5.5);
		}
		{
		\draw[draw=white] (0.5,3) rectangle node[midway] {\huge \cyan{$\vdots$}} +(1,1) ;
		
		\draw[thick, draw=oiorange ,fill=white,drop shadow] (2,0) rectangle node[midway] {\huge \orange{$L$}} +(1,5.5) ;
		}
		
	\end{scope}

	\iftoggle{figureSPNaddroundkey}{
	\draw[thick, draw=oired] (3.6, 4.5 + .2*4) -- +(0, 0.7) node[above] {\red{$rk_{0}$}};
	\draw[thick, draw=oired] (7.3, 4.5 + .2*4) -- +(0, 0.7) node[above] {\red{$rk_{1}$}};
	\draw[thick, draw=oired] (14, 4.5 + .2*4) -- +(0, 0.7) node[above] {\red{$rk_{R}$}};
	}{
	\draw[thick, draw=oired,opacity=0] (3.6, 4.5 + .2*4) -- +(0, 0.7) node[above,opacity=0] {\red{$rk_{0}$}};
	\draw[thick, draw=oired,opacity=0] (7.3, 4.5 + .2*4) -- +(0, 0.7) node[above,opacity=0] {\red{$rk_{1}$}};
	\draw[thick, draw=oired,opacity=0] (14, 4.5 + .2*4) -- +(0, 0.7) node[above,opacity=0] {\red{$rk_{R}$}};
	}
	
	\foreach \y in {0, 1.5, 4.5}
	{
		\foreach \smally in {1, 2, 3, 4} {
		\draw[dashed,oigray] (7.3, \y + .2*\smally) -- +(3, 0);
		}
	}
		\draw[draw=white] (8.2, 3) -- node[midway] {$\gray{\cdots}$} +(1, 1);
	}

	\iftoggle{figureSPNnoDetail}{}{
	\draw[draw=oicyan, thick,decorate,decoration={brace, amplitude=10pt}] (0.5, 5.6) -- node[midway, above=.3cm] {\cyan{Sbox layer}} +(1, 0);
	\draw[draw=oiorange, thick,decorate,decoration={brace, amplitude=10pt}] (3, -.2) -- node[midway, below=.3cm] {\orange{Linear layer}} +(-1, 0);
	}

\end{tikzpicture}
}
% \end{center}
% }

% \only<2>{
% \begin{center}
% \toggletrue{figureSPNnoDetail}
% \toggletrue{figureSPNaddroundkey}
% \scalebox{.6}{% !TeX root = ../slides_uvsq.tex
\begin{tikzpicture}[scale=1]
	
	\foreach \xshift in {0, 3.7, 10.4} 
	{
	\begin{scope}[xshift=\xshift cm]
		\foreach \y in {0, 1.5, 4.5}
		{
			\iftoggle{figureSPNnoDetail}{}{
			%sbox
			\draw[thick, draw=oicyan ,fill=white,drop shadow] (0.5,\y) rectangle node[midway] {\huge \cyan{$S$}} +(1,1) ;
			}
			
			%wires
			\foreach \smally in {1, 2, 3, 4} {
			\draw (-.1, \y + .2*\smally) -- +(.6, 0);
			\draw (1.5, \y + .2*\smally) -- +(.5, 0);
			\draw (3, \y + .2*\smally) -- +(.6, 0);
			
			%\ifthenelse{\not{\equal{\xshift}{0}}}{	\draw (3.7, \y + .2*\smally) -- +(.5, 0);}{}
			
			\iftoggle{figureSPNaddroundkey}{
			%xors
			\draw[thick, draw=oired] (3.6,\y + .2*\smally) circle (0.1cm);
			\draw[thick, draw=oired] (3.6, \y + .2*\smally) -- +(0, 0.1)
			(3.6, \y + .2*\smally) -- +(0, -0.1)
			(3.6, \y + .2*\smally) -- +(-.1, 0)
			(3.6, \y + .2*\smally) -- +(.1, 0);

			}{
						\draw[thick, draw=oired, opacity=0] (3.6,\y + .2*\smally) circle (0.1cm);
			\draw[thick, draw=oired, opacity=0] (3.6, \y + .2*\smally) -- +(0, 0.1)
			(3.6, \y + .2*\smally) -- +(0, -0.1)
			(3.6, \y + .2*\smally) -- +(-.1, 0)
			(3.6, \y + .2*\smally) -- +(.1, 0);
			}


			}
		}

		\iftoggle{figureSPNnoDetail}{
		\draw[thick, draw=oigray ,fill=white,drop shadow] (0.5,0) rectangle node[midway] {\huge \gray{$F$}} +(2.5, 5.5);
		}
		{
		\draw[draw=white] (0.5,3) rectangle node[midway] {\huge \cyan{$\vdots$}} +(1,1) ;
		
		\draw[thick, draw=oiorange ,fill=white,drop shadow] (2,0) rectangle node[midway] {\huge \orange{$L$}} +(1,5.5) ;
		}
		
	\end{scope}

	\iftoggle{figureSPNaddroundkey}{
	\draw[thick, draw=oired] (3.6, 4.5 + .2*4) -- +(0, 0.7) node[above] {\red{$rk_{0}$}};
	\draw[thick, draw=oired] (7.3, 4.5 + .2*4) -- +(0, 0.7) node[above] {\red{$rk_{1}$}};
	\draw[thick, draw=oired] (14, 4.5 + .2*4) -- +(0, 0.7) node[above] {\red{$rk_{R}$}};
	}{
	\draw[thick, draw=oired,opacity=0] (3.6, 4.5 + .2*4) -- +(0, 0.7) node[above,opacity=0] {\red{$rk_{0}$}};
	\draw[thick, draw=oired,opacity=0] (7.3, 4.5 + .2*4) -- +(0, 0.7) node[above,opacity=0] {\red{$rk_{1}$}};
	\draw[thick, draw=oired,opacity=0] (14, 4.5 + .2*4) -- +(0, 0.7) node[above,opacity=0] {\red{$rk_{R}$}};
	}
	
	\foreach \y in {0, 1.5, 4.5}
	{
		\foreach \smally in {1, 2, 3, 4} {
		\draw[dashed,oigray] (7.3, \y + .2*\smally) -- +(3, 0);
		}
	}
		\draw[draw=white] (8.2, 3) -- node[midway] {$\gray{\cdots}$} +(1, 1);
	}

	\iftoggle{figureSPNnoDetail}{}{
	\draw[draw=oicyan, thick,decorate,decoration={brace, amplitude=10pt}] (0.5, 5.6) -- node[midway, above=.3cm] {\cyan{Sbox layer}} +(1, 0);
	\draw[draw=oiorange, thick,decorate,decoration={brace, amplitude=10pt}] (3, -.2) -- node[midway, below=.3cm] {\orange{Linear layer}} +(-1, 0);
	}

\end{tikzpicture}
}
% \end{center}
% }


% \only<3->{
% \begin{center}
% \toggletrue{figureSPNaddroundkey}
% \scalebox{.6}{% !TeX root = ../slides_uvsq.tex
\begin{tikzpicture}[scale=1]
	
	\foreach \xshift in {0, 3.7, 10.4} 
	{
	\begin{scope}[xshift=\xshift cm]
		\foreach \y in {0, 1.5, 4.5}
		{
			\iftoggle{figureSPNnoDetail}{}{
			%sbox
			\draw[thick, draw=oicyan ,fill=white,drop shadow] (0.5,\y) rectangle node[midway] {\huge \cyan{$S$}} +(1,1) ;
			}
			
			%wires
			\foreach \smally in {1, 2, 3, 4} {
			\draw (-.1, \y + .2*\smally) -- +(.6, 0);
			\draw (1.5, \y + .2*\smally) -- +(.5, 0);
			\draw (3, \y + .2*\smally) -- +(.6, 0);
			
			%\ifthenelse{\not{\equal{\xshift}{0}}}{	\draw (3.7, \y + .2*\smally) -- +(.5, 0);}{}
			
			\iftoggle{figureSPNaddroundkey}{
			%xors
			\draw[thick, draw=oired] (3.6,\y + .2*\smally) circle (0.1cm);
			\draw[thick, draw=oired] (3.6, \y + .2*\smally) -- +(0, 0.1)
			(3.6, \y + .2*\smally) -- +(0, -0.1)
			(3.6, \y + .2*\smally) -- +(-.1, 0)
			(3.6, \y + .2*\smally) -- +(.1, 0);

			}{
						\draw[thick, draw=oired, opacity=0] (3.6,\y + .2*\smally) circle (0.1cm);
			\draw[thick, draw=oired, opacity=0] (3.6, \y + .2*\smally) -- +(0, 0.1)
			(3.6, \y + .2*\smally) -- +(0, -0.1)
			(3.6, \y + .2*\smally) -- +(-.1, 0)
			(3.6, \y + .2*\smally) -- +(.1, 0);
			}


			}
		}

		\iftoggle{figureSPNnoDetail}{
		\draw[thick, draw=oigray ,fill=white,drop shadow] (0.5,0) rectangle node[midway] {\huge \gray{$F$}} +(2.5, 5.5);
		}
		{
		\draw[draw=white] (0.5,3) rectangle node[midway] {\huge \cyan{$\vdots$}} +(1,1) ;
		
		\draw[thick, draw=oiorange ,fill=white,drop shadow] (2,0) rectangle node[midway] {\huge \orange{$L$}} +(1,5.5) ;
		}
		
	\end{scope}

	\iftoggle{figureSPNaddroundkey}{
	\draw[thick, draw=oired] (3.6, 4.5 + .2*4) -- +(0, 0.7) node[above] {\red{$rk_{0}$}};
	\draw[thick, draw=oired] (7.3, 4.5 + .2*4) -- +(0, 0.7) node[above] {\red{$rk_{1}$}};
	\draw[thick, draw=oired] (14, 4.5 + .2*4) -- +(0, 0.7) node[above] {\red{$rk_{R}$}};
	}{
	\draw[thick, draw=oired,opacity=0] (3.6, 4.5 + .2*4) -- +(0, 0.7) node[above,opacity=0] {\red{$rk_{0}$}};
	\draw[thick, draw=oired,opacity=0] (7.3, 4.5 + .2*4) -- +(0, 0.7) node[above,opacity=0] {\red{$rk_{1}$}};
	\draw[thick, draw=oired,opacity=0] (14, 4.5 + .2*4) -- +(0, 0.7) node[above,opacity=0] {\red{$rk_{R}$}};
	}
	
	\foreach \y in {0, 1.5, 4.5}
	{
		\foreach \smally in {1, 2, 3, 4} {
		\draw[dashed,oigray] (7.3, \y + .2*\smally) -- +(3, 0);
		}
	}
		\draw[draw=white] (8.2, 3) -- node[midway] {$\gray{\cdots}$} +(1, 1);
	}

	\iftoggle{figureSPNnoDetail}{}{
	\draw[draw=oicyan, thick,decorate,decoration={brace, amplitude=10pt}] (0.5, 5.6) -- node[midway, above=.3cm] {\cyan{Sbox layer}} +(1, 0);
	\draw[draw=oiorange, thick,decorate,decoration={brace, amplitude=10pt}] (3, -.2) -- node[midway, below=.3cm] {\orange{Linear layer}} +(-1, 0);
	}

\end{tikzpicture}
}
% \end{center}
% }

% \end{frame}


%\section{Differential cryptanalysis}
% {
\setbeamercolor{background canvas}{bg=ptblue}	
\begin{frame}[plain]
\vfill
\begin{center}
%\color{white} \Huge \Roman{section} -  \secname
\color{white} \Huge \secname
\end{center}
\vfill
\end{frame}
}

\begin{frame}[fragile]
\frametitle{Differential cryptanalysis}
\vspace{-.4cm}
% \begin{recap}
% $\blockcipher = \left(\black{E}_{\red{k}} \from \FFspace \bij \FFspace\right)_{\red{k} \in \keyspace}$.\hfill $[\ E \xleftarrow{\$} \mathcal{E} \ ]$  or $[ \ F \xleftarrow{\$} \mathrm{Bij}(\FF_2^{n}) \ ]$ ?
% \end{recap}
$\purple{F} \from \FFspace \to \FFspace$. 
\begin{mybox}{Principle}{}{}
Studies for each input difference $\din \neq 0$, the \emph{distribution of output differences}:
    $$\forall \ \dout \in \FFspace, \quad \PP_{x \xleftarrow{\$} \FFspace}\left[\purple{F}(x + \din) + \purple{F}(x) = \dout\right] = \ ? $$
\end{mybox}

\begin{center}

\begin{tikzcd}
\only<1-3>{
                        x  \arrow[leftrightarrow]{d}{\din}[]{} \arrow[opacity=0]{r}[name=uparrow,opacity=0]{\gray{F^{(0)}}} & \white{x^{(1)}} \arrow[leftrightarrow,opacity=0]{d}{\seconddiff} \arrow[dashed,opacity=0]{r}{} &  \white{x^{(R-1)}} \arrow[opacity=0]{r}[opacity=0]{\gray{F^{(r-1)}}} \arrow[leftrightarrow,opacity=0]{d}{}{\beforelastdiff} &  \purple{F}(x) \arrow[leftrightarrow]{d}{\dout}[]{}
                        \\
                        y \arrow[opacity=0]{r}{}[swap,name=bottomarrow,opacity=0]{\gray{F^{(0)}}} & \white{y^{(1)}}  \arrow[dashed,opacity=0]{r}{}[swap]{} &  \white{y^{(R-1)}} \arrow[swap,opacity=0]{r}{}[opacity=0]{\gray{F^{(r-1)}}} &   \purple{F}(y)
                        %\arrow[to path={(uparrow) node[midway, scale=1.2, left=0.05cm] {$\circlearrowleft$} (bottomarrow)}]{}
                        }
                        % \only<4>{      x  \arrow[leftrightarrow]{d}{\din}[]{} \arrow[opacity=1]{r}[name=uparrow,opacity=1]{\gray{F^{(0)}}} & x^{(1)} \arrow[leftrightarrow,opacity=1]{d}{\seconddiff} \arrow[dashed,opacity=1]{r}{} &  x^{(R-1)} \arrow[opacity=1]{r}[opacity=1]{\gray{F^{(r-1)}}} \arrow[leftrightarrow,opacity=1]{d}{}{\beforelastdiff} &  \purple{F}(x) \arrow[leftrightarrow]{d}{\dout}[]{}
                        % \\
                        % y \arrow[opacity=1]{r}{}[swap,name=bottomarrow,opacity=1]{\gray{F^{(0)}}} & y^{(1)}  \arrow[dashed,opacity=1]{r}{}[swap]{} &  y^{(R-1)} \arrow[swap,opacity=1]{r}{}[opacity=1]{\gray{F^{(r-1)}}} &   \purple{F}(y)
                        % }
            
\end{tikzcd}
\end{center}

\pause
\begin{mybox}{Average over all bijections}{}{}
For all $(\din \neq 0, \dout)$, the equation $\purple{F}(x + \din) + \purple{F}(x) = \dout$ has 1 solution $x$ \emph{on average}.
\end{mybox}


\pause
\begin{mybox}{Differential distinguisher\hfill {\small\purple{[BihSha91]}}}{}{}
$(\din, \dout)$ such that for many $\red{k}$, $\quad E_{\red{k}}(x + \din) + E_{\red{k}}(x) = \dout$ has many solutions $x$.
\end{mybox}

\end{frame}


\subsection{Resisting against differential attacks}
\begin{frame}\frametitle{\subsecname}
\vspace{-.5cm}

\begin{mybox}{Differential distinguisher\hfill {\small\purple{[BihSha91]}}}{}{}
$(\din, \dout)$ s.t for many $\red{k}$, $\quad E_{\red{k}}(x + \din) + E_{\red{k}}(x) = \dout$ has many solutions $x$.
\end{mybox}

\pause
\begin{mybox}{Differential resistance}{}{}
For all $(\din, \dout)$ and all keys $\red{k}$, $\quad E_{\red{k}}(x + \din) + E_{\red{k}}(x) = \dout$ has \emph{few} solutions.
\end{mybox}

% \begin{mybox}{For a random bijection \purple{$F$}}{}{}
% $\purple{F}(x + \din) + \purple{F}(x) = \dout$ has 1 solution $x$ on average.
% \end{mybox}
%If $x$ is a solution of Eq.~\eqref{eq:diff}, then so is $x + \din$.  \hfill$\implies \ $ even number of solutions $\delta_{F}(\din, \dout)$

\pause
\vspace{-.3cm}
\begin{mybox}{How to achieve this}{}{}
For all $(\din, \dout)$,  $\quad \purple{S}(x + \din) + \purple{S}(x) = \dout$ has \emph{few} solutions.
\end{mybox}
\pause

$$ \delta_{\purple{S}}(\din, \dout) = \card{\set{x \mid \purple{S}(x + \din) + \purple{S}(x) = \dout}}$$


\begin{columns}[c]
\begin{column}{0.5\textwidth}

    \scalebox{.5}{% !TeX root = ../slides_uvsq.tex
\begin{tikzpicture}[scale=1]
	
	\foreach \xshift in {0, 3.7} 
	{
	\begin{scope}[xshift=\xshift cm]
		\foreach \y in {0, 1.5, 4.5}
		{
			%sbox
			\draw[thick, draw=black ,fill=white,drop shadow] (0.5,\y) rectangle node[midway] {\large \black{$S$}} +(1,1) ;
			
			%wires
			\foreach \smally in {1, 2, 3, 4} {
			\draw (-.1, \y + .2*\smally) -- +(.6, 0);
			\draw (1.5, \y + .2*\smally) -- +(.5, 0);
			\draw (3, \y + .2*\smally) -- +(.6, 0);
			
			%\ifthenelse{\not{\equal{\xshift}{0}}}{	\draw (3.7, \y + .2*\smally) -- +(.5, 0);}{}
			
			%xors
			\draw[thick, draw=ptred] (3.6,\y + .2*\smally) circle (0.1cm);
			\draw[thick, draw=ptred] (3.6, \y + .2*\smally) -- +(0, 0.1)
			(3.6, \y + .2*\smally) -- +(0, -0.1)
			(3.6, \y + .2*\smally) -- +(-.1, 0)
			(3.6, \y + .2*\smally) -- +(.1, 0);


			}
		}
		\draw[draw=white] (0.5,3) rectangle node[midway] {\large \black{$\vdots$}} +(1,1) ;
		
		\draw[thick, draw=black ,fill=white,drop shadow] (2,0) rectangle node[midway] {\large \black{$L$}} +(1,5.5) ;
		
	\end{scope}

	\draw[thick, draw=ptred] (0, 4.5 + .2*4) -- +(0, 0.7) node[above] {\red{$rk_{0}$}};
	\draw[thick, draw=ptred] (3.6, 4.5 + .2*4) -- +(0, 0.7) node[above] {\red{$rk_{1}$}};
	\draw[thick, draw=ptred] (7.3, 4.5 + .2*4) -- +(0, 0.7) node[above] {\red{$rk_{2}$}};
			\foreach \y in {0, 1.5, 4.5}{
				\foreach \smally in {1, 2, 3, 4} {
			\draw[thick, draw=ptred] (0,\y + .2*\smally) circle (0.1cm);
			\draw[thick, draw=ptred] (0, \y + .2*\smally) -- +(0, 0.1)
			(0, \y + .2*\smally) -- +(0, -0.1)
			(0, \y + .2*\smally) -- +(-.1, 0)
			(0, \y + .2*\smally) -- +(.1, 0);
			}
			}
	
	}

	%\draw[draw=ptred, thick,decorate,decoration={brace, amplitude=5pt}] (3.3, 5.6) -- node[midway, above=.3cm] {\red{Key addition}} +(.6, 0);

\end{tikzpicture}
}


\end{column}

\begin{column}{0.5\textwidth}


$\PP[\din, \cyan{\Delta}, \dout] \leq \left(\frac{\max\limits_{a \neq 0, b} \delta_{\purple{S}}(a, b)}{2^{m}}\right)^{d(\green{L})}$


\end{column}
\end{columns}



% \vspace{.3cm}
% \begin{itemize}
%     \item[\bulletpoint] For all $\din$, there exists $\dout$ such that $\delta_{\purple{F}}(\din, \dout) >0$\pause
%     \item[\bulletpoint] For all $\din \neq 0 , \dout$, $x$ is a solution iff $x + \din$ is a solution. \hfill $\delta_{\purple{F}}(\din, \dout)$ is even.
% \end{itemize}
% \pause

% \begin{mybox}{Definition}{APN function}{\purple{\small[NybKnu92]}}\label{def:apn}
%     A function $F$ is APN if: $\quad \forall \ \din\neq 0, \dout, \quad \delta_{F}(\din, \dout) \leq 2$.
% \end{mybox}
\end{frame}

\begin{frame}\frametitle{Differentially-optimal functions}
\begin{mybox}{How to achieve this}{}{}
For all $\din\neq 0, \dout $  $\quad \delta_{\purple{S}}(\din, \dout) = \card{\set{x \mid \purple{S}(x + \din) + \purple{S}(x) = \dout}}$ should be \emph{low}.
\end{mybox}\pause

\vspace{.5cm}
\begin{itemize}
    \item[\bulletpoint] For all $\din$, there exists $\dout$ such that $\delta_{\purple{S}}(\din, \dout) >0$\pause
    \item[\bulletpoint] For all $\din \neq 0 , \dout$, $x$ is a solution iff $x + \din$ is a solution. \hfill $\delta_{\purple{S}}(\din, \dout)$ is even.
\end{itemize}
\pause

\vspace{1cm}
\begin{mybox}{Almost perfect non-linear (APN) function}{}{\purple{\small[NybKnu92]}}
    A function $\purple{F}$ is APN if: $\quad \forall \ \din\neq 0, \dout, \quad \delta_{\purple{F}}(\din, \dout) \leq 2$.
\end{mybox}

\end{frame}

\begin{frame}[t]\frametitle{Almost perfect non-linear (APN) function}

\begin{mybox}{Definition}{APN function}{\purple{\small[NybKnu92]}}
    A function $\purple{F}$ is APN if: $\quad \forall \ \din\neq 0, \dout, \quad \delta_{\purple{F}}(\din, \dout) \leq 2$.
\end{mybox}

\pause
\begin{mybox}{A typical classification problem}{}{}
\begin{itemize}
    \item[-] \green{Easy} definition
    \item[-] \red{Hard} to find new instances (even for small $n$)
    \item[-] \red{Hard} to classify the known instances
    \item[-] Lots of open problems
\end{itemize}
\end{mybox}

\pause
\begin{mybox}{Big APN problem}{}{\purple{\small[BDMW10]}}
Find $F\from \FFspace \to \FFspace$ which is APN, \emph{bijective} for an \emph{even} $n$.

\hfill A \emph{single} example is known for $n=6$.\hfill 
\end{mybox}

\end{frame}

\begin{comment}

\subsection{Finding an APN function}
\begin{frame}
\frametitle{\subsecname}
\vspace{-.5cm}

% \begin{center}
% $4$ bits $\approx$ a pair of $2$-bit words $\approx$ a $4$-bit word    
% \end{center}


\begin{mybox}{Use alternative representations}{}{}
$F \from \FF_{2}^{4} \to \FF_{2}^{4}, \begin{pmatrix}
x_{0}\\ x_{1}\\ x_{2}\\ x_{3}
\end{pmatrix} \mapsto \begin{pmatrix}
    x_0x_2 + x_0 + x_1x_2 + x_1x_3\\ 
x_0x_1 + x_0x_2 + x_2x_3 + x_3\\ 
x_0x_1 + x_0x_2 + x_0x_3 + x_1x_2 + x_1x_3 + x_2x_3 + x_2\\ 
x_1x_3 + x_1 + x_2x_3 + x_2 + x_3
\end{pmatrix}$
\pause

\vspace{.5cm}
$F \from \FF_{4}^{2} \to \FF_{4}^{2}, (x_{0}, x_{1}) \mapsto \begin{pmatrix}\orange{\alpha_{0}}x_{0}^3 + x_{0}^2x_{1} + \orange{\alpha_{1}}x_{0}x_{1}^2 + \orange{\alpha_{2}}x_{1}^3\\
\orange{\alpha_{3}}x_{0}^3 + \orange{\alpha_{4}}x_{0}^2x_{1} + \orange{\alpha_{5}}x_{0}x_{1}^2\end{pmatrix}$\hfill $\alpha_{i} \in \FF_{4}$.
\pause

\vspace{.5cm}
$F \from \FF_{16} \to \FF_{16}, X \mapsto X^{3}$\hfill 3 representations of the \emph{``same''} function
\end{mybox}

\onslide<4>{
$$  (X + \plaindiff)^{3} + X^{3} = \plaindiff X^{2} + \plaindiff^{2} X + \plaindiff^{3}$$

Quadratic equation $\implies$ at most 2 solutions $\implies$ APN ! }


\end{frame}

\subsection{A (not so welcoming) state of the art}
\begin{frame}
\frametitle{\subsecname}

% !TEX root = ../slides.tex

\begin{columns}[c]
\begin{column}{0.5\textwidth}
\renewcommand\arraystretch{1.3} 
\scalebox{.88}{
\begin{tabular}{|c|}
\toprule
\rctt\textbf{Univariate}\\
\midrule
$x^{2^s + 1} + ax^{2^{(3-i)k + s} + 2^{ik}}$\\
\rct$x^{2^s + 1} + ax^{2^{(4-i)k + s} + 2^{ik}} $\\
$ax^{2^k + 1} + x^{2^s +1} + x^{2^{s + k} + 2^k} + bx^{2^{k + s} + 1} + b^{2^{k}}x^{2^s + 2^k}$\\
\rct$x^{3} + a^{-1}\tr[\FF_{2^n}][\FF_2](a^3x^9)$\\
$x^{3} + a^{-1}\tr[\FF_{2^n}][\FF_{2^3}](a^3x^9 + a^6x^{18})$\\
\rct$x^{3} + a^{-1}\tr[\FF_{2^n}][\FF_{2^3}](a^6x^{18} + a^{12}x^{36})$\\
$ax^{2^s + 1} + a^{2^k}x^{2^{2k} + 2^{k + s}} + bx^{2^{2k} + 1} + ca^{2^k + 1}x^{2^{s} + 2^{k + s}}$\\
\rct$a^{2}x^{2^{2k + 1} + 1} + b^{2}x^{2^{k +1} + 1} + ax^{2^{2k} + 2} + bx^{2^{k} + 2} + dx^{3}$\\
$ x^3 + ax^{2^{s+i} + 2^i} + a^2x^{2^{k+1} + 2^k} + x^{2^{s + i + k} + 2^{i + k}}$\\
\rct$ a\tr[\FFfield][\subfield](bx^{2^i + 1}) + a^{2^k}\tr[\FFfield][\subfield](cx^{2^s + 1})$\\
$ L(x)^{2^k + 1} + bx^{2^k + 1} $\\
\bottomrule
\end{tabular}}
\end{column}
\begin{column}{0.5\textwidth}
% !TEX root = ../slides.tex
\renewcommand\arraystretch{.9} 
\scalebox{.77}{
\begin{tabular}{|c|}
\toprule
\rctt \multicolumn{1}{c|}{\textbf{Multivariate}}\\ 
\midrule
$(x,y) \mapsto \left(\begin{array}{c} x^{2^s + 1} + ay ^{(2^s+1)2^i}\\ xy\end{array}\right)$\\
\rct$(x,y) \mapsto \left(\begin{array}{c}x^{2^{2s} + 2^{3s}} + ax^{2^{2s}}y^{2^s} + by^{2^s+1}\\ xy\end{array}\right)$\\
$(x,y) \mapsto \left(\begin{array}{c}x^{2^s+1} + x^{2^{s + k/2}}y^{2^{k/2}} + axy^{2^s} + by^{2^s+1}\\ xy\end{array}\right)$\\
\rct$(x,y) \mapsto \left(\begin{array}{c}x^{2^s+1} + xy^{2^{s}} + y^{2^s + 1}\\ x^{2^{2s}+1} + x^{2^{2s}}y + y^{2^{2s} + 1}\end{array}\right)$\\
$(x,y) \mapsto \left(\begin{array}{c}x^{2^s+1} + xy^{2^{s}} + y^{2^s + 1}\\ x^{2^{3s}}y + xy^{2^{3s}}\end{array}\right)$\\
\rct$(x,y) \mapsto \left(\begin{array}{c}x^{2^s+1} + by^{2^s + 1}\\ x^{2^{s + k/2}}y + \frac{a}{b}xy^{2^{s + k/2}}\end{array}\right)$\\
$(x,y) \mapsto \left(\begin{array}{c}x^{2^s + 1} + xy^{2^s} + ay^{2^s +1}\\ x^{2^{2s} + 1} + ax^{2^{2s}}y + (1 + a)^{2^s}xy^{2^{2s}} + ay^{2^{2s} + 1}\end{array}\right)$\\
\rct$(x,y,z) \mapsto \left(\begin{array}{c}x^{2^s+1} + x^{2^s}z + yz^{2^s}\\x^{2^s}z + y^{2^s+1}\\xy^{2^s} + y^{2^s}z + z^{2^s+1}\end{array}\right)$\\
$(x,y,z) \mapsto \left(\begin{array}{c}x^{2^s+1} + xy^{2^s} + yz^{2^s}\\xy^{2^s} + z^{2^s+1}\\x^{2^s}z + y^{2^s+1} + y^{2^s}z\end{array}\right)$\\
\bottomrule
\end{tabular}
}
\end{column}
\end{columns}

\onslide<2->{
\begin{tikzpicture}[overlay]
     \node[fill=white] at (11.5,.6) {\inlinebox{ + a lot of sporadic examples for small $n$}};
 \end{tikzpicture}}

 \onslide<3>{
\begin{tikzpicture}[overlay]
     \node[fill=white] at (8,6) {\inlinebox{\red{Where to look for new function ?}}};

     \node[fill=white] at (8,4.5) {\inlinebox{\red{How to prove that a new $F$ is \emph{actually} new ? }}};
 \end{tikzpicture}}

\end{frame}


\subsection{Step 1: understanding the SotA}
\begin{frame}
\frametitle{\subsecname}

\begin{mybox}{Theorem}{Linear self-equivalence}{\small \purple{[\blue{B}CanteautPerrin24]}}
For (almost) any $F$ presented before, there exists $\AA, \BB$ linear bijective such that:
$$ F \comp \AA = \BB \comp F$$
\end{mybox}

\pause
\begin{corollary}
    Any of the pen-and-paper APN functions are still \emph{very close to be} monomial.
\end{corollary}
\pause
    Let $c \in \FF_{16}$ and $F \from X \mapsto X^{3}$. Then $F(cX) = c^{3}X^{3} = c^{3}F(X^{3})$

    $ \AA \from X \mapsto cX, \BB \from X \mapsto c^{3}X.$ 

\pause
\begin{mybox}{A lot of open questions}{}{}
    \begin{itemize}
        \item[-] Inherent property of APN functions or due to a Human bias ?
        \item[-] Find more APN with a related structure or without !
    \end{itemize}
\end{mybox}

\end{frame}



\subsection{Steps 2 and 3: classify then search}
\begin{frame}
\frametitle{\subsecname}

\begin{mybox}{Classification}{}{}
\begin{itemize}
    \item[-] Implementation of the pen-and-paper families \hfill(almost done)
    \item[-] \emph{Data base} of the known functions \hfill(in progress)
    \item[-] Compute all interesting \emph{invariants} (e.g existence of $\AA, \BB$. . .) \hfill (not started)
\end{itemize}
\end{mybox}
\pause

\begin{mybox}{Search}{}{}
\begin{itemize}
    \item[-] Generic tree search for APN with a fixed $\AA, \BB$ \hfill(based on \purple{[BeiBriLea21]}, done)
    \item[-] Optimization \hfill(not started)
    \item[-] Search functions with more branches (in 4/5/6 variables ?)\hfill(not started)
    \item[-] Search functions with high degree (another open problem)\hfill(not started)
\end{itemize}
\end{mybox}
\end{frame}


\subsection{Take away}
\begin{frame}
\frametitle{\subsecname}

\begin{mybox}{APN functions}{}{}
\begin{itemize}
    \item[-] \emph{Optimal} objects w.r.t \emph{differential} cryptanalysis
    \item[-] Very little is known, finding/classifying is hard
    \item[-] Not trendy, but exciting !
\end{itemize}
\end{mybox}


\begin{mybox}{Big APN Problem}{}{\purple{\small[BDMW10]}}
Find an APN bijection $F \from \FFspace \to \FFspace$ with $n > 6$ even.
\end{mybox}

\begin{mybox}{Work in progress}{}{}
\begin{itemize}
    \item[-] Data base of APN functions + invariants
    \item[-] Search for functions with more branches
    \item[-] Search for functions with high degree
\end{itemize}
\end{mybox}
\end{frame}


\subsection{References}
\begin{frame}
\frametitle{\subsecname}
\vspace{-1cm}
\begin{mybox}{My bedside readings}{}{}
\begin{itemize}
    \item[-] Anne Canteaut, \emph{Lecture Notes on Cryptographic Boolean Functions} (available online, easily-accessible course)
    \item[-] Rudolf Lidl \& Harald Niederreiter, \emph{Finite fields} (my must-read about finite fields, with all standards results (and way more))
    \item[-] Claude Carlet, \emph{Boolean Functions for Cryptography and Coding Theory} (available online, not my favorite book, but it has the benefit of pointing toward 1000+ articles)
\end{itemize}
\end{mybox}

\begin{mybox}{Papers cited in prev. slides}{}{}
\begin{itemize}
    \item[-] [BihSha91]  Biham \& Shamir, \emph{Differential Cryptanalysis of the Data Encryption Standard}
    \item[-] [NybKnu92] Nyberg \& Knudsen, \emph{Provable Security Against Differential Cryptanalysis}, CRYPTO 92
    \item[-] [BDMW10] Browning, Dillon, McQuistan \& Wolfe, \emph{An APN Permutation in Dimension Six}, Fq9 conference 2009 (Finite Fields and their Applications)
    \item[-] [BauCanPer24] Baudrin, Canteaut, Perrin, \emph{Linear self-equivalence of the known families of APN functions: a unified point of view} (submitted)
    \item[-] [BeiBriLea21] \emph{Linearly self-equivalent APN permutations in small dimension}, IEEE IT 2021
\end{itemize}
\end{mybox}

\end{frame}
\end{comment}





