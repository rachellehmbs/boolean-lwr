% !TEX root = ./slides_jc2.tex
%!TEX program = lualatex
\documentclass[t, aspectratio=1610, 10pt,dvipsnames]{beamer}

% Font + Maths packages
\usepackage[english]{babel}
\usepackage[utf8]{inputenc}
\usepackage[T1]{fontenc}
\usepackage{amsmath,amssymb,amsthm,mathtools,stmaryrd,bm} % stmaryrd for [[  ]] 
\usepackage{calc}
\usepackage{chronology}
\usepackage{array}
% Tikz and options
\usepackage{tikz}
\usepackage{tikz-cd}
\usepackage{overpic}
\usepackage{centernot}

\usetikzlibrary{cd, hobby, babel}
\usetikzlibrary{shadows, calc}
\usetikzlibrary{decorations.pathreplacing} 
\usetikzlibrary{shadows,cipher,sponge}
\usetikzlibrary{decorations.markings}

\usepackage{pgfplots}
\usepgfplotslibrary{fillbetween}

\usetikzlibrary{ decorations.markings, arrows}
%\usetikzlibrary{patterns}

\usepackage{fontawesome}


% Beamer layouts

\usepackage{float} % use for tikz figures
\usepackage[absolute,overlay]{textpos} % for textblock and absolute positioning
\usepackage{multicol, multirow}
\usepackage{adjustbox}
\usepackage{tcolorbox}
\usepackage{ifthen} %for spn figure
\usepackage{enumitem} % for leftmargin option for itemize

\usepackage{etoolbox} % for toggle and conditional variables, easy to use

\usepackage{cancel} %  pour barrer du texte
\usepackage{nth}
\usepackage{booktabs} % midrule/toprule...

\usepackage[table]{xcolor}
%\usepackage[absolute,overlay]{textpos} 

\newcommand{\Plus}{\mathord{\begin{tikzpicture}[baseline=0ex, line width=1.3, scale=0.08]
								\draw (1,0) -- (1,2);
								\draw (0,1) -- (2,1);

\end{tikzpicture}}}


%------------ Beamer colors -----------
\definecolor{oiblack}{RGB}{0, 0, 0}
\definecolor{oigreen}{RGB}{0, 158, 115}
\definecolor{oiblue}{RGB}{0, 114, 178}
\definecolor{oicyan}{RGB}{86, 180, 233}
\definecolor{oiyellow}{RGB}{240, 228, 066}
\definecolor{oiorange}{RGB}{230, 159, 0}
\definecolor{oired}{RGB}{213, 094, 0}
\definecolor{oipurple}{RGB}{204, 121, 167}

\definecolor{ptblack}{RGB}{0, 0, 0}
\definecolor{ptgreen}{RGB}{0, 153, 136}
\definecolor{ptblue}{RGB}{68, 119, 187}
\definecolor{ptcyan}{RGB}{51, 187, 238}
\definecolor{ptyellow}{RGB}{240, 228, 066}
\definecolor{ptorange}{RGB}{238, 119, 51}
\definecolor{ptred}{RGB}{204, 51, 17}
\definecolor{ptpurple}{RGB}{238, 51, 119}

\definecolor{cgreen}{RGB}{0, 158, 115}
\definecolor{cblue}{RGB}{0, 114, 178}
\definecolor{ccyan}{RGB}{86, 180, 233}
\definecolor{cyellow}{RGB}{240, 228, 066}
\definecolor{corange}{RGB}{230, 159, 0}
\definecolor{cred}{RGB}{213, 094, 0}
\definecolor{cpurple}{RGB}{204, 121, 167}

\definecolor{oigray}{RGB}{118, 118, 118}

\definecolor{purple4}{RGB}{116, 40, 129}
\definecolor{purple3}{RGB}{152, 110, 172}
\definecolor{purple2}{RGB}{195, 164, 207}
\definecolor{purple1}{RGB}{229, 212, 232}
\definecolor{green1}{RGB}{217, 241, 213}
\definecolor{green2}{RGB}{173, 212, 160}
\definecolor{green3}{RGB}{92, 174, 99}
\definecolor{green4}{RGB}{27, 121, 57}

\renewcommand{\emph}[1]{{\color{ptorange}\textit{\textbf{#1}}}} % EMPHASIZED IN GRAY

\newcommand{\white}[1]{{\color{white!0} #1}}
\newcommand{\green}[1]{{\color{ptgreen} #1}}
\newcommand{\blue}[1]{{\color{ptblue} #1}}
\newcommand{\cyan}[1]{{\color{ptcyan} #1}}
\newcommand{\yellow}[1]{{\color{oiyellow} #1}}
\newcommand{\orange}[1]{{\color{ptorange} #1}}
\newcommand{\red}[1]{{\color{ptred} #1}}
\newcommand{\purple}[1]{{\color{ptpurple} #1}}
\newcommand{\gray}[1]{{\color{oigray} #1}}
\newcommand{\black}[1]{{\color{oiblack} #1}}


%====== Table color
\colorlet{sectioncolor}{ptblue}
\colorlet{titletablecolor}{sectioncolor!40}
\colorlet{tablecolor}{sectioncolor!15}
% Colored column
\newcolumntype{a}{>{\columncolor{tablecolor}}c}
\newcolumntype{t}{>{\columncolor{titletablecolor}}c}
\newcommand{\cct}{\cellcolor{tablecolor}}
\newcommand{\ccw}{\cellcolor{white}}
\newcommand{\cctt}{\cellcolor{titletablecolor}}
\newcommand{\rct}{\rowcolor{tablecolor}}
\newcommand{\rctt}{\rowcolor{titletablecolor}}

% ----------- MY BEAMER THEME ----------------
\useoutertheme{infolines}
%\useoutertheme[subsection=false]{smoothbars}

\setbeamercolor{frametitle}{fg=ptblue,bg=white}
\setbeamercolor{subsection in head/foot}{fg=white, bg=oigreen}
\setbeamercolor{section in head/foot}{fg=ptblue, bg=white}
\setbeamercolor{author in head/foot}{fg=white, bg=oigreen}
\setbeamercolor{title in head/foot}{fg=white, bg=oigreen}
\setbeamercolor{date in head/foot}{fg=white, bg=oigreen}

\usetheme{Frankfurt}

\setbeamertemplate{headline}{}


\makeatletter
\setbeamertemplate{footline}{
  \pgfuseshading{beamer@barshade}%
  \vskip-0.05ex%
  \begin{beamercolorbox}[wd=\paperwidth,ht=1ex,dp=0ex]{empty}
  \begin{pgfpicture}
    \begin{pgfscope}%
      \ifbeamer@sb@subsection%
        \pgfsetfillcolor{subsection in head/foot.bg}%
      \else%
        \pgfsetfillcolor{section in head/foot.bg}%
      \fi%
      \pgfpathrectangle{\pgfpoint{-.5\paperwidth}{-0.5ex}}{\pgfpoint{\paperwidth}{1ex}}%
      \pgfsetfading{beamer@belowframetitlemask}{}%
      \pgfusepath{fill}%
    \end{pgfscope}%
  \end{pgfpicture}%
  \end{beamercolorbox}%
  \ifbeamer@sb@subsection%
    \vskip-9.75ex%
  \else%
    \vskip-6.85ex%
  \fi%
  \begin{beamercolorbox}[ignorebg,ht=2.25ex,dp=3.75ex]{section in head/foot}
    \insertnavigation{0.93\paperwidth}
    \raisebox{-.1cm}{\footnotesize \insertframenumber/\inserttotalframenumber}
  \end{beamercolorbox}%
  \ifbeamer@sb@subsection%
    \begin{beamercolorbox}[ignorebg,ht=2.125ex,dp=1.125ex,%
      leftskip=.3cm,rightskip=.3cm plus1fil]{subsection in head/foot}
      \usebeamerfont{subsection in head/foot}\insertsubsectionhead
    \end{beamercolorbox}%
  \fi%
}
\makeatother

\usepackage{appendixnumberbeamer}
\beamertemplatenavigationsymbolsempty

\setbeamerfont{frametitle}{size=\large}
\setbeamertemplate{frametitle}[default][center]

%----------- My color box ------------
\tcbuselibrary{skins}

\newtcolorbox{mybox}[3]{
	beamer,
	enhanced jigsaw,
	adjusted title=flush center,
	arc=0mm,
	size=fbox,
	colframe=white,
	colback=white,
	colbacktitle=white,
	opacitybacktitle=0.3,
	opacityback=0.1,
	coltitle=black,
	no shadow,
	%boxrule=.4mm,
  borderline west ={1.2pt}{-3pt}{ptblue},
  overlay unbroken and last={%
        \draw[line width=1.2pt, ptblue] (frame.south west) + (-.1cm,0)  -- ++(0:.2cm);},
	titlerule=0mm,
	top=0mm,
	boxsep=1mm,
	fonttitle=\normalsize\bfseries,
	fontlower=\normalsize,
	title={{\blue{#1}} \ifthenelse { \equal {#2} {} }{}{(#2)}\hfill#3},
	before upper={\setbeamercolor{item}{fg=black}},
	after upper={\setbeamercolor{item}{fg=normal color}},
}

\newenvironment{recap}[1][]{\mybox{Recap}{#1}{\recapsymbol}\ignorespaces}{\endmybox\ignorespacesafterend}
\renewenvironment{definition}[1][]{\mybox{Definition}{#1}{}\ignorespaces}{\endmybox\ignorespacesafterend}
\newenvironment{proposition}[1][]{\mybox{Proposition}{#1}{}\ignorespaces}{\endmybox\ignorespacesafterend}
\renewenvironment{theorem}[1][]{\mybox{Theorem}{#1}{}\ignorespaces}{\endmybox\ignorespacesafterend}
% \renewenvironment{lemma}[1][]{\mybox[Lemma]{#1}\ignorespaces}{\endmybox\ignorespacesafterend}
% \newenvironment{remark}[1][]{\mybox[Remark]{#1}\ignorespaces}{\endmybox\ignorespacesafterend}
\newenvironment{assumption}[1][]{\mybox{Assumption}{#1}{}\ignorespaces}{\endmybox\ignorespacesafterend}
% \renewenvironment{problem}[1][]{\mybox[Problem]{#1}\ignorespaces}{\endmybox\ignorespacesafterend}
\newenvironment{goal}[1][]{\mybox{Goal}{#1}{}\ignorespaces}{\endmybox\ignorespacesafterend}
\newenvironment{constraints}[1][]{\mybox{Constraints}{#1}{}\ignorespaces}{\endmybox\ignorespacesafterend}
\renewenvironment{example}[1][]{\mybox{Example}{#1}{}\ignorespaces}{\endmybox\ignorespacesafterend}
\renewenvironment{corollary}[1][]{\mybox{Corollary}{#1}{}\ignorespaces}{\endmybox\ignorespacesafterend}



\newtcbox{\inlinebox}{
hbox,
  colback=white,% background color
  on line,% don't add line breaks
 % before upper=\vphantom{Ay},% ensure constant height
}
\usetikzlibrary{patterns, crypto.symbols, arrows} 
% !TEX root = ./slides_jc2.tex

\newcommand{\bulletpoint}{\raisebox{.08cm}{\tiny $\bullet$~}}
\newcommand\scalemath[2]{\scalebox{#1}{\mbox{\ensuremath{\displaystyle #2}}}}
\newcommand{\leadstobig}{\scalemath{1.5}{\leadsto}}
\newcommand{\recapsymbol}{\green{\faHistory}}
%========= Maths sets=========
\newcommand{\NN}{\mathbb{N}}
\newcommand{\ZZ}{\mathbb{Z}}
\newcommand{\QQ}{\mathbb{Q}}
\newcommand{\RR}{\mathbb{R}}
\newcommand{\CC}{\mathbb{C}}
\newcommand{\FF}{\mathbb{F}}
\newcommand{\GG}{\mathbb{G}}
\newcommand{\LL}{\mathbb{L}}
\newcommand{\KK}{\mathbb{K}}
\newcommand{\PP}{\mathbb{P}}
\newcommand{\TT}{\mathbb{T}}
\newcommand{\XX}{\mathbb{X}}
\newcommand{\ind}{\mathbf{1}}

%multisets
\newcommand*{\ldblbrace}{\{\mskip-5mu\{} 
\newcommand*{\rdblbrace}{\}\mskip-5mu\}}

% Equivalence
\newcommand{\equivcolor}[1]{\orange{#1}}
\renewcommand{\AA}{\equivcolor{A}}
\newcommand{\BB}{\equivcolor{B}}
\newcommand{\EE}{\equivcolor{E}}
\newcommand{\DD}{\equivcolor{D}}
\newcommand{\Caff}{\equivcolor{C}}
\newcommand{\calAA}{\equivcolor{\mathcal{A}}}
\newcommand{\calLL}{\equivcolor{\mathcal{L}}}
\newcommand{\calBB}{\equivcolor{\mathcal{B}}}
\newcommand{\cstterm}{\equivcolor{c}}

\newcommand{\ccz}{\sim_{\mathrm{CCZ}}}
\newcommand{\ea}{\sim_{\mathrm{EA}}}
\newcommand{\lin}{\sim_{\mathrm{lin}}}
\newcommand{\similar}{\sim_{\mathrm{sim}}}
\newcommand{\aff}{\sim_{\mathrm{aff}}}
\newcommand{\graph}{\mathcal{G}}


% AES round functions
\newcommand{\SB}{\mathsf{SB}}
\newcommand{\SR}{\mathsf{SR}}
\newcommand{\MC}{\mathsf{MC}}
\newcommand{\SC}{\mathsf{SC}}

\newcommand{\ShiftRows}{\textsf{ShiftRows}}
\newcommand{\MixColumns}{\textsf{MixColumns}}
\newcommand{\ShuffleCells}{\textsf{ShuffleCells}}

\newcommand{\whitekey}{\mathrm{WK}}

\newcommand{\succset}{\mathrm{Succ}}
\newcommand{\precset}{\mathrm{Prec}}

% ASCON
\newcommand{\IV}{\mathrm{IV}}
\newcommand{\nonce}{N}
\newcommand{\aeadtag}{T}
\newcommand{\setascon}[1]{Z_{#1}}
\newcommand{\classa}{\blue{\mathcal{C}(a_{0}+1, e_{0})}}
\newcommand{\classb}{\orange{\mathcal{C}(a_{0}, e_{0})}}
\newcommand{\cubea}{\blue{v}}
\newcommand{\cubeb}{\orange{w}}
\newcommand{\cubeas}[1]{\blue{v \lll #1}}
\newcommand{\cubebs}[1]{\orange{w \lll #1}}
\newcommand{\mcubea}{\blue{x^{v}}}
\newcommand{\mcubeb}{\orange{x^{w}}}
\newcommand{\mcubeas}[1]{\blue{x^{v \lll #1}}}
\newcommand{\mcubebs}[1]{\orange{x^{w \lll #1}}}
\newcommand{\classai}[1]{\blue{\mathcal{C}(a_{#1}+1, e_{#1})}}
\newcommand{\classbi}[1]{\orange{\mathcal{C}(a_{#1}, e_{#1})}}
\newcommand{\rin}{r_{\mathrm{in}}}
\newcommand{\rout}{r_{\mathrm{out}}}


%============ Maths/info======
\newcommand{\re}{\mathrm{Re}}
\newcommand{\im}{\mathrm{Im}}
\newcommand{\rank}{\mathrm{rk}}
\newcommand{\xor}{\oplus}
\newcommand{\from}{\colon}
\newcommand{\set}[1]{\left\{#1\right\}} % { }
\newcommand{\spanset}[1]{\left\langle#1\right\rangle} % < >
\newcommand{\intset}[1]{\left\llbracket #1 \right\rrbracket} % [[  ]]
\newcommand{\multiset}[1]{\ldblbrace #1 \rdblbrace}
\newcommand{\abs}[1]{\left\lvert #1 \right\rvert} % absolute value | |
\newcommand{\card}[1]{\left\lvert #1 \right\rvert} % cardinal | |
\newcommand{\comp}{\circ}
\newcommand{\modadd}[1]{\boxplus_{#1}} % modular addition
%\newcommand{\aes}{\mathrm{AES}}
\newcommand{\walsh}{W}
\newcommand\extendedwalsh{\mathcal{W}}
\newcommand{\scaledwalsh}{\widetilde{W}}
\newcommand{\keysize}{\red{\kappa}}
\newcommand{\algdeg}{\deg_{\text{a}}}
\newcommand{\univdeg}{\deg_{\text{u}}}
\newcommand{\hamming}{\mathrm{wt}}
\newcommand{\blockcipher}{\mathcal{E}}
\newcommand{\bij}{\xrightarrow{\sim}}
\newcommand{\keyspace}{\FF_2^{\keysize}}
\newcommand{\randomdraw}{\xleftarrow{\scriptscriptstyle\$}}
\newcommand{\drawfrom}{\overset{\$}{\leftarrow}}
\newcommand{\transitionmat}[1]{\mathbf{T}^{#1}}
\newcommand{\quasidiff}[1]{\mathbf{D}^{#1}}
\newcommand{\correlationmat}[1]{\mathbf{C}^{#1}}

\newcommand{\bijset}{\mathrm{Bij}(\FFspace)}
\newcommand{\deriv}[1]{D_{#1}}
\newcommand{\proba}[1]{\PP\left[#1\right]}

% phi, epsilon
\renewcommand{\phi}{\varphi}
\renewcommand{\epsilon}{\varepsilon}

\newcommand{\id}{\mathrm{Id}}
\newcommand{\idn}{\mathrm{I_{n}}}

% Linear masks
\newcommand{\mask}[1]{\blue{#1}}
\newcommand{\maa}{\mask{\alpha}}
\newcommand{\aazero}{\mask{\alpha_{0}}}
\newcommand{\aaone}{\mask{\alpha^{(1)}}}
\newcommand{\aabeforelast}{\mask{\alpha^{(R-1)}}}
\newcommand{\aalast}{\mask{\alpha_{n-1}}}
\newcommand{\aain}{\mask{\alpha^{(0)}}}
%\renewcommand{\aat}{\mask{\alpha^{(r)}}} % renew because of blkarray
\newcommand{\aanext}{\mask{\alpha^{(r+1)}}}
\newcommand{\aaout}{\mask{\alpha^{(R)}}}

\newcommand{\bb}{\mask{\beta}}
\newcommand{\bbzero}{\mask{\beta_{0}}}
\newcommand{\bblast}{\mask{\beta_{n-1}}}

\newcommand{\ai}{\mask{\alpha_{i}}}
\newcommand{\bi}{\mask{\beta_{i}}}

\newcommand{\linearity}{\mathcal{L}}
\newcommand{\LAT}{\mathrm{LAT}}


% Trace functions
\NewDocumentCommand{\tr}{o o}{
	\IfNoValueTF {#1}
	{\mathrm{Tr}}
	{\mathrm{Tr}_{{#1}/{#2}}}
}
\newcommand{\trlf}{\tr[\LL][\FF]}
\newcommand{\trlfk}{\tr[\LL][\F_{2^k}]}
\newcommand{\trlftwo}{\tr[\LL][\FF_{2}]}
\newcommand{\trfftwo}{\tr[\FF][\FF_{2}]}
\newcommand{\trntwo}{\tr[\FF_{2^n}][\FF_{2}]}
\newcommand{\trnk}{\tr[\FF_{2^n}][\FF_{2^{k}}]}

\usepackage{upgreek}
\newcommand{\canon}[1]{\upxi^{(#1)}} 


% DIFFERENCES
\newcommand{\bnabla}{\blue{\nabla}}
\newcommand{\plaindiff}{\cyan{\Delta}}
\newcommand{\difference}[1]{\blue{\Delta^{(#1)}}}
\newcommand{\inputdiff}{\difference{0}}
\newcommand{\seconddiff}{\difference{1}}
\newcommand{\ithdiff}{\difference{i}}
\newcommand{\tdiff}{\difference{r}}
\newcommand{\rdiff}{\difference{r}}
\newcommand{\nextdiff}{\difference{r+1}}
\newcommand{\beforelastdiff}{\difference{r-1}} 
\newcommand{\outputdiff}{\difference{r}}      
\newcommand{\rplusonediff}{\difference{r+1}}   
\newcommand{\sthdiff}{\difference{s}}
\newcommand{\din}{\cyan{\Delta^{\mathrm{in}}}}
\newcommand{\dout}{\cyan{\Delta^{\mathrm{out}}}}
\newcommand{\smalldin}[1]{\blue{\Delta^{\mathrm{in}}_{#1}}}
\newcommand{\smalldout}[1]{\blue{\Delta^{\mathrm{out}}_{#1}}}

\newcommand{\ddiff}{\difference{d}}
\newcommand{\dminusonediff}{\difference{d-1}}

% ROUND FUNCTIONS
\newcommand{\roundF}[1]{F^{(#1)}}
\newcommand{\Ft}{\roundF{t}}
\newcommand{\Fr}{\roundF{r}}
\newcommand{\Fnext}{\roundF{t+1}}
\newcommand{\Fprev}{\roundF{t-1}}
\newcommand{\Fin}{\roundF{0}}
\newcommand{\Fout}{\roundF{r-1}}
\newcommand{\Fbeforelast}{\roundF{r-2}}



% Round values  ! warning F and x/y not the same out value !
\newcommand{\roundx}[1]{x^{(#1)}}
\newcommand{\xt}{\roundx{t}}
\newcommand{\xr}{\roundx{r}}
\newcommand{\xnext}{\roundx{t+1}}
\newcommand{\xprev}{\roundx{t-1}}
\newcommand{\xin}{\roundx{0}}
\newcommand{\xout}{\roundx{r}}
\newcommand{\xbeforelast}{\roundx{r-1}}

\newcommand{\roundy}[1]{y^{(#1)}}
\newcommand{\yt}{\roundy{t}}
\newcommand{\yr}{\roundy{r}}
\newcommand{\ynext}{\roundy{t+1}}
\newcommand{\yprev}{\roundy{t-1}}
\newcommand{\yin}{\roundy{0}}
\newcommand{\yout}{\roundy{r}}
\newcommand{\ybeforelast}{\roundy{r-1}}

\newcommand{\roundz}[1]{z^{(#1)}}

\newcommand{\roundk}[1]{k^{(#1)}}
\newcommand{\kt}{\roundk{t}}
\newcommand{\knext}{\roundk{t+1}}
\newcommand{\kprev}{\roundk{t-1}}
\newcommand{\kin}{\roundk{0}}
\newcommand{\kout}{\roundk{r}}
\newcommand{\kbeforelast}{\roundk{r-1}}

% Matrices
\newcommand{\matrixspace}[2]{\mathbf{M}_{#1}(#2)}
\newcommand{\diag}{\mathrm{diag}}
\newcommand{\GL}[2]{\mathbf{GL}_{#1}(#2)}

\newcommand{\functionproduct}[2]{#1 \times #2}

\newcommand{\add}[1]{T_{#1}}

\newcommand{\roundblockcipher}{\mathcal{F}}

\newcommand{\changevar}{\orange{G}}
\newcommand{\changevarlayer}{\orange{\mathcal{G}}}

\newcommand{\li}{N}
\newcommand{\pcomA}{\cyan{A}}
\newcommand{\pcomB}{\cyan{B}}
\newcommand{\alayer}{\cyan{\mathcal{A}}}
\newcommand{\blayer}{\blue{\mathcal{B}}}
\newcommand{\alayerpart}{\blue{\widetilde{\mathcal{A}}}}
\newcommand{\blayerpart}{\blue{\widetilde{\mathcal{B}}}}
\newcommand{\comA}[1]{\blue{A^{(#1)}}}
\newcommand{\comB}[1]{\blue{B^{(#1)}}}
\newcommand{\comAlayer}[1]{\blue{\mathcal{A}^{(#1)}}}
\newcommand{\comBlayer}[1]{\blue{\mathcal{B}^{(#1)}}}

\newcommand{\invf}{\blue{f}}
\newcommand{\invg}{\blue{g}}
\newcommand{\invh}[1]{\blue{h_{#1}}}
\newcommand{\invrho}{\blue{\rho}}
\newcommand{\invtau}{\blue{\tau}}

\NewDocumentCommand{\diff}{O{} O{\inputdiff} O{\difference{R}}}{#2 \xrightarrow{#1} #3}
\NewDocumentCommand{\trail}{O{} O{} O{\inputdiff} O{\seconddiff} O{\difference{R}}}{#3 \xrightarrow{#1} #4 \xrightarrow{} \cdots \xrightarrow{#2} #5}

\NewDocumentCommand{\diffproba}{O{\inputdiff} O{\difference{R}} m}{\PP\left[#1 \xrightarrow{#3} #2\right]}
\NewDocumentCommand{\diffprobatrail}{O{\inputdiff} O{\seconddiff} O{\difference{R}} m m}{\PP\left[#1 \xrightarrow{#4} #2 \xrightarrow{} \cdots \xrightarrow{#5} #3\right]}

\NewDocumentCommand{\edp}{O{\inputdiff} O{\difference{R}} m}{\mathbb{E}\left[#1 \xrightarrow{#3} #2\right]}
\NewDocumentCommand{\edptrail}{O{\inputdiff} O{\seconddiff} O{\difference{R}} m m}{\mathbb{E}\left[#1 \xrightarrow{#4} #2 \xrightarrow{} \cdots \xrightarrow{#5} #3\right]}

\newcommand{\solutionset}{Z}
\newcommand{\diffset}{\solutionset^{\mathrm{diff}}}
\newcommand{\commuteset}{\solutionset^{\mathrm{comm}}}
%\newcommand{\opset}{\solutionset^{\op\text{-diff}}}
\newcommand{\linset}{\solutionset^{\mathrm{lin}}}
\newcommand{\linnbsols}{z^{\mathrm{lin}}}



\newcommand{\keyschedule}{\mathrm{KS}}

\newcommand{\sboxsize}{m}
\newcommand{\sboxspace}{\FF_2^m}
\newcommand{\sboxlayer}{\mathcal{S}}
\newcommand{\llayer}{\mathcal{L}}
\newcommand{\nbs}{s} % Number of sboxes

%============ Ciphers ======
\newcommand{\cipher}[1]{\textsf{#1}} %sf, sc

\newcommand{\AES}{\cipher{AES}}
\newcommand{\AESNI}{\cipher{AES-NI}}
\newcommand{\Ascon}{\cipher{Ascon}}
\newcommand{\Asconpq}{\cipher{Ascon-80pq}}
\newcommand{\DES}{\cipher{DES}}
\newcommand{\mdfour}{\cipher{MD}4}
\newcommand{\keccak}{\cipher{Keccak}}
\newcommand{\keccakf}{\cipher{Keccak}-$f$}
\newcommand{\mimc}{\cipher{MiMC}}
\newcommand{\scream}{\cipher{Scream}}
\newcommand{\midori}{\cipher{Midori}}
\newcommand{\prince}{\cipher{Prince}}
\newcommand{\mantis}{\cipher{Mantis}}
\newcommand{\Square}{\cipher{Square}}
\newcommand{\Misty}{\cipher{Misty1}}
\newcommand{\SHAthree}{\cipher{SHA}3}
\newcommand{\midoribis}[2]{\cipher{Vert}$^{#2}_{#1}$}
\newcommand{\midoriter}[2]{\cipher{Grün}\textsubscript{#1}\textsuperscript{#2}}
\newcommand{\iscream}{\cipher{iSCREAM}}
\newcommand{\norx}{\cipher{NORX v2.0}}
\newcommand{\simpira}{\cipher{Simpira v1}}
\newcommand{\haraka}{\cipher{Haraka v1}}
\newcommand{\Robin}{\cipher{Robin}}
\newcommand{\iSCREAM}{\cipher{iSCREAM}}
\newcommand{\Zorro}{\cipher{Zorro}}
\newcommand{\boomslang}{\cipher{Boomslang}}
\newcommand{\qarma}{\cipher{QARMA}}
\newcommand{\kuznyechik}{\cipher{Kuznyechik}}

\newcommand{\op}{\diamond}
\newcommand{\opset}{Z^{\op\text{-diff}}}
\newcommand{\trans}{\mathcal{T}}
\newcommand{\linspace}[1]{\mathrm{LS}\left(#1\right)}
\newcommand{\wkset}[2]{W(#1, #2)}


\newcommand{\kim}{\kappa}
\newcommand{\FFfield}{\FF_{2^n}}
\newcommand{\subfield}{\FF_{2^{k}}}
\newcommand{\FFspace}{\FF_{2}^{n}}
\newcommand{\supp}{\mathrm{Supp}}
\newcommand{\fix}{\mathrm{Fix}}

\newcommand{\mathtitle}{\texorpdfstring}


%=========== MACROS LeMac========
% operation LeMac textsf
\newcommand{\LIN}	{\ensuremath{\mathsf{LIN}}\xspace}
\newcommand{\AK}{\ensuremath{\mathsf{AK}}\xspace}
\newcommand{\AC}{\ensuremath{\mathsf{AC}}\xspace}
\newcommand{\AT}{\ensuremath{\mathsf{AT}}\xspace}
\newcommand{\ATK}  {\ensuremath{\mathsf{ATK}}\xspace}
\newcommand{\KS}{\ensuremath{\mathsf{KS}}\xspace}
\newcommand{\AddRoundKey} {\ensuremath{\mathsf{AddRoundKey}}\xspace}
\newcommand{\SubCells}  {\ensuremath{\mathsf{SubCells}}\xspace}
\newcommand{\AddConstants}{\ensuremath{\mathsf{AddConstants}}\xspace}
\newcommand{\AddRoundTweak}  {\ensuremath{\mathsf{AddRoundTweak}}\xspace}
\newcommand{\AddRoundTweakey} {\ensuremath{\mathsf{AddRoundTweakey}}\xspace}
\newcommand\SubBytes{\ensuremath{\mathsf{SubBytes}}\xspace}

\newcommand\LFSR{\ensuremath{\mathsf{LFSR}}}
\newcommand{\transitionspace}{\matrixspace{(\ours+\ourr)\times (\ourm+\ourr)}{\FF_2}}
\newcommand{\aes}   {{\textsf{AES}}\xspace}
\newcommand{\siv}   {\textsf{SIV}\xspace}
\newcommand{\ghash}   {\textsf{GHASH}\xspace}
\newcommand{\gcm}   {\textsf{GCM}\xspace}
\newcommand{\lemac}  {\textsf{LeMac}\xspace}
\newcommand{\lepetit} {\textsf{PetitMac}\xspace}
\newcommand{\laconstruction}{\textrm{LaConstruction}}
\newcommand{\panama}   {\textsf{Panama}\xspace}
\newcommand{\radiogatun}   {\textsf{RadioGat\`un}\xspace}
\newcommand{\lux}   {\textsf{Lux}\xspace}
\newcommand{\tiaoxin}   {\textsf{Tiaoxin}\xspace}
\newcommand{\aegis}   {\textsf{AEGIS}\xspace}
\newcommand{\deoxys}   {\textsf{Deoxys}\xspace}
\newcommand{\deoxysbc}   {\textsf{Deoxys-BC}\xspace}
\newcommand{\aerion}   {\textsf{Aerion}\xspace}
\newcommand{\rocca}   {\textsf{Rocca}\xspace}
\newcommand{\roccas}   {\textsf{Rocca-S}\xspace}
\newcommand\ours{\green{s}}
\newcommand\ourS{\green{S}}
\newcommand\ourX{\green{X}}
\newcommand\ourY{\green{Y}}
\newcommand\ourZ{\green{Z}}
\newcommand\ourm{\blue{m}}
\newcommand\ourr{\orange{r}}
\newcommand\ourR{\orange{R}}
\newcommand\oura[1]{\red{a_{#1}}}
\newcommand\ourL{\red{L}}
\newcommand\ourT{\red{T}}
\newcommand\ourV{\green{V}}
\newcommand\ourW{\green{W}}
\newcommand\ourM{\blue{M}}
\newcommand\ourw{\red{w}}
\newcommand{\ourseqV}{\green{\mathcal{V}}}
\newcommand{\ourseqW}{\green{\mathcal{W}}}
\newcommand{\ourseqM}{\blue{\mathcal{M}}}
\makeatletter
\def\@eA#1{\@ifnextchar[{\@eeeA{#1}}{\@ifnextchar_{\@eeeeA{#1}}{\@ifnextchar'{\@eeeeeA{#1}}{\@eeA{#1}}}}}
\def\@eeA#1{\ifmmode\text{$\varepsilon$-A#1}\else$\varepsilon$-A#1\fi\xspace}
\def\@eeeA#1[#2]{\ifmmode\text{$#2$-A#1}\else$#2$-A#1\fi\xspace}
\def\@eeeeA#1_#2{\ifmmode\text{$\varepsilon_{#2}$-A#1}\else$\varepsilon_{#2}$-A#1\fi\xspace}
\def\@eeeeeA#1'{\ifmmode\text{$\varepsilon'$-A#1}\else$\varepsilon'$-A#1\fi\xspace}
\newcommand\eAU{\@eA{U}}
\newcommand\eAXU{\@eA{XU}}
\makeatother
\newcommand\MACauth{\mathrm{AUTH}}
\newcommand\MACverif{\mathrm{VER}}
\newcommand{\roundnb}{\rho}
\newcommand{\model}[3]{{\textsf{Model($#1$, lin=#2, output=#3)}}}


%===========TRANSISTOR=======================
\newcommand\keyspacee{\mathcal{K}}
\newcommand\outspace{\mathcal{S}}
\newcommand\keyevent{k}
\newcommand\outevent{s}
\newcommand\keyvar{K}
\newcommand\outvar{S}
\newcommand\fsmspace{\mathcal{X}}
\newcommand\fsmvar{X}
\newcommand\coolName{\texttt{Transistor}\xspace}
\newcommand\minimalLength{correlation-immune length\xspace}
%components transistor textsf
\newcommand\plaintextSpace{\mathcal{P}}
\newcommand\mainField{\FF_{p}}
\newcommand\evenRing{\mathbb{Z}_t}
\newcommand\LFSRlen{\mathsf{len}}
\newcommand\pseudoKS{\orange{\mathcal{K}}}
\newcommand\whitening{\blue{\mathcal{W}}}
\newcommand\cX{\green{X}}
\newcommand\cK{\orange{K}}
\newcommand\cW{\blue{W}}
\newcommand\cS{\red{S}}
\newcommand\thesbox{\mathsf{S}}
\newcommand\mixmat{M}
\newcommand\filter{\red{\mathsf{\phi}}}
\newcommand\fsmState[1]{\green{X}^{(#1)}}
%operations
\newcommand\subWords{\mathsf{SubDigits}}
\newcommand\mixColumns{\mathsf{MixColumns}}
\newcommand\shiftRows{\mathsf{ShiftRows}}
\newcommand\subW{\mathsf{SD}}
\newcommand\mixC{\mathsf{MC}}
\newcommand\shiftR{\mathsf{SR}}
\newcommand\pr[0]{\mathrm{Pr}}
% latex magic for eprint version
\newif\ifeprint
\eprinttrue
% \eprintfalse



%=============APN FUNCTIONS
\newcommand{\FFinter}{\mathbb{F}_{2^{k}}^{\ell}}
\newcommand\weight[2]{\mathrm{w}_{#1}\left( #2 \right)}
\newcommand{\functionspace}[2]{\mathcal{F}(#1, #2)}
\newcommand{\FFinterbis}{\mathbb{F}_{2^{k'}}^{\ell'}}
\newcommand{\linclass}{\mathcal{C}}
\newcommand{\Aut}{\mathrm{Aut}}
\newcommand{\AutLE}{\mathrm{Aut}_{\mathrm{LE}}}
\newcommand{\gologlu}{G{\"{o}}lo{\u{g}}lu}
\newcommand{\ord}{\mathrm{ord}}
\newcommand{\cycl}[1]{{\sigma}_{#1}}
\newcommand{\K}{\mathcal{K}}
\newcommand{\zeroes}{\mathcal{Z}}
\newcommand{\origins}{\mathcal{O}}
\newcommand{\gammazero}{{\gamma^{\circ}}}
\newcommand{\Gammastar}{\Gamma^{\circ}}
\newcommand{\aspb}{ASPP}
\NewDocumentCommand{\mult}{m m}{
	\mathrm{M}_{{#2}}
}
\newcommand{\Gstar}{G^{\circ}}

%============ Latin ===========
\newcommand{\eg}{\textit{e.g.\@}}
\newcommand{\cf}{\textit{c.f.\@}}
\newcommand{\ie}{\textit{i.e.\@}}
\newcommand{\etal}{\textit{et al.\@}}


% ======== Togle for midori figure =========
\newtoggle{midoriconstant}
\newtoggle{midorikey}
\newtoggle{midorisbox}
\newtoggle{midorilinear}
\newtoggle{midorishuffle}
\newtoggle{midorinumbering}
\newtoggle{midorishufflenumbering}
\newtoggle{midorishufflenumberingout}
\newtoggle{AESshiftrowsin}
\newtoggle{AESshiftrowsout}
\newtoggle{AESconstant}
\newtoggle{AESSR}
\newtoggle{stateempty}

\newtoggle{SPNaccolade}

\newtoggle{SPNtwoS}

\newtoggle{figureSPNnoDetail}
\newtoggle{figureSPNaddroundkey}
%\usepackage{bm}
\newcommand{\cFFspace}{\cyan{\FFspace}}
\newcommand{\cFFfield}{\orange{\FFfield}}
\newcommand{\cFFinter}{\purple{\FFinter}}

\usepackage{inriafontes}

%% Le changement de police peut être réalisé
%% au chargement du paquet
%% \usepackage{InriaSans]{inriafontes}
%% ou en cours d'utilisation

%\usepackage{sfmath}

% \usepackage{fontspec}
% \usepackage{unicode-math}
% \setmathfont[FakeBold=1]{Latin Modern Math}
\mathversion{bold}
%\usepackage{fourier}

\renewcommand{\familydefault}{InriaSans}
%\newfontfamily\myfont{InriaSerif}

% \title{Practical cube-attack against nonce-misused ASCON}
% \author{Jules Baudrin, Anne Canteaut, and Léo Perrin (Inria, Paris, France)}
% \date{May 2024}

\begin{document}

\author{Jules Baudrin}
\institute{UCLouvain}
\date{11/02/2025}

\begin{frame}[plain]
\vfill
\begin{center}
\color{ptblue} \huge Linear self-equivalence : a unifying point-of-view on the known families of APN functions
\end{center}

\begin{center}
\vspace{.6cm}
\color{oiblack} {\Large Jules Baudrin}\\ \vspace{.2cm} based on a joint work with
Anne Canteaut \& Léo Perrin
\vfill
\includegraphics[scale=.3]{figures/logo_uclouvain}
\vfill
{Journées Codage \& Cryptographie,  April 3rd, 2025}
\end{center}

\begin{textblock*}{7cm}(9.5cm,9.5cm) % {block width} (coords)
 Contact: \href{mailto:jules.baudrin@uclouvain.be}{jules.baudrin@uclouvain.be}
\end{textblock*}

\end{frame}


% % !TEX root = ../slides.tex

\begin{frame}
	\frametitle{Block ciphers in practice}
	\vfill


\begin{definition}[hihi]\label{def:xxx}
    asd
\end{definition}

\begin{proposition}[hihi]\label{def:xxx}
    asd
\end{proposition}

\begin{lemma}[hihi]\label{def:xxx}
    asd
\end{lemma}

\begin{theorem}[hihi]\label{def:xxx}
    asd
\end{theorem}

\begin{remark}[ASD]\label{rem:}
asd
\end{remark}


\end{frame}


\begin{frame}[plain, label=title]
\vfill

\blue{This is a test}


\cyan{This is a test}


\yellow{This is a test}


\orange{This is a test}


\red{This is a test}


\purple{This is a test}

		\vspace{-10pt}
		$$ \orange{y} = E_{\red{k}}(\purple{x}) \quad \quad \iff \quad \quad \purple{x} = (E_{\red{k}})^{-1}(\orange{y}) $$
\vspace{5pt}		
\begin{mybox}[Definition]{Indistinguishability}
$[\ E \xleftarrow{\$} \orange{\mathcal{E}} \ ]$ The block is \emph{indistinguishable} from a random bijection if $[ \ F \xleftarrow{\$} \blue{\mathrm{Bij}}(\FF_2^{n}) \ ].$
	\end{mybox}

\begin{mybox}[Theorem]{Major constraints}
$\card{\red{K}} = 2^{128}$, $n = 64, 128$
\begin{itemize}
	\item[-] $\orange{\mathcal{E}}$ should be \emph{easily implemented},
	\item[-] $\orange{\mathcal{E}}$ should be \emph{``easily'' analyzed}.
\end{itemize}
\end{mybox}

\end{frame}




% !TEX root = ../slides_jc2.tex

\section{Introduction}


\begin{frame}
\frametitle{Linear self-equivalence : a unifying PoV on the known families of APN functions}

% !TEX root = ../slides.tex

\begin{columns}[c]
\begin{column}{0.5\textwidth}
\renewcommand\arraystretch{1.3} 
\scalebox{.88}{
\begin{tabular}{|c|}
\toprule
\rctt\textbf{Univariate}\\
\midrule
$x^{2^s + 1} + ax^{2^{(3-i)k + s} + 2^{ik}}$\\
\rct$x^{2^s + 1} + ax^{2^{(4-i)k + s} + 2^{ik}} $\\
$ax^{2^k + 1} + x^{2^s +1} + x^{2^{s + k} + 2^k} + bx^{2^{k + s} + 1} + b^{2^{k}}x^{2^s + 2^k}$\\
\rct$x^{3} + a^{-1}\tr[\FF_{2^n}][\FF_2](a^3x^9)$\\
$x^{3} + a^{-1}\tr[\FF_{2^n}][\FF_{2^3}](a^3x^9 + a^6x^{18})$\\
\rct$x^{3} + a^{-1}\tr[\FF_{2^n}][\FF_{2^3}](a^6x^{18} + a^{12}x^{36})$\\
$ax^{2^s + 1} + a^{2^k}x^{2^{2k} + 2^{k + s}} + bx^{2^{2k} + 1} + ca^{2^k + 1}x^{2^{s} + 2^{k + s}}$\\
\rct$a^{2}x^{2^{2k + 1} + 1} + b^{2}x^{2^{k +1} + 1} + ax^{2^{2k} + 2} + bx^{2^{k} + 2} + dx^{3}$\\
$ x^3 + ax^{2^{s+i} + 2^i} + a^2x^{2^{k+1} + 2^k} + x^{2^{s + i + k} + 2^{i + k}}$\\
\rct$ a\tr[\FFfield][\subfield](bx^{2^i + 1}) + a^{2^k}\tr[\FFfield][\subfield](cx^{2^s + 1})$\\
$ L(x)^{2^k + 1} + bx^{2^k + 1} $\\
\bottomrule
\end{tabular}}
\end{column}
\begin{column}{0.5\textwidth}
% !TEX root = ../slides.tex
\renewcommand\arraystretch{.9} 
\scalebox{.77}{
\begin{tabular}{|c|}
\toprule
\rctt \multicolumn{1}{c|}{\textbf{Multivariate}}\\ 
\midrule
$(x,y) \mapsto \left(\begin{array}{c} x^{2^s + 1} + ay ^{(2^s+1)2^i}\\ xy\end{array}\right)$\\
\rct$(x,y) \mapsto \left(\begin{array}{c}x^{2^{2s} + 2^{3s}} + ax^{2^{2s}}y^{2^s} + by^{2^s+1}\\ xy\end{array}\right)$\\
$(x,y) \mapsto \left(\begin{array}{c}x^{2^s+1} + x^{2^{s + k/2}}y^{2^{k/2}} + axy^{2^s} + by^{2^s+1}\\ xy\end{array}\right)$\\
\rct$(x,y) \mapsto \left(\begin{array}{c}x^{2^s+1} + xy^{2^{s}} + y^{2^s + 1}\\ x^{2^{2s}+1} + x^{2^{2s}}y + y^{2^{2s} + 1}\end{array}\right)$\\
$(x,y) \mapsto \left(\begin{array}{c}x^{2^s+1} + xy^{2^{s}} + y^{2^s + 1}\\ x^{2^{3s}}y + xy^{2^{3s}}\end{array}\right)$\\
\rct$(x,y) \mapsto \left(\begin{array}{c}x^{2^s+1} + by^{2^s + 1}\\ x^{2^{s + k/2}}y + \frac{a}{b}xy^{2^{s + k/2}}\end{array}\right)$\\
$(x,y) \mapsto \left(\begin{array}{c}x^{2^s + 1} + xy^{2^s} + ay^{2^s +1}\\ x^{2^{2s} + 1} + ax^{2^{2s}}y + (1 + a)^{2^s}xy^{2^{2s}} + ay^{2^{2s} + 1}\end{array}\right)$\\
\rct$(x,y,z) \mapsto \left(\begin{array}{c}x^{2^s+1} + x^{2^s}z + yz^{2^s}\\x^{2^s}z + y^{2^s+1}\\xy^{2^s} + y^{2^s}z + z^{2^s+1}\end{array}\right)$\\
$(x,y,z) \mapsto \left(\begin{array}{c}x^{2^s+1} + xy^{2^s} + yz^{2^s}\\xy^{2^s} + z^{2^s+1}\\x^{2^s}z + y^{2^s+1} + y^{2^s}z\end{array}\right)$\\
\bottomrule
\end{tabular}
}
\end{column}
\end{columns}

% \onslide<2->{
% \begin{tikzpicture}[overlay]
%      \node[fill=white] at (11.5,.6) {\inlinebox{ + a lot of sporadic examples for small $n$}};
%  \end{tikzpicture}}

 \onslide<2>{
\begin{tikzpicture}[overlay]
    %  \node[fill=white] at (8,6) {\inlinebox{\red{Where to look for a new function ?}}};
    % \node[fill=white] at (8,3) {\inlinebox{\red{Intersection between families ?}}};
     \node[fill=white] at (8,4.5) {\inlinebox{\orange{\large Hopefully clearer in 20 min ?}}};
 \end{tikzpicture}}

\end{frame}





% \begin{frame}\frametitle{Searching for ideal components}

% \begin{center}
% \begin{tikzpicture}
%     %   \scriptsize     
%         \coordinate (nw) at (0, 5);
%         \coordinate (sw) at (0, 0);
%         \coordinate (ne) at (6, 5);
%         \coordinate (se) at (6, 0);

%             \draw[very thick,oiblue,rounded corners=30pt] ($(nw)+(2, -.6)$) rectangle ($(se)+(-.2, .6)$) ;
%         \draw[very thick,oiorange,rounded corners=10pt] ($(nw)+(2.2, -1.5)$) rectangle ($(se)+(-3.1, 2.5)$) ;

%         \draw[very thick,oired,rounded corners=2pt] ($(nw)+(2.3, -2.1)$) rectangle + (.2, -.2) ;

%         % \draw (3.5,2.7) node[] {{\tiny X} $E_{\red{k}}$};

        
%         \draw[very thick,oiblue] ($ (ne) + (-.25, -1.4) $) --++(.6,0.7) node[right=1]{Cryptography};
%         \draw[very thick,oiorange] ($ (nw)+(2.3, -1.6) $) --++(-.6,0.7) node[left=1]{Symmetric cryptography};
%         \draw[very thick,oired] ($ (nw)+(2.3, -2.2) $) --++(-.6,-0.7) node[left=1]{Cryptographic Boolean functions};
% \end{tikzpicture}
% \end{center}

% \begin{mybox}{Using optimal components}{}{}
% \begin{itemize}
%     \item[-] to reach a high security at \emph{lower costs}
%     \item[-] to achieve ideal properties \emph{assumed in security proofs}
% \end{itemize}
% \end{mybox}

% \begin{columns}[t]
% \begin{column}{0.3\textwidth}
% \begin{mybox}{Cryptanalysis}{}{}
% \begin{itemize}
%     \item[-] A specific attack\dots
%     \item[-] \dots then generalized
% \end{itemize}
% \end{mybox}

% \end{column}
% \begin{column}{0.4\textwidth}
% \onslide<2>{
%     \begin{mybox}{Theoretical study}{}{}
% \begin{itemize}
% \item[-] Definition of ``resistance''
% \item[-] Study of optimal objects
% \end{itemize}
% \end{mybox}
% }
% \end{column}

% \begin{column}{0.3\textwidth}
% \begin{mybox}{Design}{}{}
% \begin{itemize}
%     \item[-] Security arguments
% \end{itemize}
% \end{mybox}
% \end{column}
% \end{columns}

%\end{frame}

\begin{frame}\frametitle{Outline}

\begin{itemize}
    \item[\bulletpoint] From Differential cryptanalysis to APN functions
    \item[\bulletpoint] Polynomial representations of vectorial Boolean functions
    \item[\bulletpoint] APN state of the art
    \item[\bulletpoint] Our unified point of view on the known APN functions

\end{itemize}

\end{frame}

% \section{Symmetric encryption}
% %========================= Encryption ========================
% \subsection{}
% \begin{frame}
% \frametitle{Symmetric encryption}
% \vspace{-.3cm}
% \begin{goal}
% Ensure \emph{confidentiality} under the assumption of a \emph{shared secret} \red{\faKey}.
% \end{goal}

% \begin{center}
% \begin{tikzpicture}
% \node (A) at (0,0)
%     {\includegraphics[height=1.24cm]{figures/shadok_A3} \blue{A}, \red{\faKey}};
% \node (B) at (6,0)
%     {\purple{B},\red{\faKey} \includegraphics[height=1.25cm]{figures/shadok_B3}};

% \draw[<->,thick] (A.east) -- (B.west) node[midway] (mid) {} ;

% \onslide<2>{
% \draw[thick, draw=oigray ,fill=white,drop shadow] (.6,-2.2) rectangle node[midway] {\large \black{$E$}} +(.8,.8) ;
% \draw[->, thick] (1,-.8) -- +(0,-.5);
% \draw[draw=oigray, fill=oigray] (.8, -1.4) -- (1, -1.55) -- (1.2, -1.4);

% \draw[thick, draw=oigray ,fill=white,drop shadow] (5.2,-2.2) rectangle node[midway] {\large \black{$E^{-1}$}} +(.8,.8) ;
% \draw[->, thick] (5.6,-.8) -- +(0,-.5);
% \draw[draw=oigray, fill=oigray] (5.4, -1.4) -- (5.6, -1.55) -- (5.8, -1.4);


% \node at (-.5, -1.8) {\blue{\faFileTextO}};
% \node at (7.1, -1.8) {\blue{\faFileTextO}};

% \draw[->, thick] (-.25, -1.8) -- +(0.8,0);
% \draw[<-, thick] (6.9, -1.8) -- +(-0.8,0);
% \draw[->, thick] (1.5, -1.8) -- node[midway, fill=white] {\red{\faEnvelope}} +(3.6,0);
% }
% \end{tikzpicture}

% \end{center}

% \onslide<2>{
% \begin{constraints}
% \begin{itemize}
%     \item[\bulletpoint] Secure
%     \item[\bulletpoint] Easily implemented
%     \item[\bulletpoint] Arbitrary-long messages
% \end{itemize}
% \end{constraints}
% }

% \end{frame}

% \begin{frame}\frametitle{Building a symmetric encryption scheme}

% \begin{center}
%     \begin{tikzpicture}
%     %[thick, draw=oigray ,fill=white,drop shadow] % prev rounded corners=1ex,fill=red!20,draw
%         \foreach \x in {0, 1, 2} {
%             \node (f\x) at ($\x*(2.5cm,0)$) [minimum size=1cm,thick, draw=oigray, fill=white, drop shadow] {$E_{\red{k}}$};
%             \node (p\x) [above of=f\x, node distance=1.5cm, circle, draw] {};
%             \node (m\x) [above of=p\x, node distance=1cm] {$m_\x$};
%             %\node (k\x) [left of=f\x, node distance=1.5cm] {$k$};
%             \node (c\x) [below of=f\x, node distance=1.5cm] {$c_\x$};
%             \draw[-] (p\x.north) -- (p\x.south);
%             \draw[-] (p\x.east) -- (p\x.west);
%             \draw[-latex] (m\x) -- (p\x);
%             \draw[-latex] (p\x) -- (f\x);
%             %\draw[-latex] (k\x) -- (f\x);
%             \draw[-latex] (f\x) -- (c\x);
%         }
%         \node (iv) [left of=p0, node distance=1.5cm] {$IV$};
%         \draw[-latex] (iv) -- (p0);

%         \foreach \x in {0, 1} {
%         \draw[-latex] ($(c\x) + (0,0.6cm)$) -| +(0.8cm,2.4cm) -- ($(p\x) + (2.4cm,0)$);

%         \begin{scope}
%             \node at (6.4,0) {$\cdots\cdots$};
%         \end{scope}

%         \begin{scope}
%             \node (f) at (9.6cm,0) [minimum size=1cm,thick, draw=oigray, fill=white, drop shadow] {$E_{\red{k}}$};
%             \node (p) [above of=f, node distance=1.5cm, circle, draw] {};
%             \node (m) [above of=p, node distance=1cm] {$m_n$};
%             %\node (k) [left of=f, node distance=1.5cm] {$k$};
%             \node (c) [below of=f, node distance=1.5cm] {$c_n$};
%             \draw[-] (p.north) -- (p.south);
%             \draw[-] (p.east) -- (p.west);
%             \draw[-latex] (m) -- (p);
%             \draw[-latex] (p) -- (f);
%             %\draw[-latex] (k) -- (f);
%             \draw[-latex] (f) -- (c);
%             \draw[-] ($(c) + (-2.5,0.6cm)$) -- ($(c) + (-1.7,0.6cm)$);
%             \draw[-latex] ($(c) + (-1.7,0.6cm)$) |- + (0cm,2.4cm) -- (p);
%         \end{scope}
%         }

%     \end{tikzpicture}
%     \end{center}

%     \begin{mybox}{Ingredients}{}{}
%         \begin{itemize}
%             \item[\bulletpoint] a \red{key-dependent} transformation of $n$-bit words (\textit{e.g.} $n = 128$).\hfill \emph{Block cipher}
%             \item[\bulletpoint] a chaining method to handle arbitrary-long messages \hfill \emph{Mode of operation}
%         \end{itemize}
%     \end{mybox}

% \end{frame}


\section{From differential cryptanalysis to APN functions}
\subsection{}
\begin{frame}
\frametitle{Security of block ciphers}

\vspace{-.8cm}

\vspace{.5cm}
\begin{mybox}{Block cipher}{}{}
A family of bijections $\blockcipher$ of $\FFspace$.
$$\blockcipher = \left(E_{\red{k}} \from \FFspace \bij \FFspace\right)_{\red{k} \in \keyspace}$$
\end{mybox}

\begin{center}
\begin{tikzpicture}

\draw[thick, draw=oigray ,fill=white,drop shadow] (.6,-2.2) rectangle node[midway] {\large $E$} +(.8,.8) ;
\draw[->, thick] (1,-.8) node[above] {$\red{k} = $ \red{\faKey}} -- +(0,-.5);
\draw[draw=oigray, fill=oigray] (.8, -1.4) -- (1, -1.55) -- (1.2, -1.4);

\draw[thick, draw=oigray ,fill=white,drop shadow] (5.2,-2.2) rectangle node[midway] {\large \black{$E^{-1}$}} +(.8,.8) ;
\draw[->, thick] (5.6,-.8) node[above] {\red{$k$}} -- +(0,-.5);
\draw[draw=oigray, fill=oigray] (5.4, -1.4) -- (5.6, -1.55) -- (5.8, -1.4);


\node at (-.5, -1.8) {\gray{\faFileTextO}};
\node at (7.1, -1.8) {\gray{\faFileTextO}};

\draw[->, thick] (-.25, -1.8) -- +(0.8,0);
\draw[<-, thick] (6.9, -1.8) -- +(-0.8,0);
\draw[->, thick] (1.5, -1.8) -- node[midway, fill=white] {\red{\faEnvelope}} +(3.6,0);
\end{tikzpicture}

\end{center}\pause

\begin{mybox}{Ideal block cipher}{}{}
A \emph{random} family of bijections.

\vspace{.4cm}
In practice, $\blockcipher$ should be \emph{indistinguishable} from a random family of bijections

\vspace{-.1cm}
\begin{itemize}
    \item[\bulletpoint] to satisfy assumptions of security proofs
    \item[\bulletpoint] to avoid stronger attack (\eg{} key recoveries)
\end{itemize}
\end{mybox}


\end{frame}

\begin{comment}
    
\subsection{Security of block ciphers}
\begin{frame}\frametitle{\subsecname}
\vspace{-.5cm}
\only<1>{
\begin{definition}[Block cipher]
A family of bijections $\blockcipher = \left(\black{E}_{\red{k}} \from \FFspace \bij \FFspace\right)_{\red{k} \in \keyspace}$
\end{definition}
}
\only<2->{
\begin{definition}[\emph{Secure} block cipher]
A family of bijections $\blockcipher = \left(\black{E}_{\red{k}} \from \FFspace \bij \FFspace\right)_{\red{k} \in \keyspace}$ which \emph{behaves ideally}.
\end{definition}
}

\onslide<3>{
\begin{center}

\scalebox{.8}{
\begin{tikzpicture}
    %   \scriptsize     
        \coordinate (nw) at (0, 5);
        \coordinate (sw) at (0, 0);
        \coordinate (ne) at (6, 5);
        \coordinate (se) at (6, 0);
        


            \draw[very thick,oipurple,rounded corners=30pt] ($(nw)+(2, -.6)$) rectangle ($(se)+(-.2, .6)$) ;
        \draw[very thick,oiorange,rounded corners=15pt] ($(nw)+(3, -1.5)$) rectangle ($(se)+(-1.9, 2.3)$) ;

        \draw (3.5,2.7) node[] {{\tiny X} $E_{\red{k}}$};

        
        \draw[very thick,oipurple] ($ (ne) + (-.25, -1.4) $) --++(.6,0.7) node[right=1]{$\mathrm{Bij}(\FF_2^n)$};
        \node[very thick,oiorange] at ($ (ne) + (-1.6, -2) $) {\orange{$\mathcal{E}$}};
\end{tikzpicture}
}
\end{center}

\begin{definition}[Indistinguishability]
A \emph{random-looking} family: $[\ \orange{E \xleftarrow{\$} \mathcal{E}} \ ]$ \emph{indistinguishable} from  $[ \ \purple{F \xleftarrow{\$} \mathrm{Bij}(\FF_2^{n})} \ ].$

Otherwise:
\vspace{-.1cm}
\begin{itemize}
    \item[\bulletpoint] contradicts the assumptions of modes
    \item[\bulletpoint] leads to key recoveries.
\end{itemize}
\end{definition}
}
\end{frame}
\end{comment}



% \subsection{Iterated block ciphers}
% \begin{frame}
% \frametitle{\subsecname}
% \vspace{-.5cm}
% \begin{mybox}{Block cipher}{}{}
% A family of bijections $\blockcipher = \left(\black{E}_{\red{k}} \from \FFspace \bij \FFspace\right)_{\red{k} \in \keyspace}$.
% \end{mybox}
% %\scalebox{.65}{% !TeX root = ../these.tex
\begin{tikzpicture}[scale=1]
	
	\foreach \xshift in {0, 2.5, 8} 
	{
	\begin{scope}[xshift=\xshift cm]

			
			%wires
			\begin{scope}[decoration={
    markings,
    mark=at position 0.8 with {\arrow{Stealth}}}
    ] 

			\draw[postaction={decorate}, thick] (.5, 1.7) -- +(0, -.7);		

			\draw[postaction={decorate}, thick] (-1.5, .5) -- +(1.5, 0);

			\ifthenelse{\not{\equal{\xshift}{0}}}{	\draw[postaction={decorate}, thick] (1, .5) -- +(1.5, 0);}

			\end{scope}

		
			%sbox
			\draw[thick, draw=oigray ,fill=white,drop shadow] (0,0) rectangle node[midway] {\huge \gray{$F$}} +(1 , 1) ;

	\end{scope}

	\draw[dashed, thick] (5, .5) -- +(1.5, 0);
				\begin{scope}[decoration={
    markings,
    mark=at position 0.5 with {\arrow{Stealth}}}
    ] 
			% \draw[postaction={decorate}] (1, 1.5) -- +(2, 0);
			\end{scope}


}
	\begin{scope}[decoration={
    markings,
    mark=at position 0.5 with {\arrow{Stealth}}}
    ] 
	\draw[postaction={decorate}, thick] (5, 2.8) node[above]{\red{$k$}} -- +(0, -.5);
	\end{scope}

	% \draw[draw=oired, thick,decorate,decoration={brace, amplitude=10pt}] (1.25, -.25) -- node[midway, below=.3cm] {{Round $0$}} +(-1.5, 0);
	% \draw[draw=oired, thick,decorate,decoration={brace, amplitude=10pt}] (3.75, -.25) -- node[midway, below=.3cm] {{Round $1$}} +(-1.5, 0);
	% \draw[draw=oired, thick,decorate,decoration={brace, amplitude=10pt}] (9.25, -.25) -- node[midway, below=.3cm] {{Round $R-1$}} +(-1.5, 0);

	\node[oired] at (1, 1.35) {$\roundk{0}$};
	\node[oired] at (3.5, 1.35) {$\roundk{1}$};
	\node[oired] at (9.25, 1.35) {$\roundk{R-1}$};

	\node at (-1.8, 0.5) {\Large $\gray{m}$};
	\node at (10.7, 0.5) {\Large $\gray{c}$};

	\draw[thick, draw=oired ,fill=white,drop shadow] (0,1.7) -- node[midway, above=.01] {\large \red{Key schedule}} (9,1.7) -- (8, 2.3) -- (1, 2.3) -- (0,1.7); 
	% rectangle node[midway] {\huge \green{$F$}} +(1 , 3) ;
	\draw[draw=oigray, thick,decorate,decoration={brace, amplitude=10pt}] (9, -.3) -- node[midway, below=.3cm] {\huge $\gray{E_{\red{k}}} = \gray{F}_{\red{k^{(R-1)}}}  \comp \cdots \comp \gray{F}_{\red{k^{(1)}}} \comp \gray{F}_{\red{k^{(0)}}} $} +(-9, 0);
\end{tikzpicture}
} 
% \vspace{.3cm}
% \only<1>{
% \begin{center}
% \toggletrue{figureSPNnoDetail}
% \scalebox{.6}{% !TeX root = ../slides_uvsq.tex
\begin{tikzpicture}[scale=1]
	
	\foreach \xshift in {0, 3.7, 10.4} 
	{
	\begin{scope}[xshift=\xshift cm]
		\foreach \y in {0, 1.5, 4.5}
		{
			\iftoggle{figureSPNnoDetail}{}{
			%sbox
			\draw[thick, draw=oicyan ,fill=white,drop shadow] (0.5,\y) rectangle node[midway] {\huge \cyan{$S$}} +(1,1) ;
			}
			
			%wires
			\foreach \smally in {1, 2, 3, 4} {
			\draw (-.1, \y + .2*\smally) -- +(.6, 0);
			\draw (1.5, \y + .2*\smally) -- +(.5, 0);
			\draw (3, \y + .2*\smally) -- +(.6, 0);
			
			%\ifthenelse{\not{\equal{\xshift}{0}}}{	\draw (3.7, \y + .2*\smally) -- +(.5, 0);}{}
			
			\iftoggle{figureSPNaddroundkey}{
			%xors
			\draw[thick, draw=oired] (3.6,\y + .2*\smally) circle (0.1cm);
			\draw[thick, draw=oired] (3.6, \y + .2*\smally) -- +(0, 0.1)
			(3.6, \y + .2*\smally) -- +(0, -0.1)
			(3.6, \y + .2*\smally) -- +(-.1, 0)
			(3.6, \y + .2*\smally) -- +(.1, 0);

			}{
						\draw[thick, draw=oired, opacity=0] (3.6,\y + .2*\smally) circle (0.1cm);
			\draw[thick, draw=oired, opacity=0] (3.6, \y + .2*\smally) -- +(0, 0.1)
			(3.6, \y + .2*\smally) -- +(0, -0.1)
			(3.6, \y + .2*\smally) -- +(-.1, 0)
			(3.6, \y + .2*\smally) -- +(.1, 0);
			}


			}
		}

		\iftoggle{figureSPNnoDetail}{
		\draw[thick, draw=oigray ,fill=white,drop shadow] (0.5,0) rectangle node[midway] {\huge \gray{$F$}} +(2.5, 5.5);
		}
		{
		\draw[draw=white] (0.5,3) rectangle node[midway] {\huge \cyan{$\vdots$}} +(1,1) ;
		
		\draw[thick, draw=oiorange ,fill=white,drop shadow] (2,0) rectangle node[midway] {\huge \orange{$L$}} +(1,5.5) ;
		}
		
	\end{scope}

	\iftoggle{figureSPNaddroundkey}{
	\draw[thick, draw=oired] (3.6, 4.5 + .2*4) -- +(0, 0.7) node[above] {\red{$rk_{0}$}};
	\draw[thick, draw=oired] (7.3, 4.5 + .2*4) -- +(0, 0.7) node[above] {\red{$rk_{1}$}};
	\draw[thick, draw=oired] (14, 4.5 + .2*4) -- +(0, 0.7) node[above] {\red{$rk_{R}$}};
	}{
	\draw[thick, draw=oired,opacity=0] (3.6, 4.5 + .2*4) -- +(0, 0.7) node[above,opacity=0] {\red{$rk_{0}$}};
	\draw[thick, draw=oired,opacity=0] (7.3, 4.5 + .2*4) -- +(0, 0.7) node[above,opacity=0] {\red{$rk_{1}$}};
	\draw[thick, draw=oired,opacity=0] (14, 4.5 + .2*4) -- +(0, 0.7) node[above,opacity=0] {\red{$rk_{R}$}};
	}
	
	\foreach \y in {0, 1.5, 4.5}
	{
		\foreach \smally in {1, 2, 3, 4} {
		\draw[dashed,oigray] (7.3, \y + .2*\smally) -- +(3, 0);
		}
	}
		\draw[draw=white] (8.2, 3) -- node[midway] {$\gray{\cdots}$} +(1, 1);
	}

	\iftoggle{figureSPNnoDetail}{}{
	\draw[draw=oicyan, thick,decorate,decoration={brace, amplitude=10pt}] (0.5, 5.6) -- node[midway, above=.3cm] {\cyan{Sbox layer}} +(1, 0);
	\draw[draw=oiorange, thick,decorate,decoration={brace, amplitude=10pt}] (3, -.2) -- node[midway, below=.3cm] {\orange{Linear layer}} +(-1, 0);
	}

\end{tikzpicture}
}
% \end{center}
% }

% \only<2>{
% \begin{center}
% \toggletrue{figureSPNnoDetail}
% \toggletrue{figureSPNaddroundkey}
% \scalebox{.6}{% !TeX root = ../slides_uvsq.tex
\begin{tikzpicture}[scale=1]
	
	\foreach \xshift in {0, 3.7, 10.4} 
	{
	\begin{scope}[xshift=\xshift cm]
		\foreach \y in {0, 1.5, 4.5}
		{
			\iftoggle{figureSPNnoDetail}{}{
			%sbox
			\draw[thick, draw=oicyan ,fill=white,drop shadow] (0.5,\y) rectangle node[midway] {\huge \cyan{$S$}} +(1,1) ;
			}
			
			%wires
			\foreach \smally in {1, 2, 3, 4} {
			\draw (-.1, \y + .2*\smally) -- +(.6, 0);
			\draw (1.5, \y + .2*\smally) -- +(.5, 0);
			\draw (3, \y + .2*\smally) -- +(.6, 0);
			
			%\ifthenelse{\not{\equal{\xshift}{0}}}{	\draw (3.7, \y + .2*\smally) -- +(.5, 0);}{}
			
			\iftoggle{figureSPNaddroundkey}{
			%xors
			\draw[thick, draw=oired] (3.6,\y + .2*\smally) circle (0.1cm);
			\draw[thick, draw=oired] (3.6, \y + .2*\smally) -- +(0, 0.1)
			(3.6, \y + .2*\smally) -- +(0, -0.1)
			(3.6, \y + .2*\smally) -- +(-.1, 0)
			(3.6, \y + .2*\smally) -- +(.1, 0);

			}{
						\draw[thick, draw=oired, opacity=0] (3.6,\y + .2*\smally) circle (0.1cm);
			\draw[thick, draw=oired, opacity=0] (3.6, \y + .2*\smally) -- +(0, 0.1)
			(3.6, \y + .2*\smally) -- +(0, -0.1)
			(3.6, \y + .2*\smally) -- +(-.1, 0)
			(3.6, \y + .2*\smally) -- +(.1, 0);
			}


			}
		}

		\iftoggle{figureSPNnoDetail}{
		\draw[thick, draw=oigray ,fill=white,drop shadow] (0.5,0) rectangle node[midway] {\huge \gray{$F$}} +(2.5, 5.5);
		}
		{
		\draw[draw=white] (0.5,3) rectangle node[midway] {\huge \cyan{$\vdots$}} +(1,1) ;
		
		\draw[thick, draw=oiorange ,fill=white,drop shadow] (2,0) rectangle node[midway] {\huge \orange{$L$}} +(1,5.5) ;
		}
		
	\end{scope}

	\iftoggle{figureSPNaddroundkey}{
	\draw[thick, draw=oired] (3.6, 4.5 + .2*4) -- +(0, 0.7) node[above] {\red{$rk_{0}$}};
	\draw[thick, draw=oired] (7.3, 4.5 + .2*4) -- +(0, 0.7) node[above] {\red{$rk_{1}$}};
	\draw[thick, draw=oired] (14, 4.5 + .2*4) -- +(0, 0.7) node[above] {\red{$rk_{R}$}};
	}{
	\draw[thick, draw=oired,opacity=0] (3.6, 4.5 + .2*4) -- +(0, 0.7) node[above,opacity=0] {\red{$rk_{0}$}};
	\draw[thick, draw=oired,opacity=0] (7.3, 4.5 + .2*4) -- +(0, 0.7) node[above,opacity=0] {\red{$rk_{1}$}};
	\draw[thick, draw=oired,opacity=0] (14, 4.5 + .2*4) -- +(0, 0.7) node[above,opacity=0] {\red{$rk_{R}$}};
	}
	
	\foreach \y in {0, 1.5, 4.5}
	{
		\foreach \smally in {1, 2, 3, 4} {
		\draw[dashed,oigray] (7.3, \y + .2*\smally) -- +(3, 0);
		}
	}
		\draw[draw=white] (8.2, 3) -- node[midway] {$\gray{\cdots}$} +(1, 1);
	}

	\iftoggle{figureSPNnoDetail}{}{
	\draw[draw=oicyan, thick,decorate,decoration={brace, amplitude=10pt}] (0.5, 5.6) -- node[midway, above=.3cm] {\cyan{Sbox layer}} +(1, 0);
	\draw[draw=oiorange, thick,decorate,decoration={brace, amplitude=10pt}] (3, -.2) -- node[midway, below=.3cm] {\orange{Linear layer}} +(-1, 0);
	}

\end{tikzpicture}
}
% \end{center}
% }


% \only<3->{
% \begin{center}
% \toggletrue{figureSPNaddroundkey}
% \scalebox{.6}{% !TeX root = ../slides_uvsq.tex
\begin{tikzpicture}[scale=1]
	
	\foreach \xshift in {0, 3.7, 10.4} 
	{
	\begin{scope}[xshift=\xshift cm]
		\foreach \y in {0, 1.5, 4.5}
		{
			\iftoggle{figureSPNnoDetail}{}{
			%sbox
			\draw[thick, draw=oicyan ,fill=white,drop shadow] (0.5,\y) rectangle node[midway] {\huge \cyan{$S$}} +(1,1) ;
			}
			
			%wires
			\foreach \smally in {1, 2, 3, 4} {
			\draw (-.1, \y + .2*\smally) -- +(.6, 0);
			\draw (1.5, \y + .2*\smally) -- +(.5, 0);
			\draw (3, \y + .2*\smally) -- +(.6, 0);
			
			%\ifthenelse{\not{\equal{\xshift}{0}}}{	\draw (3.7, \y + .2*\smally) -- +(.5, 0);}{}
			
			\iftoggle{figureSPNaddroundkey}{
			%xors
			\draw[thick, draw=oired] (3.6,\y + .2*\smally) circle (0.1cm);
			\draw[thick, draw=oired] (3.6, \y + .2*\smally) -- +(0, 0.1)
			(3.6, \y + .2*\smally) -- +(0, -0.1)
			(3.6, \y + .2*\smally) -- +(-.1, 0)
			(3.6, \y + .2*\smally) -- +(.1, 0);

			}{
						\draw[thick, draw=oired, opacity=0] (3.6,\y + .2*\smally) circle (0.1cm);
			\draw[thick, draw=oired, opacity=0] (3.6, \y + .2*\smally) -- +(0, 0.1)
			(3.6, \y + .2*\smally) -- +(0, -0.1)
			(3.6, \y + .2*\smally) -- +(-.1, 0)
			(3.6, \y + .2*\smally) -- +(.1, 0);
			}


			}
		}

		\iftoggle{figureSPNnoDetail}{
		\draw[thick, draw=oigray ,fill=white,drop shadow] (0.5,0) rectangle node[midway] {\huge \gray{$F$}} +(2.5, 5.5);
		}
		{
		\draw[draw=white] (0.5,3) rectangle node[midway] {\huge \cyan{$\vdots$}} +(1,1) ;
		
		\draw[thick, draw=oiorange ,fill=white,drop shadow] (2,0) rectangle node[midway] {\huge \orange{$L$}} +(1,5.5) ;
		}
		
	\end{scope}

	\iftoggle{figureSPNaddroundkey}{
	\draw[thick, draw=oired] (3.6, 4.5 + .2*4) -- +(0, 0.7) node[above] {\red{$rk_{0}$}};
	\draw[thick, draw=oired] (7.3, 4.5 + .2*4) -- +(0, 0.7) node[above] {\red{$rk_{1}$}};
	\draw[thick, draw=oired] (14, 4.5 + .2*4) -- +(0, 0.7) node[above] {\red{$rk_{R}$}};
	}{
	\draw[thick, draw=oired,opacity=0] (3.6, 4.5 + .2*4) -- +(0, 0.7) node[above,opacity=0] {\red{$rk_{0}$}};
	\draw[thick, draw=oired,opacity=0] (7.3, 4.5 + .2*4) -- +(0, 0.7) node[above,opacity=0] {\red{$rk_{1}$}};
	\draw[thick, draw=oired,opacity=0] (14, 4.5 + .2*4) -- +(0, 0.7) node[above,opacity=0] {\red{$rk_{R}$}};
	}
	
	\foreach \y in {0, 1.5, 4.5}
	{
		\foreach \smally in {1, 2, 3, 4} {
		\draw[dashed,oigray] (7.3, \y + .2*\smally) -- +(3, 0);
		}
	}
		\draw[draw=white] (8.2, 3) -- node[midway] {$\gray{\cdots}$} +(1, 1);
	}

	\iftoggle{figureSPNnoDetail}{}{
	\draw[draw=oicyan, thick,decorate,decoration={brace, amplitude=10pt}] (0.5, 5.6) -- node[midway, above=.3cm] {\cyan{Sbox layer}} +(1, 0);
	\draw[draw=oiorange, thick,decorate,decoration={brace, amplitude=10pt}] (3, -.2) -- node[midway, below=.3cm] {\orange{Linear layer}} +(-1, 0);
	}

\end{tikzpicture}
}
% \end{center}
% }

% \end{frame}


%\section{Differential cryptanalysis}
% {
\setbeamercolor{background canvas}{bg=ptblue}	
\begin{frame}[plain]
\vfill
\begin{center}
%\color{white} \Huge \Roman{section} -  \secname
\color{white} \Huge \secname
\end{center}
\vfill
\end{frame}
}

\begin{frame}[fragile]
\frametitle{Differential cryptanalysis}
\vspace{-.4cm}
% \begin{recap}
% $\blockcipher = \left(\black{E}_{\red{k}} \from \FFspace \bij \FFspace\right)_{\red{k} \in \keyspace}$.\hfill $[\ E \xleftarrow{\$} \mathcal{E} \ ]$  or $[ \ F \xleftarrow{\$} \mathrm{Bij}(\FF_2^{n}) \ ]$ ?
% \end{recap}
$\purple{F} \from \FFspace \to \FFspace$. 
\begin{mybox}{Principle}{}{}
Studies for each input difference $\din \neq 0$, the \emph{distribution of output differences}:
    $$\forall \ \dout \in \FFspace, \quad \PP_{x \xleftarrow{\$} \FFspace}\left[\purple{F}(x + \din) + \purple{F}(x) = \dout\right] = \ ? $$
\end{mybox}

\begin{center}

\begin{tikzcd}
\only<1-3>{
                        x  \arrow[leftrightarrow]{d}{\din}[]{} \arrow[opacity=0]{r}[name=uparrow,opacity=0]{\gray{F^{(0)}}} & \white{x^{(1)}} \arrow[leftrightarrow,opacity=0]{d}{\seconddiff} \arrow[dashed,opacity=0]{r}{} &  \white{x^{(R-1)}} \arrow[opacity=0]{r}[opacity=0]{\gray{F^{(r-1)}}} \arrow[leftrightarrow,opacity=0]{d}{}{\beforelastdiff} &  \purple{F}(x) \arrow[leftrightarrow]{d}{\dout}[]{}
                        \\
                        y \arrow[opacity=0]{r}{}[swap,name=bottomarrow,opacity=0]{\gray{F^{(0)}}} & \white{y^{(1)}}  \arrow[dashed,opacity=0]{r}{}[swap]{} &  \white{y^{(R-1)}} \arrow[swap,opacity=0]{r}{}[opacity=0]{\gray{F^{(r-1)}}} &   \purple{F}(y)
                        %\arrow[to path={(uparrow) node[midway, scale=1.2, left=0.05cm] {$\circlearrowleft$} (bottomarrow)}]{}
                        }
                        % \only<4>{      x  \arrow[leftrightarrow]{d}{\din}[]{} \arrow[opacity=1]{r}[name=uparrow,opacity=1]{\gray{F^{(0)}}} & x^{(1)} \arrow[leftrightarrow,opacity=1]{d}{\seconddiff} \arrow[dashed,opacity=1]{r}{} &  x^{(R-1)} \arrow[opacity=1]{r}[opacity=1]{\gray{F^{(r-1)}}} \arrow[leftrightarrow,opacity=1]{d}{}{\beforelastdiff} &  \purple{F}(x) \arrow[leftrightarrow]{d}{\dout}[]{}
                        % \\
                        % y \arrow[opacity=1]{r}{}[swap,name=bottomarrow,opacity=1]{\gray{F^{(0)}}} & y^{(1)}  \arrow[dashed,opacity=1]{r}{}[swap]{} &  y^{(R-1)} \arrow[swap,opacity=1]{r}{}[opacity=1]{\gray{F^{(r-1)}}} &   \purple{F}(y)
                        % }
            
\end{tikzcd}
\end{center}

\pause
\begin{mybox}{Average over all bijections}{}{}
For all $(\din \neq 0, \dout)$, the equation $\purple{F}(x + \din) + \purple{F}(x) = \dout$ has 1 solution $x$ \emph{on average}.
\end{mybox}


\pause
\begin{mybox}{Differential distinguisher\hfill {\small\purple{[BihSha91]}}}{}{}
$(\din, \dout)$ such that for many $\red{k}$, $\quad E_{\red{k}}(x + \din) + E_{\red{k}}(x) = \dout$ has many solutions $x$.
\end{mybox}

\end{frame}


\subsection{Resisting against differential attacks}
\begin{frame}\frametitle{\subsecname}
\vspace{-.5cm}

\begin{mybox}{Differential distinguisher\hfill {\small\purple{[BihSha91]}}}{}{}
$(\din, \dout)$ s.t for many $\red{k}$, $\quad E_{\red{k}}(x + \din) + E_{\red{k}}(x) = \dout$ has many solutions $x$.
\end{mybox}

\pause
\begin{mybox}{Differential resistance}{}{}
For all $(\din, \dout)$ and all keys $\red{k}$, $\quad E_{\red{k}}(x + \din) + E_{\red{k}}(x) = \dout$ has \emph{few} solutions.
\end{mybox}

% \begin{mybox}{For a random bijection \purple{$F$}}{}{}
% $\purple{F}(x + \din) + \purple{F}(x) = \dout$ has 1 solution $x$ on average.
% \end{mybox}
%If $x$ is a solution of Eq.~\eqref{eq:diff}, then so is $x + \din$.  \hfill$\implies \ $ even number of solutions $\delta_{F}(\din, \dout)$

\pause
\vspace{-.3cm}
\begin{mybox}{How to achieve this}{}{}
For all $(\din, \dout)$,  $\quad \purple{S}(x + \din) + \purple{S}(x) = \dout$ has \emph{few} solutions.
\end{mybox}
\pause

$$ \delta_{\purple{S}}(\din, \dout) = \card{\set{x \mid \purple{S}(x + \din) + \purple{S}(x) = \dout}}$$


\begin{columns}[c]
\begin{column}{0.5\textwidth}

    \scalebox{.5}{% !TeX root = ../slides_uvsq.tex
\begin{tikzpicture}[scale=1]
	
	\foreach \xshift in {0, 3.7} 
	{
	\begin{scope}[xshift=\xshift cm]
		\foreach \y in {0, 1.5, 4.5}
		{
			%sbox
			\draw[thick, draw=black ,fill=white,drop shadow] (0.5,\y) rectangle node[midway] {\large \black{$S$}} +(1,1) ;
			
			%wires
			\foreach \smally in {1, 2, 3, 4} {
			\draw (-.1, \y + .2*\smally) -- +(.6, 0);
			\draw (1.5, \y + .2*\smally) -- +(.5, 0);
			\draw (3, \y + .2*\smally) -- +(.6, 0);
			
			%\ifthenelse{\not{\equal{\xshift}{0}}}{	\draw (3.7, \y + .2*\smally) -- +(.5, 0);}{}
			
			%xors
			\draw[thick, draw=ptred] (3.6,\y + .2*\smally) circle (0.1cm);
			\draw[thick, draw=ptred] (3.6, \y + .2*\smally) -- +(0, 0.1)
			(3.6, \y + .2*\smally) -- +(0, -0.1)
			(3.6, \y + .2*\smally) -- +(-.1, 0)
			(3.6, \y + .2*\smally) -- +(.1, 0);


			}
		}
		\draw[draw=white] (0.5,3) rectangle node[midway] {\large \black{$\vdots$}} +(1,1) ;
		
		\draw[thick, draw=black ,fill=white,drop shadow] (2,0) rectangle node[midway] {\large \black{$L$}} +(1,5.5) ;
		
	\end{scope}

	\draw[thick, draw=ptred] (0, 4.5 + .2*4) -- +(0, 0.7) node[above] {\red{$rk_{0}$}};
	\draw[thick, draw=ptred] (3.6, 4.5 + .2*4) -- +(0, 0.7) node[above] {\red{$rk_{1}$}};
	\draw[thick, draw=ptred] (7.3, 4.5 + .2*4) -- +(0, 0.7) node[above] {\red{$rk_{2}$}};
			\foreach \y in {0, 1.5, 4.5}{
				\foreach \smally in {1, 2, 3, 4} {
			\draw[thick, draw=ptred] (0,\y + .2*\smally) circle (0.1cm);
			\draw[thick, draw=ptred] (0, \y + .2*\smally) -- +(0, 0.1)
			(0, \y + .2*\smally) -- +(0, -0.1)
			(0, \y + .2*\smally) -- +(-.1, 0)
			(0, \y + .2*\smally) -- +(.1, 0);
			}
			}
	
	}

	%\draw[draw=ptred, thick,decorate,decoration={brace, amplitude=5pt}] (3.3, 5.6) -- node[midway, above=.3cm] {\red{Key addition}} +(.6, 0);

\end{tikzpicture}
}


\end{column}

\begin{column}{0.5\textwidth}


$\PP[\din, \cyan{\Delta}, \dout] \leq \left(\frac{\max\limits_{a \neq 0, b} \delta_{\purple{S}}(a, b)}{2^{m}}\right)^{d(\green{L})}$


\end{column}
\end{columns}



% \vspace{.3cm}
% \begin{itemize}
%     \item[\bulletpoint] For all $\din$, there exists $\dout$ such that $\delta_{\purple{F}}(\din, \dout) >0$\pause
%     \item[\bulletpoint] For all $\din \neq 0 , \dout$, $x$ is a solution iff $x + \din$ is a solution. \hfill $\delta_{\purple{F}}(\din, \dout)$ is even.
% \end{itemize}
% \pause

% \begin{mybox}{Definition}{APN function}{\purple{\small[NybKnu92]}}\label{def:apn}
%     A function $F$ is APN if: $\quad \forall \ \din\neq 0, \dout, \quad \delta_{F}(\din, \dout) \leq 2$.
% \end{mybox}
\end{frame}

\begin{frame}\frametitle{Differentially-optimal functions}
\begin{mybox}{How to achieve this}{}{}
For all $\din\neq 0, \dout $  $\quad \delta_{\purple{S}}(\din, \dout) = \card{\set{x \mid \purple{S}(x + \din) + \purple{S}(x) = \dout}}$ should be \emph{low}.
\end{mybox}\pause

\vspace{.5cm}
\begin{itemize}
    \item[\bulletpoint] For all $\din$, there exists $\dout$ such that $\delta_{\purple{S}}(\din, \dout) >0$\pause
    \item[\bulletpoint] For all $\din \neq 0 , \dout$, $x$ is a solution iff $x + \din$ is a solution. \hfill $\delta_{\purple{S}}(\din, \dout)$ is even.
\end{itemize}
\pause

\vspace{1cm}
\begin{mybox}{Almost perfect non-linear (APN) function}{}{\purple{\small[NybKnu92]}}
    A function $\purple{F}$ is APN if: $\quad \forall \ \din\neq 0, \dout, \quad \delta_{\purple{F}}(\din, \dout) \leq 2$.
\end{mybox}

\end{frame}

\begin{frame}[t]\frametitle{Almost perfect non-linear (APN) function}

\begin{mybox}{Definition}{APN function}{\purple{\small[NybKnu92]}}
    A function $\purple{F}$ is APN if: $\quad \forall \ \din\neq 0, \dout, \quad \delta_{\purple{F}}(\din, \dout) \leq 2$.
\end{mybox}

\pause
\begin{mybox}{A typical classification problem}{}{}
\begin{itemize}
    \item[-] \green{Easy} definition
    \item[-] \red{Hard} to find new instances (even for small $n$)
    \item[-] \red{Hard} to classify the known instances
    \item[-] Lots of open problems
\end{itemize}
\end{mybox}

\pause
\begin{mybox}{Big APN problem}{}{\purple{\small[BDMW10]}}
Find $F\from \FFspace \to \FFspace$ which is APN, \emph{bijective} for an \emph{even} $n$.

\hfill A \emph{single} example is known for $n=6$.\hfill 
\end{mybox}

\end{frame}

\begin{comment}

\subsection{Finding an APN function}
\begin{frame}
\frametitle{\subsecname}
\vspace{-.5cm}

% \begin{center}
% $4$ bits $\approx$ a pair of $2$-bit words $\approx$ a $4$-bit word    
% \end{center}


\begin{mybox}{Use alternative representations}{}{}
$F \from \FF_{2}^{4} \to \FF_{2}^{4}, \begin{pmatrix}
x_{0}\\ x_{1}\\ x_{2}\\ x_{3}
\end{pmatrix} \mapsto \begin{pmatrix}
    x_0x_2 + x_0 + x_1x_2 + x_1x_3\\ 
x_0x_1 + x_0x_2 + x_2x_3 + x_3\\ 
x_0x_1 + x_0x_2 + x_0x_3 + x_1x_2 + x_1x_3 + x_2x_3 + x_2\\ 
x_1x_3 + x_1 + x_2x_3 + x_2 + x_3
\end{pmatrix}$
\pause

\vspace{.5cm}
$F \from \FF_{4}^{2} \to \FF_{4}^{2}, (x_{0}, x_{1}) \mapsto \begin{pmatrix}\orange{\alpha_{0}}x_{0}^3 + x_{0}^2x_{1} + \orange{\alpha_{1}}x_{0}x_{1}^2 + \orange{\alpha_{2}}x_{1}^3\\
\orange{\alpha_{3}}x_{0}^3 + \orange{\alpha_{4}}x_{0}^2x_{1} + \orange{\alpha_{5}}x_{0}x_{1}^2\end{pmatrix}$\hfill $\alpha_{i} \in \FF_{4}$.
\pause

\vspace{.5cm}
$F \from \FF_{16} \to \FF_{16}, X \mapsto X^{3}$\hfill 3 representations of the \emph{``same''} function
\end{mybox}

\onslide<4>{
$$  (X + \plaindiff)^{3} + X^{3} = \plaindiff X^{2} + \plaindiff^{2} X + \plaindiff^{3}$$

Quadratic equation $\implies$ at most 2 solutions $\implies$ APN ! }


\end{frame}

\subsection{A (not so welcoming) state of the art}
\begin{frame}
\frametitle{\subsecname}

% !TEX root = ../slides.tex

\begin{columns}[c]
\begin{column}{0.5\textwidth}
\renewcommand\arraystretch{1.3} 
\scalebox{.88}{
\begin{tabular}{|c|}
\toprule
\rctt\textbf{Univariate}\\
\midrule
$x^{2^s + 1} + ax^{2^{(3-i)k + s} + 2^{ik}}$\\
\rct$x^{2^s + 1} + ax^{2^{(4-i)k + s} + 2^{ik}} $\\
$ax^{2^k + 1} + x^{2^s +1} + x^{2^{s + k} + 2^k} + bx^{2^{k + s} + 1} + b^{2^{k}}x^{2^s + 2^k}$\\
\rct$x^{3} + a^{-1}\tr[\FF_{2^n}][\FF_2](a^3x^9)$\\
$x^{3} + a^{-1}\tr[\FF_{2^n}][\FF_{2^3}](a^3x^9 + a^6x^{18})$\\
\rct$x^{3} + a^{-1}\tr[\FF_{2^n}][\FF_{2^3}](a^6x^{18} + a^{12}x^{36})$\\
$ax^{2^s + 1} + a^{2^k}x^{2^{2k} + 2^{k + s}} + bx^{2^{2k} + 1} + ca^{2^k + 1}x^{2^{s} + 2^{k + s}}$\\
\rct$a^{2}x^{2^{2k + 1} + 1} + b^{2}x^{2^{k +1} + 1} + ax^{2^{2k} + 2} + bx^{2^{k} + 2} + dx^{3}$\\
$ x^3 + ax^{2^{s+i} + 2^i} + a^2x^{2^{k+1} + 2^k} + x^{2^{s + i + k} + 2^{i + k}}$\\
\rct$ a\tr[\FFfield][\subfield](bx^{2^i + 1}) + a^{2^k}\tr[\FFfield][\subfield](cx^{2^s + 1})$\\
$ L(x)^{2^k + 1} + bx^{2^k + 1} $\\
\bottomrule
\end{tabular}}
\end{column}
\begin{column}{0.5\textwidth}
% !TEX root = ../slides.tex
\renewcommand\arraystretch{.9} 
\scalebox{.77}{
\begin{tabular}{|c|}
\toprule
\rctt \multicolumn{1}{c|}{\textbf{Multivariate}}\\ 
\midrule
$(x,y) \mapsto \left(\begin{array}{c} x^{2^s + 1} + ay ^{(2^s+1)2^i}\\ xy\end{array}\right)$\\
\rct$(x,y) \mapsto \left(\begin{array}{c}x^{2^{2s} + 2^{3s}} + ax^{2^{2s}}y^{2^s} + by^{2^s+1}\\ xy\end{array}\right)$\\
$(x,y) \mapsto \left(\begin{array}{c}x^{2^s+1} + x^{2^{s + k/2}}y^{2^{k/2}} + axy^{2^s} + by^{2^s+1}\\ xy\end{array}\right)$\\
\rct$(x,y) \mapsto \left(\begin{array}{c}x^{2^s+1} + xy^{2^{s}} + y^{2^s + 1}\\ x^{2^{2s}+1} + x^{2^{2s}}y + y^{2^{2s} + 1}\end{array}\right)$\\
$(x,y) \mapsto \left(\begin{array}{c}x^{2^s+1} + xy^{2^{s}} + y^{2^s + 1}\\ x^{2^{3s}}y + xy^{2^{3s}}\end{array}\right)$\\
\rct$(x,y) \mapsto \left(\begin{array}{c}x^{2^s+1} + by^{2^s + 1}\\ x^{2^{s + k/2}}y + \frac{a}{b}xy^{2^{s + k/2}}\end{array}\right)$\\
$(x,y) \mapsto \left(\begin{array}{c}x^{2^s + 1} + xy^{2^s} + ay^{2^s +1}\\ x^{2^{2s} + 1} + ax^{2^{2s}}y + (1 + a)^{2^s}xy^{2^{2s}} + ay^{2^{2s} + 1}\end{array}\right)$\\
\rct$(x,y,z) \mapsto \left(\begin{array}{c}x^{2^s+1} + x^{2^s}z + yz^{2^s}\\x^{2^s}z + y^{2^s+1}\\xy^{2^s} + y^{2^s}z + z^{2^s+1}\end{array}\right)$\\
$(x,y,z) \mapsto \left(\begin{array}{c}x^{2^s+1} + xy^{2^s} + yz^{2^s}\\xy^{2^s} + z^{2^s+1}\\x^{2^s}z + y^{2^s+1} + y^{2^s}z\end{array}\right)$\\
\bottomrule
\end{tabular}
}
\end{column}
\end{columns}

\onslide<2->{
\begin{tikzpicture}[overlay]
     \node[fill=white] at (11.5,.6) {\inlinebox{ + a lot of sporadic examples for small $n$}};
 \end{tikzpicture}}

 \onslide<3>{
\begin{tikzpicture}[overlay]
     \node[fill=white] at (8,6) {\inlinebox{\red{Where to look for new function ?}}};

     \node[fill=white] at (8,4.5) {\inlinebox{\red{How to prove that a new $F$ is \emph{actually} new ? }}};
 \end{tikzpicture}}

\end{frame}


\subsection{Step 1: understanding the SotA}
\begin{frame}
\frametitle{\subsecname}

\begin{mybox}{Theorem}{Linear self-equivalence}{\small \purple{[\blue{B}CanteautPerrin24]}}
For (almost) any $F$ presented before, there exists $\AA, \BB$ linear bijective such that:
$$ F \comp \AA = \BB \comp F$$
\end{mybox}

\pause
\begin{corollary}
    Any of the pen-and-paper APN functions are still \emph{very close to be} monomial.
\end{corollary}
\pause
    Let $c \in \FF_{16}$ and $F \from X \mapsto X^{3}$. Then $F(cX) = c^{3}X^{3} = c^{3}F(X^{3})$

    $ \AA \from X \mapsto cX, \BB \from X \mapsto c^{3}X.$ 

\pause
\begin{mybox}{A lot of open questions}{}{}
    \begin{itemize}
        \item[-] Inherent property of APN functions or due to a Human bias ?
        \item[-] Find more APN with a related structure or without !
    \end{itemize}
\end{mybox}

\end{frame}



\subsection{Steps 2 and 3: classify then search}
\begin{frame}
\frametitle{\subsecname}

\begin{mybox}{Classification}{}{}
\begin{itemize}
    \item[-] Implementation of the pen-and-paper families \hfill(almost done)
    \item[-] \emph{Data base} of the known functions \hfill(in progress)
    \item[-] Compute all interesting \emph{invariants} (e.g existence of $\AA, \BB$. . .) \hfill (not started)
\end{itemize}
\end{mybox}
\pause

\begin{mybox}{Search}{}{}
\begin{itemize}
    \item[-] Generic tree search for APN with a fixed $\AA, \BB$ \hfill(based on \purple{[BeiBriLea21]}, done)
    \item[-] Optimization \hfill(not started)
    \item[-] Search functions with more branches (in 4/5/6 variables ?)\hfill(not started)
    \item[-] Search functions with high degree (another open problem)\hfill(not started)
\end{itemize}
\end{mybox}
\end{frame}


\subsection{Take away}
\begin{frame}
\frametitle{\subsecname}

\begin{mybox}{APN functions}{}{}
\begin{itemize}
    \item[-] \emph{Optimal} objects w.r.t \emph{differential} cryptanalysis
    \item[-] Very little is known, finding/classifying is hard
    \item[-] Not trendy, but exciting !
\end{itemize}
\end{mybox}


\begin{mybox}{Big APN Problem}{}{\purple{\small[BDMW10]}}
Find an APN bijection $F \from \FFspace \to \FFspace$ with $n > 6$ even.
\end{mybox}

\begin{mybox}{Work in progress}{}{}
\begin{itemize}
    \item[-] Data base of APN functions + invariants
    \item[-] Search for functions with more branches
    \item[-] Search for functions with high degree
\end{itemize}
\end{mybox}
\end{frame}


\subsection{References}
\begin{frame}
\frametitle{\subsecname}
\vspace{-1cm}
\begin{mybox}{My bedside readings}{}{}
\begin{itemize}
    \item[-] Anne Canteaut, \emph{Lecture Notes on Cryptographic Boolean Functions} (available online, easily-accessible course)
    \item[-] Rudolf Lidl \& Harald Niederreiter, \emph{Finite fields} (my must-read about finite fields, with all standards results (and way more))
    \item[-] Claude Carlet, \emph{Boolean Functions for Cryptography and Coding Theory} (available online, not my favorite book, but it has the benefit of pointing toward 1000+ articles)
\end{itemize}
\end{mybox}

\begin{mybox}{Papers cited in prev. slides}{}{}
\begin{itemize}
    \item[-] [BihSha91]  Biham \& Shamir, \emph{Differential Cryptanalysis of the Data Encryption Standard}
    \item[-] [NybKnu92] Nyberg \& Knudsen, \emph{Provable Security Against Differential Cryptanalysis}, CRYPTO 92
    \item[-] [BDMW10] Browning, Dillon, McQuistan \& Wolfe, \emph{An APN Permutation in Dimension Six}, Fq9 conference 2009 (Finite Fields and their Applications)
    \item[-] [BauCanPer24] Baudrin, Canteaut, Perrin, \emph{Linear self-equivalence of the known families of APN functions: a unified point of view} (submitted)
    \item[-] [BeiBriLea21] \emph{Linearly self-equivalent APN permutations in small dimension}, IEEE IT 2021
\end{itemize}
\end{mybox}

\end{frame}
\end{comment}






% !TEX root = ../slides_jc2.tex
\section{Polynomial representations of Boolean functions}

\begin{frame}
\frametitle{Linear self-equivalence : a unifying PoV on the known families of APN functions}

% !TEX root = ../slides.tex

\begin{columns}[c]
\begin{column}{0.5\textwidth}
\renewcommand\arraystretch{1.3} 
\scalebox{.88}{
\begin{tabular}{|c|}
\toprule
\rctt\textbf{Univariate}\\
\midrule
$x^{2^s + 1} + ax^{2^{(3-i)k + s} + 2^{ik}}$\\
\rct$x^{2^s + 1} + ax^{2^{(4-i)k + s} + 2^{ik}} $\\
$ax^{2^k + 1} + x^{2^s +1} + x^{2^{s + k} + 2^k} + bx^{2^{k + s} + 1} + b^{2^{k}}x^{2^s + 2^k}$\\
\rct$x^{3} + a^{-1}\tr[\FF_{2^n}][\FF_2](a^3x^9)$\\
$x^{3} + a^{-1}\tr[\FF_{2^n}][\FF_{2^3}](a^3x^9 + a^6x^{18})$\\
\rct$x^{3} + a^{-1}\tr[\FF_{2^n}][\FF_{2^3}](a^6x^{18} + a^{12}x^{36})$\\
$ax^{2^s + 1} + a^{2^k}x^{2^{2k} + 2^{k + s}} + bx^{2^{2k} + 1} + ca^{2^k + 1}x^{2^{s} + 2^{k + s}}$\\
\rct$a^{2}x^{2^{2k + 1} + 1} + b^{2}x^{2^{k +1} + 1} + ax^{2^{2k} + 2} + bx^{2^{k} + 2} + dx^{3}$\\
$ x^3 + ax^{2^{s+i} + 2^i} + a^2x^{2^{k+1} + 2^k} + x^{2^{s + i + k} + 2^{i + k}}$\\
\rct$ a\tr[\FFfield][\subfield](bx^{2^i + 1}) + a^{2^k}\tr[\FFfield][\subfield](cx^{2^s + 1})$\\
$ L(x)^{2^k + 1} + bx^{2^k + 1} $\\
\bottomrule
\end{tabular}}
\end{column}
\begin{column}{0.5\textwidth}
% !TEX root = ../slides.tex
\renewcommand\arraystretch{.9} 
\scalebox{.77}{
\begin{tabular}{|c|}
\toprule
\rctt \multicolumn{1}{c|}{\textbf{Multivariate}}\\ 
\midrule
$(x,y) \mapsto \left(\begin{array}{c} x^{2^s + 1} + ay ^{(2^s+1)2^i}\\ xy\end{array}\right)$\\
\rct$(x,y) \mapsto \left(\begin{array}{c}x^{2^{2s} + 2^{3s}} + ax^{2^{2s}}y^{2^s} + by^{2^s+1}\\ xy\end{array}\right)$\\
$(x,y) \mapsto \left(\begin{array}{c}x^{2^s+1} + x^{2^{s + k/2}}y^{2^{k/2}} + axy^{2^s} + by^{2^s+1}\\ xy\end{array}\right)$\\
\rct$(x,y) \mapsto \left(\begin{array}{c}x^{2^s+1} + xy^{2^{s}} + y^{2^s + 1}\\ x^{2^{2s}+1} + x^{2^{2s}}y + y^{2^{2s} + 1}\end{array}\right)$\\
$(x,y) \mapsto \left(\begin{array}{c}x^{2^s+1} + xy^{2^{s}} + y^{2^s + 1}\\ x^{2^{3s}}y + xy^{2^{3s}}\end{array}\right)$\\
\rct$(x,y) \mapsto \left(\begin{array}{c}x^{2^s+1} + by^{2^s + 1}\\ x^{2^{s + k/2}}y + \frac{a}{b}xy^{2^{s + k/2}}\end{array}\right)$\\
$(x,y) \mapsto \left(\begin{array}{c}x^{2^s + 1} + xy^{2^s} + ay^{2^s +1}\\ x^{2^{2s} + 1} + ax^{2^{2s}}y + (1 + a)^{2^s}xy^{2^{2s}} + ay^{2^{2s} + 1}\end{array}\right)$\\
\rct$(x,y,z) \mapsto \left(\begin{array}{c}x^{2^s+1} + x^{2^s}z + yz^{2^s}\\x^{2^s}z + y^{2^s+1}\\xy^{2^s} + y^{2^s}z + z^{2^s+1}\end{array}\right)$\\
$(x,y,z) \mapsto \left(\begin{array}{c}x^{2^s+1} + xy^{2^s} + yz^{2^s}\\xy^{2^s} + z^{2^s+1}\\x^{2^s}z + y^{2^s+1} + y^{2^s}z\end{array}\right)$\\
\bottomrule
\end{tabular}
}
\end{column}
\end{columns}

% \onslide<2->{
% \begin{tikzpicture}[overlay]
%      \node[fill=white] at (11.5,.6) {\inlinebox{ + a lot of sporadic examples for small $n$}};
%  \end{tikzpicture}}

 \onslide<2>{
\begin{tikzpicture}[overlay]
    %  \node[fill=white] at (8,6) {\inlinebox{\red{Where to look for a new function ?}}};
    % \node[fill=white] at (8,3) {\inlinebox{\red{Intersection between families ?}}};
     \node[fill=white] at (8,4.5) {\inlinebox{\orange{\large Hopefully clearer in 12 min ?}}};
 \end{tikzpicture}}

\end{frame}

{
\setbeamercolor{background canvas}{bg=ptblue}	
\begin{frame}[plain]
\vfill
\begin{center}
%\color{white} \Huge \Roman{section} -  \secname
\color{white} \Huge \secname
\end{center}
\vfill
\end{frame}
}
% \begin{frame}\frametitle{Representing a vectorial Boolean function}
% \vspace{-10pt}
% $$F \from \FFspace \to \FFspace, \begin{pmatrix}x_{1}\\  \vdots\\  x_{n}
	
% \end{pmatrix} \mapsto \begin{pmatrix}F_{1}(x_{1}, \dotsc, x_{n})\\  \vdots\\  F_{n}(x_{1}, \dotsc, x_{n})
	
% \end{pmatrix}.$$

% Each $F_{i}: \FFspace \to \FF_{2}$ is a \emph{coordinate}.\pause

% \vspace{.5cm}
% A \emph{component} of $F$ is a linear combination of coordinate: $\alpha \cdot F \vcentcolon= \sum_{i=0}^{n-1} \alpha_{i} F_{i}$.\pause

% \vspace{.5cm}
% \begin{mybox}{Representations we won't look at}{}{}
% \begin{itemize}
% 	\item[\bulletpoint] Truth table / \emph{graph} of $F$: $\graph_{F} = \set{(x, F(x)), x \in \FFspace}$
% 	\item[\bulletpoint] \emph{Walsh transform}: Fourier transform of all components $\alpha \cdot F: \FFspace \to \FF_{2}  \subset \CC $
% \end{itemize}
% \end{mybox}

% \end{frame}


\begin{frame}\frametitle{Polynomial representations (1/2)}
\vspace{-10pt}
$$F \from \cFFspace \to \cFFspace, \begin{pmatrix}x_{1}\\  \vdots\\  x_{n}
    
\end{pmatrix} \mapsto \begin{pmatrix}F_{1}(x_{1}, \dotsc, x_{n})\\  \vdots\\  F_{n}(x_{1}, \dotsc, x_{n})
    
\end{pmatrix}.$$
\pause
\begin{theorem}[Lagrange multivariate interpolation]
$f \from (\FF_{q})^{m} \to \FF_{q}$ admits a polynomial representation in $\FF_{q}[X_{1}, \dotsc, X_{m}]/ (X_{1}^{q} + X_{1}, \dotsc, X_{m}^{q} + X_{m})$.
\end{theorem}\pause


\begin{mybox}{Algebraic Normal Form (ANF)}{}{}
$(q = 2, m=n)$. Each coordinate is a polynomial of $\FF_{2}[X_{1}, \dotsc, X_{n}]/ (X_{1}^{2} + X_{1}, \dotsc, X_{n}^{2} + X_{n})$
\end{mybox}\pause

\begin{center}
$F \from \cyan{\FF_{2}^{4}} \to \cyan{\FF_{2}^{4}}, \begin{pmatrix}
x_{0}\\ x_{1}\\ x_{2}\\ x_{3}
\end{pmatrix} \mapsto \begin{pmatrix}
    x_0x_2 + x_0 + x_1x_2 + x_1x_3\\ 
x_0x_1 + x_0x_2 + x_2x_3 + x_3\\ 
x_0x_1 + x_0x_2 + x_0x_3 + x_1x_2 + x_1x_3 + x_2x_3 + x_2\\ 
x_1x_3 + x_1 + x_2x_3 + x_2 + x_3
\end{pmatrix}$
\end{center}

%\emph{Algebraic degree} : $\algdeg(F) \vcentcolon= \max\limits_{1 \leq i \leq n} \deg(F_{i})$.\hfill Here $\algdeg(F) = 2$

\end{frame}


\begin{frame}\frametitle{Polynomial representations (2/2)}
\vspace{-.5cm}
\begin{theorem}[Lagrange multivariate interpolation]
$f \from (\FF_{q})^{m} \to \FF_{q}$ admits a polynomial representation in $\FF_{q}[X_{1}, \dotsc, X_{m}]/ (X_{1}^{q} + X_{1}, \dotsc, X_{m}^{q} + X_{m})$.
\end{theorem}\pause


\begin{mybox}{$\FF_{2}$-space isomorphisms}{}{}
$$\quad \quad \cFFspace \quad \simeq \quad \cFFfield \quad \simeq \quad \cFFinter, \text{ with } n = \ell k.$$
\end{mybox}\pause

\begin{columns}[t]
\begin{column}{0.5\textwidth}
\begin{mybox}{Univariate representations}{$q = 2^{n}, m = 1$}{}
$F \from \cFFspace \to \cFFspace$ can be seen as $\widetilde{F} \from \cFFfield \to \cFFfield$.


\begin{align*}
	\widetilde{F} \from \orange{\FF_{2^{4}}} &\to \orange{\FF_{2^{4}}}\\ 
	X &\mapsto \orange{\alpha_{0}}X^{12} + \orange{\alpha_{1}}X^{6} + \orange{\alpha_{2}}X^{3}
\end{align*}
\end{mybox}

\end{column}\pause
\begin{column}{0.5\textwidth}

\begin{mybox}{Multivariate representations}{$q = 2^{k}, m = \ell$}{}
$F \from \cFFspace \to \cFFspace$ can be seen as $\widetilde{F} \from \cFFinter \to \cFFinter$.


\begin{align*}\widetilde{F} \from \purple{\FF_{2^{2}}^{2}} &\to \purple{\FF_{2^{2}}^{2}}\\ \begin{pmatrix}x_{0}\\  x_{1}\end{pmatrix} &\mapsto \begin{pmatrix}\purple{\alpha_{0}}x_{0}^3 + x_{0}^2x_{1} + \purple{\alpha_{1}}x_{0}x_{1}^2 + \purple{\alpha_{2}}x_{1}^3\\
\purple{\alpha_{3}}x_{0}^3 + \purple{\alpha_{4}}x_{0}^2x_{1} + \purple{\alpha_{5}}x_{0}x_{1}^2\end{pmatrix}
\end{align*}
\end{mybox}
\end{column}
\end{columns}

\begin{center}
    \emph{Up to a choice of bases!}
\end{center}
\end{frame}

\begin{frame}\frametitle{Polynomial representations and APN functions}
\vspace{-.5cm}
 $$ \delta_{F}(\din, \dout) = \card{\set{x, F(x + \din) + F(x) = \dout}} $$\pause

 $\orange{A} \from (\FFspace, +) \to (U, \black{+_{_{U}}})$ and $\orange{B} \from (V, \black{+_{_{V}}}) \to (\FFspace, +)$ linear bijective mappings.

 Then $\orange{A}\comp F\comp \orange{B} \from (V, \black{+_{_{V}}}) \to (U, \black{+_{_{U}}})$\pause



% $$
% \begin{array}{ccccc}
% \orange{A}\comp F\comp \orange{B}(x \ \black{+_{_{V}}} \ \din) \ &\black{+_{_{U}}}& \  \orange{A}\comp F\comp \orange{B}(x) &=& \dout\\\pause 
% %\orange{G} (F\comp \red{H}(x \ \red{+_{_{V}}} \ \din) &+& F\comp \red{H}(x)) &=& \dout \\ 
%  F\comp \orange{B}(x \ \black{+_{_{V}}} \ \din) &+& F\comp \orange{B}(x) &=& \orange{A}^{-1}(\dout) \\\pause
%   F( \orange{B}(x) + \orange{B}(\din)) &+& F\comp \orange{B}(x) &=& \orange{A}^{-1}(\dout) \\\pause
%     %F( y + \red{H}(\din)) &+& F(y) &=& \orange{G}^{-1}(\dout)
% \end{array}
% $$

\begin{proposition}
\begin{itemize}
	\item[\bulletpoint] $\forall \din, \dout, \quad \delta_{F}(\orange{B}(\din), \orange{A}^{-1}(\dout)) = \delta_{\orange{A}F\orange{B}} (\din, \dout)$ 
	\item[\bulletpoint] $F$ is APN if and only if $\orange{A}\comp F\comp \orange{B}$ is APN.
\end{itemize}
\end{proposition}

\begin{definition}[Linear equivalence]
    $F_{1} \lin F_{2}$ if $\quad \exists \ \orange{A}, \orange{B}$, bijective linear s.t. $\quad \orange{A} \comp F_{1} \comp \orange{B} = F_{2}$.
\end{definition}

% \begin{corollary}[Freedom of choice]
% 	Sometimes easier to prove that $F$ is APN using \emph{another representation}.
% \end{corollary}

\end{frame}

\begin{comment}
    


\subsection{Equivalence relations}
\begin{frame}
\frametitle{\subsecname}
\vspace{-.4cm}

\begin{columns}[c]
\begin{column}{0.7\textwidth}

\begin{mybox}{Linear equivalence}{}{}
$F_{1} \lin F_{2}$ if $\quad \exists \ \orange{A}, \orange{B}$, bijective \emph{linear} s.t. $\quad \orange{A} \comp F_{1} \comp \orange{B} = F_{2}$.
\end{mybox}

\onslide<2->{
\begin{mybox}{Affine equivalence}{}{}
$F_{1} \aff F_{2}$ if $\quad \exists \ \blue{A}, \blue{B}$, bijective \emph{affine} s.t. $\quad \blue{A} \comp F_{1} \comp \blue{B} = F_{2}$.
\end{mybox}
}
\onslide<3->{
\begin{mybox}{CCZ equivalence}{}{\purple{\small[CCZ98]}}
$F_{1} \from \FFspace \to \FFspace$  $\ccz$  $F_{2} \from \FFspace \to \FFspace$ if: $\exists \ \purple{\mathcal{A}} \from \FFspace \times \FFspace \to \FFspace \times \FFspace$

 bijective \emph{affine} s.t.
\vspace{-.2cm}
$$ \purple{\mathcal{A}} \left(\graph_{F_{1}}\right) = \graph_{F_{2}},$$

where $\graph_{F} \vcentcolon= \set{(x, F(x), x \in \FFspace)}$.
\end{mybox}}

\onslide<4->{
\begin{proposition}
    If  $F_{1} \ccz F_{2}$, then $\quad F_{1}$ APN $\iff$ $F_{2}$ APN.
\end{proposition}
}

\end{column}

\begin{column}{0.3\textwidth}

\vspace{.6cm}
\begin{tikzpicture}[scale=.8]
    %   \scriptsize     
        % \coordinate (nw) at (0, 5);
        % \coordinate (sw) at (0, 0);
        % \coordinate (ne) at (6, 5);
        % \coordinate (se) at (6, 0);
        
        \onslide<2->{
        	\draw[very thick,oiblue, rounded corners=5pt] (0, -.2) rectangle node {Affine} (4, -3.8) ;
            \draw[very thick,oiblue,rounded corners=5pt] (0, 0) rectangle (4,4) ;
         \draw[very thick,oiorange,rounded corners=5pt] (.1, 2.05) rectangle +(3.8,.875) ;
          \draw[very thick,oiorange,rounded corners=5pt] (.1, 3.025) rectangle +(3.8,.875) ;
        \draw[very thick,oiorange,rounded corners=5pt] (.1, 1.075) rectangle +(3.8,.875) ;
        \draw[very thick,oiorange,rounded corners=5pt] (.1, 0.1) rectangle node {Linear} +(3.8,.875) ;
        }

       	\onslide<3->{
        \draw[very thick,oipurple,rounded corners=5pt] (-0.2, 4.2) rectangle (4.2,-4) ;
        \draw[very thick, oipurple] (0.2, -4) -- +(.5, -.5) node[right=.4] {CCZ};}

        % \draw[very thick,oiorange,rounded corners=15pt] ($(nw)+(3, -1.5)$) rectangle ($(se)+(-1.3, 1.5)$) ;


        % \draw (3.7,1.9) node[] {{\tiny X} $E_{\red{k}}$};

        
        % \draw[very thick,oiblue] ($ (ne) + (-.25, -1.4) $) --++(.6,0.7) node[right=1]{$\mathrm{aff}(\FF_2^n)$};
        % \node[very thick,oiorange] at ($ (ne) + (-1.1, -2) $) {\orange{$\mathcal{E}$}};
\end{tikzpicture}


\end{column}

\end{columns}

\end{frame}
\end{comment}


\subsection{Proper representatives for easier proofs}
\begin{frame}
\frametitle{\subsecname}

% \begin{center}
% $4$ bits $\approx$ a pair of $2$-bit words $\approx$ a $4$-bit word    
% \end{center}
\vspace{-.5cm}
\begin{mybox}{4 linearly-equivalent functions}{}{}

$F \from \cyan{\FF_{2}^{4}} \to \cyan{\FF_{2}^{4}}, \begin{pmatrix}
x_{0}\\ x_{1}\\ x_{2}\\ x_{3}
\end{pmatrix} \mapsto \begin{pmatrix}
    x_0x_2 + x_0 + x_1x_2 + x_1x_3\\ 
x_0x_1 + x_0x_2 + x_2x_3 + x_3\\ 
x_0x_1 + x_0x_2 + x_0x_3 + x_1x_2 + x_1x_3 + x_2x_3 + x_2\\ 
x_1x_3 + x_1 + x_2x_3 + x_2 + x_3
\end{pmatrix}$
\vspace{.5cm}

$F \from \purple{\FF_{4}^{2}} \to \purple{\FF_{4}^{2}},  \begin{pmatrix}x_{0}\\  x_{1}\end{pmatrix} \mapsto \begin{pmatrix}\purple{\alpha_{0}}x_{0}^3 + x_{0}^2x_{1} + \purple{\alpha_{1}}x_{0}x_{1}^2 + \purple{\alpha_{2}}x_{1}^3\\
\purple{\alpha_{3}}x_{0}^3 + \purple{\alpha_{4}}x_{0}^2x_{1} + \purple{\alpha_{5}}x_{0}x_{1}^2\end{pmatrix}$

\vspace{.5cm}
$F \from \orange{\FF_{16}} \to \orange{\FF_{16}}, X \mapsto \orange{\alpha_{0}}X^{12} + \orange{\alpha_{1}}X^{6} + \orange{\alpha_{2}}X^{3}$

\onslide<2->{
\vspace{.5cm}
$F \from \orange{\FF_{16}} \to \orange{\FF_{16}}, X \mapsto X^{3}$}
\end{mybox}
\vspace{-1cm}
\begin{align*}
	\onslide<3->{F(X+ \plaindiff) + F(X) = \dout}\\
	\onslide<4->{(X + \plaindiff)^{3} + X^{3} = \dout}\\ 
	\onslide<5->{\plaindiff X^{2} + \plaindiff^{2} X + \plaindiff^{3} + \dout = 0}
\end{align*}
%$$  = (X + \plaindiff)^{3} + X^{3} = \plaindiff X^{2} + \plaindiff^{2} X + \plaindiff^{3}$$

\onslide<6>{$\implies$ at most 2 solutions $\implies$ APN !}
\end{frame}

\begin{frame}
\frametitle{Linear self-equivalence : a unifying PoV on the known families of APN functions}

% !TEX root = ../slides.tex

\begin{columns}[c]
\begin{column}{0.5\textwidth}
\renewcommand\arraystretch{1.3} 
\scalebox{.88}{
\begin{tabular}{|c|}
\toprule
\rctt\textbf{Univariate}\\
\midrule
$x^{2^s + 1} + ax^{2^{(3-i)k + s} + 2^{ik}}$\\
\rct$x^{2^s + 1} + ax^{2^{(4-i)k + s} + 2^{ik}} $\\
$ax^{2^k + 1} + x^{2^s +1} + x^{2^{s + k} + 2^k} + bx^{2^{k + s} + 1} + b^{2^{k}}x^{2^s + 2^k}$\\
\rct$x^{3} + a^{-1}\tr[\FF_{2^n}][\FF_2](a^3x^9)$\\
$x^{3} + a^{-1}\tr[\FF_{2^n}][\FF_{2^3}](a^3x^9 + a^6x^{18})$\\
\rct$x^{3} + a^{-1}\tr[\FF_{2^n}][\FF_{2^3}](a^6x^{18} + a^{12}x^{36})$\\
$ax^{2^s + 1} + a^{2^k}x^{2^{2k} + 2^{k + s}} + bx^{2^{2k} + 1} + ca^{2^k + 1}x^{2^{s} + 2^{k + s}}$\\
\rct$a^{2}x^{2^{2k + 1} + 1} + b^{2}x^{2^{k +1} + 1} + ax^{2^{2k} + 2} + bx^{2^{k} + 2} + dx^{3}$\\
$ x^3 + ax^{2^{s+i} + 2^i} + a^2x^{2^{k+1} + 2^k} + x^{2^{s + i + k} + 2^{i + k}}$\\
\rct$ a\tr[\FFfield][\subfield](bx^{2^i + 1}) + a^{2^k}\tr[\FFfield][\subfield](cx^{2^s + 1})$\\
$ L(x)^{2^k + 1} + bx^{2^k + 1} $\\
\bottomrule
\end{tabular}}
\end{column}
\begin{column}{0.5\textwidth}
% !TEX root = ../slides.tex
\renewcommand\arraystretch{.9} 
\scalebox{.77}{
\begin{tabular}{|c|}
\toprule
\rctt \multicolumn{1}{c|}{\textbf{Multivariate}}\\ 
\midrule
$(x,y) \mapsto \left(\begin{array}{c} x^{2^s + 1} + ay ^{(2^s+1)2^i}\\ xy\end{array}\right)$\\
\rct$(x,y) \mapsto \left(\begin{array}{c}x^{2^{2s} + 2^{3s}} + ax^{2^{2s}}y^{2^s} + by^{2^s+1}\\ xy\end{array}\right)$\\
$(x,y) \mapsto \left(\begin{array}{c}x^{2^s+1} + x^{2^{s + k/2}}y^{2^{k/2}} + axy^{2^s} + by^{2^s+1}\\ xy\end{array}\right)$\\
\rct$(x,y) \mapsto \left(\begin{array}{c}x^{2^s+1} + xy^{2^{s}} + y^{2^s + 1}\\ x^{2^{2s}+1} + x^{2^{2s}}y + y^{2^{2s} + 1}\end{array}\right)$\\
$(x,y) \mapsto \left(\begin{array}{c}x^{2^s+1} + xy^{2^{s}} + y^{2^s + 1}\\ x^{2^{3s}}y + xy^{2^{3s}}\end{array}\right)$\\
\rct$(x,y) \mapsto \left(\begin{array}{c}x^{2^s+1} + by^{2^s + 1}\\ x^{2^{s + k/2}}y + \frac{a}{b}xy^{2^{s + k/2}}\end{array}\right)$\\
$(x,y) \mapsto \left(\begin{array}{c}x^{2^s + 1} + xy^{2^s} + ay^{2^s +1}\\ x^{2^{2s} + 1} + ax^{2^{2s}}y + (1 + a)^{2^s}xy^{2^{2s}} + ay^{2^{2s} + 1}\end{array}\right)$\\
\rct$(x,y,z) \mapsto \left(\begin{array}{c}x^{2^s+1} + x^{2^s}z + yz^{2^s}\\x^{2^s}z + y^{2^s+1}\\xy^{2^s} + y^{2^s}z + z^{2^s+1}\end{array}\right)$\\
$(x,y,z) \mapsto \left(\begin{array}{c}x^{2^s+1} + xy^{2^s} + yz^{2^s}\\xy^{2^s} + z^{2^s+1}\\x^{2^s}z + y^{2^s+1} + y^{2^s}z\end{array}\right)$\\
\bottomrule
\end{tabular}
}
\end{column}
\end{columns}

% \onslide<2->{
% \begin{tikzpicture}[overlay]
%      \node[fill=white] at (11.5,.6) {\inlinebox{ + a lot of sporadic examples for small $n$}};
%  \end{tikzpicture}}

 \onslide<2>{
\begin{tikzpicture}[overlay]
    %  \node[fill=white] at (8,6) {\inlinebox{\red{Where to look for a new function ?}}};
    % \node[fill=white] at (8,3) {\inlinebox{\red{Intersection between families ?}}};
     \node[fill=white] at (8,4.5) {\inlinebox{\orange{\large Hopefully clearer in 5 min ?}}};
 \end{tikzpicture}}

\end{frame}

\begin{comment}
\begin{frame}\frametitle{The APN family tree}
\vspace{-.5cm}
\begin{mybox}{A common descent}{}{\small \purple{[Nyberg93]}}
The function $F \from \FFfield \to \FFfield, X \mapsto X^{3}$ is APN.
\begin{itemize}
	\item[\bulletpoint] $F$ is a \blue{power mapping}
	\item[\bulletpoint] $F$ is \orange{quadratic}: $\algdeg(F) = \hamming(3) = 2$
\end{itemize}
\end{mybox}\pause
\vspace{-.5cm}

\begin{chronology}[5]{1992}{2024}{15cm}[\textwidth]
\eventpoint{1992}{ \ \ [NybKnu92]}[black][1][1]
\eventpoint{1993}{ \ \ [Nyberg93]}[black][1][1]
\eventspan{1992}{2001}{}[oiblue][.7][.2]
\eventpoint{2006}{ \ \ [EKP06]}[black][1][1] %BCP06, 
\eventspan{2006}{2024}{}[oiorange][.7][.2]
\end{chronology}

\begin{mybox}{Descendants}{}{}
\begin{itemize}
	\item[\bulletpoint] 6 infinite families of APN \blue{power mappings}, some are \emph{not quadratic}. %\small \purple{[Nyberg93, Dobbertin99a,99b,01]
	\item[\bulletpoint] About 20 infinite families of \orange{quadratic} APN mappings.
% 	\item[\bulletpoint] A lot of sporadic examples for $n \leq 9$
\end{itemize}
\end{mybox}\pause

% \begin{mybox}{Descendant branch \#2: quadratic mappings}{}{}
% \begin{itemize}
% 	\item[\bulletpoint] About 20 infinite families of quadratic APN mappings.
% 	\item[\bulletpoint] A lot of sporadic examples for $n \leq 9$
% \end{itemize}
% \end{mybox}

\begin{mybox}{A single counter-example}{}{\small \purple{[BriLea08,EdePot09]}}
A single APN function \emph{inequivalent} to a power mapping or a quadratic mapping is known.
\end{mybox}

\end{frame}
\end{comment}


\begin{comment}
    

\begin{frame}
\frametitle{Infinite families of quadratic APN mappings}

% !TEX root = ../slides.tex

\begin{columns}[c]
\begin{column}{0.5\textwidth}
\renewcommand\arraystretch{1.3} 
\scalebox{.88}{
\begin{tabular}{|c|}
\toprule
\rctt\textbf{Univariate}\\
\midrule
$x^{2^s + 1} + ax^{2^{(3-i)k + s} + 2^{ik}}$\\
\rct$x^{2^s + 1} + ax^{2^{(4-i)k + s} + 2^{ik}} $\\
$ax^{2^k + 1} + x^{2^s +1} + x^{2^{s + k} + 2^k} + bx^{2^{k + s} + 1} + b^{2^{k}}x^{2^s + 2^k}$\\
\rct$x^{3} + a^{-1}\tr[\FF_{2^n}][\FF_2](a^3x^9)$\\
$x^{3} + a^{-1}\tr[\FF_{2^n}][\FF_{2^3}](a^3x^9 + a^6x^{18})$\\
\rct$x^{3} + a^{-1}\tr[\FF_{2^n}][\FF_{2^3}](a^6x^{18} + a^{12}x^{36})$\\
$ax^{2^s + 1} + a^{2^k}x^{2^{2k} + 2^{k + s}} + bx^{2^{2k} + 1} + ca^{2^k + 1}x^{2^{s} + 2^{k + s}}$\\
\rct$a^{2}x^{2^{2k + 1} + 1} + b^{2}x^{2^{k +1} + 1} + ax^{2^{2k} + 2} + bx^{2^{k} + 2} + dx^{3}$\\
$ x^3 + ax^{2^{s+i} + 2^i} + a^2x^{2^{k+1} + 2^k} + x^{2^{s + i + k} + 2^{i + k}}$\\
\rct$ a\tr[\FFfield][\subfield](bx^{2^i + 1}) + a^{2^k}\tr[\FFfield][\subfield](cx^{2^s + 1})$\\
$ L(x)^{2^k + 1} + bx^{2^k + 1} $\\
\bottomrule
\end{tabular}}
\end{column}
\begin{column}{0.5\textwidth}
% !TEX root = ../slides.tex
\renewcommand\arraystretch{.9} 
\scalebox{.77}{
\begin{tabular}{|c|}
\toprule
\rctt \multicolumn{1}{c|}{\textbf{Multivariate}}\\ 
\midrule
$(x,y) \mapsto \left(\begin{array}{c} x^{2^s + 1} + ay ^{(2^s+1)2^i}\\ xy\end{array}\right)$\\
\rct$(x,y) \mapsto \left(\begin{array}{c}x^{2^{2s} + 2^{3s}} + ax^{2^{2s}}y^{2^s} + by^{2^s+1}\\ xy\end{array}\right)$\\
$(x,y) \mapsto \left(\begin{array}{c}x^{2^s+1} + x^{2^{s + k/2}}y^{2^{k/2}} + axy^{2^s} + by^{2^s+1}\\ xy\end{array}\right)$\\
\rct$(x,y) \mapsto \left(\begin{array}{c}x^{2^s+1} + xy^{2^{s}} + y^{2^s + 1}\\ x^{2^{2s}+1} + x^{2^{2s}}y + y^{2^{2s} + 1}\end{array}\right)$\\
$(x,y) \mapsto \left(\begin{array}{c}x^{2^s+1} + xy^{2^{s}} + y^{2^s + 1}\\ x^{2^{3s}}y + xy^{2^{3s}}\end{array}\right)$\\
\rct$(x,y) \mapsto \left(\begin{array}{c}x^{2^s+1} + by^{2^s + 1}\\ x^{2^{s + k/2}}y + \frac{a}{b}xy^{2^{s + k/2}}\end{array}\right)$\\
$(x,y) \mapsto \left(\begin{array}{c}x^{2^s + 1} + xy^{2^s} + ay^{2^s +1}\\ x^{2^{2s} + 1} + ax^{2^{2s}}y + (1 + a)^{2^s}xy^{2^{2s}} + ay^{2^{2s} + 1}\end{array}\right)$\\
\rct$(x,y,z) \mapsto \left(\begin{array}{c}x^{2^s+1} + x^{2^s}z + yz^{2^s}\\x^{2^s}z + y^{2^s+1}\\xy^{2^s} + y^{2^s}z + z^{2^s+1}\end{array}\right)$\\
$(x,y,z) \mapsto \left(\begin{array}{c}x^{2^s+1} + xy^{2^s} + yz^{2^s}\\xy^{2^s} + z^{2^s+1}\\x^{2^s}z + y^{2^s+1} + y^{2^s}z\end{array}\right)$\\
\bottomrule
\end{tabular}
}
\end{column}
\end{columns}

% \onslide<2->{
% \begin{tikzpicture}[overlay]
%      \node[fill=white] at (11.5,.6) {\inlinebox{ + a lot of sporadic examples for small $n$}};
%  \end{tikzpicture}}

 \onslide<2>{
\begin{tikzpicture}[overlay]
     \node[fill=white] at (8,6) {\inlinebox{\red{Where to look for a new function ?}}};
  	\node[fill=white] at (8,3) {\inlinebox{\red{Intersection between families ?}}};
     \node[fill=white] at (8,4.5) {\inlinebox{\red{How to prove that a new $F$ is actually new ? }}};
 \end{tikzpicture}}

\end{frame}
\end{comment}

% !TEX root = ../slides_jc2.tex
\section[A unified PoV on the known APN functions]{A unified point-of-view on the known APN functions}

{
\setbeamercolor{background canvas}{bg=ptblue}	
\begin{frame}[plain]
\vfill
\begin{center}
%\color{white} \Huge \Roman{section} -  \secname
\color{white} \Huge \secname
\end{center}
\vfill
\end{frame}
}


\subsection{One of the first non-power functions}

\begin{frame}
\frametitle{\subsecname}
\vspace{-.6cm}
% \begin{mybox}{Power function}{}{\small\purple{[Gold68, Nyberg94]}}
% $$F \from \FF_{2^{12}} \to \FF_{2^{12}} \quad x \mapsto x^{3}$$
% % $\blue{\lambda} \in \FFfield^{*}$. $F(\blue{\lambda} x) = \blue{\lambda}^{3} F(x)$ $\leadsto$ $\cyan{A} \comp F \comp \cyan{B} = F$ \hfill $\cyan{B}(x) = \blue{\lambda} x$, $\cyan{A}(x) = \blue{\lambda}^{-3} x$
% \end{mybox}
% \pause
\begin{mybox}{An APN binomial}{}{\small \purple{[BudCarLea08]}}
$$ G \from \orange{\FF_{2^{12}}} \to \orange{\FF_{2^{12}}} \quad x \mapsto x^3 + \orange{\alpha} x^{528}$$

$G(x) = x^{3}(1 + x^{525}) = x^{3}P(x^{15})$, where $P = 1+ X^{35}$ \hfill($525 = 35\times 15$)
%$\hspace*{5.5cm}F(x) =  x^3 + \alpha x^{528} = x^{3}P(x^{15}) \hfill P = 1 + x^{35}$


% \onslide<3->{
% \vspace{.3cm}
% $\blue{\lambda} \in \blue{\FF_{2^{4}}^{*}}$ (i.e. $\blue{\lambda}^{15} = 1$).  \hspace*{1.5cm} $F(\blue{\lambda})  =\blue{\lambda}^{3}P(\blue{\lambda}^{15}) = \blue{\lambda}^{3}P(1)$
% }


% \onslide<4->{
% \vspace{.3cm}
% \bulletpoint $F$ behaves as $x \mapsto x^{\red{3}}$ on each coset $\gamma\blue{\FF_{2^{4}}}$
% }

% \onslide<5->{
% \vspace{.1cm}
% \bulletpoint Multivariate point of view\hfill
% $\widetilde{F} \from (\blue{\FF_{2^{4}}})^{3} \to (\blue{\FF_{2^{4}}})^{3}\  (v_{1}, v_{2}, v_{3}) \mapsto \left(\widetilde{F_{1}}(v), \widetilde{F_{2}}(v), \widetilde{F_{3}}(v)\right)$
% }



% \onslide<6->{
% \vspace{.1cm}
% \hfill All coordinates $\widetilde{F}_{i}$ are \emph{homogeneous} of the \emph{same} \red{order 3}
% }

\end{mybox}\pause

\vspace{.2cm}
$\purple{\FF_{2^{4}}^{*}} \subset \orange{\FF_{2^{12}}^{*}}$. %$\quad\FF_{2^{12}}^{*} = \bigsqcup\limits_{\gamma \in \Gamma} \gamma\purple{\FF_{2^{4}}^{*}}$ for some system of representatives $\Gamma$.
\pause

$$\forall \ \purple{\phi} \in \purple{\FF_{2^{4}}^{*}}, \quad G(\purple{\phi}) = \purple{\phi}^{3}P(\purple{\phi}^{15}) = \purple{\phi}^{3}P(1).$$\pause
\vspace{.2cm}

\begin{proposition}
    For any $\gamma \in  \orange{\FF_{2^{12}}^{*}}$, the restriction of $G|_{\gamma\purple{\FF_{2^{4}}^{*}}}$ is (up to a constant) the power mapping $x \mapsto x^{3}$.
\end{proposition}


% \onslide<7->{
% \begin{mybox}{One of the first bivariate functions}{}{\purple{\small[ZhoPot13]}}
% $\hspace*{5.5cm}F \from \FF_{64}^{2} \to \FF_{64}^{2}, (x, y) \mapsto (xy, x^{3} + ay^{3})$


% % \vspace{.2cm}
% % $F_{0}$ homogeneous of order 2, $F_{1}$ homogeneous of order 3

% % \vspace{.2cm}
% % $\leadsto \cyan{A} \comp F \comp \cyan{B} = F$ with $\quad \cyan{B}(x, y) = (\blue{\lambda} x, \blue{\lambda} y), \cyan{A}(x, y) = (\blue{\lambda}^{-2} x, \blue{\lambda}^{-3} y)$
% \end{mybox}
% }

\end{frame}
\begin{frame}\frametitle{The multiplicative point of view}

\begin{mybox}{An APN binomial}{}{\small \purple{[BudCarLea08]}}
\begin{itemize}
    \item[\bulletpoint] $G \from \orange{\FF_{2^{12}}} \to \orange{\FF_{2^{12}}} \quad x \mapsto x^3 + \orange{\alpha} x^{528}$
    \item[\bulletpoint] $G|_{\purple{\FF_{2^{4}}}} : \purple{\phi} \mapsto c \purple{\phi}^{3}$
\end{itemize}
\end{mybox}\pause

\begin{mybox}{Multivariate point-of-view}{}{}
$G$ is linearly equivalent to $\widetilde{G} \from (\purple{\FF_{2^{4}}})^{3} \to (\purple{\FF_{2^{4}}})^{3}\  (x_{1}, x_{2}, x_{3}) \mapsto \left(\widetilde{G_{1}}(x), \widetilde{G_{2}}(x), \widetilde{G_{3}}(x)\right)$.

$$\widetilde{G}_{1}(x) = {\color{ptpurple!50}?}x_{1}^2x_{2} + {\color{ptpurple!50}?}x_{1}x_{2}^{2} + {\color{ptpurple!50}?}x_{2}^{3}+ {\color{ptpurple!50}?}x_{1}^2x_{3} + {\color{ptpurple!50}?}x_{2}^2x_{3} + {\color{ptpurple!50}?}x_{1}x_{3}^2 + {\color{ptpurple!50}?}x_{2}x_{3}^2 + {\color{ptpurple!50}?}x_{3}^3. $$

All coordinates of $\widetilde{G}$ are homogeneous of the same degree $3$.
\end{mybox}\pause

\begin{mybox}{An APN bivariate functions}{}{\purple{\small[ZhoPot13]}}
$$H \from \green{\FF_{64}^{2}} \to \green{\FF_{64}^{2}}, (x, y) \mapsto (xy, x^{3} + \green{a}y^{3})$$

\begin{itemize}
    \item[\bulletpoint{}] $H_{1}$ homogeneous of order 2.
    \item[\bulletpoint{}] $H_{2}$ homogeneous of order 3.
\end{itemize}
\end{mybox}


\end{frame}


\subsection{Linear self-equivalence}
\begin{frame}
\frametitle{\subsecname}
\vspace{-.7cm}
\begin{mybox}{Power mapping}{}{}
$$F(x) = x^{e}$$
Let $ \blue{\lambda} \in \FFfield^{*}$. Then for all $x$, $F(\blue{\lambda} x) = \blue{\lambda}^{e}x^{e} = \blue{\lambda}^{e}F(x)$.

Thus
 $\orange{A} \comp F \comp \orange{B} = F\ $ \hfill with $\orange{B}(x)\vcentcolon= \blue{\lambda} x$, $\ \orange{A}(x) \vcentcolon= \blue{\lambda}^{-e} x$. 
\end{mybox}
\pause

\vspace{-.1cm}
\begin{mybox}{Cyclotomic mapping w.r.t a subfield}{}{\purple{\small[Wang07]}}
$$ G(x) = x^{e}P\left(x^{2^{k} - 1}\right), n = \ell k$$
Let $ \purple{\phi} \in \purple{\FF_{2^{k}}}$. Then for all $x $, $G(\purple{\phi} x) = \purple{\phi}^{e}x^{e}P\left(x^{2^{k}-1}\right) = \purple{\phi}^{e}G(x)$.

Thus
 $\orange{A} \comp G \comp \orange{B} = G\ $ \hfill with  $ \orange{B}(x)\vcentcolon= \purple{\phi} x$, $\ \orange{A}(x) \vcentcolon= \purple{\phi}^{-e} x$. 


% $\cyan{A} \comp F \comp \cyan{B} = F$ with $\quad \cyan{B}(x) = \blue{\lambda} x$, $\quad\cyan{A}(x) = \blue{\lambda}^{-e} x$ for any $\lambda \in \blue{\FF_{2^{k}}^{*}}$

% \vspace{.2cm}
% $\widetilde{\cyan{A}} \comp \widetilde{F} \comp \widetilde{\cyan{B}} = \widetilde{F}$ with $\quad\widetilde{\cyan{B}}(v) = (\blue{\lambda} v_{1}, \dotsc , \blue{\lambda} v_{\ell}),$ $\quad \widetilde{\cyan{A}}(v) = (\blue{\lambda}^{-e} v_{1}, \dotsc , \blue{\lambda}^{-e} v_{\ell})$
\end{mybox}
\pause

\vspace{-.1cm}
\begin{mybox}{$\ell$-projective mapping}{}{\purple{\small[\blue{B}CP24,Göloğlu22]}}

$$ H \from \FFinter \to \FFinter \ (x_{1}, \dotsc, x_{\ell}
) \mapsto (H_{1}(x), \dotsc, H_{\ell}(x)),$$
$ \forall \ i$, $H_{i}$ is homogeneous of order $\cyan{e_{i}}$. 

Thus $\orange{A} \comp H \comp \orange{B} = H$ \hfill with $\quad\orange{B}(x) = (\purple{\phi} x_{1}, \dotsc , \purple{\phi} x_{\ell}),$

\hfill $\orange{A}(x) = (\purple{\phi}^{-\cyan{e_{1}}} x_{1}, \dotsc , \purple{\phi}^{-\cyan{e_{\ell}}} x_{\ell})$
\end{mybox}
\end{frame}

\subsection{Our main result (1/2)}
\begin{frame}
\frametitle{\subsecname}

\begin{center}
\vspace{-.5cm}
Among the 22 known infinite APN families, 19 consist entirely of 
\emph{cyclotomic} or \emph{$\ell$-projective} mappings, \emph{up to linear equivalence}.
% !TEX root = ../slides.tex
\renewcommand\arraystretch{1.3} 
\begin{tabular}{c<{\onslide<2->}c<{\onslide}}
% \visible<2->{\toprule}
\rctt\textbf{Univariate} &  \textbf{Observations}\\
% \midrule
$x^{2^s + 1} + ax^{2^{(3-i)k + s} + 2^{ik}}$ & cyclotomic\\
\rct$x^{2^s + 1} + ax^{2^{(4-i)k + s} + 2^{ik}} $ & cyclotomic\\
$ax^{2^k + 1} + x^{2^s +1} + x^{2^{s + k} + 2^k} + bx^{2^{k + s} + 1} + b^{2^{k}}x^{2^s + 2^k}$ & $\lin$  biprojective\\
\rct$x^{3} + a^{-1}\tr[\FF_{2^n}][\FF_2](a^3x^9)$ & cyclotomic/($\lin$) frob.\\
$x^{3} + a^{-1}\tr[\FF_{2^n}][\FF_{2^3}](a^3x^9 + a^6x^{18})$ & cyclotomic/($\lin$) frob.\\
\rct$x^{3} + a^{-1}\tr[\FF_{2^n}][\FF_{2^3}](a^6x^{18} + a^{12}x^{36})$ &  cyclotomic/($\lin$) frob.\\
$ax^{2^s + 1} + a^{2^k}x^{2^{2k} + 2^{k + s}} + bx^{2^{2k} + 1} + ca^{2^k + 1}x^{2^{s} + 2^{k + s}}$ & cyclotomic\\
\rct$a^{2}x^{2^{2k + 1} + 1} + b^{2}x^{2^{k +1} + 1} + ax^{2^{2k} + 2} + bx^{2^{k} + 2} + dx^{3}$ & cyclotomic\\
$ x^3 + ax^{2^{s+i} + 2^i} + a^2x^{2^{k+1} + 2^k} + x^{2^{s + i + k} + 2^{i + k}}$ & $\lin$ biprojective\\
\rct$ a\tr[\FFfield][\subfield](bx^{2^i + 1}) + a^{2^k}\tr[\FFfield][\subfield](cx^{2^s + 1})$ & $\lin$ biprojective\\
$ L(x)^{2^k + 1} + bx^{2^k + 1} $ & ?\\
% \visible<2->{\bottomrule}
\end{tabular}
\end{center}
% % !TEX root = ../slides.tex
\renewcommand\arraystretch{.9} 
\begin{tabular}{c|c|}
% \toprule
\rctt\textbf{Multivariate} &  \textbf{Observations}\\
% \midrule
$(x,y) \mapsto \left(\begin{array}{c} x^{2^s + 1} + ay ^{(2^s+1)2^i}\\ xy\end{array}\right)$ & $\lin$ biprojective\\
\rct$(x,y) \mapsto \left(\begin{array}{c}x^{2^{2s} + 2^{3s}} + ax^{2^{2s}}y^{2^s} + by^{2^s+1}\\ xy\end{array}\right)$ & $\lin$ biprojective\\
$(x,y) \mapsto \left(\begin{array}{c}x^{2^s+1} + x^{2^{s + k/2}}y^{2^{k/2}} + axy^{2^s} + by^{2^s+1}\\ xy\end{array}\right)$ & $\lin$ 4-projective\\
\rct$(x,y) \mapsto \left(\begin{array}{c}x^{2^s+1} + xy^{2^{s}} + y^{2^s + 1}\\ x^{2^{2s}+1} + x^{2^{2s}}y + y^{2^{2s} + 1}\end{array}\right)$ & biprojective\\
$(x,y) \mapsto \left(\begin{array}{c}x^{2^s+1} + xy^{2^{s}} + y^{2^s + 1}\\ x^{2^{3s}}y + xy^{2^{3s}}\end{array}\right)$ & biprojective\\
\rct$(x,y) \mapsto \left(\begin{array}{c}x^{2^s+1} + by^{2^s + 1}\\ x^{2^{s + k/2}}y + \frac{a}{b}xy^{2^{s + k/2}}\end{array}\right)$ &biprojective\\
$(x,y) \mapsto \left(\begin{array}{c}x^{2^s + 1} + xy^{2^s} + ay^{2^s +1}\\ x^{2^{2s} + 1} + ax^{2^{2s}}y + (1 + a)^{2^s}xy^{2^{2s}} + ay^{2^{2s} + 1}\end{array}\right)$ & biprojective\\
\rct$(x,y,z) \mapsto \left(\begin{array}{c}x^{2^s+1} + x^{2^s}z + yz^{2^s}\\x^{2^s}z + y^{2^s+1}\\xy^{2^s} + y^{2^s}z + z^{2^s+1}\end{array}\right)$ & $\begin{array}{c}\text{3-projective}\\ \lin \text{cyclotomic}\end{array}$\\
$(x,y,z) \mapsto \left(\begin{array}{c}x^{2^s+1} + xy^{2^s} + yz^{2^s}\\xy^{2^s} + z^{2^s+1}\\x^{2^s}z + y^{2^s+1} + y^{2^s}z\end{array}\right)$ & $\begin{array}{c}\text{3-projective}\\ \lin \text{cyclotomic}\end{array}$\\
% \bottomrule
\end{tabular}
\end{frame}

\subsection{Our main result (2/2)}
\begin{frame}
\frametitle{\subsecname}

\begin{center}
\vspace{-.5cm}
Among the 22 known infinite APN families, 19 consist entirely of 

\emph{cyclotomic} or \emph{$\ell$-projective} mappings, \emph{up to linear equivalence}.
\scalebox{.75}{
% !TEX root = ../slides.tex
\renewcommand\arraystretch{.9} 
\begin{tabular}{c|c|}
% \toprule
\rctt\textbf{Multivariate} &  \textbf{Observations}\\
% \midrule
$(x,y) \mapsto \left(\begin{array}{c} x^{2^s + 1} + ay ^{(2^s+1)2^i}\\ xy\end{array}\right)$ & $\lin$ biprojective\\
\rct$(x,y) \mapsto \left(\begin{array}{c}x^{2^{2s} + 2^{3s}} + ax^{2^{2s}}y^{2^s} + by^{2^s+1}\\ xy\end{array}\right)$ & $\lin$ biprojective\\
$(x,y) \mapsto \left(\begin{array}{c}x^{2^s+1} + x^{2^{s + k/2}}y^{2^{k/2}} + axy^{2^s} + by^{2^s+1}\\ xy\end{array}\right)$ & $\lin$ 4-projective\\
\rct$(x,y) \mapsto \left(\begin{array}{c}x^{2^s+1} + xy^{2^{s}} + y^{2^s + 1}\\ x^{2^{2s}+1} + x^{2^{2s}}y + y^{2^{2s} + 1}\end{array}\right)$ & biprojective\\
$(x,y) \mapsto \left(\begin{array}{c}x^{2^s+1} + xy^{2^{s}} + y^{2^s + 1}\\ x^{2^{3s}}y + xy^{2^{3s}}\end{array}\right)$ & biprojective\\
\rct$(x,y) \mapsto \left(\begin{array}{c}x^{2^s+1} + by^{2^s + 1}\\ x^{2^{s + k/2}}y + \frac{a}{b}xy^{2^{s + k/2}}\end{array}\right)$ &biprojective\\
$(x,y) \mapsto \left(\begin{array}{c}x^{2^s + 1} + xy^{2^s} + ay^{2^s +1}\\ x^{2^{2s} + 1} + ax^{2^{2s}}y + (1 + a)^{2^s}xy^{2^{2s}} + ay^{2^{2s} + 1}\end{array}\right)$ & biprojective\\
\rct$(x,y,z) \mapsto \left(\begin{array}{c}x^{2^s+1} + x^{2^s}z + yz^{2^s}\\x^{2^s}z + y^{2^s+1}\\xy^{2^s} + y^{2^s}z + z^{2^s+1}\end{array}\right)$ & $\begin{array}{c}\text{3-projective}\\ \lin \text{cyclotomic}\end{array}$\\
$(x,y,z) \mapsto \left(\begin{array}{c}x^{2^s+1} + xy^{2^s} + yz^{2^s}\\xy^{2^s} + z^{2^s+1}\\x^{2^s}z + y^{2^s+1} + y^{2^s}z\end{array}\right)$ & $\begin{array}{c}\text{3-projective}\\ \lin \text{cyclotomic}\end{array}$\\
% \bottomrule
\end{tabular}
}
\end{center}
\end{frame}

\begin{comment}

\subsection{Sketch of proof}
\begin{frame}
\frametitle{\subsecname}
\vspace{-.7cm}
% \begin{recap}[Conjugacy, again]
% The conjugate of a composition is the composition of the conjugates.
% $$ F = F_{3} \comp F_{2} \comp F_{1} \quad \iff \quad F^{\changevar} = F_{3}^{\changevar} \comp F_{2}^{\changevar} \comp F_{1}^{\changevar}$$
% \end{recap}

\onslide<1->{
\begin{mybox}{Linear self-equivalence \& conjugacy}{}{}
Let $F$ be linearly self-equivalent: $\quad F = \orange{A} \comp F \comp \orange{B}$.

Let $G$ be linearly equivalent to $F$: $\quad G = \cyan{P} \comp F \comp \cyan{Q}$.

\vspace{.4cm}
Then $G$ is linearly self-equivalent:
\vspace{-.2cm}
 $$ G = (\cyan{P} \comp \orange{A} \comp \cyan{P})^{-1} \comp G \comp (\cyan{Q}^{-1}\comp \orange{B} \comp \cyan{Q}) $$

\onslide<2->{
Furthermore, $\orange{A}$ and  $\cyan{P} \comp \orange{A} \comp \cyan{P}^{-1}$ are \emph{similar} and thus share the same \emph{elementary divisors}. }
\end{mybox}}

\onslide<3->{
$$G = \cyan{P} \comp F \comp \cyan{Q} = \cyan{P} \comp \orange{A} \comp F \comp \orange{B} \comp \cyan{Q}  = \cyan{P} \comp \orange{A} \comp \cyan{P}^{-1} \comp G \comp \cyan{Q}^{-1}\comp \orange{B} \comp \cyan{Q} $$}

\onslide<4>{
\begin{theorem}[Alternative formulation]
Most of the known infinite APN families are made of \emph{linearly self-equivalent mappings} with \emph{very specific} mappings $\orange{A}, \orange{B}$. This can be detected independently of the representation.
\end{theorem}
}
\end{frame}

\begin{frame}[t]\frametitle{Example: Cyclotomic mappings}
\begin{recap}
$$ F(x) = x^{e}P\left(x^{2^{k} - 1}\right), n = \ell k$$


Univariate: $\orange{A} \comp F \comp \orange{B} = F$ with $\quad \orange{B}(x) = \blue{\lambda} x$, $\quad\orange{A}(x) = \blue{\lambda}^{-e} x$ for any $\lambda \in \blue{\FF_{2^{k}}^{*}}$

\vspace{.2cm}
Multivariate: $\widetilde{\orange{A}} \comp \widetilde{F} \comp \widetilde{\orange{B}} = \widetilde{F}$ with $\quad\widetilde{\orange{B}}(v) = (\blue{\lambda} v_{1}, \dotsc , \blue{\lambda} v_{\ell}),$ $\quad \widetilde{\orange{A}}(v) = (\blue{\lambda}^{-e} v_{1}, \dotsc , \blue{\lambda}^{-e} v_{\ell})$
\end{recap}\pause

\begin{proposition}[Up to linear equivalence]
    $F \from \FFspace \to \FFspace$. $F$ is linearly equivalent to a cyclotomic mapping w.r.t a subfield $\FF_{2^{k}}$ iff:

    \vspace{.2cm}
     $\exists \ \orange{A},\orange{B}$ such that $\orange{A} \comp F \comp \orange{B} = F$ and:

    \begin{itemize}
        \item[\bulletpoint] $\min(\orange{A}), \min(\orange{B})$ are \emph{irreducible} polynomials
        \item[\bulletpoint] $\ord(\orange{B}) = 2^{k} - 1$ and $\ord(\orange{A}) \mid \ord(\orange{B})$
    \end{itemize}
\end{proposition}




\end{frame}

\begin{frame}\frametitle{Linear self-equivalence and APN functions}

\begin{mybox}{Sum up}{}{}
\begin{itemize}
    \item[\bulletpoint] \emph{Pen-and-paper} functions:  linearly self-equivalent with \emph{very specific} $\orange{A},\orange{B}$
    \item[\bulletpoint] From \emph{computer searches}: most are linearly self-equivalent with \emph{less structured} $\orange{A},\orange{B}$.
\end{itemize}
\end{mybox}\pause

\begin{mybox}{The only solution to the big APN problem}{}{}
A single bijective APN mapping is known when $n$ is even. It is \emph{CCZ-equivalent} to the ``Kim mapping'':
$$ \kappa \from \FF_{2^{6}} \to \FF_{2^{6}}, X \mapsto X^{3} + X^{10} + uX^{24},$$
for some specific $u \in \FF_{2^{6}}$.

\pause
\vspace{.3cm}
$\kappa(X) = X^{3}(1 + X^{7} + uX^{21}) = X^{3}P(X^{2^{3}-1})$ \hfill \emph{cyclotomic w.r.t $\FF_{2^{3}}$}.
\end{mybox}


\end{frame}
\end{comment}

\begin{comment}
\begin{frame}[t]\frametitle{A (re)open problem}


\begin{mybox}{Question}{}{}
For an APN function $F$, does there always exist a \emph{CCZ-equivalent} function $G$ which is linear self-equivalent ($\orange{A} \comp G \comp \orange{B} = G$) ?
\end{mybox}\pause

\begin{mybox}{Element of answers}{}{}
\begin{itemize}
    \item[\bulletpoint] A \emph{data base} of the known functions (sporadic / infinite families) for small $n$.
    \item[\bulletpoint] Some of the properties of $\orange{A},\orange{B}$ are still preserved by \emph{affine and CCZ equivalences}.
\end{itemize}

\end{mybox}

\end{frame}



\begin{frame}\frametitle{More self-equivalent APN functions ?}

% \begin{mybox}{Linear self-equivalences of APN functions}{}{}
% \begin{itemize}
%     \item[\bulletpoint] (almost) All pen-and-paper functions are  linearly self-equivalent with \emph{very specific} $\orange{A},\orange{B}$
%     \item[\bulletpoint] A lot (all ?) functions from \emph{computer searches} are linearly self-equivalent with \emph{less structured} $\orange{A},\orange{B}$.
% \end{itemize}
% \end{mybox}

\begin{mybox}{Previous works}{}{}
Linearly self-equivalence to \emph{speed up searches} \hfill \purple{\small [BeiBriLea21,BeiLea22]}.
\end{mybox}\pause

\begin{mybox}{Toward new APN functions ?}{}{}
\begin{itemize}
    \item[\bulletpoint] \emph{Non-quadratic }linearly self-equivalent functions for $n =6$ ?
    \item[\bulletpoint] Cyclotomic mappings $F(x) = x^{e}P\left(x^{2^{k} - 1}\right)$ with \emph{non-quadratic} $e$ ?
    \item[\bulletpoint] $\ell$-projective mappings with \emph{$\ell > 4$} ?
\end{itemize}

\end{mybox}

\end{frame}
% \onslide<3->{
% \begin{mybox}{Sketch theorem}{}{}
% $F$ $\lin$ to a cyclotomic mapping iff \quad $\exists \ \cyan{A}, \cyan{B}$, such that $F = \cyan{A} \comp F \comp \cyan{B}$ with:

% \vspace{.1cm}
% $\cyan{B}$ (resp. $\cyan{A}$) with elementary divisors of $M_{\blue{\lambda}} : x \mapsto \blue{\lambda} x$ (resp. $M_{\blue{\lambda^{e}}}$ )
% % \begin{itemize}
% % 	\item[-] 
% % 	\item[-] $\cyan{A}$ with elementary divisors of $M_{\blue{\lambda^{e}}}: x \mapsto \blue{\lambda^{e}} x$ for some $e$.
% % \end{itemize}

% \onslide<4>{
% \vspace{.1cm}
% \hfill$\leadsto$ related to \emph{minimal polynomials} of $\cyan{A}, \cyan{B}$.
% }
% \end{mybox}
% }
\end{comment}


\subsection{Take away}
\begin{frame}
\frametitle{\subsecname}

\begin{theorem}[]
Among the 22 known infinite APN families, 19 consist entirely of 

\emph{cyclotomic} or \emph{$\ell$-projective} mappings, \emph{up to linear equivalence}.
\end{theorem}

\begin{mybox}{Sum up}{}{}
\begin{itemize}
\item[-] Characterization of \emph{very specific} self-equivalences
\item[-] Unify most of the approaches
\item[-] Partial answer to the \emph{detection} of such structures up to equivalence
\end{itemize}
\end{mybox}

\pause
\begin{mybox}{Open questions}{}{}
\begin{itemize}
\item[-] Link between self-equivalence and APN-ness\hfill\purple{\small [BeiBriLea21, Conjecture 1]}
% \item[-] Characterization up to CCZ equivalence?
\item[-] Cyclotomic mappings outside the known classes? (from \emph{non-quadratic} APN monomial)
\item[-] Projective mappings outside the known classes? (with \emph{more} coordinates)

\end{itemize}
\end{mybox}

\end{frame}



\appendix

\begin{frame}\frametitle{About the naming}

\begin{mybox}{Definition}{APN function}{\purple{\small[NybKnu92]}}\label{def:apn}
    A function $F$ is APN if: $\quad \forall \ \din\neq 0, \dout, \quad \delta_{F}(\din, \dout) \leq 2$.
\end{mybox}\pause

\begin{mybox}{The linear case}{}{}
$F$ linear.
$$ F(x + \din) + F(x) \quad  = \quad F(x) + F(\din) + F(x) \quad  = \quad F(\din)$$

$\din \neq 0$. $\quad \quad \delta_{F}(\din, \dout) = \left\{
    \begin{array}{ll}
        2^{n} & \text{ if } \dout = F(\din)\\
        0 & \text{ otherwise.}
    \end{array}
\right.$
\end{mybox}
\pause
\begin{mybox}{The APN case}{}{}
$F$ APN. Then $ \forall \ \din\neq 0, \quad \card{\set{\dout, \ \ \delta_{F}(\din, \dout) > 0 }} = 2^{n-1}$.
\end{mybox}



\end{frame}
% % !TEX root = ../slides.tex
\section{Appendix}
{
\setbeamercolor{background canvas}{bg=ptblue}	
\begin{frame}[plain]
\vfill
\begin{center}
%\color{white} \Huge \Roman{section} -  \secname
\color{white} \Huge \secname
\end{center}
\vfill
\end{frame}
}

	\subsection{The permutation}
	\begin{frame}[label=page3]
		\frametitle{\subsecname}
		\vspace{-.3cm}
		\begin{columns}
			\begin{column}{0.5\textwidth}
				\begin{mybox}{\centering A confusion/diffusion structure\dots}{}{}
				
						\begin{textblock*}{7cm}(.2cm,2.4cm)
							\centering
							{\scriptsize The state}
							\scalebox{0.4}{%%%%%%%%%%%%%%%%%%%%%%%%%%%%%%%%%%%%%%%%%%%%%%%%%%%%%%%%%%%%%%%%%%%%%%%%%%%%%%%%%%
% The Ascon state
%
% public domain (CC0 1.0 https://creativecommons.org/publicdomain/zero/1.0/)
%%%%%%%%%%%%%%%%%%%%%%%%%%%%%%%%%%%%%%%%%%%%%%%%%%%%%%%%%%%%%%%%%%%%%%%%%%%%%%%%%%



\newif\ifsans
\newif\iftext
\newif\ifcolor

%%% CONFIGURATION %%%%%%%%%%%%%%%%%%%%%%%%%%%%%%%%%%%%%%%%%%%%%%%%%%%%%%%%%%%%%%%%
%\sanstrue   % for sans-serif fonts (slides, web)
\sansfalse % for serif fonts (article)

\texttrue   % include phase description
%\textfalse % no phase description

\colortrue   % use color for highlights
%\colorfalse % no color

%\horizontaltrue % highlight horizontal word
%\verticaltrue % highlight vertical slice
%\constanttrue % highlight vertical slice
%%%%%%%%%%%%%%%%%%%%%%%%%%%%%%%%%%%%%%%%%%%%%%%%%%%%%%%%%%%%%%%%%%%%%%%%%%%%%%%%%%

\usetikzlibrary{shadows}
\ifcolor
  \definecolor{webred}{HTML}{D35400}
\else
  \definecolor{webred}{HTML}{444444}
\fi

\begin{tikzpicture}[thick, scale=0.5]
  \draw[fill=white, drop shadow] (0,0) rectangle (32,3.5);
  \draw (0,.7) -- (32,.7)
                (0,1.4) -- (32,1.4)
                (0,2.1) -- (32,2.1)
                (0,2.8) -- (32,2.8);
                
  \foreach \x in {1,...,63}
  \draw (0.5*\x, 0) -- (.5*\x, 3.5);
  
  \iftext
    \draw (33,.35) node {$X_4$}
         ++(0,.7) node {$X_3$}
         ++(0,.7) node {$X_2$}
         ++(0,.7) node {$X_1$}
         ++(0,.7) node {$X_0$};
  \fi

\end{tikzpicture}}
							\vspace{-.2cm}
							\[  p = {\color{oigreen} p_L} \comp {\color{oiorange} p_S}  \comp {\color{oiblue} p_C} \]
						\end{textblock*}
			
						\begin{textblock*}{7cm}(0.2cm,4.5cm)
							\centering
							\scriptsize The constant addition {\color{oiblue} $ p_C $}
							\scalebox{0.4}{%%%%%%%%%%%%%%%%%%%%%%%%%%%%%%%%%%%%%%%%%%%%%%%%%%%%%%%%%%%%%%%%%%%%%%%%%%%%%%%%%%
% The Ascon state
%
% public domain (CC0 1.0 https://creativecommons.org/publicdomain/zero/1.0/)
%%%%%%%%%%%%%%%%%%%%%%%%%%%%%%%%%%%%%%%%%%%%%%%%%%%%%%%%%%%%%%%%%%%%%%%%%%%%%%%%%%

\usetikzlibrary{shadows}


\begin{tikzpicture}[thick, scale=0.5]
	  
	
  \draw[thick, fill = white, drop shadow] (0,0) rectangle (32,3.5);
  \draw (0,.7) -- (32,.7)
                (0,1.4) -- (32,1.4)
                (0,2.1) -- (32,2.1)
                (0,2.8) -- (32,2.8);
                
  \foreach \x in {1,...,63}
  \draw (0.5*\x, 0) -- (.5*\x, 3.5);
  
    \draw (33,.35) node {$X_4$}
         ++(0,.7) node {$X_3$}
         ++(0,.7) node {$X_2$}
         ++(0,.7) node {$X_1$}
         ++(0,.7) node {$X_0$};

 
  \draw[line width =2, color=cblue,fill=white,drop shadow={shadow yshift=-.7ex}] (27.9,1.3) rectangle +(4.2,.9);
  \foreach \x in {1,...,7}
  \draw[line width = 1.5, color=cblue] (32-\x*0.5,1.3) -- (32-\x*0.5,2.2);

  \foreach \x in {1,...,8}
  \draw[color=cblue] (27.75+\x*0.5,1.75) node { $ \Plus $};

\end{tikzpicture}}
						\end{textblock*}
			

						\begin{textblock*}{7cm}(.2cm,5.9cm)
							\centering
							\scriptsize The substitution layer {\color{oiorange} $ p_S $} \scalebox{0.4}{%%%%%%%%%%%%%%%%%%%%%%%%%%%%%%%%%%%%%%%%%%%%%%%%%%%%%%%%%%%%%%%%%%%%%%%%%%%%%%%%%%
% The Ascon state
%
% public domain (CC0 1.0 https://creativecommons.org/publicdomain/zero/1.0/)
%%%%%%%%%%%%%%%%%%%%%%%%%%%%%%%%%%%%%%%%%%%%%%%%%%%%%%%%%%%%%%%%%%%%%%%%%%%%%%%%%%
\usetikzlibrary{shadows}

\begin{tikzpicture}[thick, scale=0.5]
  \draw[fill=white, drop shadow] (0,0) rectangle (32,3.5);
  \draw (0,.7) -- (32,.7)
                (0,1.4) -- (32,1.4)
                (0,2.1) -- (32,2.1)
                (0,2.8) -- (32,2.8);
                
  \foreach \x in {1,...,63}
  \draw (0.5*\x, 0) -- (.5*\x, 3.5);
  
    \draw (33,.35) node {$X_4$}
         ++(0,.7) node {$X_3$}
         ++(0,.7) node {$X_2$}
         ++(0,.7) node {$X_1$}
         ++(0,.7) node {$X_0$};
  	
    \draw[line width=2, color=corange,fill=white,drop shadow={shadow xshift=.7ex}] (2.4,-.6) rectangle +(.7, 4.7);
    \draw[line width =1.5, color=corange,<->] (2.75, -.1) -- (2.75, 3.6);
\end{tikzpicture}}

						\end{textblock*}
										
				
						\begin{textblock*}{7cm}(.2cm,7.4cm)
							\centering
							\scriptsize The linear layer {\color{oigreen} $ p_L $}
							\scalebox{0.4}{%%%%%%%%%%%%%%%%%%%%%%%%%%%%%%%%%%%%%%%%%%%%%%%%%%%%%%%%%%%%%%%%%%%%%%%%%%%%%%%%%%
% The Ascon state
%
% public domain (CC0 1.0 https://creativecommons.org/publicdomain/zero/1.0/)
%%%%%%%%%%%%%%%%%%%%%%%%%%%%%%%%%%%%%%%%%%%%%%%%%%%%%%%%%%%%%%%%%%%%%%%%%%%%%%%%%%

\usetikzlibrary{shadows}

\begin{tikzpicture}[thick, scale=0.5]
  \draw[fill=white, drop shadow] (0,0) rectangle (32,3.5);
  \draw (0,.7) -- (32,.7)
                (0,1.4) -- (32,1.4)
                (0,2.1) -- (32,2.1)
                (0,2.8) -- (32,2.8);
                
  \foreach \x in {1,...,63}
  \draw (0.5*\x, 0) -- (.5*\x, 3.5);
  

    \draw (33,.35) node {$X_4$}
         ++(0,.7) node {$X_3$}
         ++(0,.7) node {$X_2$}
         ++(0,.7) node {$X_1$}
         ++(0,.7) node {$X_0$};

    \draw[line width=2 , color=cgreen,fill=white,drop shadow={shadow yshift=-.7ex}] (-.4,2.05) rectangle +(32.8,.9);
    \draw[line width=1.5,color=cgreen,<->] (-.2, 2.5) -- (32.2, 2.5);
\end{tikzpicture}}
						\end{textblock*}
					
				\end{mybox}
			\end{column}

			\begin{column}{0.36\textwidth}
				\begin{mybox}{\centering \dots studied algebraically}{}{}
					\tiny
					\vspace{15pt}
					\begin{align*}
						y_0 &= {\color{oired} \bm{x_4x_1}} + x_3 + {\color{oired} \bm{x_2x_1}} + x_2 + {\color{oired} \bm{x_1x_0}} + x_1 + x_0 \\
						y_1 &= x_4 + {\color{oired} \bm{x_3x_2}} + {\color{oired} \bm{x_3x_1}} + x_3 + {\color{oired} \bm{x_2x_1}} + x_2 + x_1 + x_0 \\
						y_2 &= {\color{oired} \bm{x_4x_3}} + x_4 + x_2 + x_1 + 1\\
						y_3 &= {\color{oired} \bm{x_4x_0}} + x_4 + {\color{oired} \bm{x_3x_0}} + x_3 + x_2 + x_1 + x_0 \\
						y_4 &= {\color{oired} \bm{x_4x_1}} + x_4 + x_3 + {\color{oired} \bm{x_1x_0}} + x_1
					\end{align*}
					%\vspace{-22pt}
					\begin{center}
						\scriptsize Algebraic Normal Form (ANF) of the {\color{oiorange} S-box}
					\end{center}
					%\pause
					\vspace{5pt}
					\begin{align*}
						X_0 &= X_0 \xor \ (X_0 \ggg 19) \ \xor \ (X_0 \ggg 28) \\
						X_1 &= X_1 \xor \ (X_1 \ggg 61) \ \xor \ (X_1 \ggg 39)  \\
						X_2 &= X_2 \xor \ (X_2 \ggg 1) \ \xor \ (X_2 \ggg 6)  \\
						X_3 &= X_3 \xor \ (X_3 \ggg 10) \ \xor \ (X_3 \ggg 17)  \\
						X_4 &= X_4 \xor \ (X_4 \ggg 7) \ \xor \ (X_4 \ggg 41)
					\end{align*}
					%\vspace{-22pt}
					\begin{center}
						\scriptsize ANF of the linear layer {\color{oigreen} $ p_L $}
					\end{center}

				\end{mybox}
			
			\end{column}
		\end{columns}

	\end{frame}

	\subsection{The nonce-misuse scenario}
	\begin{frame}[label=page4]
		\frametitle{\subsecname}
		\begin{mybox}{Simplified setting of \Ascon-128}{}{}
			\begin{center}
				\scalebox{0.7}{\begin{tikzpicture}
	
	\definecolor{cblue}{HTML}{3F71A1}
	\definecolor{corange}{HTML}{E09F1F}
	\definecolor{cgreen}{HTML}{36C190}
	\definecolor{cred}{HTML}{C2554F}
	\definecolor{deepmagenta}{HTML}{9F0162}
	
	 \node[draw] at (8.5, 0) (init) [rounded corners=1ex, align=center,draw]  { \begin{tabular}{c}
			\color{cgreen} $ v_0, \ \cdots, v_{63} $	\\ \hline \hline
			\color{corange} $ a_0, \ \cdots, a_{63}$ 	\\ \hline
			\color{corange}$ b_0, \ \cdots, b_{63} $  	\\ \hline
			\color{corange}$ c_0, \ \cdots, c_{63} $  	\\ \hline
			\color{corange}$ d_0, \ \cdots, d_{63} $  	\\	
	\end{tabular}};


	\node (capacity) [align=center] [right of=init, node distance=3.3cm] {\small Unknown internal state};
	\node (externalstate) [align=center] [above of=capacity, node distance=.9cm] {\small Chosen external state};
	\node (caption) [spongephase, align = center, below of = capacity, left of = externalstate, left=0.4cm, below = 1.67cm] {{\color{cred}$ \Sigma_{\mathrm{E}} $} State before encryption};

    \SpongeInitInner{$k \| N$}{}
   \draw (P.south) +(-.6,-1) node[spongephase] (phase) {Initialization};
   %\SpongeXorInner{$\hskip-4pt 0^* \| k$}
   \SpongeStep\SpongePhaseSep[cred]{}
   
\SpongeEncrypt{$ \mathbf{P_0 }$~~}{~~$C_0 $}{$ p^6 $}{}{}
\draw (P.south) +(-1.5,-0.63) node[spongephase] (phase) {\color{cred}{$ \Sigma_{\mathrm{E}} $}};
\draw (P.south) +(0,-1.04) node[spongephase] (phase) {Encryption};
\SpongeEncrypt{$ 0^* $~~}{~~$ \mathbf{C_1} $}{}{}{}



\end{tikzpicture}
}
			\end{center}
		\vspace{10pt}
			\begin{itemize}[leftmargin=0cm]
				\item[-] Many reuse of the {\color{oiorange} same $ (k, N) $ pair}.
				\item[-] State recovery = {\color{oired} compromised confidentiality without interaction}.
				\item[-] {\color{oigreen} No trivial key-recovery nor forgery} in that case.
				\item[-] Different from the generic attack [ACNS:VauViz18].

			\end{itemize}
		\end{mybox}
	
	\end{frame}


\subsection{Conditional cube}
\begin{frame}
	\frametitle{\subsecname}
	\vspace{10pt}
%	\begin{center}
		\begin{itemize}[leftmargin=0.3cm]
			\item[-] We look for $ {\color{oiorange}\alpha_u} $ with a {\color{oiblue} simple divisor}: $ \color{oiorange}\beta_0 $.
			\item[-]  $\color{oiorange} \alpha_u $ {\color{oired} mostly unknown}, but we still get: $ {\color{oiorange}\alpha_u } = 1 \implies {\color{oiorange}\beta_0 } = 1$. 
			\item[-] If $ \color{oiorange}\beta_0  $ is linear, we get a { \color{oiblue} linear system}.
			
		\end{itemize}
%	\end{center}

		% \pause
				
		% 	\only<2>{\begin{center}
		% 		\scalebox{0.8}{\input{figures/trees_4.tex}}\end{center}}
		\begin{center}
				\scalebox{0.8}{\begin{tikzpicture}
	\node[inner sep=0pt] (s5) at (3.5,0) {${\color{oiorange}\alpha_u} v_0v_1v_2v_3v_4v_5v_{6}v_{7} $};
	\node[inner sep=0pt] (s52) at (3.4,-1.1) {${\color{oiorange}\alpha_u} = {\color{oiorange}\beta_0} \left( {\color{oiblue} \prod\limits_{i = 1}^7  ?} + {\color{oired} \prod\limits_{i = 1}^7  ?}  + \cdots \right)$};
	\node[inner sep=0pt] (r4) at (3.5,-2.7) {$ R_4 $};
	
	
	
	\node[inner sep=0pt] (trail0) at (0,2.5) {Trail $ t_0 $};
	\node[inner sep=0pt] (s41) at (0,1.2) {$ v_0v_1v_2v_3 $};
	\node[inner sep=0pt] (s42) at (0,-1.2) {$ v_4v_5v_6v_7 $};
	\node[inner sep=0pt] (r3l) at (0,-2.7) {$ R_3 $};
	
	
	\node[inner sep=0pt] (trail1) at (7,2.5) {Trail $ t_1 $};
	\node[inner sep=0pt] (s43) at (7,1.2) {$ v_0v_1v_6v_7 $};
	\node[inner sep=0pt] (s44) at (7,-1.2) {$ v_2v_3v_4v_5 $};
	\node[inner sep=0pt] (r3r) at (7,-2.7) {$ R_3 $};
	
	
	
	\node[inner sep=0pt] (s31) at (-2,1.8) {$ v_0v_1 $};
	\node[inner sep=0pt] (s32) at (-2,0.6) {$ v_2v_3 $};
	
	\node[inner sep=0pt] (s33) at (-2,-0.6) {$ v_4v_5 $};
	\node[inner sep=0pt] (s34) at (-2,-1.8) {$ v_6v_7 $};
	
	\node[inner sep=0pt] (r2l) at (-2,-2.7) {$ R_2 $};
	
	
	
	
	\node[inner sep=0pt] (s35) at (9,1.8) {$ v_0v_7 $};
	\node[inner sep=0pt] (s36) at (9,0.6) {$ v_1v_6 $};
	
	\node[inner sep=0pt] (s37) at (9,-1.8) {$ v_2v_5 $};
	\node[inner sep=0pt] (s38) at (9,-0.6) {$ v_3v_4 $};
	
	\node[inner sep=0pt] (r2r) at (9,-2.7) {$ R_2 $};
	
	
	
	
	\node[inner sep=0pt] (v01) at (-3.2,2) {$ {\color{oiorange} \beta_{0}}v_0 $};
	\node[inner sep=0pt] (v11) at (-3.2,1.6) {{\color{oiblue} ?}$ v_1 $};
	\draw[-,thick] (v01.east) -- (s31.west)
	node[above=10pt,midway,] {};
	\draw[-,thick] (v11.east) -- (s31.west)
	node[above=10pt,midway,] {};
	
	\node[inner sep=0pt] (v21) at (-3.2,0.8) {{\color{oiblue} ?}$v_2 $};
	\node[inner sep=0pt] (v31) at (-3.2,0.4) {{\color{oiblue} ?}$v_3 $};
	\draw[-,thick] (v21.east) -- (s32.west)
	node[above=10pt,midway,] {};
	\draw[-,thick] (v31.east) -- (s32.west)
	node[above=10pt,midway,] {};
	
	\node[inner sep=0pt] (v41) at (-3.2,-0.4) {{\color{oiblue} ?}$ v_4 $};
	\node[inner sep=0pt] (v51) at (-3.2,-0.8) {{\color{oiblue} ?}$ v_5 $};
	\draw[-,thick] (v41.east) -- (s33.west)
	node[above=10pt,midway,] {};
	\draw[-,thick] (v51.east) -- (s33.west)
	node[above=10pt,midway,] {};
	
	\node[inner sep=0pt] (v61) at (-3.2,-2) {{\color{oiblue} ?}$ v_6 $};
	\node[inner sep=0pt] (v71) at (-3.2,-1.6) {{\color{oiblue} ?}$ v_7 $};
	\draw[-,thick] (v61.east) -- (s34.west)
	node[above=10pt,midway,] {};
	\draw[-,thick] (v71.east) -- (s34.west)
	node[above=10pt,midway,] {};
	
	\node[inner sep=0pt] (r1l) at (-3.2,-2.7) {$ R_1 $};
	
	
	
	
	
	\node[inner sep=0pt] (v02) at (10.2,2) {${\color{oiorange} \beta_{0}} v_0 $};
	\node[inner sep=0pt] (v12) at (10.2,1.6) {{\color{oired} ?}$ v_7 $};
	\draw[-,thick] (v02.west) -- (s35.east)
	node[above=10pt,midway,] {};
	\draw[-,thick] (v12.west) -- (s35.east)
	node[above=10pt,midway,] {};
	
	\node[inner sep=0pt] (v22) at (10.2,0.8) {{\color{oired} ?}$v_1 $};
	\node[inner sep=0pt] (v32) at (10.2,0.4) {{\color{oired} ?}$ v_6 $};
	\draw[-,thick] (v22.west) -- (s36.east)
	node[above=10pt,midway,] {};
	\draw[-,thick] (v32.west) -- (s36.east)
	node[above=10pt,midway,] {};
	
	\node[inner sep=0pt] (v42) at (10.2,-0.4) {{\color{oired} ?}$ v_3 $};
	\node[inner sep=0pt] (v52) at (10.2,-0.8) {{\color{oired} ?}$ v_4 $};
	\draw[-,thick] (v42.west) -- (s38.east)
	node[above=10pt,midway,] {};
	\draw[-,thick] (v52.west) -- (s38.east)
	node[above=10pt,midway,] {};
	
	\node[inner sep=0pt] (v62) at (10.2,-2) {{\color{oired} ?}$ v_2 $};
	\node[inner sep=0pt] (v72) at (10.2,-1.6) {{\color{oired} ?}$ v_5$};
	\draw[-,thick] (v62.west) -- (s37.east)
	node[above=10pt,midway,] {};
	\draw[-,thick] (v72.west) -- (s37.east)
	node[above=10pt,midway,] {};
	
	\node[inner sep=0pt] (r1r) at (10.2,-2.7) {$ R_1 $};
	
	
	\draw[-,thick] (s31.east) -- (s41.west)
	node[above=10pt,midway,] {};
	\draw[-,thick] (s32.east) -- (s41.west)
	node[above=10pt,midway,] {};
	
	\draw[-,thick] (s33.east) -- (s42.west)
	node[above=10pt,midway,] {};
	\draw[-,thick] (s34.east) -- (s42.west)
	node[above=10pt,midway,] {};
	
	\draw[-,thick] (s35.west) -- (s43.east)
	node[above=10pt,midway,] {};
	\draw[-,thick] (s36.west) -- (s43.east)
	node[above=10pt,midway,] {};
	
	\draw[-,thick] (s37.west) -- (s44.east)
	node[above=10pt,midway,] {};
	\draw[-,thick] (s38.west) -- (s44.east)
	node[above=10pt,midway,] {};
	
	
	\draw[-,thick] (s41.east) -- (s5.west)
	node[above=10pt,midway,] {};
	\draw[-,thick] (s42.east) -- (s5.west)
	node[above=10pt,midway,] {};
	
	\draw[-,thick] (s43.west) -- (s5.east)
	node[above=10pt,midway,] {};
	\draw[-,thick] (s44.west) -- (s5.east)
	node[above=10pt,midway,] {};
	
	
		\end{tikzpicture}}\end{center}

\end{frame}

	\subsection{Choosing conditional cubes by forcing linear divisors}
	\begin{frame}[label=page5]
		\frametitle{\subsecname}
		\vspace{-.7cm}
		\begin{mybox}{\nth{1} round}{}{}
				\begin{center}
				\scalebox{0.85}{
    \begin{tikzpicture}
	
	
	\definecolor{cblue}{HTML}{3F71A1}
	\definecolor{corange}{HTML}{E09F1F}
	\definecolor{cgreen}{HTML}{36C190}
	\definecolor{cred}{HTML}{C2554F}
        
        \node (init)[rounded corners=1ex, align=center,draw]  [node distance=1.5cm]{\footnotesize \begin{tabular}{c}
        	{\color{cgreen} $ v_0 $}	\\ \hline
        	{\color{corange} $ a_0 $} 	\\
        	{\color{corange} $ b_0 $}  	\\ 
        	{\color{corange} $ c_0 $}   	\\ 
        	{\color{corange} $ d_0 $}   	\\	
        \end{tabular}};
        
        % \node (capacity) [align=center] [left of=init, node distance=2.2cm] {\scriptsize{{Unknown capacity}}};
   % \node (externalstate) [,align=center] [above of=capacity, node distance=.7cm] {\scriptsize{Chosen external state}};

    
    \node (s1)[right of = init, rounded corners=1ex, align=center,draw]  [node distance=3cm]{\footnotesize\begin{tabular}{c}
    		{\color{corange}$ (a_0 + 1)$}{\color{cgreen}$ v_0 $} $+ \cdots $		\\ \hline
    		{\color{cgreen}$ v_0 $} $ + \cdots$ 				\\ \hline
    		$ \cdots $  					\\ \hline
    		{\color{corange}$ (c_0 + d_0 + 1)$}{\color{cgreen}$ v_0$} $ + \cdots $  	\\ \hline
    		{\color{corange}$ a_0$}{\color{cgreen}$ v_0 $}$ + \cdots $
    \end{tabular}};

    
    \draw[-latex] ($ (s1.east) + (0.9, -0.3)$) -- ($ (s1.east) + (0.3, -0.3)$)  node[midway, right]{$\ \ \ {\color{corange}  \gamma_0} \vcentcolon = {\color{corange}c_0 + d_0 + 1} $};
        \draw[-latex] ($ (s1.east) + (0.9, 0.6)$) -- ($ (s1.east) + (0.3, 0.6)$)  node[midway, right]{$\ \ \  {\color{corange} \beta_0 } \vcentcolon = {\color{corange} a_0 + 1} $};
   

\draw[-latex] ($ (init.east) + (0.1,0) $ )-- ($ (s1.west) - (0.1, 0)$) node[midway, below]{ \footnotesize $ S_1 $};
	\end{tikzpicture}}\\
			\end{center}
		\end{mybox}
	\vspace{-8pt}
	\pause
		\begin{mybox}{\nth{2} round}{}{}
			\begin{itemize}[leftmargin=0.3cm]
				\item[-]For any $\color{oigreen} v_0v_i $, $  i \neq 0 $:
				$ {\color{oiorange} \beta_0}P + {\color{oiorange} 1}Q  + {\color{oiorange} \gamma_0}R +  ({\color{oiorange}\beta_0 + 1})S$. 
				\item[-] But for {\color{oired} some} $ i $: $ {\color{oiorange} \beta_0}P $ or $ {\color{oiorange} \gamma_0}R $.
			\end{itemize}
			
%			\begin{itemize}[leftmargin=0.5cm]
%				\item[-] For any $\color{oigreen} v_0v_i $, $  i \neq 0 $:
%				$ {\color{oiorange} \beta_0}P + {\color{oiorange} 1}Q  + {\color{oiorange} \gamma_0}R +  ({\color{oiorange}\beta_0 + 1})S$.
%				\vspace{5pt}
%				\pause
%				\item[-] coefficient for {\color{oired} some} $ i $: $ {\color{oiorange} \beta_0}P $ or $ {\color{oiorange} \gamma_0}R $.
%			\end{itemize}
		\end{mybox}



	\end{frame}


	\begin{frame}
	    
	
	\begin{mybox}{\nth{6} round}{}{}
		\frametitle{\subsecname}

					\begin{itemize}[leftmargin=0.3cm]
				\item[-] With {\color{oired} chosen u},
				$ {\color{oiorange}\alpha_{u, \ j}} = {\color{oiorange}\beta_0}(\dots) + {\color{oiorange}\gamma_0}(\dots) $ , for all output coordinates. % ($ j \in \intset{0, 63}$).
				\pause
%				\vspace{5pt}
				\item[-] $ \left({\color{oiorange}\alpha_{u, 0} }, \cdots,{\color{oiorange}\alpha_{u, 63} }\right) \neq (0,\cdots, 0) \implies {\color{oiorange}\beta_0} = 1 $ or ${\color{oiorange}\gamma_0} = 1  $
				\pause
				\item[-] In practice, {\color{oired} reciprocal also true!} $ [\,{\color{oiorange}\alpha_{u, \ j}} = 0, \ \forall \ j\ ]\,\ \implies {\color{oiorange}\beta_0} = 0 $ and ${\color{oiorange}\gamma_0 } = 0   $
				\vspace{5pt}

				\begin{columns}
			\begin{column}{5cm}
				\includegraphics[width=5cm]{figures/test0} %\\
				%\scriptsize \centering Individual cancellations of each $ \color{oiorange} \alpha_{u,j} $\\ (1000 random internal states)
			\end{column}
			
			\begin{column}{5cm}
			\includegraphics[width=5cm]{figures/test1} %\\
%				\scriptsize \centering Hamming weight of the cube-sum vectors \\(1000 random internal states)
			\end{column}
		\end{columns}

\end{itemize}
		\end{mybox}
	\end{frame}


\begin{frame}\frametitle{Midori}
\includegraphics[width=14.5cm]{figures/midori_1}
\end{frame}
\begin{frame}\frametitle{Midori}
\includegraphics[width=14.5cm]{figures/midori_2}
\end{frame}
\begin{frame}\frametitle{Midori}
\includegraphics[width=14.5cm]{figures/midori_3}
\end{frame}

\begin{frame}\frametitle{Walsh spectrum of cyclotomic mappings}
\includegraphics[width=14.5cm]{figures/wcc_1}
\end{frame}
\begin{frame}\frametitle{Streebog}
\includegraphics[width=14.5cm]{figures/wcc_2}
\end{frame}

\begin{frame}\frametitle{}
\includegraphics[width=14.5cm]{figures/thesis_1}
\end{frame}
\begin{frame}\frametitle{}
\includegraphics[width=14.5cm]{figures/thesis_2}
\end{frame}
\begin{frame}\frametitle{}
\includegraphics[width=14.5cm]{figures/thesis_3}
\end{frame}

\begin{frame}\frametitle{Brinckmann-Leander-Edel-Pott APN cubic}
\begin{itemize}
	\item[-] 7 non-trivial automorphisms
	\item[-] Elementary divisors for $\calLL$ : $(X+1) $multiplicity 2, $(X+1)^{2}$ multiplicity 5
	\item[-]  If $\calLL \sim \diag(A, B)$ then $(X+1)^{2}$ is among the elementary divisors of $A$ $\implies$ $\min(A) = (X+1)^{2}$ not irreducible.
	\item[-] Cannot be cyclotomic mappings nor $\ell$-projective mappings.
\end{itemize}
\end{frame}

\begin{frame}\frametitle{}
\includegraphics[width=13cm]{figures/thesis_4}
\end{frame}
\begin{frame}\frametitle{}
\includegraphics[width=13cm]{figures/thesis_5}
\end{frame}	

\end{document}