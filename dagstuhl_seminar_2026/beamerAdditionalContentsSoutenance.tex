\section{Additional slides}


\begin{frame}{Two additional improvements}
\flushleft
\small
\vspace{-3mm}
\begin{enumerate}
\item {\bf Sieving technique.}

\small
\begin{alertblock}{}
\centering
Use the differential constraints to filter out pairs that \ver{cannot follow the differential}, \rouge{regardless of the value of the key}.
\end{alertblock}
\smallskip 
{\bf Pre-sieving}: Apply a sieve on all \rouge{S-boxes of the external rounds}
\smallskip

$\quad\rightarrow$ The key recovery is performed on \ver{$N' \leq N$ pairs}.
\small 

\bigskip
\small

\item {\bf Precomputing partial solutions.}

\small
\begin{alertblock}{}
\centering
\ver{Precompute} the partial solutions to some \rouge{subgraph}.
\end{alertblock}

\begin{itemize}
\item \bleu{Impact} on the \ver{memory complexity} and the \ver{offline time} of the attack.
\smallskip
\item The \rouge{optimal key recovery strategy} depends on the memory and offline time allowed. 
\end{itemize}
\end{enumerate}

\end{frame}

%%%%%%%%%%

\begin{frame}{Sieving}

%\begin{exampleblock}{Idea behind the pre-sieving}
%{\color{blue} Reduce the number of pairs} as quickly as possible to only keep the $N' \leq N$ pairs that satisfy the \rouge{differential constraints}.
%\end{exampleblock}
\bigskip

\begin{alertblock}{}
\centering
{\bf Idea:} Use the differential constraints to filter out pairs that \ver{cannot follow the differential}, \rouge{regardless of the value of the key}.
\end{alertblock}

\medskip
\begin{minipage}{3cm}
\begin{itemize}
\item Example:
\end{itemize}
\end{minipage}
\begin{minipage}{5cm}
${\tt (x_3,x'_3,x_2,x'_2,x_1\oplus x'_1,x_0\oplus x'_0)}$ 

\medskip
{\color{blue} Filter}: $\displaystyle  36/2^6 = \rouge{2^{-0.83}}.$
\end{minipage}
\begin{minipage}{3cm}
\medskip
\begin{center}
\includegraphics[width=1.2cm]{figures/presieving.png}
\end{center}
\end{minipage}

\bigskip

\pause
\begin{exampleblock}{Pre-sieving}
Apply a sieve on all \rouge{S-boxes of the external rounds}. \\
\smallskip
\centering
{\bf Advantage :} The key recovery is performed on \ver{$N' \leq N$ pairs}.
\end{exampleblock}
\end{frame}

%%%%%%%%%%

\begin{frame}{Precomputing partial solutions}
\begin{block}{Idea}
{\color{blue} Precompute} the partial solutions to some \rouge{subgraph}.
\end{block}

\begin{center}
\includegraphics[width=2cm]{figures/precomputation1.png}
\end{center}

\begin{itemize}
\item {\color{blue} Impact} on the \ver{memory complexity} and the \ver{offline time} of the attack.
\bigskip
\item The \rouge{optimal key recovery strategy} depends on how much memory and offline time are allowed. 
\end{itemize}
\end{frame}

%%%%%%%%%%%

\begin{frame}{Other contributions of our paper [BHLS24]}

\flushleft
\footnotesize

\vspace{-2mm}
\begin{table}[t!]
\centering
\begin{tabular}{cccc}
\toprule
Target & Type & Time & Source \\
\midrule
\multirow{5}*{XOR} & \multirow{5}*{Preimage} & $2^{5n/6} \approx 2^{0.833n} $ & [EC:LeuWan15] \\
& & $2^{2n/3} \approx 2^{0.667n} $ & [EC:Dinur16] \\
& & $2^{5n/8} \approx 2^{0.625n} $ & [C:BWGG17] \\
& & $2^{11n/18} \approx 2^{0.611n} $ & [JC:BDGLW20] \\
& & \rouge{$2^{3n/5} \approx 2^{0.6n}$} & \rouge{Our work} \\
\midrule
\multirow{3}*{Zipper} & Second preimage, $L' \leq 2^{n/2}$ & $2^{5n/8} \approx 2^{0.625n} $ & [C:BWGG17] \\
& Second preimage & $2^{3n/5} \approx 2^{0.6n}$ & [C:BWGG17] \\
& Second preimage & \rouge{$2^{7n/12} \approx 2^{0.583n}$}& \rouge{Our work} \\
\midrule
\multirow{3}*{Hash-Twice} & \multirow{3}*{Second preimage} & $2^{2n/3} \approx 2^{0.667n}$ & [SAC:ABSK09] \\
& & $2^{13n/22} \approx 2^{0.591n}$ & [JC:BDGLW20] \\
& & \rouge{$2^{15n/26} \approx 2^{0.577n}$}& \rouge{Our work} \\
%& \multirow{2}*{Second preimage, quantum} & $2^{n/2}$ & Brute-force \\
%& & \rouge{$2^{3n/7} \approx 2^{0.459n}$} & \rouge{Our work} \\
\bottomrule
\end{tabular}
\end{table}

\end{frame}

%%%%%%%%%%

\begin{frame}{Nesting exceptional functions}
\small
\vspace{-2mm}

\begin{enumerate} 
\setlength{\itemsep}{4pt}
\item Find $\cipherblock$ s.t. $h_{\cipherblock}$ has cycle length $ \leq 2^{\mu}$. \hfill \colorbox{rouge3!20}{$2^{c - \mu}$}.
\item Find $\gamma$ s.t. $g_{\cipherblock,\altblock}$ has cycle length $\approx 2^{\nu}$.  \hfill \colorbox{rouge3!20}{$2^{c/2 + \mu - \nu}$}
\begin{itemize}
\setlength{\itemsep}{2pt}
\item $g_{\cipherblock,\altblock}$ has a domain of size $2^{\mu}$ $\rightarrow$ try about \colorbox{rouge3!20}{$2^{\mu/2 - \nu}$} different $\gamma$'s.
\item Checking the cycle length: \colorbox{rouge3!20}{$\approx 2^{\mu/2}$} applications of $g_{\cipherblock,\altblock}$.
\item One application of $g_{\cipherblock,\altblock}$ costs \colorbox{rouge3!20}{$L \approx 2^{c/2}$} applications of $h$. 
\end{itemize}
\item One must try \colorbox{rouge3!20}{$2^{\nu}$} tags using messages of length \colorbox{rouge3!20}{$2^{(c+\mu)/2}$}.\hfill \colorbox{rouge3!20}{$2^{c/2 +  \mu/2 + \nu}$.}
\end{enumerate}

\begin{exampleblock}{\small Optimal trade-off}
\centering
\small For $\mu = 2c/7$ and $\nu = c/14$, we obtain a balanced total complexity \bleu{$2^{5c/7} < 2^{3c/4}$}.
\end{exampleblock}

\end{frame}

\begin{frame}{A new improvement [BHLS24]}

\small
\begin{alertblock}{}
\centering
Define \rouge{\it nested exceptional functions} from the cycle $\mathcal{C}$ of $h_{\beta}$ to the cycle.
\end{alertblock}

\footnotesize

\begin{itemize}
\item When $L \gg 2^{c/2}$, processing ciphertexts of the form:
\end{itemize}

\vspace{-5mm}
\begin{center}
 \begin{tikzpicture}[xscale=.6,yscale=.6,
    level distance=1.2cm, sibling distance=1cm,
    edge from parent/.style={draw,<-}]
    
  \foreach \x in {0,1,2,3}{
 
    	\foreach \i in {0,1,...,7} {
      		\node[point] (b\i\x) at ($(7*0.9*\x + \i*0.9,0)$) {};
		\node (braised\i\x) at ($(7*0.9*\x + \i*0.9,0.77)$) {};
    	}    
    	 \foreach \i in {1,2,4,5,6} {
           	\node (bb\i\x) at ($(7*0.9*\x + 0.5+\i*0.9,0.5)$) {{\scriptsize$\cipherblock$}};
    	}  
   	 \foreach \i in {0} {
           	\node (bb\i\x) at ($(7*0.9*\x + 0.5+\i*0.9,0.5)$) {{\scriptsize\ver{$\gamma$}}};
    	}    
    	\foreach \i\j in {1/2,2/3,4/5,5/6,6/7} {
     		 \draw[->] (b\i\x) to (b\j\x);
    	}
	\foreach \i\j in {3/4} {
		\draw[dotted] (b\i\x) to (b\j\x) ;
   	 }
      	\foreach \i\j in {0/1} {
      		\draw[vert3,->] (b\i\x) to (b\j\x);
   	 }
	 
	\draw[line width=0.5pt,densely dotted,-latex,in=-160,out=-20] (b0\x) to (b7\x) ;
	\node at ($ (b0\x)!0.5!(b7\x) $) [below=3.8mm] {\scriptsize $g_{\beta, \ver{\gamma}}$};
	
	\draw[densely dashed, decorate, decoration={brace, amplitude=5pt}, thick, color=NavyBlue!70] (braised4\x) to (braised7\x); 
	\node at ($ (braised4\x)!0.5!(braised7\x) $) [above=2mm] {\scriptsize \bleu{$\in \mathcal{C}$}};
   }
   \onslide<3>{
     \foreach \x in {2,3}{
   		\node[rouge3,point] (blow7\x) at ($(7*0.9*\x + 7*0.9,-1.3)$) {};
   }
   	\draw[densely dashed, decorate, decoration={brace, amplitude=5pt}, thick, color=rouge3!70] (blow73.east) to (blow72.west); 
		\node at ($ (blow72)!0.5!(blow73) $) [below=2mm] {\scriptsize \rouge{$\in \mathcal{C'}$}};
%   	\draw[densely dashed, decorate, decoration={brace, amplitude=5pt}, thick, color=NavyBlue!70] (braised4\x) to (braised7\x); 
}
\end{tikzpicture}
\end{center}

\vspace{-5mm}
\flushleft $\quad$ iterates the function

\vspace{-6mm}
\begin{align*}
g_{\cipherblock,\ver{\altblock}}: \bleu{\mathcal{C}} &\to \bleu{\mathcal{C}} \\
  x &\mapsto h^*( \ver{\altblock} \| \cipherblock^{\cipherlength},x) \,.
 \end{align*}
%\end{exampleblock}
\pause

\vspace{-3mm}
\begin{itemize}
\setlength{\itemsep}{4pt}
\item Finding an `exceptional' $\beta$ i.e. s.t. \bleu{$\mathcal{C}$} has length $2^{\mu}$ costs \colorbox{NavyBlue!20}{$2^{c-\mu}$}. \text{[GKHR23]}
\pause
\item Eventually, the \rouge{cycle $\mathcal{C'}$} of $g_{\cipherblock,\ver{\altblock}}$ is reached.
\end{itemize}
\end{frame}

%%%%%%%%%%

\begin{frame}{Nested exceptional functions}

\footnotesize

\begin{figure}
  \centering
  % \colorlet{colorA}{red!0!green!50!white}
  % \colorlet{colorB}{red!33!green!50!white}
  % \colorlet{colorC}{red!67!green!50!white}
  % \colorlet{colorD}{red!100!green!50!white}
  
  \colorlet{colorA}{red!50!white}
  \colorlet{colorB}{colorA>wheel,1,4}
  \colorlet{colorC}{colorA>wheel,2,4}
  \colorlet{colorD}{colorA>wheel,3,4}
  
  \begin{tikzpicture}[xscale=.6,yscale=.6,
    level distance=1.2cm, sibling distance=1cm,
    edge from parent/.style={draw,<-}]
    \foreach \i in {0,1,...,6} {
      \node[point] (x\i) at ($(\i*360/7:1.25cm)$) {};
      \node at ($(\i*360/7:0.85cm)$) {{\scriptsize$x_{\i}$}};
    }    
    \foreach \i\j in {0/1,1/2,2/3,3/4,4/5,5/6,6/0} {
      \draw[->] (x\i) to (x\j);
    }
    \node[ipoint] at (x1) {} child[grow=right,shift={(0,-.3)}]
    { node[point] (tx6) {}
      child { node[point] (tx5) {} child { node[point] {} } child { node[point] (tx4) {} } }
      child { node[point] {} }
    };
    \node[ipoint] at (x2) {} child[grow=north east]
    { node[point] (tx3) {}
      child { node[point] {} }
      child { node[point] (tx2) {} child { node[point] {} } }
    };
    \node[ipoint] at (x3) {} child[grow=north west]
    { node[point] {}
      child { node[point] {} child { node[point] {} } }
    };
    \node[ipoint] at (x5) {} child[grow=west,shift={(0,-0.2)}]
    { node[point] (tx7) {}
      child { node[point] {} child { node[point] {} } child { node[point] {} } }
    };
    \node[ipoint] at (x6) {} child[grow=south]
    { node[point] (tx1) {}
      child { node[point] {} } child { node[point] (tx0) {} }
    };

    \foreach \i in {0,1,...,6} {
      \node[point] (xx\i) at ($(12,0)+(\i*360/7:1.25cm)$) {};
      \node at ($(12,0)+(\i*360/7:1.65cm)$) {{\scriptsize $x_{\i}$}};
    }    
    
    \begin{scope}[on background layer]
    
    \only<2->{
      \draw[line width=2pt,colorA,->] (xx0) -- (xx4);
      }
      \only<3->{
      \draw[line width=2pt,colorB,->] (xx4) -- (xx2);
      }
      \only<4->{
      \draw[line width=2pt,colorC,->] (xx2) -- (xx1);
      }
      \only<5->{
      \draw[line width=2pt,colorD,->,in=-45,out=15,looseness=25] (xx1) to (xx1);
}
    \only<2->{
      \node[point,minimum size=6pt,fill=colorA] at (x0) {};
      \draw[dashed,->,line width=2pt,colorA,bend right=40] (x0) to node[fill,circle] (g1) {} (tx2);
      \draw[line width=2pt,colorA,->] (tx2) -- (tx3) -- (x2) -- (x3) -- (x4);
}
\only<3->{
      \node[point,minimum size=6pt,fill=colorB] at (x4) {};
      \draw[dashed,->,line width=2pt,colorB,bend right=75,looseness=1.2] (x4) to node[fill,circle] (g2) {} (tx4);
      \draw[line width=2pt,colorB,->] (tx4) -- (tx5) -- (tx6) -- (x1) -- (x2);
}
\only<4->{
      \node[point,minimum size=6pt,fill=colorC] at (x2) {};
      \draw[dashed,->,line width=2pt,colorC,bend right=20,looseness=1] (x2) to node[pos=0.3,fill,circle] (g3) {} (tx0);
      \draw[line width=2pt,colorC] (tx0) -- (tx1) -- (x6);
      \draw[line width=2pt,colorC,transform canvas={shift={(154.3:-2pt)}}] (x6) -- (x0);
      \draw[line width=2pt,colorC,arrows={-To[harpoon,swap]},transform canvas={shift={(-154.3:-2pt)}}] (x0) -- (x1);
}
\only<5->{
      \node[point,minimum size=6pt,fill=colorD] at (x1) {};
      \draw[dashed,->,line width=2pt,colorD,bend right=90,looseness=1.75] (x1) to node[pos=0.6,fill,circle] (g4) {} (tx7);
      \draw[line width=2pt,colorD] (tx7) -- (x5) -- (x6);
      \draw[line width=2pt,colorD,transform canvas={shift={(154.3:2pt)}}] (x6) -- (x0);
      \draw[line width=2pt,colorD,arrows={-To[harpoon]},transform canvas={shift={(-154.3:2pt)}}] (x0) -- (x1);
      }
    \end{scope}
\only<2->{
    \node at (g1) {$\gamma$};
    }
    \only<3->{
    \node at (g2) {$\gamma$};
    }
    \only<4->{
    \node at (g3) {$\gamma$};
    }
    \only<5->{
    \node at (g4) {$\gamma$};
    }

\only<2->{
    \node[fill=colorA] at (7.5, 3  ) {{\scriptsize$g_{\cipherblock,\altblock}(x_0)=x_4$}};
    }
    \only<3->{
    \node[fill=colorB] at (7.5, 1.5) {{\scriptsize$g_{\cipherblock,\altblock}(x_4)=x_2$}};
    }
    \only<4->{
    \node[fill=colorC] at (7.5, 0  ) {{\scriptsize$g_{\cipherblock,\altblock}(x_2)=x_1$}};
    }
    \only<5->{
    \node[fill=colorD] at (7.5,-1.5) {{\scriptsize$g_{\cipherblock,\altblock}(x_1)=x_1$}};
    }
    \node at (12.3,3) {{\scriptsize Graph of $g_{\cipherblock,\altblock}$}};
    \node at (-1,3.5) {{\scriptsize Graph of $h_\cipherblock$}};

  \end{tikzpicture}



\end{figure}





\end{frame}

%%%%%%%%%%

